\documentclass[11pt,a4paper, 
swedish, english %% Make sure to put the main language last!
]{article}
\pdfoutput=1

%% Andréas's custom package 
%% (Will work for most purposes, but is mainly focused on physics.)
\usepackage{../custom_as}
\usepackage{cancel}
%% Figures can now be put in a folder: 
\graphicspath{ {figures/} %{some_folder_name/}
}

%% If you want to change the margins for just the captions
\usepackage[]{caption}

%% To add todo-notes in the pdf
\usepackage[%disable  %%this will hide all notes
]{todonotes} 

%% Change the margin in the documents
\usepackage[
%            top    = 3cm,              %% top margin
%            bottom = 3cm,              %% bottom margin
%            left   = 3cm, right  = 3cm %% left and right margins
]{geometry}

\newcommand{\enull}{\ensuremath{\varepsilon_{0}}}
\newcommand{\lD}{\ensuremath{\lambda_{\text{D}}}}
\newcommand{\wc}{\ensuremath{\omega_{\text{c}}}}
\newcommand{\rL}{\ensuremath{r_{\text{L}}}}
\newcommand{\vL}{\ensuremath{v_{\text{L}}}}
\newcommand{\mee}{\ensuremath{m_{\text{e}}}}
\newcommand{\nee}{\ensuremath{{n_{\text{e}}}}}
\newcommand{\nuen}{\ensuremath{{\nu_{\text{en}}}}}
\newcommand{\ve}{\ensuremath{{\vb*{v}_{\text{e}}}}}
\newcommand{\wpe}{\ensuremath{{\omega_{\text{p}}}}}

%% If you want to chage the formating of the section headers
\renewcommand{\thesubsection}{\arabic{section}.\Alph{subsection})}



%%%%%%%%%%%%%%%%%%%%%%%%%%%%%%%%%%%%%%%%%%%%%%%%%%%%%%%%%%%%%%%%%%%%%%
\begin{document}%% v v v v v v v v v v v v v v v v v v v v v v v v v v
%%%%%%%%%%%%%%%%%%%%%%%%%%%%%%%%%%%%%%%%%%%%%%%%%%%%%%%%%%%%%%%%%%%%%%


%%%%%%%%%%%%%%%%%%%% vvv Internal title page vvv %%%%%%%%%%%%%%%%%%%%%
\title{Assignment 2 \\
{\Large Plasma Physics -- RRY085}}
\author{Andréas Sundström}
\date\today%{2017-09-17}

\maketitle

%%%%%%%%%%%%%%%%%%%% ^^^ Internal title page ^^^ %%%%%%%%%%%%%%%%%%%%%
%% If you want a list of all todos
%\todolist

\section{Collisional damping}
In this problem, we study how high frequency, transvers EM-waves
propagating through a cold, weakly ionized, unmagnetized, static and
uniform plasma are affected by collisions with neutrals in the
plasma. These collisons manifest them selves through a collisional term
$-\mee\nee\nuen\ve$ in the fluid EOM, where $\nuen$ is the the
electron-neutral collision rate and is assumed to be constant.

The electron fluid equation now becomes
\begin{align}
\pdv{\nee}{t}+\div\qty[\nee\ve]&=0\\
\label{eq1:EOM0}
\mee\nee\qty[\pdv{\ve}{t}+\ve\vdot\grad\ve]
&=-e\nee\qty[\vb*E+\ve\cross\vb*B]-\grad P_\ee 
-\mee\nee\nuen\ve\\
\dv{P_\ee}{t}&=\frac{\gamma_\ee P_\ee}{\nee^{\gamma_\ee}}\dv{\nee}{t}
\end{align}
Since the plasma is assumed to be cols and unmagnetized we can brop
the preassure and magnetic terms in the EOM. Next, we assume that 
$\ve=\ve_0+\ve_1$, where $\ve_1$ is a \emph{small} correction, and we
linearize. Since the plasma was static and uniform, any derivative of
zeroth order variables vanish, and if we also discard any higher than
linear order correction we get the EOM
\begin{equation}\label{eq1:EOM}
\mee\nee\pdv{\vee_1}{t} = -e\nee_0\vb*E_1-\mee\nee_0\nuen\vb*\ve_1.
\end{equation}

Using the Fourier transform
\begin{equation}
g(\vb*x, t) = \oldint_{\vb*k}\oldint_{\omega}
g(\vb*k, \omega)\ee^{\ii\vb*k\vdot\vb*x-\ii\omega t}
\id^3k\id\omega,
\end{equation}
and denoting the Fouriertansforms of the first order correction with
tildes, we can write \eqref{eq1:EOM} as
\begin{equation}
\qty[-\ii\omega\mee\nee + \mee\nee_0\nuen]
\tilde{\vb*v}_\ee = -e\nee\widetilde{\vb*E}.
\end{equation}
This can be rewritten as
\begin{equation}
\tilde{\vb*v}_\ee 
=\frac{-e\widetilde{\vb*E}}{-\ii\omega\mee + \mee\nuen}
=\frac{-e\ii\widetilde{\vb*E}}{\mee(\omega + \ii\nuen)}.
\end{equation}

Then the current can be written as
\begin{equation}
\widetilde{\vb*J} = -e\nee_0\tilde{\vb*v}_\ee 
=\frac{+\ii e^2\nee_0 \widetilde{\vb*E}}{\mee(\omega + \ii\nuen)}
=\frac{\ii \enull\wpe^2}{\omega + \ii\nuen}\widetilde{\vb*E}
=\sigma\widetilde{\vb*E},
\end{equation}
where the ions have been neglected due to them being much heavier than
the electrons. The dielectric tensor then becomes
\begin{equation}
\epsilon_{ij}=\delta_{ij}+\frac{\ii \sigma_{ij}}{\enull\omega}
=\qty(1-\frac{\wpe^2}{\omega^2(1 + \ii\nuen/\omega)})\delta_{ij}.
\end{equation}
The dispersion relation for the transverse waves can be found through
the wave equation
\begin{equation}
\epsilon\vb*E_\perp = \frac{k^2c^2}{\omega^2}\vb*E_\perp,
\end{equation}
resulting in
\begin{equation}
\begin{aligned}
0=&\omega^2-\frac{\wpe^2}{1 + \ii\nuen/\omega}-k^2c^2\\
\qty{\text{High freq.}}\approx&
\omega^2-\wpe^2\qty(1-\ii\frac\nuen\omega)-k^2c^2.
\end{aligned}
\end{equation}
Now we can assume that $\omega=\omega_0+\eta\omega_1$, where
$\eta=\nuen/\omega_0\ll1$. Keeping only terms of first order
in $\eta$, we get
\begin{equation}
0=\omega_0^2+2\eta\omega_0\omega_1
-\wpe^2\bigg[1-
\overbrace{\frac{\ii\nuen}{\omega_0}}^{=\ii\eta}
\qty(1-\cancel{\eta\frac{\omega_1}{\omega_0}})\bigg]
-k^2c^2+\order{\eta^2}.
\end{equation}
We now get two equations for te different orders of $\eta$:
\begin{equation}\label{eq1:omega0}
\omega_0^2-\wpe^2-k^2c^2=0
\quad\Longrightarrow\quad
\omega_0=\sqrt{\wpe^2+k^2c^2}
\end{equation}
and
\begin{equation}
2\omega_0\omega_1 +\ii\wpe^2=0
\quad\Longrightarrow\quad
\omega_1=-\ii\frac{\wpe^2}{2\omega_0}.
\end{equation}
We see that the introduction of a collisoin term has made the
frequency complex through
\begin{equation}
\omega=\omega_0-\ii\frac{\nuen\wpe^2}{2\omega_0^2}
+\order{\frac{\nuen^2}{\omega_0^2}},
\end{equation}
where $\omega_0$ is given in \eqref{eq1:omega0}, which results in
an exponential decay of the wave. 

\section{Two-stream instability (fluid model)}


\section{Two-stream instability (kinematic model)}




%%%%%%%%%%%%%%%%%%%%%%%%%%%%%%%%%%%%%%%%%%%%%%%%%%%%%%%%%%%%%%%%%%%%%%
\end{document}%% ^ ^ ^ ^ ^ ^ ^ ^ ^ ^ ^ ^ ^ ^ ^ ^ ^ ^ ^ ^ ^ ^ ^ ^ ^ ^ ^
%%%%%%%%%%%%%%%%%%%%%%%%%%%%%%%%%%%%%%%%%%%%%%%%%%%%%%%%%%%%%%%%%%%%%%
%  LocalWords:  Debye Laramor
