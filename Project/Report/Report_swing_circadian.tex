\documentclass[11pt,letter, english,%twocolumn
]{article}
\pdfoutput=1

\usepackage[oldint]{custom_as}

\graphicspath{ {figures/} }

%%Drar in tabell och figurtexter
\usepackage[margin=10 pt]{caption}
%%För att lägga in 'att göra'-noteringar i texten
\usepackage{todonotes} %\todo{...}

%%För att själv bestämma marginalerna. 
\usepackage[
            top    = 2cm,
            bottom = 2cm,
            left   = 2cm, right  = 2cm
]{geometry}

\DeclareMathAlphabet{\mathpzc}{OT1}{pzc}{m}{it}
\newcommand{\oh}{\ensuremath\mathpzc{o}}




\newcommand{\T}{{\!\mathsfrm{T}}}
\newcommand{\I}{{\mathbfrm{I}}}
\newcommand{\PC}{P_{\text{C}}}
\newcommand{\PN}{P_{\text{N}}}



\usepackage{fancyhdr}
\pagestyle{fancy}
\lhead{AMATH\,732}
\chead{Swings and Biorhythms}
\rhead{Avery, Sundstr{\"o}m}

\usepackage{indentfirst}

\begin{document}

%%%%%%%%%%%%%%%%% vvv Inbyggd titelsida vvv %%%%%%%%%%%%%%%%%

\title{\vspace{-1cm}
Swings and Biorhythms --- Parametrically forced oscillators}
\author{Leon Avery \and Andréas Sundstr{\"o}m}
\date{November 20, 2016}

\maketitle

\begin{abstract}\noindent
This report presents two different scenarios, which both can be
modeled as parametrically forced oscillators: pumping a playground
swing, and keeping the circadian rhythm in sync with the
environment. 
Children's swings and biological clocks are very different, but they
present a common problem. Each is a natural oscillator with an
intrinsic free-running rhythm. And each is driven by a forcing
function that works, not by directly pushing the oscillator itself,
but by periodically modifying the parameters of the oscillator.
The swing is modeled as a point-mass pendulum with a time
dependent length. We investigate conditions needed to pump the
swing. Under those conditions we show that the amplitude, to a first
approximation, will increase exponentially.
In the case of the biological clock, the forcing is external light,
which modifies the rate of biochemical reactions such as the synthesis
of messenger RNA. The response of the clock to light is described by a
Phase Response Curve. If it satisfies certain conditions, robust
entrainment will occur.
Mathematically, both of these problems are
united by Floquet Theory: a general analysis of linear differential
equations with periodic parametric forcing. Although Floquet Theory
does not directly give solutions to the differential equations, it
sharply constrains the form these solutions can take. In particular,
it helps to place bounds on the frequencies that will effectively pump
a swing, and it predicts the existence of periodic solutions to the
phase gradient differential equation for the circadian clock.  

\end{abstract}

%%%%%%%%%%%%%%%%% ^^^ Inbyggd titelsida ^^^ %%%%%%%%%%%%%%%%%

\input{Swing.input.tex}




\section{Introduction to Floquet Theory}
In the previous sections we showed that, in the linear approximation, a
swing with periodically varying length has solutions whose amplitude
increases or decreases exponentially in time. Although we didn’t show
this, there are also solutions that show no steady change in amplitude
with time. Which type of solutions one gets depends on the ratio of
the frequency $\omega$ of the length oscillation to the free-running
frequency $\omega_0$ of the unperturbed swing, and also on the magnitude  of the
length oscillation. As a general rule, exponential growth occurs when
the forcing entrains the oscillation of the swing in such a way that
the motions that deliver energy occur at the most effective phase of
each cycle. This is more likely to happen if the ratio $\omega/\omega_0$ is an integer
(as we showed, 2 is particularly effective), and when $\epsilon$ is larger.  

Floquet Theory is a general approach to understanding systems of
linear ODEs with periodic parametric forcing. Specifically, it applies
to systems of the form 
\begin{equation}\label{eq:floquetsys}%1
\dv{\vb{x}}{t}=\vb{A}(t)\vb{x}(t)
\qcomma \vb{x}(t_0)=:\vb{x}_0,
\end{equation}
where the matrix $\vb{A}(t)$ is a $T$-periodic function of time: 
\begin{equation}\label{eq:2}
\vb{A}(t+nT)=\vb{A}(t)\qcomma
n\in\Z.
\end{equation}
Every solution of \eqref{eq:floquetsys} is encompassed in its principal
fundamental matrix solution $\vb{U}(t,t_0)$, defined as the solution
to the matrix IVP  
\begin{equation}
\dv{\vb{U}(t,t_0)}{t}=\vb{A}(t)\vb{U}(t,t_0)\qcomma
\vb{U}(t_0,t_0)=\I
\end{equation}
Here  $\I$ is the  $n\times n$ identity matrix. Specifically, the
solution of \eqref{eq:floquetsys} is  $\vb{x}\left(t\right)=\vb{U}\left(t,{t}_{0}\right){x}_{0}$.

\bigskip
\noindent
Define the monodromy matrix $\vb{B}=\vb{U}({t}_{0}+T,{t}_{0})$. The monodromy
matrix turns out to be independent of  ${t}_{0}$, so it can be
computed as  $\vb{B}={U}(T,0)$. The eigenvalues of  $\vb{B}$,  $\rho_{1},\rho_{2},{\dots},\rho_{n}$, are called the characteristic
multipliers. The determinant of  $\vb{B}$ (the product of the
characteristic multipliers) is given by, 
\begin{equation}
\det(\vb{B})=\exp \left(
\int_{0}^{T}{\tr[\vb{A}(s)]\id{s}}
\right)
\end{equation}
The multipliers determine the character (e.g. stable or exponentially
growing) of solutions of \eqref{eq:floquetsys}. In particular, 
\begin{equation}
\vb{U}(t,{t}_{0})=
\vb{U}\Big((t-{t}_{0})\;\;\text{mod} T,\,{t}_{0}\Big)
{\vb{B}}^{\lfloor(t-{t}_{0})/T\rfloor}
\end{equation}

The characteristic exponents  
$\mu_{1},\mu_{2},{\ldots},\mu_{n}$, defined by  $\ee^{\mu_{i}T}=\rho_{i}$ determine the
growth rate of solutions. In particular, if  $\Re(\mu_{i})<0$,
solutions associated with it will decay exponentially; if
$\Re(\mu_{i})>0$, solutions associated with it will grow
exponentially, and if  $\Re(\mu_{i})=0$, solutions 
associated with it will be stable. (Strictly speaking, the previous
statements hold only if the geometric multiplicity of each eigenvalue
of  $\vb{B}$ equals its algebraic multiplicity. When this is not the case,
polynomial factors can also arise.) There will be, associated with
each characteristic exponent  $\mu$, a solution of the form 
\begin{equation}
\vb{x}(t)=\ee^{\mathit{\mu t}}\vb{p}(t)
\end{equation}
where  $\vb{p}(t)$ is a  $T$-periodic vector function of time. 

A particularly interesting case arises when  $\vb{A}(t)$ is a periodic
perturbation of a constant matrix that itself has 
${T}_{0}${}-periodic solutions, i.e.
\begin{equation}
\vb{A}(t)={\vb{A}}_{0}+\epsilon \widetilde{{\vb{A}}}(t)
\end{equation}
with  $\widetilde{{\vb{A}}}(t)$  $T${}-periodic. For instance, the varying
length swing is of this form, with
\begin{equation}
\begin{aligned}
\vb{A}_0&=
\begin{pmatrix}0&1\\-{\eta }^{2}&0\end{pmatrix}
\\
\widetilde{{\vb{A}}}(\tau)&=
\begin{pmatrix}0&0\\
0&-2\frac{\dot{{r}}(\tau)}{r(\tau)}
\end{pmatrix}
=
\begin{pmatrix}0&0\\
0&-2\pd_{\tau}\log(r(\tau))
\end{pmatrix}.
\end{aligned}
\end{equation}
${\vb{A}}_{0}$ is traceless, and since  $r$ is periodic, the integral of 
$\tr(\widetilde{{\vb{A}}})$ over one cycle is  
$-2\log({\tau}_{0}+2\pi )+2\log({\tau }_{0})=0$. Thus the
monodromy matrix  $\vb{B}$ for the swing has determinant 0. The character
of the solutions can be determined from  $\tr(\vb{B})$.
If  $\abs{\rho_{1}+\rho_{2}}=|\tr(\vb{B})|<2$, the multipliers  
$\rho_{1},\rho_{2}$ form a complex conjugate pair, each with absolute
value 1, and solutions are stable. If, however,
$\abs{\rho_{1}+\rho_{2}}=|\tr(\vb{B})|>2$, the multipliers
are real and different, and since their product is 1, one must be
greater than one. In this case, there is an exponentially growing
solution. Finally, in the border case,  $|\rho_{1}+\rho_{2}|=\tr(\vb{B})=2$, the multipliers are equal
, either both 1 or both  $-1$. The values of  $\eta,\,\epsilon $ for
which that holds define the boundaries of the Floquet tongues, within
which exponentially growing solutions occur. 





%%% Local Variables: 
%%% mode: latex
%%% TeX-master: "Report_swing_circadian"
%%% End: 


\input{Circadian_rythms.input.tex}


\section*{Bibliography}
\noindent
Burns, J.A. (1970). \textit{More on Pumping a Swing}. Am. J. Phys. \textbf{38}, 920-2

\bigskip\noindent
Ko, C.H., and Takahashi, J.S. (2006). \textit{Molecular components of the
mammalian circadian clock}. Hum. Mol. Genet. R271-7.

\bigskip\noindent
Konopka, R.J., and Benzer, S. (1971). \textit{Clock Mutants of Drosophila
melanogaster}. \textbf{68}, 2112–2116.

\bigskip\noindent
Partch, C.L., Green, C.B., Takahashi (2014).
\textit{Molecular architecture of the mammalian circadian clock}. Trends Cell
Biol. \textbf{24}, 90–99.

\bigskip\noindent
Pfeuty, B., Thommen, Q., and Lefranc, M. (2011). \textit{Robust Entrainment of
Circadian Oscillators Requires Specific Phase Response Curves}.

\bigskip\noindent
Rosato, E., Tauber, E., and Kyriacou, C.P. (2006). \textit{Molecular genetics
of the fruit-fly circadian clock}. Eur. J. Hum. Genet. \textbf{14},
729–738.

\bigskip\noindent
Strub, D.C. (2009). \textit{How do children swing?}  M.Eng. thesis, University
of Bristol, Dept. of Eng. Math.%ematics

\bigskip\noindent
Tea, P.L. Jr. and Falk, H. (1968). \textit{Pumping on a Swing}. Am. J. Phys. 
\textbf{36}, 1165-6

\bigskip\noindent
Ward, M.J. \textit{Basic Floquet Theory}
(http://www.math.ubc.ca/\~{}ward/teaching/m605/every2\_floquet1.pdf).


\end{document}





%%% Local Variables: 
%%% mode: latex
%%% TeX-master: "Report_swing_circadian"
%%% End: 
