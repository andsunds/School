%----------------------------------------------------------------------------------------
%	PACKAGES AND THEMES
%----------------------------------------------------------------------------------------
\pdfoutput=1
\documentclass[swedish, english]{beamer}



\usepackage[utf8]{inputenc}
%\usepackage[scaled]{helvet} %helvetica i sliden blir bra
\usepackage[T1]{fontenc}

\usepackage{graphicx} % Allows including images
\usepackage{booktabs} % Allows the use of \toprule, \midrule and \bottomrule in tables
\usepackage{comment}
\usepackage[swedish,english]{babel}
\usepackage{graphicx}
\usepackage{graphics}
\usepackage{units}
\usepackage{comment}
\usepackage{epstopdf}
\usepackage{tikz}
\usepackage{eso-pic}
\usepackage{wallpaper} %For background pictures
\usepackage{multimedia}
\usepackage{media9}

\usepackage{physics}

\usefonttheme[onlymath]{serif}

\graphicspath{ 
{presentation_figures/} %path to figures
}



\mode<presentation> {

% The Beamer class comes with a number of default slide themes
% which change the colors and layouts of slides. Below this is a list
% of all the themes, uncomment each in turn to see what they look like.

\usetheme{default}


% As well as themes, the Beamer class has a number of color themes
% for any slide theme. Uncomment each of these in turn to see how it
% changes the colors of your current slide theme.

\usecolortheme{beaver}

}







\renewcommand{\thefootnote}{\fnsymbol{footnote}}

%----------------------------------------------------------------------------------------
%	TITLE PAGE
%----------------------------------------------------------------------------------------



\title[]{Project introduction, \\parametrically driven oscillators}
\subtitle{} % optional
\author{Leon Avery \and Andréas Sundström} % [short author (optional)]{many authors}
\institute[UW]{University of Waterloo}
\date{October 19, 2016}

%\date{\vspace{-0.25cm}\today}




\begin{document}

\begin{frame}[plain]
\linethickness{0.075mm}
  \titlepage
\end{frame}
% }




%%%%%%%%%%%%%%%%%%%%%%%%%%%%%%%%%%%%%%%%%%%%%%%%


% \begin{frame} %A frame with table of contents
% \frametitle{Contents}
% \tableofcontents
% \end{frame}



% \begin{frame}
% \frametitle{Sample frame}
%  \begin{columns}[c]
% \column{.7\textwidth} % Left column and width
% \begin{figure}
% \includegraphics[width=1\textwidth]{Circadian_Rhythm_with_label.jpg}
%  %\caption{}
% \end{figure}
% \vspace{3mm}
% \footnotesize \url{http://www.livescience.com/images/i/000/015/173/original/Circadian_Rhythm_with_label.jpg}
% \column{.5\textwidth} % Right column and width
% \vspace{8mm}
% \begin{figure}
% \includegraphics[width=1\textwidth]{ode_solutionfield.jpg}
%  %\caption{}
% \end{figure}
% \vspace{5mm}
% \footnotesize\url{http://minitorn.tlu.ee/~jaagup/uk/dynsys/ds2/limit/Poincare/Image502.gif}
% \end{columns}
% \end{frame}



% \begin{frame}
% \frametitle{Parametric forcing as a model for a child’s swing}

% \begin{columns}[c]

% \column{.4\textwidth} % Left column and width
% \centerline{
% \resizebox{.7\textwidth}{!}{
% \input{presentation_figures/pendulum.pdf_t}}
% }


% \column{.7\textwidth} % Right column and width
% \begin{itemize}
% % \item Simple model:
% % $l(t) = \bar{l}[1+\epsilon\cos(\omega t)]$.
% \item Governing equation:
% $l\phi'' + 2l'\phi' + l_0\omega_0^2\sin(\phi)=0$.\\
% % Can be transformed to:
% % $z''+\qty[\eta^2+\epsilon\qty(1-\eta^2)\cos(\tau)]z=0+\order{\epsilon^2}$,
% % $\eta^2=(\omega_0/\omega)^2$.
% % \item This is an example of a Mathieu equation.
% \vspace{5mm}
% \item Would be interesting to study:
% $l(\phi)=l_0\Big[1+\cos(a\phi)\Big]$
% \end{itemize}

% \end{columns}
% \end{frame}



\begin{frame}
\frametitle{Project outline}

\begin{columns}[c]
\column{.5\textwidth} % Left column and width
\textbf{Swing}
\begin{itemize}
\item Explain physics
\item Governing equation:
$l\ddot\phi + 2\dot{l}\dot{\phi} + l_0\omega_0^2\sin(\phi)=0$.
\item Would be interesting to study:
$l(\phi)=l_0\Big[1+\cos(a\phi)\Big]$
\item Form of solutions
\end{itemize}

\column{.5\textwidth} % Right column and width
\textbf{Circadian Rythms}
\begin{itemize}
\item Explain the biology
\item Entrainment
\item Phase reduction technique
\item Phase response curve
\item Robustness
\end{itemize}

\end{columns}
\end{frame}





% \begin{frame}
% \frametitle{Sample frame}
%  \begin{columns}[c]
% \column{.4\textwidth} % Left column and width
% Text here.
% \begin{itemize}
%     \item Individual points.
%     \item Feel free to remove this list.
% \end{itemize}
% \column{.6\textwidth} % Right column and width
% This slide has two columns
% \begin{figure}
% %\includegraphics[width=1\textwidth]{}
% % \caption{}
% \end{figure}
% \end{columns}
% \end{frame}

\end{document}
