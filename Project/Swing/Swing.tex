\documentclass[11pt,letter, swedish, english
]{article}
\pdfoutput=1

\usepackage{../../custom_as}

\graphicspath{ {figures/} }

%\renewcommand{\thesubsection}{\arabic{section} (\alph{subsection})}

%\renewcommand{\thesubsubsection}{\arabic{section} (\alph{subsection},\,\roman{subsubsection})}


%%Drar in tabell och figurtexter
\usepackage[margin=10 pt]{caption}
%%För att lägga in 'att göra'-noteringar i texten
\usepackage{todonotes} %\todo{...}

%%För att själv bestämma marginalerna. 
\usepackage[
%            top    = 3cm,
%            bottom = 3cm,
%            left   = 3cm, right  = 3cm
]{geometry}

\DeclareMathAlphabet{\mathpzc}{OT1}{pzc}{m}{it}
\newcommand{\oh}{\ensuremath\mathpzc{o}}

\newcommand{\as}{\qcomma\text{as }}

%\renewcommand{\thefootnote}{\fnsymbol{footnote}}

\newcommand{\tE}{\tilde{E}}


\begin{document}

%%%%%%%%%%%%%%%%% vvv Inbyggd titelsida vvv %%%%%%%%%%%%%%%%%

% \title{\vspace{-2.5cm}Energy argument}
% \author{Andréas Sundström}
% \date{\today,\;\; \texttt{v\,1.4}}

%\maketitle

%%%%%%%%%%%%%%%%% ^^^ Inbyggd titelsida ^^^ %%%%%%%%%%%%%%%%%


\section{How to pump a swing}
A pendulums is one of the simplest physical systems -- we can all
understand, in broad terms, the behaviour of pendulum set in
motion. However, a more intriguing scenario is when a child starts off
on a swing with vritually zero amplitute, and after just a few minutes
are swinging high and fast; and that was without anyone pushing.
Consequently the main objective of this section is to explore how it
is possible to pump a swing \emph{without} any external forcing. 

One of the simplest model, put forth by Tea \& Falk in
1968~\cite{Tea_Falk_1968}, is the \emph{parametrically forced
  pedulum}, where the length of the pendulum varies to increas the
amplitude of the swing. The physical interpretation of this model is
that the child moves his center of mass (CM) up and down along the radial
direction of the swing. Also the child and the swing is viewed as a
point mass on a mass less connection to the swing suspension. 

In this section we will study the equation of motion for such a system,
then we study the energy of the swing in more detail to arive at a
physcal understanding of how the swing can gain ernergy through this
parametric forcing. We will study the cases where the length of the
pendulum is either directly dependent on time, or directly dependent
of the phase of the swing.
After that, we will use 
\todo{Specify later!}
perturbation techniques to study the motion of the swing.

\subsection{The equation of motion}
One of the most interesting paramameters will be the angle, $\phi$,
the swing makes against the vertical as in \figref{fig:pendulum}. 
A simple and elegant way to find $\phi(t)$ is to use Lagrangian
formalism. We therefore need to find the Lagrangian $\mathcal{L}=T-V$,
where $T$ is the kinetic energy and $V$ is the potential energy of the
swing. 

To find the kinetic energy, we will need to know the velocity of the
CM. Since we are looking for the angle, polar coordinates
are prefered. In polar coordinates the velocity is given by
$\vb{v}=\dot{l}\vu{r}+l\dot{\phi}\vu{\phi}$. Thus the kinetic energy is
\begin{equation}
T=\frac{m}{2}\vb{v}^2=\frac{m}{2}\qty(\dot{l}^2(t)+l^2(t)\dot{\phi}^2(t)).
\end{equation}
Next the potential energy can be written as
\begin{equation}
V=-mg\,l(t)\,\cos\phi(t),
\end{equation}
where the zero point of the potential is set at the swing suspension. 

\begin{figure}\centering
\resizebox{.2\textwidth}{!}{\input{figures/pendulum.pdf_t}}
\caption{A sketch of the pendulum and it's movement under influence of
parametric forcing. the length of the pendulum is assumed to be of the
form $l(t)=l_0[1+\epsilon\,\delta(t)]$, 
}
\label{fig:pendulum}
\end{figure}

By now using the Euler-Lagrange equation on $\mathcal{L}=T-V$, with
with respect to $\phi$ nad $\dot\phi$, we get
\begin{equation}
0=\dv{t}\qty[\pdv{\mathcal{L}}{\dot{\phi}}]-\pdv{\mathcal{L}}{\phi}
=ml^2\ddot\phi +2ml\dot{l}\dot\phi +mgl\sin\phi.
\end{equation}
This can be cleaned up by dividing through by $m$ and $l$. 

To further simplify the equation of motion (EOM), we
non-dimensionalize by introducing
\begin{equation}
r=\frac{l}{l_0}\qcomma\text{and}\quad
\tau=\omega t.
\end{equation}
Here $l_0$ is the average length of the pendulum, and $\omega$ is a
frequency that will be used in the forcing later -- i.e. $r(t)$ is a
periodic function with period $2\pi/\omega$. 
With the non-dimensionalized parameters we get the EOM:
\begin{equation}\label{eq:eom2}
%\mathtt{eom312}=
r(\tau)\phi'(\tau)+2r'(\tau)\phi'(\tau) + \eta^2\sin\phi(\tau)=0,
\end{equation}
where ``$(')=(\dv*{\tau})$'', and
$\eta^2=\omega_0^2/\omega=(g/l_0)^2/\omega^2$ is the square of the
ratio of the unperturbed frequency $\omega_0$ to $\omega$.



\subsection{The swing energy}
%The non-dimensionalized energy of the swing is
%\begin{equation}
%E=\frac{1}{2}\vb{v}^2 - \eta^2 r \cos\phi.
%\end{equation}
%The velocity, $\vb{v}$, in polar coordinates is
%\begin{equation}
%\vb{v} = x'\vu{x}+y'\vu{y} = r'\vu{r}+r\phi'\vu{\phi}.
%\end{equation}
%Therefore the energy is
Non-dimenzionalizing $T$ and $V$ with $r$ and $\tau$, and aslo noing
that $m$ have no significance, we get the non-dimensionalized energy
\begin{equation}
\tE=\frac{1}{2}\qty[{r'}^2 + {r}^2{\phi'}^2] - \eta^2 r \cos\phi.
\end{equation}
meaning that the time-derivative of the energy is
\begin{equation}
\begin{aligned}
\dv{E}{\tau}%=& \qty[r'r'' + rr'{\phi'}^2+r^2\phi'\phi'']
%+ \eta^2r\sin(\phi)\phi' - \eta^2r'\cos(\phi)
%\\
=&r'r''+r\phi'\qty(r\phi'' + r'\phi' + \eta^2\sin\phi) 
-\eta^2r'\cos\phi.
%\\
%=&r'r''+r\phi'\qty(\mathtt{eom312}-r'\phi') -\eta^2r'\cos\phi
%\\
%=&r'r'' 
%- r'\underbrace{\qty(r{\phi'}^2 + \eta^2\cos\phi)
%}_{\ge0,\;\;\text{as long as } \phi\le\pi/2}.
\end{aligned}
\end{equation}
No we note that the expression in parenthesis is just the LHS of
\eqref{eq:eom2} with a term $r'\phi'$ missing. Therefore we can write
\begin{equation}\label{eq:dE/dtau1}
\dv{E}{\tau}
=r'r''+r\phi'\qty(0-r'\phi')-\eta^2r'\cos\phi
=r'r'' - r'%\underbrace{
\qty(r{\phi'}^2 + \eta^2\cos\phi)
%}_{\ge0,\;\;\text{as long as } \phi\le\pi/2}.
\end{equation}

% To get the pumped energy we integrate this over one
% period\footnotemark{}. But
% \begin{equation}
% \int_0^T\!\rd\tau r' r''=\qty[\frac{1}{2}{r'}^2]_0^T\approx0,
% \end{equation}
% meaning that we are left with:
% \begin{equation}\label{eq:int_dE1}
% \Delta{E}=-\int_0^T\!\rd\tau\;
% r'\qty(r{\phi'}^2 + \eta^2\cos\phi).
% \end{equation}

% \footnotetext{This is a little iffy, since we're not dealing with
%   perfectly periodic behavior, but hopefully it's manageable. }



\subsubsection{Physical interpretation}
This has a really neat physical interpretation. When the child is
raising himself on the swing he has to do work against two forces:
the centrifugal and the gravitational force. Then since there's no
dissipation of energy, all the work done by the child must go into the
motion of the swing. 
The first term in the parenthesis in \eqref{eq:dE/dtau1} corresponds
to the centrifugal force and the second term to the gravitational
force.\footnotemark{}  
\footnotetext{
The last term, $r'r''$, corresponds to the power needed
to accelerate the CM along the line of the swing. (We are in a
``swing-fixed'' frame of reference, which means that $r''$ is the
only term for acceleration needed here.) 
%Furthermore the rotating frame of reference is needed to talk about a
%\emph{centrifugal} force.  
Then the same amount of energy is given back when the mass
decelerates, resulting in no net contribution to the swing energy over
one cycle. }

By only looking at the reasoning above, without the need of the
analysis resulting in \eqref{eq:dE/dtau1}, one can derive that in
order to pump the swing,
the child must move inwards ($r'<0$) when the swing is as 
\emph{low as possible}, and outwards ($r'>0$) when the swing is as
\emph{high as possible}. 
Because that would correspond to the child having to do the most
amount of work against the two forces, and then letting the forces do
the least amount of work. 

This is the reason why it's possible for the swing to gain energy by
only changing the length of the pendulum, and not having any external
forces. The child actaully has to do work when moving the CM up and
down.

Furthermore, by actually studying the quantitive result,
\eqref{eq:dE/dtau1}, we can conclude that the most effective pumping
is if $r$ abruply decreases when the swing is at its lowest and then
abrupltly increases when the swing is at its higest. This strategy
will result in an $r(t)$ that has a 
\todo{Is is ok to use ``frequency'' (not sinusiodal)?}
frequency of about double that of the unperturbed swing --
i.e. $\omega\approx2\omega_0$ and $\eta\approx1/4$. 



% \subsubsection{Small angle approximation}
% If we limit ourselves to the small angle approximation, then
% $\cos(\phi)\approx1$ and 
% \begin{equation}
% \int_0^T\!\rd\tau\;r'\cos\phi
% \approx\int_0^T\!\rd\tau\;r'=0.
% \end{equation}
% The pumped energy now reduces to
% \begin{equation}
% \Delta{E}=-\int_0^T\!\rd\tau\;
% r'r{\phi'}^2.
% \end{equation}
% With $r=(1+\epsilon\,\delta{r})$ and $r'=\epsilon\,\delta{r'}$, we get
% \begin{equation}\label{eq:int_dE2}
% \Delta{E}=-\int_0^T\!\rd\tau\;r'{\phi'}^2
% +\order{\epsilon^2}.
% \end{equation}

\subsection{Studying solutions}




\begin{thebibliography}{9}

\bibitem{Tea_Falk_1968}
Tea, P.L. Jr. and Falk, H. (1968). Pumping on a Swing. Am. J. Phys. 
\textit{36, 1165-6}

 

\end{thebibliography}


\end{document}



