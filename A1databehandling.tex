\chapter{Databehandling}


\section{Datainsamling}
\todo[inline]{Lägg in information i resp. kap.}
Datan som behandlats i denna studie har inte samlats in i samband med denna studie utan tillhandahölls från andra källor. Hur denna data där samlats in och vad den beskriver presenteras mer utförligt nedan.

\subsubsection{Datan för partikelrörelse i celler}
Datan som studerats för partikelrörelser är den samma som \cite{Midtveldt_etal2016} använde och utgörs av mätningar av positionen för fluorescerande partiklar i jästceller. Jästcellerna hade genmodifierats till att producera fluorescerande protein som lätt bildar kluster. 
Dessa kluster brukar vara av storleksordning 100--500\,nm, vilket kan jämföras med själva cellernas storlek på omkring 5\,\micro{m}.

Data för ett hundratal partiklar från olika jästceller ingick i mätserien, både för aktiva celler och celler som försatts i dvala med sänkt metabol aktivitet. Mätningen genomfördes med 100 bilder per sekund.


\subsubsection{Datan för strängrörelse i vätska}

Datan som analyserats för strängrörelse i vätska kommer från \todo{Var kommer datan från? Fråga Daniel?}... och består av filmer av aktinfilament som tillåts röra sig i en vätska. Dessa strängar hade en längd kring 10--30\,\micro{m} och befann sig i kanaler av olika bredd. Datan hade redan behandlats något så att strängens läge gav av en uppsättning vita pixlar mot en svart bakgrund.

Mätningar hade utförts på två typer av strängar: fria strängar i breda kanaler och inneslutna strängar i smala skåror. Det fanns två filmer för vardera strängtyp. Alla fyra hade filmats med 10 bilder per sekund. Rörelsen utfördes till största del i två dimensioner då skårornas djup var litet i förhållande till skårornas och filamentens bredd.


\section{Polynomanpassning för strängarna (bättre rubrik?)}
\todo{Förslag: Parametrisering av strängdata}







%Bara en liten kodsnutt som behövs när man kompilerar lokalt
%%% Local Variables: 
%%% mode: latex
%%% TeX-master: "main.tex"
%%% End: 