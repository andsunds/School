\documentclass[11pt,letter, swedish, english
]{article}
\pdfoutput=1

\usepackage{../custom_as}
\usepackage[makeroom
]{cancel}
\graphicspath{{figures/}}

\swapcommands{\Delta}{\varDelta}
\swapcommands{\Omega}{\varOmega}

%%Drar in tabell och figurtexter
\usepackage[margin=10 pt]{caption}
%%För att lägga in 'att göra'-noteringar i texten
\usepackage{todonotes} %\todo{...}

%%För att själv bestämma marginalerna. 
\usepackage[
%            top    = 2.5cm,
%            bottom = 3cm,
%            left   = 3cm, right  = 3cm
]{geometry}

%%För att ändra hur rubrikerna ska formateras
\renewcommand{\thesubsection}{\arabic{section} (\alph{subsection})}

\renewcommand{\thesubsubsection}{\arabic{section} (\alph{subsection},\,\roman{subsubsection})}

\renewcommand{\thefootnote}{\fnsymbol{footnote}}

\newcommand{\Tc}{\ensuremath{T_{\text{c}}}}
\newcommand{\eF}{\ensuremath{\epsilon_{\text{F}}}}
\newcommand{\wD}{\ensuremath{\omega_{\text{D}}}}

%\usepackage{tikz}

\begin{document}

%\tikzstyle{every picture}+=[remember picture]
%\tikzstyle{na} = [shape=rectangle,inner sep=0pt,text depth=0pt]



%%%%%%%%%%%%%%%%% vvv Inbyggd titelsida vvv %%%%%%%%%%%%%%%%%

\title{Statistical Physics -- PHYS\,704 \\
Assignment 5}
\author{Andréas Sundström}
\date{\today}

\maketitle

%%%%%%%%%%%%%%%%% ^^^ Inbyggd titelsida ^^^ %%%%%%%%%%%%%%%%%

\section{A phase transition}

We have the Landau free energy
\begin{equation}\label{eq:f}
f(t,\,m)=%-hm+
tm^2+sm^4+um^6,
\end{equation}
where $u>0$ is a fixed constant.
We can also write $f$ on a (semi) factorized form
\begin{equation}\label{eq:f_fact}
f(t,\,m)=um^2
\qty(m^2+\frac{s-\sqrt{s^2-4tu}}{2u})
\qty(m^2+\frac{s+\sqrt{s^2-4tu}}{2u}).
\end{equation}
%Also this function is obviously even in $m$, so we only need to study
%for instance, positive $m$.


All of this problem concerns finding the minimum of $f$ with respect
to $m_0$. So we might as well find the derivative
\begin{equation}
f'_m:=\pdv{f}{m}=2tm+4sm^3+6um^5=2m\qty(t+2sm^2+3um^4)
\end{equation}
which is zero at $m=0$, or at 
\begin{equation}\label{eq:m1}
m_1^2=\frac{-s\pm\sqrt{s^2-3ut}}{3u}.
\end{equation}
We can also semi-factorize the derivative
\begin{equation}\label{eq:f'_fact}
f'_m=6um\qty[m^2-m_{1,-}^2]\qty[m^2-m_{1,+}^2],
\end{equation}
where $m_{1,+}^2$ is \eqref{eq:m1} with the ``+'' in the ``$\pm$'', and
similar for $m_{1, -}^2$. Note that $m_{1,+}^2>m_{1,-}^2$.

\subsection{$t>0$, $s>-\sqrt{3ut}$}
We begin by studying the even narrower case:
$|s|<\sqrt{3ut}$. Obviously \eqref{eq:m1} lacks a real solution now,
since $(s^2-3ut)<0$.

Next we tackle the rest of this case, which is: $s>\sqrt{3ut}$. This
time we see that the RHS of \eqref{eq:m1} is real, but negative. That
is $m^2$ has to be negative. 

So there's no real solutions here, except for $m=0$. This is a
\emph{minimum}, since $m=0$ is the only zero point of the derivative, and
$f\to+\infty$ as $m\to\pm\infty$ (because $u>0$).

\subsection{$t>0$, $-\sqrt{4ut}<s<-\sqrt{3ut}$}
Now that $s<-\sqrt{3ut}$, \eqref{eq:m1} has real solutions. We get
two values for $m^2$, the smaller of those values correspond to a
local \emph{maximum}. We therefore conclude that we get two new
minimums at
\begin{equation}\label{eq:1b_m1+}
m_{1,+}=\pm\qty(\frac{-s+\sqrt{s^2-3ut}}{3u})^{1/2}.
\end{equation}
To show that this is an actual minimum, we look at
\eqref{eq:f'_fact}, under the condition that
$m>0$\footnotemark{}. There we see that $f'_m$ is negative when 
$m_{1,-}^2<m^2<m_{1,+}^2$, i.e. between the two new zero points of
$f'_m$, and $f'_m$ becomes positive for $m^2>m_{1,+}^2$.
\footnotetext{Since $f$ clearly is even, we only need to study, for
  instance, $m>0$. The behavior on the other side follows (but
  mirrored).} 

To show that $f(m_{1,+})>f(m_0=0)=0$, for $-\sqrt{4ut}<s\le-\sqrt{3ut}$,
we look at \eqref{eq:f_fact} and see that for $s^2<4ut$, the only real
zero point is at $m=0$. Thus we know that $f(m)>f(m_0=0)=0$ for any
$m\neq0$ in this interval of $s$, including $m_{1,+}$.
  
\subsection{$t>0$, $s=-\sqrt{4ut}$}
In this case, all the reasoning from the previous problem still
applies. But we can write \eqref{eq:1b_m1+} as
\begin{equation}
m_{1,+}=\pm\qty(\frac{+\sqrt{4ut}+\sqrt{4ut-3ut}}{3u})^{1/2}
=\pm\qty(\frac{3\sqrt{ut}}{3u})^{1/2}
=\pm\qty(\frac{t}{u})^{1/4}.
\end{equation}
Furthermore, we can rewrite \eqref{eq:f_fact} as
\begin{equation}
f(t,\,m)=um^2
\qty(m^2+\frac{-\sqrt{4tu}}{2u})
\qty(m^2+\frac{-\sqrt{4tu}}{2u})
=um^2\qty(m^2+\sqrt{\frac{t}{u}})^2.
\end{equation}
And we see that $f$ is indeed~0 at $m_{1,+}$ for $s=-\sqrt{4ut}$. So
the two minima are equal: $f(m_{1,+})=f(0)=0$. 


\subsection{$t>0$, $s<-\sqrt{4ut}$}
Once again, the reasoning from 1b is still valid. And the minimum is
still given by \eqref{eq:1b_m1+}. 

Now, when $s<-\sqrt{4ut}$, we do however get negative values
for $f$. By studying \eqref{eq:f_fact} we see that the expressions in
the parentheses are real, so we have got more (distinct) zeros for
$f$. More distinct zeros (that are not minima themselves) means that
$f$ has to be \emph{negative} at some intervals. Now since $f$ can be
negative and $f(m=0)=0$, means that $m=0$ is \emph{not} a global
minimum. Therefore $m_{1,+}$ is the global minimum\footnotemark{},
where $f(m_{1,+})<0=f(m=0)$.  
\footnotetext{Because $f(m)\to+\infty$ as $m\to\pm\infty$. }


\subsection{$t=0$, $s<0$}
At $t=0$, \eqref{eq:m1} becomes
\begin{equation}\label{eq:m1_t=0}
m_1^2=\frac{-s\pm\abs{s}}{3u}.
\end{equation}
For $s<0$ the ``$+$'' value becomes $m_{1,+}^2=2\abs{s}/(3u)$, and
the ``$-$'' value becomes $m_{1,-}^2=0$. And the factorization of the
derivative becomes
\begin{equation}%\label{eq:f'_fact_t=0}
f'_m=6um^3\qty[m^2-\frac{2\abs{s}}{3u}].
\end{equation}

We now see that $m=0$ corresponds to a local \emph{maximum}, because
the sign of $f'_m$ changes from positive to negative when $m$ passes~0
from below. Therefore $m_{1,+}=\pm\sqrt{2\abs{s}/(3u)}$ are the only
two minima of $f$. 

\subsection{$t<0$, any $s$}
As always, the factorization \eqref{eq:f_fact} still holds. What's new
is that, with $t<0$, $m_{1,+}$ and $m_{1,-}$ will have different
signs. This fact is independent of the value $s$. 

Define $(m_1^{(+)})^2$ as the positive of $m_{1,+}^2$ or $m_{1,-}^2$, and
$(m_1^{(-)})^2$ as the negative of the two. We can still write down
\eqref{eq:f_fact}:
\begin{equation}\label{eq:f'_fact}
f'_m=6um\qty[m^2-\qty(m_{1}^{(+)})^2]\qty[m^2-\qty(m_{1}^{(-)})^2].
\end{equation}
We see once again, that the sign of $f'_m$ changes from positive to
negative when $m$ passes~0 from below. Therefore $m=0$ is a local
maximum, and $(m_{1}^{(+)})^2$ corresponds to the two minima of
$f$.\footnote{The factor containing $m_{1,-}^2$, doesn't give any real
zeros for the derivative since the factor has $m^2$, which can't be
negative, as its other term.}


For $s>0$ and $t$ near 0, the minima can be approximated by
\begin{equation}
\qty(m_{1}^{(+)})^2=\frac{-s+\sqrt{s^2-3ut}}{3u}
=\frac{s}{3u}\qty[-1+\qty(1-\frac{3ut}{2s}+\order{t^2})]
\approx \frac{-t}{2},%\to0\;\text{as}\;t\to0.
\end{equation}
which goes to $0$ as $t\to\0^-$.

\subsection{$t=0$, $s>0$}
We can still use \eqref{eq:m1_t=0}. But for $s>0$ the ``$+$'' value to
becomes $m_{1,+}^2=0$, and the ``$-$'' value becomes
$m_{1,-}^2=-2\abs{s}/(3u)<0$. This last equation lacks real solutions
for $m_{1,-}$. 
The factorization of the derivative instead becomes
\begin{equation}%\label{eq:f'_fact_t=0}
f'_m=6um^3\qty[m^2+\frac{2\abs{s}}{3u}].
\end{equation}
We now see that $m=0$ is again the only minimum of $f$, and is indeed the
only real solution to $f'_m=0$.



\end{document}


%  LocalWords:  Pathria  idealities bosonic Bogoliubov Beale BCS
%  LocalWords:  Laplace's
