\documentclass[11pt,letter, swedish, english
]{article}
\pdfoutput=1

\usepackage{../custom_as}
\usepackage{cancel}
\graphicspath{ {figures/} }

%%Drar in tabell och figurtexter
\usepackage[margin=10 pt]{caption}
%%För att lägga in 'att göra'-noteringar i texten
\usepackage{todonotes} %\todo{...}

%%För att själv bestämma marginalerna. 
\usepackage[
%            top    = 3cm,
%            bottom = 3cm,
%            left   = 3cm, right  = 3cm
]{geometry}

%%För att ändra hur rubrikerna ska formateras
\renewcommand{\thesubsection}{\arabic{section} (\alph{subsection})}

\renewcommand{\thesubsubsection}{\arabic{section} (\alph{subsection},\,\roman{subsubsection})}



\swapcommands{\varPhi}{\Phi}
\swapcommands{\varPi}{\Pi}
\swapcommands{\varOmega}{\Omega}



\begin{document}

%%%%%%%%%%%%%%%%% vvv Inbyggd titelsida vvv %%%%%%%%%%%%%%%%%
% \begin{titlepage}
\title{Quantum Mechanics -- PHYS\,701 \\
Assignment 5}
\author{Andréas Sundström}
\date{\today}

\maketitle

%%%%%%%%%%%%%%%%% ^^^ Inbyggd titelsida ^^^ %%%%%%%%%%%%%%%%%

%Om man vill ha en lista med vilka todo:s som finns.
%\todolist

\section{Optical theorem}
The optical theorem relates the total cross-section to the imaginary
part of the forward scattering amplitude. 

To prove this theorem we use the scattering aplitude representd using
phase shift:
\begin{equation}
f(\theta)=\frac{1}{k}\sum_{l=0}^\infty
(2l+1)\ee^{\ii\delta_l}\sin(\delta_l)\,P_l(\cos\theta).
\end{equation}
Now the forward scattering will be at $\theta=0$. Due to the way
we ``normalize'' Legendre polynomials\footnotemark{}, we have $P_l(1)=1$
for all $l$. So that the forward scatering amplitude is
\begin{equation}
f(0)=\frac{1}{k}\sum_{l=0}^\infty
(2l+1)\ee^{\ii\delta_l}\sin(\delta_l),
\end{equation}
and its imaginary part is
\begin{equation}\label{eq:1_Im(f0)}
\Im[f(0)]=\frac{1}{k}\sum_{l=0}^\infty
(2l+1)\Im[\ee^{\ii\delta_l}]\sin(\delta_l)
=\frac{1}{k}\sum_{l=0}^\infty (2l+1)\sin^2(\delta_l).
\end{equation}

\footnotetext{The ``normalization'' is:
$\oldint_{-1}^1 P_n(x)P_m(x)\id{x}=\frac{2}{2n+1}\delta_{m,\,n}$,
see any mathematical handbook or collection of formulas,
e.g. Råde~\&~Westergren, \textit{Mathematics Handbook for Science and
  Engineering}, ed. 5:12.}

Next up is the total scattering cross-section
\begin{equation}
\sigma=\frac{4\pi}{k^2}\sum_{l=0}^\infty
(2l+)\sin^2(\delta_l),
\end{equation}
which only differs from \eqref{eq:1_Im(f0)} by a factor of
$4\pi/k$. In other words
\begin{equation}
\sigma=\frac{4\pi}{k}\Im[f(0)].
\end{equation}
This is the optical theorem.
\qed



\section{Gaussian potential scattering}
In this probelm we're concerned with the total scattering
cross-section from a Gaussian potential:
\begin{equation}\label{eq:2_V}
V(\vb*{r})=A\ee^{-\mu r^2},
\end{equation}
where $\mu>0$ is some constant defining the length scale of the potential.

We will use the Born approximation to calculate the scattering
amplitude
\begin{equation}
f(\theta)\approx -\frac{2m}{\hbar^2q}
\int_0^\infty r V(r) \sin(qr)\id{r},
\end{equation}
where $q=2k\sin(\theta/2)$. Here, we've also use the fact that the
potential is spherically symmetric.
Substituting \eqref{eq:2_V} into this, gives
\begin{equation}
f(\theta)\approx -\frac{2mA}{\hbar^2q}
\int_0^\infty r \ee^{-\mu r^2} \sin(qr)\id{r}.
\end{equation}
This can either be solved by convering the sine to a complex
exponential an taking the imaginary part of the integral, as we did in
the last assignment. Or we could let \textit{Mathematica} handle this,
which results in
\begin{equation}
f(\theta)\approx -\frac{2mA}{\hbar^2q}\,
\frac{\ee^{-\frac{q^2}{4\mu}}\sqrt{\pi}q}{4\mu^{3/2}}
=-\frac{\sqrt{\pi}mA}{2\hbar^2\mu^{3/2}}\,\exp(-\frac{k^2\sin^2(\theta/2)}{\mu}).
\end{equation}
Here, we used $q=2k\sin(\theta/2)$ in the last step.

To then get the total scattering cross-section we have to integrate
over the unit sphere:
\begin{equation}
\sigma=\int_0^{2\pi}\rd{\phi}
\int_0^\pi\rd\theta\,\sin(\theta) \abs{f(\theta)}^2
=\qty(\frac{\sqrt{\pi}mA}{2\hbar^2\mu^{3/2}})^2 2\pi 
\int_0^\pi\rd\theta\,\sin(\theta)
\exp(-\frac{2k^2}{\mu}\sin^2\qty(\frac{\theta}{2}))
\end{equation}
The last integral is easily evaluated with the change of variables
\begin{equation}
u=\sin^2\qty(\frac{\theta}{2})
\quad\Longrightarrow\quad
\rd{u}=
2\sin(\frac{\theta}{2})\frac{1}{2}\cos(\frac{\theta}{2})\id\theta
=\frac{1}{2}\sin(\theta)\id\theta.
\end{equation}
So we get
\begin{equation}
\begin{aligned}
\sigma=&
\frac{\pi^2mA}{2\hbar^4\mu^{3}}  
\int_0^1 2\rd{u}\,\exp(-\frac{2k^2}{\mu}u)
=\frac{\pi^2mA}{\hbar^4\mu^{3}} \frac{\mu}{2k^2}
\qty[1-\exp(-\frac{2k^2}{\mu})]\\
=&\frac{\pi^2mA}{2\hbar^4\mu^{2}k^2}
\qty[1-\exp(-\frac{2k^2}{\mu})].
\end{aligned}
\end{equation}

\subsection*{Validity of the Born approximation}
To discuss the validity of the Born approximation we need a length
scale and a ``depth'' scale of hte potential. Clearly the length scale
is given by $a=\mu^{-1/2}$, and the energy scale is just $A$.

In the case of low energy incoming particles, $\abs{k}\mu^{-1/2}\ll1$,
we need  
\begin{equation}
\frac{ma^2}{\hbar^2}A=\frac{m}{\hbar^2\mu}A\ll1,
\end{equation}
for the Born approximation to be valid.

And in the case of high energy incoming particles we need the incoming
particle momentum to be high enought so that
\begin{equation}
\frac{mA}{\hbar^2k\sqrt{\mu}}\ll1,
\end{equation}
for the Born approximation to be valid.

\section{H-Kr scattering}


\section{More scattering}



\end{document}



