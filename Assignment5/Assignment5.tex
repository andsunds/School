\documentclass[11pt,letter, swedish, english
]{article}
\pdfoutput=1

\usepackage{../custom_as}
\usepackage{cancel}
\graphicspath{ {figures/} }

%%Drar in tabell och figurtexter
\usepackage[margin=10 pt]{caption}
%%För att lägga in 'att göra'-noteringar i texten
\usepackage{todonotes} %\todo{...}

%%För att själv bestämma marginalerna. 
\usepackage[
%            top    = 3cm,
%            bottom = 3cm,
%            left   = 3cm, right  = 3cm
]{geometry}

%%För att ändra hur rubrikerna ska formateras
\renewcommand{\thesubsection}{\arabic{section} (\alph{subsection})}

\renewcommand{\thesubsubsection}{\arabic{section} (\alph{subsection},\,\roman{subsubsection})}



\swapcommands{\varPhi}{\Phi}
\swapcommands{\varPi}{\Pi}
\swapcommands{\varOmega}{\Omega}



\begin{document}

%%%%%%%%%%%%%%%%% vvv Inbyggd titelsida vvv %%%%%%%%%%%%%%%%%
% \begin{titlepage}
\title{Quantum Mechanics -- PHYS\,701 \\
Assignment 5}
\author{Andréas Sundström}
\date{\today}

\maketitle

%%%%%%%%%%%%%%%%% ^^^ Inbyggd titelsida ^^^ %%%%%%%%%%%%%%%%%

%Om man vill ha en lista med vilka todo:s som finns.
%\todolist

\section{Optical theorem}
The optical theorem relates the total cross-section to the imaginary
part of the forward scattering amplitude. 

To prove this theorem we use the scattering aplitude represented using
phase shift:
\begin{equation}
f(\theta)=\frac{1}{k}\sum_{l=0}^\infty
(2l+1)\ee^{\ii\delta_l}\sin(\delta_l)\,P_l(\cos\theta).
\end{equation}
Now the forward scattering will be at $\theta=0$. Due to the way
we ``normalize'' Legendre polynomials\footnotemark{}, we have $P_l(1)=1$
for all $l$. So that the forward scatering amplitude is
\begin{equation}
f(0)=\frac{1}{k}\sum_{l=0}^\infty
(2l+1)\ee^{\ii\delta_l}\sin(\delta_l),
\end{equation}
and its imaginary part is
\begin{equation}\label{eq:1_Im(f0)}
\Im[f(0)]=\frac{1}{k}\sum_{l=0}^\infty
(2l+1)\Im[\ee^{\ii\delta_l}]\sin(\delta_l)
=\frac{1}{k}\sum_{l=0}^\infty (2l+1)\sin^2(\delta_l).
\end{equation}

\footnotetext{The ``normalization'' is:
$\oldint_{-1}^1 P_n(x)P_m(x)\id{x}=\frac{2}{2n+1}\delta_{m,\,n}$,
see any mathematical handbook or collection of formulas,
e.g. Råde~\&~Westergren, \textit{Mathematics Handbook for Science and
  Engineering}, ed. 5:12.}

Next up is the total scattering cross-section
\begin{equation}\label{eq:1_sigma}
\sigma=\frac{4\pi}{k^2}\sum_{l=0}^\infty
(2l+)\sin^2(\delta_l),
\end{equation}
which only differs from \eqref{eq:1_Im(f0)} by a factor of
$4\pi/k$. In other words
\begin{equation}
\sigma=\frac{4\pi}{k}\Im[f(0)].
\end{equation}
This is the optical theorem.
\qed



\section{Gaussian potential scattering}
In this probelm we're concerned with the total scattering
cross-section from a Gaussian potential:
\begin{equation}\label{eq:2_V}
V(\vb*{r})=A\ee^{-\mu r^2},
\end{equation}
where $\mu>0$ is some constant defining the length scale of the potential.

We will use the Born approximation to calculate the scattering
amplitude
\begin{equation}
f(\theta)\approx -\frac{2m}{\hbar^2q}
\int_0^\infty r V(r) \sin(qr)\id{r},
\end{equation}
where $q=2k\sin(\theta/2)$. Here, we've also use the fact that the
potential is spherically symmetric.
Substituting \eqref{eq:2_V} into this, gives
\begin{equation}
f(\theta)\approx -\frac{2mA}{\hbar^2q}
\int_0^\infty r \ee^{-\mu r^2} \sin(qr)\id{r}.
\end{equation}
This can either be solved by convering the sine to a complex
exponential an taking the imaginary part of the integral, as we did in
the last assignment. Or we could let \textit{Mathematica} handle this,
which results in
\begin{equation}
f(\theta)\approx -\frac{2mA}{\hbar^2q}\,
\frac{\ee^{-\frac{q^2}{4\mu}}\sqrt{\pi}q}{4\mu^{3/2}}
=-\frac{\sqrt{\pi}mA}{2\hbar^2\mu^{3/2}}\,\exp(-\frac{k^2\sin^2(\theta/2)}{\mu}).
\end{equation}
Here, we used $q=2k\sin(\theta/2)$ in the last step.

To then get the total scattering cross-section we have to integrate
over the unit sphere:
\begin{equation}
\sigma=\int_0^{2\pi}\rd{\phi}
\int_0^\pi\rd\theta\,\sin(\theta) \abs{f(\theta)}^2
=\qty(\frac{\sqrt{\pi}mA}{2\hbar^2\mu^{3/2}})^2 2\pi 
\int_0^\pi\rd\theta\,\sin(\theta)
\exp(-\frac{2k^2}{\mu}\sin^2\qty(\frac{\theta}{2}))
\end{equation}
The last integral is easily evaluated with the change of variables
\begin{equation}
u=\sin^2\qty(\frac{\theta}{2})
\quad\Longrightarrow\quad
\rd{u}=
2\sin(\frac{\theta}{2})\frac{1}{2}\cos(\frac{\theta}{2})\id\theta
=\frac{1}{2}\sin(\theta)\id\theta.
\end{equation}
So we get
\begin{equation}\label{eq:2_final}
\begin{aligned}
\sigma=&
\frac{\pi^2mA}{2\hbar^4\mu^{3}}  
\int_0^1 2\rd{u}\,\exp(-\frac{2k^2}{\mu}u)
=\frac{\pi^2mA}{\hbar^4\mu^{3}} \frac{\mu}{2k^2}
\qty[1-\exp(-\frac{2k^2}{\mu})]\\
=&\frac{\pi^2mA}{2\hbar^4\mu^{2}k^2}
\qty[1-\exp(-\frac{2k^2}{\mu})].
\end{aligned}
\end{equation}

\subsection*{Validity of the Born approximation}
To discuss the validity of the Born approximation we need a length
scale and a ``depth'' scale of hte potential. Clearly the length scale
is given by $a=\mu^{-1/2}$, and the energy scale is just $A$.

In the case of low energy incoming particles, $\abs{k}\mu^{-1/2}\ll1$,
we need  
\begin{equation}
\frac{ma^2}{\hbar^2}A=\frac{m}{\hbar^2\mu}A\ll1,
\end{equation}
for the Born approximation to be valid. (Also note that the expression
\eqref{eq:2_final} has a finite limit as $k\to0$. So there's in no
singularity to wory about as $k\to0$ either.)

And in the case of high energy incoming particles we need the incoming
particle momentum to be high enought so that
\begin{equation}
\frac{mA}{\hbar^2k\sqrt{\mu}}\ll1,
\end{equation}
for the Born approximation to be valid.








\section{H-Kr scattering}
\newcommand{\Er}{E_{\text{r}}}
\newcommand{\BW}{\text{(BW)}}

In this problem we will be using the Breit-Wigner formula
\begin{equation}\label{eq:3_BW}
\sigma_l=
\frac{4\pi}{k^2} (2l+1) \frac{(\varGamma/2)^2}{(E-\Er)^2+(\varGamma/2)^2}
\end{equation}
for the cross-section of the $l$th partial wave, eqn. (6.7.9) in
Sakurai\,\&\,Napolitano, \textit{Modern Quantum Mechanics},
ed. 2. Here $\varGamma$ is the width of the peak at half the
cross-section and can be calculated using
\begin{equation}
-\frac{2}{\varGamma} = \eval{\dv{\cot(\delta_l)}{E}}_{E=\Er},
\end{equation}
eqn. (6.7.8)  in Sakurai\,\&\,Napolitano, 
\textit{Modern Quantum Mechanics}, ed. 2. This can be rewritten as
\begin{equation}\label{eq:3_Gamma}
\varGamma=2\sin^2(\delta_l) \qty(\dv{\delta_l}{E})^{-1}.
\end{equation}


\subsection{Comparison with more elaborate calculations}
From the slides on the course web-site we get the values in
\tabref{tab:3}, and we can just start to use and compare these values
with the theory above.

\begin{table}\centering
\caption{Characteristics of the $l=4$ partial wave, from the slieds on
H-Kr cross section on the couse web-page.}
\label{tab:3}
\begin{tabular}{|c|c|c|c|}\hline
$\sigma_\text{peak}\;/[\unit{Å^2}]$  & $\Er\;/[\unit{meV}]$ &
$\varGamma\;/[\unit{meV}]$  & $\dv{\delta_l}{E}\;/[\unit{rad/meV}]$ 
\\ \hline
480 & 0.48 & 0.058 & 31.4\\ \hline
\end{tabular}
\end{table}

We can begin with the peak width. According to \eqref{eq:3_Gamma} the
peak width should be
\begin{equation}
\varGamma^\BW=2\overbrace{\sin^2(\delta_l)}^{=1}
\qty(\dv{\delta_l}{E})^{-1}
=\unit[0.064]{meV},
\end{equation}
which is reasonably close (just over 10\,\% off) to the value in
\tabref{tab:3}. Also, in both the slide and by B-W theory,
$\delta_l=\pi/2$ at resonance; so that's why the sine is set to~$1$ here.

Next up is the peak cross-section. We see that \eqref{eq:3_BW} reduces
to
\begin{equation}
\sigma_l=\frac{4\pi}{k^2} (2l+1),
\end{equation}
at $E=\Er$. Next using $E=\hbar^2k^2/(2m)$, we can express the
cross-section, at $E=\Er$, as
\begin{equation}\label{eq:3_BW_Er}
\sigma_l=\frac{2\pi\hbar^2}{m\Er} (2l+1).
\end{equation}
From here it's just plugging in the values from \tabref{tab:3} and
using $l=4$, as was given in the slides. We get
\begin{equation}
\sigma_4^{\BW}=\unit[54.3]{Å^2}\times9\approx\unit[490]{Å^2},
\end{equation}
which again is close (around 2\,\% off) to the complete numerical
calculations. 

Over all the Breit-Wigner theroy performed quite well. The discrepancy
in chiefly the width of the peak shouldn't bee too worrying. 

For the
peak width there are two main reasons that come to mind regarding the
dircrepancy. The first reason is that this is a peak having a
background of scattering from other values of $l$, so exactly where
the half point of the peak isn't very well defined. The second thing
is that \eqref{eq:3_Gamma} relies on the derivative, but the slope
given in the slides are probably rounded off to near nearest
$\unit[100]{deg/meV}$. 

For the peak cross-section, most of the discrepancy probably aslo
comes from rounding in the slide. And the background plays a role here
as well. 

\subsection{Using the B-W formula to determine $l$}
\newcommand{\ER}{E_{\text{R}}}
Given a scattering resonance of (spin-less non-relativistic) particles
of mass $\mu$, at energy $\ER$, with peak cross-section $\sigma_{\max}$, we could
easily figure out the $l$ number using \eqref{eq:3_BW_Er}. The angular
momentum would therefore be given by
\begin{equation}
l\approx\frac{\sigma_{\max} \mu \ER}{4\pi\hbar^2} - \frac{1}{2}
\qcomma\text{or}\quad
l=\left\lfloor\frac{\sigma_{\max} \mu \ER}{4\pi\hbar^2}\right\rfloor.
\end{equation}

The fact that the particle is spin-less helps us not getting tangeled
up in any coupeling between the sipn and the angular momentum or the
scattering potential.






\section{More scattering}
We have ben given (scaled) graphs of the numerically calculated
wavefunctions from scattering off of the potential
\begin{equation}
V(r)=-V_0\exp(-\frac{2r}{b}).
\end{equation}
All of the calculations have been made using the dimensionless
strength of the potentail: $g=8.8$, where
\begin{equation}
g^2=\frac{V_0}{\hbar^2/(2mb^2)}
=\frac{2V_0 mb^2}{\hbar^2}.
\end{equation}
The dimensionless particles energy is analogously 
\begin{equation}
f=\frac{2Emb^2}{\hbar^2}=k^2b^2.
\end{equation}
Note that the value of $f$ gives the reltion between $k$ and $b$.

\subsection{Total elastic cross-section at $f=0.2$}
To calculate the total cross-section we will use the formalism of
phase shifts, \eqref{eq:1_sigma}. 
And since we're at a relatively long wavelength (low energy) compared
to the potetial range, we can approximate
\begin{equation}
\sigma\approx\frac{4\pi}{k^2}\sum_{l=0}^1(2l+1)\sin^2(\delta_l).
\end{equation}
We just need to find the phase shifts $\delta_l$. From the lecture
notes we know that
\begin{equation}\label{eq:4_u_approx}
u_l(r)\approx \sin(kr-\frac{l\pi}{2} + \delta_l),
\end{equation}
for large $r$.

The free particle wavelength, outside the reach of the potential, is 
\begin{equation}
\lambda=\frac{2\pi}{k}=\frac{2\pi b}{\sqrt{f}}
\approx 14 b.
\end{equation}
This could be measured from the plots as well, and 14 seems to agree
well with the graphs. We also see that the wavelength is indeed
considerably larger than the lengthscale of the potential. 

For the s-wave, using \eqref{eq:4_u_approx}, this would mean that the,
for instance, ``third'' period should have started at
$r=2\lambda\approx28b$ if $\delta_0=0$. Instead it starts at
$x_0=r/b\approx26$, which tells us that the phse shift for the s-wave
is  
\begin{equation}
\delta_0=k(2\lambda-x_0)\approx\frac{\sqrt{0.2}}{b}(28-26)b
=0.89.
\end{equation}


The same rutine applies for the p-wave. Since $f=0.2$ hasn't changed,
the wavelength is still $\lambda\approx14b$. This time however the,
for instance, ``fourth'' period should start at
$r=(3+1/4)\lambda=45.5b$ due to the new $\pi/2$ in
\eqref{eq:4_u_approx}. Instead it starts at $x_0=r/b\approx46$.
The phase shift is therefore
\begin{equation}
\delta_1=k(3.25\lambda-x_0)=k\cdot(-0.5b)= -0.22.
\end{equation}


So we end up getting
\begin{equation}
\sigma\approx\frac{4\pi b^2}{f}
\qty(\sin^2(\delta_0)+3\sin^2(\delta_1))
\approx\frac{4\pi b^2}{0.2}\qty(0.60+3\times0.05)
\approx 47 b^2.
\end{equation}


\subsection{Total elastic cross-section at $f=0$}
For zero energy (long wavelength limit) scattering the cross-section
is just
\begin{equation}
\sigma=4\pi a^2,
\end{equation}
where $a$ is the scattering length, i.e. where the
wqvefunctionintersects the $r$-axis. 

In our case we have $a\approx 13b$, so the cross-section becomes
\begin{equation}
\sigma(f{=}0) \approx 2100b^2,
\end{equation}
which is realy huge compared to what we would've some kind of
classical expectation. Such very large values are however not unheard
of, and as $a\to\infty$ we create a new bound sate.

\subsection{Higher angular momentum}
We see that the contribution, in (a), to the cross-section was much
smaller for the p-wave than the s-wave. This is due to the
\emph{threshold behaviour}, where
\begin{equation}
\delta_l\sim k^{2l+1},
\end{equation}
see eqn. (6.6.3) in Sakurai\,\&\,Napolitano, \textit{Modern
  Quantum Mechanics}, ed. 2. 

So since $k$ is quite small, this effect will kill off the
contribution from higher values of $l$. This is why we could neglect
contributions from d-wave, and higher, to the total cross-section. The
``exponential'' decay of $\delta_l$ will of course also outweigh the
linear increase in the coefficient in front of the sine in
\eqref{eq:1_sigma} 


\subsection{Bound states}
\todo[inline]{What has $l=0$ to do with the bound states. How do we
  know if there are any $l\neq0$ bound states?}
A new bound sate is creates for each time the scattering length
$a\to\infty$, we get when setting the potential strength from~0 to
$V_0$ (or 0 to $g=8.8$ in this case).This happens every time the
wavefunction gets a local maximum or minimum. 

In our case this would correspond to having \emph{three} bound states
in this potential. We goth this number from the fact that the
wavefunction has three turning-points (max. or min.), which each of
them would correspond to a new bound state.

\todo{I would like some more justification here.}
The weakest bound state will have a binding energy of
\begin{equation}
E_\text{bound}=\frac{\hbar^2}{2ma^2}\approx\frac{\hbar^2}{2m\times (13b)^2},
\end{equation}
according to eqn. (6.6.18) in Sakurai\,\&\,Napolitano, \textit{Modern
  Quantum Mechanics}, ed. 2.
This binding energy corresponds to 
\begin{equation}
f_\text{bound} = \frac{2mE_\text{bound}}{\hbar^2}
\approx \frac{1}{13^2} \approx 0.0059.
\end{equation}


\end{document}




%  LocalWords:  Sakurai Napolitano eqn
