\documentclass[11pt,letter, swedish, english
]{article}
\pdfoutput=1

\usepackage{../custom_as}
\usepackage{cancel}
\graphicspath{ {figures/} }

%%Drar in tabell och figurtexter
\usepackage[margin=10 pt]{caption}
%%För att lägga in 'att göra'-noteringar i texten
\usepackage{todonotes} %\todo{...}

%%För att själv bestämma marginalerna. 
\usepackage[
%            top    = 3cm,
%            bottom = 3cm,
%            left   = 3cm, right  = 3cm
]{geometry}

%%För att ändra hur rubrikerna ska formateras
\renewcommand{\thesubsection}{\arabic{section} (\alph{subsection})}

\renewcommand{\thesubsubsection}{\arabic{section} (\alph{subsection},\,\roman{subsubsection})}



\swapcommands{\varPhi}{\Phi}
\swapcommands{\varPi}{\Pi}
\swapcommands{\varOmega}{\Omega}



\begin{document}

%%%%%%%%%%%%%%%%% vvv Inbyggd titelsida vvv %%%%%%%%%%%%%%%%%
% \begin{titlepage}
\title{Quantum Mechanics -- PHYS\,701 \\
Assignment 5}
\author{Andréas Sundström}
\date{\today}

\maketitle

%%%%%%%%%%%%%%%%% ^^^ Inbyggd titelsida ^^^ %%%%%%%%%%%%%%%%%

%Om man vill ha en lista med vilka todo:s som finns.
%\todolist

\section{Optical theorem}
The optical theorem relates the total cross-section to the imaginary
part of the forward scattering amplitude. 

To prove this theorem we use the scattering aplitude representd using
phase shift:
\begin{equation}
f(\theta)=\frac{1}{k}\sum_{l=0}^\infty
(2l+1)\ee^{\ii\delta_l}\sin(\delta_l)\,P_l(\cos\theta).
\end{equation}
Now the forward scattering will be at $\theta=0$. Due to the way
we ``normalize'' Legendre polynomials\footnotemark{}, we have $P_l(1)=1$
for all $l$. So that the forward scatering amplitude is
\begin{equation}
f(0)=\frac{1}{k}\sum_{l=0}^\infty
(2l+1)\ee^{\ii\delta_l}\sin(\delta_l),
\end{equation}
and its imaginary part is
\begin{equation}\label{eq:1_Im(f0)}
\Im[f(0)]=\frac{1}{k}\sum_{l=0}^\infty
(2l+1)\Im[\ee^{\ii\delta_l}]\sin(\delta_l)
=\frac{1}{k}\sum_{l=0}^\infty (2l+1)\sin^2(\delta_l).
\end{equation}

\footnotetext{The ``normalization'' is:
$\oldint_{-1}^1 P_n(x)P_m(x)\id{x}=\frac{2}{2n+1}\delta_{m,\,n}$.}

Next up is the total scattering cross-section
\begin{equation}
\sigma=\frac{4\pi}{k^2}\sum_{l=0}^\infty
(2l+)\sin^2(\delta_l),
\end{equation}
which only differs from \eqref{eq:1_Im(f0)} by a factor of
$4\pi/k$. In other words
\begin{equation}
\sigma=\frac{4\pi}{k}\Im[f(0)].
\end{equation}
This is the optical theorem.
\qed

\section{Gaussian potential scattering}



\section{H-Kr scattering}


\section{More scattering}



\end{document}



