\documentclass[11pt,letter, swedish, english
]{article}
\pdfoutput=1

\usepackage{../custom_as}
\usepackage[makeroom
]{cancel}
\graphicspath{{figures/}}

\swapcommands{\Omega}{\varOmega}
\swapcommands{\Lambda}{\varLambda}

%%Drar in tabell och figurtexter
\usepackage[margin=10 pt]{caption}
%%För att lägga in 'att göra'-noteringar i texten
\usepackage{todonotes} %\todo{...}

%%För att själv bestämma marginalerna. 
\usepackage[
%            top    = 2.5cm,
%            bottom = 3cm,
%            left   = 3cm, right  = 3cm
]{geometry}

%%För att ändra hur rubrikerna ska formateras
%\renewcommand{\thesubsection}{\arabic{section} (\roman{subsection})}
\renewcommand{\thesubsection}{\arabic{section} (\alph{subsection})}
\renewcommand{\thesubsubsection}{\arabic{section} (\alph{subsection},\,\roman{subsubsection})}

%\renewcommand{\thefootnote}{\fnsymbol{footnote}}

\newcommand{\Tc}{\ensuremath{T_{\text{c}}}}
\newcommand{\sign}{\ensuremath{\text{sign}}}

%\usepackage{tikz}

\begin{document}

%\tikzstyle{every picture}+=[remember picture]
%\tikzstyle{na} = [shape=rectangle,inner sep=0pt,text depth=0pt]



%%%%%%%%%%%%%%%%% vvv Inbyggd titelsida vvv %%%%%%%%%%%%%%%%%

\title{Statistical Physics 2 -- PHYS\,705 \\
Assignment 5}
\author{Andréas Sundström}
\date{\today}

\maketitle

%%%%%%%%%%%%%%%%% ^^^ Inbyggd titelsida ^^^ %%%%%%%%%%%%%%%%%

\section{Wavefunction renormalization}
Here we want to find the renormalization, $z$, for $\sigma$ and
$\vb*\pi$ in the non-linear $\sigma$ model. It is defined by 
$\sigma'(\vb*k') = z^{-1}\sigma_<(\vb*k)$ and
$\vb*\pi'(\vb*k') = z^{-1}\vb*\pi_<(\vb*k)$. What we want to show is
that
\begin{equation}\label{eq:1_want}
\frac{z}{b^{d}} = 1 -\frac{n-1}{2}I_1,
\end{equation}
where
\begin{equation}
I_1:= \int_{\Lambda/b}^\Lambda \frac{\rd^dq}{(2\pi)^d}G_0(\vb*q)
= \int_{\Lambda/b}^\Lambda \frac{\rd^dq}{(2\pi)^d} \frac{T}{q^2}
= \frac{\mathcal{S}_d T}{(2\pi)^d} \Lambda^{d-2} \Delta{l},
\end{equation}
where $\mathcal{S}_d$ is the area of a unit sphere in $d$ dimensions
and $\ee^{\Delta{l}} = b$.


\section{The Kosterlitz recursion relations}

We are given the Kosterlitz recursion relations in $d=2+\epsilon$
dimensions:
\begin{align}
\label{eq:2_dT/dl}
\dv{T}{l} =& \epsilon T +4\pi^3y^2\\
\label{eq:2_dy/dl}
\dv{y}{l} =& \qty(2 -\frac{\pi}{T})y.
\end{align}











\end{document}




%  LocalWords:  MFT MF Ising Ornstein Zernike Stratonovich GLW RG
%  LocalWords:  rescale quartic rescaled anisotropy
