\documentclass[11pt,letter, swedish, english
]{article}
\pdfoutput=1

\usepackage{../custom_as}
\usepackage[makeroom
]{cancel}
\graphicspath{{figures/}}

\swapcommands{\Omega}{\varOmega}
\swapcommands{\Lambda}{\varLambda}

%%Drar in tabell och figurtexter
\usepackage[margin=10 pt]{caption}
%%För att lägga in 'att göra'-noteringar i texten
\usepackage{todonotes} %\todo{...}

%%För att själv bestämma marginalerna. 
\usepackage[
%            top    = 2.5cm,
%            bottom = 3cm,
%            left   = 3cm, right  = 3cm
]{geometry}

%%För att ändra hur rubrikerna ska formateras
%\renewcommand{\thesubsection}{\arabic{section} (\roman{subsection})}
\renewcommand{\thesubsection}{\arabic{section} (\alph{subsection})}
\renewcommand{\thesubsubsection}{\arabic{section} (\alph{subsection},\,\roman{subsubsection})}

%\renewcommand{\thefootnote}{\fnsymbol{footnote}}

\newcommand{\Tc}{\ensuremath{T_{\text{c}}}}
\newcommand{\sign}{\ensuremath{\text{sign}}}

%\usepackage{tikz}

\begin{document}

%\tikzstyle{every picture}+=[remember picture]
%\tikzstyle{na} = [shape=rectangle,inner sep=0pt,text depth=0pt]



%%%%%%%%%%%%%%%%% vvv Inbyggd titelsida vvv %%%%%%%%%%%%%%%%%

\title{Statistical Physics 2 -- PHYS\,705 \\
Assignment 5}
\author{Andréas Sundström}
\date{\today}

\maketitle

%%%%%%%%%%%%%%%%% ^^^ Inbyggd titelsida ^^^ %%%%%%%%%%%%%%%%%

\section{Wavefunction renormalization}
Here we want to find the renormalization, $z$, for $\sigma$ and
$\vb*\pi$ in the non-linear $\sigma$ model. It is defined by 
$\sigma'(\vb*k') = z^{-1}\sigma_<(\vb*k)$ and
$\vb*\pi'(\vb*k') = z^{-1}\vb*\pi_<(\vb*k)$. What we want to show is
that
\begin{equation}\label{eq:1_want}
\frac{z}{b^{d}} = 1 -\frac{n-1}{2}I_1,
\end{equation}
where
\begin{equation}
I_1:= \int_{\Lambda/b}^\Lambda \frac{\rd^dq}{(2\pi)^d}G_0(\vb*q)
= \int_{\Lambda/b}^\Lambda \frac{\rd^dq}{(2\pi)^d} \frac{T}{q^2}
= \frac{\mathcal{S}_d T}{(2\pi)^d} \Lambda^{d-2} \Delta{l},
\end{equation}
where $\mathcal{S}_d$ is the area of a unit sphere in $d$ dimensions
and $\ee^{\Delta{l}} = b$.







By Wick's theorem, we know that
\begin{equation}
\ev{\pi_a(\vb*k_1)\pi_b(\vb*k_2)}
= \delta_{a, b} G_0(\vb*k_1) \,(2\pi)^d \delta(\vb*k_1 + \vb*k_2).
\end{equation}
We can use that to calculate
\begin{equation}
\begin{aligned}
\ev{(\vb*\pi(\vb*x))^2} =& \frac{1}{Z_0}
\int \mathcal{D}\vb*\pi\; 
[\vb*\pi(\vb*x)\vdot\vb*\pi(\vb*x)] \ee^{-S[\vb*\pi]}\\
=& \frac{1}{Z_0} \int \mathcal{D}\vb*\pi
\qty[\int\frac{\rd^dk_1}{(2\pi)^d}\frac{\rd^dk_2}{(2\pi)^d}
\vb*\pi(\vb*k_1)\vdot\vb*\pi(\vb*k_2) \ee^{\ii\vb*x\vdot(\vb*k_1+\vb*k_2)}]
\ee^{-S[\vb*\pi]} \\
%\end{aligned}
%\end{equation}
%To take care of the Fourier exponential, we can set
%$\vb*x=\vb*0$. This is since 
%\begin{equation}
%\begin{aligned}
%\ev{(\vb*\pi(\vb*0))^2}
=& \int\frac{\rd^dk_1}{(2\pi)^d}\frac{\rd^dk_2}{(2\pi)^d}
\ee^{\ii\vb*x\vdot(\vb*k_1+\vb*k_2)}
\underbrace{\frac{1}{Z_0} \int \mathcal{D}\vb*\pi\; 
\vb*\pi(\vb*k_1)\vdot\vb*\pi(\vb*k_2) 
\ee^{-S[\vb*\pi]}}_{=\sum_a\ev{\pi_a(\vb*k_1)\pi_a(\vb*k_2)}}\\
=& \sum_a \int\frac{\rd^dk}{(2\pi)^d}G_0(\vb*k)
= (n-1) \int\frac{\rd^dk}{(2\pi)^d}G_0(\vb*k).
\end{aligned}
\end{equation}
The factor $(n-1)$ comes from the fact that $\vb*\pi$ has $(n-1)$
components. 









\section{The Kosterlitz recursion relations}
\newcommand{\TKT}{T_{\text{KT}}}

We are given the Kosterlitz recursion relations in $d=2+\epsilon$
dimensions:
\begin{align}
\label{eq:2_dT/dl}
\beta_T := \dv{T}{l} =& \epsilon T +4\pi^3y^2\\
\label{eq:2_dy/dl}
\beta_y := \dv{y}{l} =& \qty(2 -\frac{\pi}{T})y.
\end{align}


\subsection{The finite-temperature fixed point}
The fixed point, at $T>0$, is given by setting
\begin{equation}
\epsilon T +4\pi^3y^2 = 0
\quad\Longrightarrow\quad
y = \pm\sqrt{\frac{\epsilon T}{4\pi^3}} \neq0,
\end{equation}
which gives
\begin{equation}
T^* = \TKT = \frac{\pi}{2}
\end{equation}
and 
\begin{equation}
y^* = \pm\frac{\sqrt{\epsilon}}{2^{5/2},\pi}.
\end{equation}
\todo{Is $y\ge0$, so that we can drop the $\pm$.}

\subsection{Scaling dimensions}
To get the scaling dimensions we need the eigenvalues of 
\begin{equation}
M = \eval{
\begin{pmatrix}
\pdv{\beta_T}{T} & \pdv{\beta_T}{y}\\
\pdv{\beta_y}{T} & \pdv{\beta_y}{y}
\end{pmatrix}
}_{T^*, y^*}
=
\begin{pmatrix}
-\epsilon & 8\pi^3 y^*\\
\frac{\pi y^*}{\TKT^2} & 2-\frac{\pi}{\TKT}
\end{pmatrix}
=
\begin{pmatrix}
-\epsilon & \pm2^{3/2}\pi^2\sqrt{\epsilon}\\
\pm \frac{\sqrt{2\epsilon}}{\pi^2} & 0
\end{pmatrix}.
\end{equation}
To get the eigenvalues we use the characteristic equation
\begin{equation}
0=
\begin{vmatrix}
(-\epsilon -\lambda) & \pm2^{3/2}\pi^2\sqrt{\epsilon}\\
\pm \frac{\sqrt{2\epsilon}}{\pi^2} & 0-\lambda
\end{vmatrix}
= \lambda^2 + \epsilon\lambda - 4\epsilon,
\end{equation}
which has the solutions
\begin{equation}
\lambda_\pm = \frac{1}{2}\qty[ -\epsilon
\pm\sqrt{\epsilon^2 + 16\epsilon}]
= \frac{1}{2}\qty[ -\epsilon
\pm4\sqrt{\epsilon}\sqrt{1+\frac{\epsilon}{16}}].
\end{equation}
\todo[inline]{Is the lowes order $\sqrt{\epsilon}$?}







\subsection{Estimate of $\nu$ and $\alpha$}








\end{document}




%  LocalWords:  MFT MF Ising Ornstein Zernike Stratonovich GLW RG
%  LocalWords:  rescale quartic rescaled anisotropy
