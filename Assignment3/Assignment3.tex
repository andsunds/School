\documentclass[11pt,letter, swedish, english
]{article}
\pdfoutput=1

\usepackage{../custom_as}

\graphicspath{ {figures/} }

%%Drar in tabell och figurtexter
\usepackage[margin=10 pt]{caption}
%%För att lägga in 'att göra'-noteringar i texten
\usepackage{todonotes} %\todo{...}

%%För att själv bestämma marginalerna. 
\usepackage[
%            top    = 3cm,
%            bottom = 3cm,
%            left   = 3cm, right  = 3cm
]{geometry}

%%För att ändra hur rubrikerna ska formateras
\renewcommand{\thesubsection}{\arabic{section} (\alph{subsection})}

\renewcommand{\thesubsubsection}{\arabic{section} (\alph{subsection},\,\roman{subsubsection})}



\swapcommands{\varPhi}{\Phi}



\begin{document}

%%%%%%%%%%%%%%%%% vvv Inbyggd titelsida vvv %%%%%%%%%%%%%%%%%
% \begin{titlepage}
\title{Quantum Mechanics -- PHYS\,701 \\
Assignment 3}
\author{Andréas Sundström}
\date{\today}

\maketitle

%%%%%%%%%%%%%%%%% ^^^ Inbyggd titelsida ^^^ %%%%%%%%%%%%%%%%%

%Om man vill ha en lista med vilka todo:s som finns.
%\todolist

\section{A special case of the Baker-Campbell-Hausdorf theorem}
Here we want to show that
\begin{equation}\label{eq:1_want_this}
\ee^{A+B} = \ee^A\ee^B\ee^{-\frac{1}{2}\comm{A}{B}}
\end{equation}
for two operators $A$ and $B$ wich both commute with $\comm{A}{B}$,
but not necessarily with each other.

We begin by studying
\begin{equation}
\comm{B}{\ee^{x A}}=\sum_{n=0}^\infty \frac{x^n}{n!}\comm{B}{A^n},
\end{equation}
where $x$ is just a number (\emph{not} the operator). To expand this
sum we can use the formula derived in the last problem
set\footnotemark{}:
\begin{equation}
\comm{B}{A^{m}} =\sum_{i=0}^{m-1} A^{i}\comm{B}{A}A^{m-1-i}.
\end{equation}
In this case we aslo know that $A$ commutes with $\comm{B}{A}$, which
means that we can write
\begin{equation}\label{eq:1_comm}
\comm{B}{\ee^{x A}}=\sum_{n=0}^\infty \frac{x^n}{n!}
\sum_{k=0}^{n-1}A^{n-1}\comm{B}{A}
=\sum_{n=0}^\infty \frac{x^n}{n!}
nA^{n-1}\comm{B}{A}
=x\ee^{xA}\comm{B}{A}.
\end{equation}
Here the case $n=0$ should be regraded as the whole term beeing equal
to $0$, since $k$ starts above the summation limit.
\footnotetext{I have attached a copy of the relavent derivation in Appendix~\ref{sec:comm}}

Nex up we study the function
\begin{equation}\label{eq:1_G_def}
G(x)=\ee^{xA}\ee^{xB}
=\sum_{i=0}^\infty\frac{x^iA^i}{i!}\sum_{j=0}^\infty\frac{x^jB^j}{j!}.
\end{equation}
It's derivative is
\begin{equation}
\begin{aligned}
\dv{G}{x}=&
\sum_{i=1}^\infty\frac{x^{i-1}A^i}{(i-1)!}\sum_{j=0}^\infty\frac{x^jB^j}{j!}
+\sum_{i=0}^\infty\frac{x^iA^i}{i!}\sum_{j=1}^\infty\frac{x^{j-1}B^j}{(j-1)!}\\
=&A\ee^{xA}\ee^{xB} + \ee^{xA}B\ee^{xB}.
\end{aligned}
\end{equation}
Now we use the relation \eqref{eq:1_comm} to rewrite
\begin{equation}
\ee^{xA}B = B\ee^{xA}-\comm{B}{\ee^{xA}}
=B\ee^{xA}-x\ee^{xA}\comm{B}{A}.
\end{equation}
All in all we get
\vspace{-4mm}
\begin{equation}
\dv{G}{x}=A\ee^{xA}\ee^{xB}+\qty(B\ee^{xA}-x\ee^{xA}\comm{B}{A})\ee^{xB}
=\Big(A+B+x\comm{A}{B}\Big)\overbrace{\ee^{xA}\ee^{xB}}^{=G(x)},
\end{equation}
which is a differential equation for $G$ and it results in
\begin{equation}
G(x)=\exp[x(A+B)+\frac{1}{2}x^2\comm{A}{B}].
\end{equation}
But with \eqref{eq:1_G_def} we have
\begin{equation}\label{eq:1_almost}
\ee^{xA}\ee^{xB}=\exp[x(A+B)+\frac{1}{2}x^2\comm{A}{B}]
=\ee^{x(A+B)}\ee^{\frac{1}{2}x^2\comm{A}{B}}.
\end{equation}
The last step is possible since $\comm{A}{B}$ commutes with both $A$
and $B$. By choosing $x=1$ we're almost where we want to be. 

All that's left is to move $\ee^{\frac{1}{2}x^2\comm{A}{B}}$ to the
other side. To do this we note that for any operator $\Xi$
\begin{equation}
\ee^{\Xi}\ee^{-\Xi}=\sum_{i\ge0}\sum_{j\ge0}(-1)^j\frac{\Xi^{i+j}}{i!j!}
=\sum_i\sum_{j=2n}\frac{\Xi^{i+2n}}{i!(2n)!}
-\sum_i\sum_{j=2n+1}\frac{\Xi^{i+2n+1}}{i!(2n+1)!}.
\end{equation}
Now since $i$ is still free to run over all values in both sums, we
will get two sums where they almost cancel each other out. The
exception is thet the second double sum can't produce any $\Xi^0$
term. Therefore
\begin{equation}
\ee^{\Xi}\ee^{-\Xi}=1,
\end{equation}
in the sence of $\Xi^0$ beeing the identity operator. 

With this result its now safe to ``multiply'' both sides of
\eqref{eq:1_almost} by $\ee^{-\frac{1}{2}x^2\comm{A}{B}}$, and we get
\begin{equation}
\ee^{A}\ee^{B}\ee^{-\frac{1}{2}\comm{A}{B}} = \ee^{A+B},
\end{equation}
with $x=1$.
\qed


\section{Coherent states}
A coherent state of the 1D harmonic oscillator is an eigenstate of the
anihilation operator:
\begin{equation}
a\ket{\lambda}=\lambda\ket{\lambda}\qcomma
\lambda\in\C.
\end{equation}

\subsection{A normalized coherent state}
Here we're going to prove that
\begin{equation}\label{eq:2_coherent}
\ket\lambda=\ee^{-|\lambda|^2/2}\ee^{\lambda a^\dagger}\ket0
\end{equation}
is a normalized coherent state.
It will be helpful to express $\ket\lambda$ as it's power series
\begin{equation}\label{eq:2_lambda_power}
\ket\lambda = \ee^{-|\lambda|^2/2}\sum_{k}
\frac{\lambda^k}{k!} (a^\dagger)^k \ket{0}
= \ee^{-|\lambda|^2/2}\sum_{k}
\frac{\lambda^k}{k!}\sqrt{k!}\ket{k}.
\end{equation}
Here we used the property of the creation operator that
\begin{equation}
a^\dagger\ket{n-1}=\sqrt{n}\ket{n}.
\end{equation}

\subsubsection{Normalized}
To prove that $\ket\lambda$ is normalized we just use
\eqref{eq:2_lambda_power} to express
\begin{equation}
\ip{\lambda} = \ee^{-|\lambda|^2}\sum_{m, n}
\bra{m}\frac{(\lambda^\dagger)^m}{\sqrt{m!}}
\frac{\lambda^n}{\sqrt{n!}}\ket{n}.
\end{equation}
From here we use $\ip{m}{n}=\delta_{m, n}$ and get
\begin{equation}
\ip{\lambda} = \ee^{-|\lambda|^2}
\sum_n\frac{(\lambda^\dagger\lambda)^n}{n!}
=\ee^{-|\lambda|^2}\ee^{+|\lambda|^2}=1.
\end{equation}
\qed

\subsubsection{Eigenket}
We apply $a$ to $\ket\lambda$:
\begin{equation}\label{eq:2_eigenket}
\begin{aligned}
a\ket\lambda =& \ee^{-|\lambda|^2/2}\sum_{k=0}^\infty
\frac{\lambda^k}{\sqrt{k!}} a \ket{k}
= \ee^{-|\lambda|^2/2}\sum_{k=0}^\infty
\lambda\frac{\lambda^{k}}{\sqrt{k!}}\sqrt{k} \ket{k-1}\\
=& \ee^{-|\lambda|^2/2}\sum_{k=1}^\infty
\lambda\frac{\lambda^{k-1}}{\sqrt{(k-1)!}} \ket{k-1}\\
&\hspace{-24pt}\stackrel{\{n=k-1\}}{=} \lambda \ee^{-|\lambda|^2/2}\sum_{n=0}^\infty
\lambda\frac{\lambda^{n}}{\sqrt{n!}} \ket{n}
&=\lambda\ket{\lambda}.
\end{aligned}
\end{equation}
\qed

\subsection{Uncertainty relation}
Here we're going to prove the minimum uncertainty relation
\begin{equation}\label{eq:2_unc}
\ev**{(\Delta{x})^2}{\lambda}\ev**{(\Delta{p})^2}{\lambda}
=\frac{\hbar^2}{4}.
\end{equation}
To do this we use the relations
\begin{equation}\label{eq:2_xp}
x=\sqrt{\frac{\hbar}{2m_e\omega}}\qty(a+a^\dagger)\qcomma
p=\ii\sqrt{\frac{m_e\hbar\omega}{2}}\qty(-a+a^\dagger),
\end{equation}
which follows directly from the definitions of $a$ and $a^\dagger$.

Now all we have to do is evaluate the four different expectation
values. We begin with
\begin{equation}
\ev{x}{\lambda}=\ee^{-|\lambda|^2}
\sum_{m=0}^\infty\sum_{n=0}^\infty\sqrt{\frac{\hbar}{2m_e\omega}}
\bra{m}\frac{(\lambda^\dagger)^m\lambda^{n}}{\sqrt{m!n!}}\qty(a+a^\dagger)
\ket{n}
\end{equation}
We evaluate the terms:
\begin{equation}
\mel{m}{a+a^\dagger}{n}=\mel{m}{\sqrt{n}}{n-1}+\mel{m}{\sqrt{n+1}}{n+1}
=\sqrt{n}\delta_{m, n-1}+\sqrt{n+1}\delta_{m, n+1}.
\end{equation}
This gives
\begin{equation}
\begin{aligned}
\ev{x}{\lambda}=&\ee^{-|\lambda|^2}
\sqrt{\frac{\hbar}{2m_e\omega}}\sum_{n=0}^\infty
\qty[
\lambda\frac{|\lambda|^{n-1}}{\sqrt{(n-1)!n!}}\sqrt{n} 
+\lambda^\dagger\frac{|\lambda|^{n}}{\sqrt{(n+1)!n!}}\sqrt{n+1} 
]\\
=&\ee^{-|\lambda|^2}\sqrt{\frac{\hbar}{2m_e\omega}}
\qty[
\sum_{n=1}^\infty\lambda\frac{|\lambda|^{n-1}}{(n-1)!}
+\sum_{n=0}^\infty\lambda^\dagger\frac{|\lambda|^{n}}{n!}
]\\
=&\ee^{-|\lambda|^2}\sqrt{\frac{\hbar}{2m_e\omega}}
\qty[\lambda\ee^{|\lambda|^2}
+\lambda^\dagger\ee^{|\lambda|^2}]
=2\Re[\lambda]\sqrt{\frac{\hbar}{2m_e\omega}}.
\end{aligned}
\end{equation}
The expectation value of $p$ follows in a completly analogous way: 
\begin{equation}
\ev{p}{\lambda}=\ii\sqrt{\frac{m_e\hbar\omega}{2}}\qty(\lambda^\dagger-\lambda)
=2\Im[\lambda]\sqrt{\frac{m_e\hbar\omega}{2}}.
\end{equation}

For $\ev{x^2}$ and $\ev{p^2}$ we just use \eqref{eq:2_xp} to get
\begin{equation}
x^2=\frac{\hbar}{2m_e\omega}
\qty(a^2+(a^\dagger)^2+aa^\dagger+a^{\dagger}a)\qcomma
p^2=\frac{m_e\hbar\omega}{2}
\qty(-a^2+-(a^\dagger)^\dagger + aa^\dagger + a^{\dagger}a),
\end{equation}
and
\begin{equation}
\begin{aligned}
\mel**{m}{\pm{}a^2\pm(a^\dagger)^2+aa^\dagger+a^{\dagger}a}{n}=&
\pm\sqrt{n(n-1)}\delta_{m, n-2} \pm\sqrt{(n+1)(n+2)}\delta_{m, n+2}\\
&+(2n+1)\delta_{m, n}. 
\end{aligned}
\end{equation}
With this we get, after some calculations,
\begin{equation}
\ev{x^2}{\lambda}=\ldots=
\frac{\hbar}{2m_e\omega}\qty(2\Re[\lambda^2] + 2|\lambda|^2 + 1)
\end{equation}
and
\begin{equation}
\ev{p^2}{\lambda}=\ldots=
\frac{m_e\hbar\omega}{2}\qty(1+2|\lambda|^2 - 2\Re[\lambda^2]).
\end{equation}

Now we just have to evaluate the uncertainty product
\begin{equation}
\begin{aligned}
\qty(\ev{x^2}{\lambda}-\ev{x}{\lambda}^2)
\qty(\ev{p^2}{\lambda}-\ev{p}{\lambda}^2)
=&\frac{\hbar^2}{4}
\qty(2\Re[\lambda^2] + 2|\lambda|^2 + 1 - 2\Re[\lambda])\\
&\times\qty(1+2|\lambda|^2 - 2\Re[\lambda^2]-2\Im[\lambda]).
\end{aligned}
\end{equation}
To be frank, at this point I don't eaven want to try to do this by
hand. So we turn to \emph{Mathematica}, which actually says that
\begin{equation}
\qty(2\Re[\lambda^2] + 2|\lambda|^2 + 1 - 2\Re[\lambda])
\times\qty(1+2|\lambda|^2 - 2\Re[\lambda^2]-2\Im[\lambda])
=1\qcomma\forall\lambda\in\C.
\end{equation}
We do therefore have the minimum uncertainty relation
\eqref{eq:2_unc}. 
\qed

\subsection{Poisson distributed energy}
A poisson distribution is given by
\begin{equation}
\mathcal{P}_\Lambda^{(\text{poisson})}(k)=\frac{\Lambda^k\ee^{-\Lambda}}{k!}
\end{equation}
for some parameter $\Lambda$. The expaectation value of a poisson
distribution is given by $\Lambda$ and the value with highest
probability to occur is $\lfloor\Lambda\rfloor$ ($\Lambda$ rounded
down to the nearest integer).

By writing
\begin{equation}
\ket{\lambda}=\sum_n f(n)\ket{n}
\end{equation}
and using \eqref{eq:2_lambda_power}, we see that
\begin{equation}
\abs{f(n)}^2=\frac{\qty(|\lambda|^2)^n\ee^{-|\lambda|^2}}{n!}
=\mathcal{P}_\lambda(n).
\end{equation}
This is clearly the same distribution as above with
$\Lambda=|\lambda|^2$. Therefore the expectation value of the state's
energy is
\begin{equation}
\ev{E}{\lambda}="E_{|\lambda|^2}"=\hbar\omega|\lambda|^2+\frac{1}{2}
\end{equation}
and the most probable enegry is
\begin{equation}
\hbar\omega\left\lfloor|\lambda|^2\right\rfloor+\frac{1}{2}.
\end{equation}
\qed

\subsection{Translating the ground state}
We want to show that an in space translated ground state,
\begin{equation}
\exp[-\ii\frac{p\ell}{\hbar}]\ket{0}=\ee^{\eta(-a+a^\dagger)}\ket{0}
\end{equation}
where $\eta=\sqrt{m_E\ell^2\omega/(2\hbar)}$, $p$ is the momentum
operator and $\ell$ is a finite displacement, is a coherent state. All
we have to do is to prove that this is an eigenstate of $a$.

To do this we will use the relation \eqref{eq:1_want_this} to write
\begin{equation}
\ee^{-\eta a + \eta a^\dagger}=\ee^{-\eta a}\ee^{\eta a^\dagger}
\ee^{-\frac{\eta^2}{2}\comm{-a}{a^\dagger}}.
\end{equation}
We can do this since $\comm{a}{a^\dagger}=1$ obviuosly commutes with
both $a$ and $a^\dagger$. This makes it posible to write the
translated ground state as
\begin{equation}
\exp[-\ii\frac{p\ell}{\hbar}]\ket{0}
=\ee^{-\eta a}\qty(\ee^{\eta^2/2}\ee^{\eta a^\dagger}\ket{0}),
\end{equation}
where the expression in the parenthesis is of the same form as in
\eqref{eq:2_coherent} appart from the sign of $\eta^2/2$, but the sign
only affects the normalization, not the fact that it's an eigenstate
of $a$. Furthermore, $a$ commutes with $\ee^{-\eta a}$, so 
\begin{equation}
\begin{aligned}
a\exp[-\ii\frac{p\ell}{\hbar}]\ket{0}&=
\ee^{-\eta a}a\qty(\ee^{\eta^2/2}\ee^{\eta a^\dagger}\ket{0})\\
&\hspace{-2pt}\stackrel{\eqref{eq:2_eigenket}}{=}
\ee^{-\eta a}\eta\qty(\ee^{\eta^2/2}\ee^{\eta a^\dagger}\ket{0})
=\eta\exp[-\ii\frac{p\ell}{\hbar}]\ket{0}.
\end{aligned}
\end{equation}
Or in other words, the translated ground state is an eigenstate of the
anihilations operator $a$, which means that it's also a coherent
state. 



\section{Harmonic oscillator and the Feynman path integral}




\section{The Feynman path integral and EM potentials}
Here we're going to show the equivalence of the Feynman pathintegral
to the Schrödinger equation in the presence of electromagnetic
potentials. We're also going to show that the mid-point prescrition is
required. 

First off is some definitions of the variables used here.
The Laplacian
\begin{equation}
\mathcal{L}=\frac{1}{2}m v^2 - q\varphi(x, t) + qvA(x, t),
\end{equation}
which will be exponentiated an integrated. We will restrict oursevles
to the 1D case, so no vector notetion needed. Then there's
\begin{equation}
\epsilon=\Delta{t}\to0
\end{equation}
and
\begin{equation}
\eta=\Delta{x}=x''-x'
\quad\Longrightarrow\quad
v=\frac{\eta}{\epsilon}.
\end{equation}
And lastly we will be using the parameter $\alpha\in[0, 1]$ in
\begin{equation}
x'_\alpha=x'+\alpha\eta=\alpha x'' + (1-\alpha)x'
\end{equation}
to determine where the potentials shall be evaluated.

Next a few word on where we're heading. Since we want to show the
equivalence between the shrödinger equatoin and the Feynman
pathintegral, we will need to look at the derivative
\begin{equation}
\pdv{\psi(x'', t')}{t'} = \frac{\psi(x'', t'+\epsilon)-\psi(x'', t')}{\epsilon}.
\end{equation}
this means that we will need to expand the integrand in the pathintegral
\begin{equation}
\begin{aligned}
\psi(x'', t'+\epsilon)=&\frac{1}{w(\epsilon)}\int_{-\infty}^{\infty}\!\rd{x'}
\exp[\frac{\ii}{\hbar}\epsilon
\mathcal{L}\qty(\frac{\eta}{\epsilon}, x'_\alpha, t')]\psi(x', t')\\
=&\frac{1}{w(\epsilon)}\int_{-\infty}^{\infty}\!\rd{x'}
\exp[\frac{\ii}{\hbar}\frac{m}{2}\frac{\eta^2}{\epsilon}]
\exp[-\frac{\ii}{\hbar}\epsilon q\varphi(x'_\alpha)]
\exp[\frac{\ii}{\hbar} q\eta A(x'_\alpha)]
\psi(x', t')
\end{aligned}
\end{equation}
in power series in $\epsilon$. I've also for brevity not written out
the explicit time dependence of $\varphi$ and $A$. It turns out that
the best way to do this expansion is to leav the first exponential
intact and then expand the rest of the fractors in the integrand in
terms of both $\epsilon$ and $\eta$.

The expansions are simple to do, so I'll just list them below:
\begin{enumerate}[label=(\Roman*)]
\item $\displaystyle
\psi(x', t')=\sum_{j}  \frac{(-\eta)^j}{j!}\eval{\pdv[j]{\psi}{{x'}}}_{x'', t'}
$
\item $\displaystyle
\exp[-\frac{\ii}{\hbar}\epsilon q\varphi(x'_\alpha)]
=\sum_k \frac{1}{k!} 
\qty[-\frac{\ii q\epsilon}{\hbar} \varphi(x'_\alpha)]^k
=\sum_k \frac{1}{k!} 
\qty[-\frac{\ii q\epsilon}{\hbar} 
\sum_l \frac{[(\alpha-1)\eta]^l}{l!}\eval{\pdv[l]{\varphi}{{x'}}}_{x''}
]^k
$
\item $\displaystyle
\exp[\frac{\ii}{\hbar} q\eta A(x'_\alpha)]=\ldots
=\sum_m \frac{1}{m!} 
\qty[-\frac{\ii q\eta}{\hbar} 
\sum_n \frac{[(\alpha-1)\eta]^n}{n!}\eval{\pdv[n]{A}{{x'}}}_{x''}
]^m
$
\end{enumerate}
The only detail that need som clarification is that all of the $x'$
derivatives are evaluated at $x''$, which is independent of the
integrationvariable. This also means the last two Taylor expansions
should contain the expression $(x'_\alpha-x'')^a$, but that can be
rewritten as $[(\alpha-1)\eta]^a$.

After this we now see that the pathintegral can be done over $\eta$
instead of $x'$.
The integral will furthermore be constituted of sums of integrals of
the form 
\begin{equation}\label{eq:4_int_form}
\int_{-\infty}^{\infty}\!\rd{\eta}\,\eta^d
\exp[\frac{\ii}{\hbar}\frac{m}{2}\frac{\eta^2}{\epsilon}]
=\begin{cases}
\sqrt{\frac{2\pi\ii\epsilon}{m}},&d=0\\
0,&d=1\\
\frac{\sqrt{\pi}}{2}\qty(\frac{2\ii\hbar\epsilon}{m})^{2/3},&d=2.
\end{cases}
\end{equation}
It's worth noting that the integral in reality doesnt converge for
since ther's an $\ii$ in the exponetial, but we disregrad that
caveat and soldier on. It's also not woth worrying about larger $d$'s
since $\epsilon$ is infinitesimal. 

The weight function is
\begin{equation}
\frac{1}{w(\epsilon)}=\sqrt{\frac{m}{2\pi\ii\epsilon}}.
\end{equation}
And since we will only expand the pathintegral to $\order{\epsilon}$,
we can list the cases which will contribute to the zeroth and first
order terms:
\begin{enumerate}[label=(\roman*)]
\item $(j, k, l, m, n)=(0, 0, 0, 0, 0)$ giving an $\order{1}$ term and a
$d=0$ integral. 
\item $(j, k, l, m, n)=(0, 1, 0, 0, 0)$ giving an $\order{\epsilon}$ term
and a $d=0$ integral. 
\item $(j, k, l, m, n)=(2, 0, 0, 0, 0)$ giving an $\order{\epsilon}$ term
and a $d=2$ integral.
\item $(j, k, l, m, n)=(1, 0, 0, 1, 0)$ giving an $\order{\epsilon}$ term
and a $d=2$ integral.
\item $(j, k, l, m, n)=(0, 0, 0, 1, 1)$ giving an $\order{\epsilon}$ term
and a $d=2$ integral.
\item $(j, k, l, m, n)=(0, 0, 0, 2, 0)$ giving an $\order{\epsilon}$ term
and a $d=2$ integral. 
\end{enumerate}
All of these integrals will just be of the form in
\eqref{eq:4_int_form} with a prefactor determined by $1/w(x)$ and the
factors from the relavant term in the expansion. The details of each
of the six integrations from the cases above is of little
interest. All of the calculations are very similar, so I will write
out the details of the first integral. Then I'll just list the rest of
the results below:
\begin{enumerate}[label=(\roman*)]
\item $\displaystyle
\sqrt{\frac{m}{2\pi\ii\epsilon}} \int_{-\infty}^{\infty}\!\rd{\eta}
\exp[\frac{\ii}{\hbar}\frac{m}{2}\frac{\eta^2}{\epsilon}]
\psi(x'', t')
=\sqrt{\frac{m}{2\pi\ii\epsilon}}\psi(x'', t')
\sqrt{\frac{2\pi\ii\epsilon}{m}}
=\psi(x'', t')
$
\item $\displaystyle
\epsilon\frac{q}{\ii\hbar}\varphi(x'')\psi(x'', t')
$
\item $\displaystyle
\epsilon\frac{\ii\hbar}{2m}\pdv[2]{\psi(x'', t')}{{x''}}
$
\item $\displaystyle
\epsilon\frac{q}{m}A(x'')\pdv{\psi(x'', t')}{x''}
$
\item $\displaystyle
\epsilon(1-\alpha)\frac{q}{m}\pdv{A(x'')}{x''}\psi(x'', t')
$
\item $\displaystyle
\epsilon\frac{q^2}{\ii\hbar2m}A^2(x'')\psi(x'', t')
$
\end{enumerate}
Note especially that (v) has a factor $(1-\alpha)$, originating from
expansion (III) and the fact that $m=1$ and $n=1$.


Now we just construct the pathintegral fron these terms:
\begin{equation}
\begin{aligned}
\psi(x'', t'+\epsilon)
=&\frac{1}{w(\epsilon)}\int_{-\infty}^{\infty}\!\rd{x'}
\exp[\frac{\ii}{\hbar}\frac{m}{2}\frac{\eta^2}{\epsilon}]
\exp[-\frac{\ii}{\hbar}\epsilon q\varphi(x'_\alpha)]
\exp[\frac{\ii}{\hbar} q\eta A(x'_\alpha)]
\psi(x', t')\\
=& \psi(x'', t') + \epsilon \Bigg\{
\frac{q}{\ii\hbar}\varphi(x'')\psi(x'', t')
+\frac{\ii\hbar}{2m}\pdv[2]{\psi(x'', t')}{{x''}}
+\frac{q}{m}A(x'')\pdv{\psi(x'', t')}{x''}\\
&\phantom{\psi(x'', t') + \epsilon\{ }
+(1-\alpha)\frac{q}{m}\pdv{A(x'')}{x''}\psi(x'', t')
+\frac{q^2}{\ii\hbar2m}A^2(x'')\psi(x'', t')
\Bigg\}.
\end{aligned}
\end{equation}
This can be rewritten to%\todo{there's a sign error some
%  where. Probably in the $x'\to\eta$ integration}
\begin{equation}
\begin{aligned}
=\ii\hbar\frac{\psi(x'', t'+\epsilon)-\psi(x'', t')}{\epsilon}
=&  \Bigg\{
q\varphi(x'')
+\frac{(\ii\hbar)^2}{2m}\pdv[2]{{x''}}
+\frac{q\ii\hbar}{m}A(x'')\pdv{x''}\\
&\phantom{(}
+(1-\alpha)\frac{q\ii\hbar}{m}\pdv{A(x'')}{x''}
+\frac{q^2}{2m}A^2(x'')
\Bigg\}\psi(x'', t')
\end{aligned}
\end{equation}
or in other words
\begin{equation}
\begin{aligned}
\ii\hbar\pdv{\psi}{{t'}}
=&  \Bigg\{\frac{1}{2m}\Bigg[
(\ii\hbar)^2\pdv[2]{{x''}}
+2q\ii\hbar A(x'')\pdv{x''}\\
&\phantom{(2m)}
+2(1-\alpha)q\ii\hbar\pdv{A(x'')}{x''}
+q^2A^2(x'')\Bigg]
+q\varphi(x'')\Bigg\}\psi(x'', t')
\end{aligned}
\end{equation}
We want this to be equivalent to 
\begin{equation}
\begin{aligned}
\ii\hbar\pdv{\psi}{{t'}}
=H\psi
=&\qty{\frac{1}{2m}\qty[p^2 - qpA-qAp+q^2A^2] + q\varphi}\psi\\
=&\qty{\frac{1}{2m}\qty[(-\ii\hbar)^2\pdv[2]{{x''}} 
- q(-\ii\hbar) \pdv{x''}[A\,\cdot]
-qA(-\ii\hbar)\pdv{x''}+q^2A^2] + q\varphi}\psi\\
=&\qty{\frac{1}{2m}\qty[(\ii\hbar)^2\pdv[2]{{x''}} 
+ \ii\hbar q\qty(\pdv{A}{x''}+A\pdv{x''})
+\ii\hbar qA\pdv{x''}+q^2A^2] + q\varphi}\psi.
\end{aligned}
\end{equation}
And this will be achived only by setting $\alpha=1/2$.
\qed

\section{Hollow cylindrical shell}
\newcommand{\E}{\ensuremath{\mathcal{E}}}

\begin{figure}\centering
\resizebox{.3\textwidth}{!}{\input{figures/cylinder.pdf_t}}
\caption{The dimensions of the cylindrical shell wherein the electron
  is confined together with the cylindrical coordinates. The
  wavefunction vanishes on the perifery of the shell.}
\label{fig:cylinder}
\end{figure}

An electron is confined in a hollow cylindrical shell as in
\figref{fig:cylinder}. The boundry conditions on the wave function is
\begin{equation}
\begin{cases}
\psi(\rho_a, \phi, z)=\psi(\rho_b, \phi, z)=0\\
\psi(\rho, \phi, 0)=\psi(\rho, \phi, L)=0\\
\psi(\rho_a, \phi+k2\pi, z)=\psi(\rho, \phi, z)\qcomma k\in\Z.
\end{cases}
\end{equation}

\subsection{Free particle (no magnetic field)}
For a free electron, under influence of no potential, we have the
Hamiltonian 
\begin{equation}
H=\frac{1}{2m_e}\vb{p}\vdot\vb{p} \,\dot{=}-\frac{\hbar^2}{2m_e}\laplacian
=-\frac{\hbar^2}{2m_e}\qty[
\frac{1}{\rho}\pdv{\rho}\qty(\rho\pdv{\rho})
+\frac{1}{\rho^2}\pdv[2]{\phi} + \pdv[2]{z}],
\end{equation}
where we have written the Laplacian in cylindrical coordinates.
The time independent Schrödinger equation now becomes
\begin{equation}\label{eq:5a_SE}
\qty[
\frac{1}{\rho}\pdv{\rho}\qty(\rho\pdv{\rho})
+\frac{1}{\rho^2}\pdv[2]{\phi} + \pdv[2]{z}
+\E]\psi(\rho, \phi, z)=0,
\end{equation}
where $\E=2m_eE/\hbar^2$.

To tackle this problem we implement the technique of variable
separtion. Write
\begin{equation}
\psi(\rho, \phi, z)=R(\rho)\Phi(\phi)Z(z).
\end{equation}
Now we get
\begin{equation}
\frac{1}{\rho}\pdv{\rho}\qty(\rho\pdv{R}{\rho})\Phi Z
+\frac{1}{\rho^2}\pdv[2]{\Phi}{\phi}RZ + 
\pdv[2]{Z}{z}R\Phi
+\E R\Phi Z=0,
\end{equation}
which leads us to
\begin{equation}
-\frac{1}{Z}\pdv[2]{Z}{z}
=\frac{1}{\rho R}\pdv{\rho}\qty(\rho\pdv{R}{\rho})
+\frac{1}{\rho^2\Phi}\pdv[2]{\Phi}{\phi}
+\E = \zeta^2
\end{equation}
Now the LHS only depends on $z$ while the RHS only depends on $\rho$
and $\phi$, meaning that they must both be constant; call that
constant $\zeta^2$. 

The differential equation for $Z$ now becomes
\begin{equation}
\pdv[2]{Z}{z}=-\zeta^2Z\qcomma
Z(0)=Z(L)=0.
\end{equation}
With the boundry conditions we get
\begin{equation}
Z(z)=A\sin(\zeta_l z)\qcomma 
\zeta_l=\frac{l\pi}{L}\qcomma l\in\Z.
\end{equation}

Next we can get the differential equation for $\Phi$ by writing
\begin{equation}
-\frac{1}{\Phi}\pdv[2]{\Phi}{\phi} 
= \rho^2\qty[\frac{1}{\rho R}\pdv{\rho}
\qty(\rho\pdv{R}{\rho})+\E-\zeta_l^2]
=m^2.
\end{equation}
Same as before, LHS only depends on $\phi$ and RHS only depends on
$\rho$, meaning that both sides are constant; call that constant
$m^2$. This gives
\begin{equation}
\pdv[2]{\Phi}{\phi} =-m^2\Phi\qcomma
\Phi(\phi+k2\pi)=\Phi(\phi),\;k\in\Z.
\end{equation}
With the boundry conditions we get
\begin{equation}
\Phi(\phi)=B\cos(m\phi+\phi_0)\qcomma
m\in\Z.
\end{equation}
We can furthermore argue that $\phi_0$ only depends on the choice of
where to put the $x$~axis; thus we are free to set $\phi_0=0$ without
changing any physics.

And finaly we get to the equation for $R$. By now what we have is
\begin{equation}
m^2=\rho^2\qty[\frac{1}{\rho R}\pdv{\rho}
\qty(\rho\pdv{R}{\rho})+\E-\zeta_l^2]
=\frac{1}{R}\qty[\rho^2\pdv[2]{R}{\rho}
+\rho\pdv{R}{\rho}+\rho^2\qty(\E-\zeta_l^2)R],
\end{equation}
which in turn gives us
\begin{equation}
\rho^2\pdv[2]{R}{\rho}
+\rho\pdv{R}{\rho}+\qty[\rho^2\qty(\E-\zeta_l^2)-m^2]R=0.
\end{equation}
This is the equation for the Bessel functions of the first kind, which
has the solution
\begin{equation}
R(\rho)=CJ_m(\mathcal{K}\rho)+DN_m(\mathcal{K}\rho)
\end{equation}
where
\begin{equation}\label{eq:5a_kappa}
\mathcal{K}=%\mathcal{K}_l=
\sqrt{\E-\zeta_l^2}=\sqrt{\frac{2m_eE}{\hbar^2}-\qty(\frac{l\pi}{L})^2}.
\end{equation}
Now it's time to invoke the boundry conditions:
\begin{equation}\label{eq:5a_R_boundry}
\begin{cases}
CJ_m(\mathcal{K}\rho_a)+DN_m(\mathcal{K}\rho_a)=0\\
CJ_m(\mathcal{K}\rho_b)+DN_m(\mathcal{K}\rho_b)=0
\end{cases}
\quad\Longleftrightarrow\quad
\begin{cases}
CJ_m(\mathcal{K}\rho_a)=-DN_m(\mathcal{K}\rho_a)\\
CJ_m(\mathcal{K}\rho_b)=-DN_m(\mathcal{K}\rho_b)
\end{cases}
\end{equation}
Now divide both sides with each other and we get
\begin{equation}
\frac{J_m(\mathcal{K}\rho_a)}{J_m(\mathcal{K}\rho_b)}=
\frac{N_m(\mathcal{K}\rho_a)}{N_m(\mathcal{K}\rho_b)}.
\end{equation}
This is an equation for $\mathcal{K}$
\begin{equation}\label{eq:5a_K}
J_m(\mathcal{K}\rho_a)N_m(\mathcal{K}\rho_b)-J_m(\mathcal{K}\rho_b)N_m(\mathcal{K}\rho_a)=0
\end{equation}
which has infinatly many solutions. Call those solutions
$\mathcal{K}_{m,n}$. We also need to find constants. This can be done
by adding the two equaitons in \eqref{eq:5a_R_boundry}:
\begin{equation}
C\qty[J_m(\mathcal{K}_{m,n}\rho_a)+J_m(\mathcal{K}_{m,n}\rho_b)]
=-D\qty[N_m(\mathcal{K}_{m,n}\rho_a)+N_m(\mathcal{K}_{m,n}\rho_b)],
\end{equation}
which results in an equation for $C/D$. But the only physics in the
two constants are in this ratio; so we can set $D=1$. And thus
\begin{equation}\label{eq:5a_C}
C=
\frac{J_m(\mathcal{K}_{m,n}\rho_a)+J_m(\mathcal{K}_{m,n}\rho_b)}
{N_m(\mathcal{K}_{m,n}\rho_a)+N_m(\mathcal{K}_{m,n}\rho_b)}
=:\mathcal{C}_{m,n}.
\end{equation}


By now we have an expression for the energy eigenstates:
\begin{equation}
\psi_{l,m,n}(\rho, \phi, z)=
\mathcal{N}_{l,m,n}\sin(\frac{l\pi}{L}z)\cos(m\phi)
\Big[\mathcal{C}_{m,n}J_m\qty(\mathcal{K}_{m,n}\rho)+
N_m\qty(\mathcal{K}_{m,n}\rho) \Big],
\end{equation}
where $l, m\in\Z$, $\mathcal{N}_{l,m,n}$ is a normaization factor, 
$\mathcal{C}_{m,n}$ is given by \eqref{eq:5a_C}, and
$\mathcal{K}_{m,n}$ is the $n$'th root of \eqref{eq:5a_K}. 
The energy of this state is given by \eqref{eq:5a_kappa} to be
\begin{equation}
E_{l,m,n}=\frac{\hbar^2}{2m_e}\qty[\mathcal{K}_{m,n}^2 + \qty(\frac{l\pi}{L})^2].
\end{equation}




\subsection{Electron under influence of a magnetic potential}
We will now allow for a uniform magnetic field inside the inner radius of
the shell: $\vb{B}=B\hat{z}$ for $0\le\rho<\rho_a$ and $\vb{B}=\vb{0}$
for $\rho\ge\rho_a$. The Hamiltonian is 
\begin{equation}
H=\frac{1}{2m_e}\qty[\vb{p}\vdot\vb{p}
-q\vb{p}\vdot\vb{A}-q\vb{A}\vdot\vb{p}+q^2\vb{A}\vdot\vb{A}] 
+ q\varphi.
\end{equation}
Since we're only concerned about a static magnetic field, we can
disregrad $\varphi$.

It's a good idea to calculate $\vb{A}$ before we continue. By the
fact that $\curl{\vb{A}}=\vb{B}$ and
$\div\vb{A}=0$ (in the Coulomb gauge), we conclude that
\begin{equation}
2\pi\rho' A_\phi(\rho')=\oint\limits_{\rho=\rho'}\!\vb{A}\vdot\rd{\vb{r}}
=\iint\limits_{\rho\le\rho'}(\curl\vb{A})\vdot\rd\vb{S}
=\iint\limits_{\rho\le\rho'}\vb{B}\vdot\rd\vb{S}
=\varPhi_B\stackrel{\rho'\ge\rho_a}{=}\pi\rho_a^2B.
\end{equation}
Furthermore, by cylindrical symmetry $\vb{A}$ must be
\begin{equation}
\vb{A}=A_\phi(\rho)\hat\phi=\frac{\varPhi_B}{2\pi\rho}
\stackrel{\rho\ge\rho_a}{=}\frac{\rho_a^2B}{2\rho}.
\end{equation}
With this we can write
\begin{equation}
\vb{p}\vdot\vb{A}=p_\phi A_\phi 
= -\ii\hbar\frac{1}{\rho}\pdv{\phi}\Big[A_\phi\cdot\Big]
=-\ii\hbar\frac{\Phi_B}{2\pi\rho^2}\pdv{\phi}
\end{equation}
and
\begin{equation}
\vb{A}\vdot\vb{p}= A_\phi p_\phi
=-\ii\hbar\frac{\Phi_B}{2\pi\rho^2}\pdv{\phi}.
\end{equation}

Finaly we can write the Hamiltonian as
\begin{equation}
\begin{aligned}
H=&\frac{1}{2m_e}\qty{-\hbar^2\qty[
\frac{1}{\rho}\pdv{\rho}\qty(\rho\pdv{\rho})
+\frac{1}{\rho^2}\pdv[2]{\phi} + \pdv[2]{z}]
+\ii\hbar q\frac{\Phi_B}{\pi\rho^2}\pdv{\phi}
+\frac{q^2\Phi_B^2}{4\pi^2\rho^2} }.
\end{aligned}
\end{equation}
And the Schrödinger equation becomes
\begin{equation}
\qty{\hbar^2 \qty[
\frac{1}{\rho}\pdv{\rho}\qty(\rho\pdv{\rho})
+\frac{1}{\rho^2}\pdv[2]{\phi} + \pdv[2]{z}]
-\ii\hbar q\frac{\Phi_B}{\pi\rho^2}\pdv{\phi}
-\frac{q^2\Phi_B^2}{4\pi^2\rho^2} +2m_eE}\psi=0.
\end{equation}

\subsection{Flux quantaization}




\section{Spinless particle in a uniform magnetic field}





\section{Particle in a uniform magnetic field}

\newpage
\appendix


\section{Excerpt from my last hand-in}\label{sec:comm}
\textit{All of the following appendix is copied from my last hand-in.}

Before we set out to prove the two commutator relations, it will be
helpfull to derive an intermediary result first. We're going to prove
the formula
\begin{equation} \label{eq:comm_power}
\comm{A}{B^n} = \sum_{i=0}^{n-1} B^{i}\comm{A}{B}B^{n-1-i}.
\end{equation}
This formula is readily derived with a proof by induction. 

To start off with, we can clearly see that \eqref{eq:comm_power} holds
in the base case of $n=1$. Then assume that it also holds for all
intergers up to some value $m$. Does it then also hold for $m+1$?
To show that it does, use the elementary commutator
indentity
\begin{equation} \label{eq:comm_A_BC}
\comm{A}{BC} = \comm{A}{B}C+B\comm{A}{C},
\end{equation}
which gives
\begin{equation}
\begin{aligned}
\comm{A}{B^{m+1}} &= \comm{A}{B^{m}B} \\
&\hspace{-3pt}\stackrel{\eqref{eq:comm_A_BC}}{=} \comm{A}{B^{m}}B + B^m\comm{A}{B}\\
&\hspace{-14pt}\stackrel{\stackrel{\text{\tiny induction}}{\text{\tiny assumption}}}{=}
\left(\sum_{i=0}^{m} B^{i}\comm{A}{B}B^{m-1-i}\right) B + B^m\comm{A}{B}\\
&=\sum_{i=0}^{m-1} B^{i}\comm{A}{B}B^{m-i} + B^m\comm{A}{B}&=\sum_{i=0}^{m} B^{i}\comm{A}{B}B^{m-i}.
\end{aligned}
\end{equation}
The last expression is just \eqref{eq:comm_power} with $n=m+1$. This
concludes the proof by induction. \qed



\end{document}





%% På svenska ska citattecknet vara samma i både början och slut.
%% Använd två apostrofer (två enkelfjongar): ''.


%% Inkludera PDF-dokument
\includepdf[pages={1-}]{filnamn.pdf} %Filnamnet får INTE innehålla 'mellanslag'!

%% Figurer inkluderade som pdf-filer
\begin{figure}\centering
\centerline{ %centrerar även större bilder
\includegraphics[width=1\textwidth]{filnamn.pdf}
}
\caption{}
\label{fig:}
\end{figure}

%% Figurer inkluderade med xfigs "Combined PDF/LaTeX"
\begin{figure}\centering
\resizebox{.8\textwidth}{!}{\input{filnamn.pdf_t}}
\caption{}
\label{fig:}
\end{figure}

%% Figurer roterade 90 grader
\begin{sidewaysfigure}\centering
\centerline{ %centrerar även större bilder
\includegraphics[width=1\textwidth]{filnamn.pdf}
}
\caption{}
\label{fig:}
\end{sidewaysfigure}


%%Om man vill lägga till något i TOC
\stepcounter{section} %Till exempel en 'section'
\addcontentsline{toc}{section}{\Alph{section}\hspace{8 pt}Labblogg} 

