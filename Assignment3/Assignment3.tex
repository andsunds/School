\documentclass[11pt,letter, swedish, english
]{article}
\pdfoutput=1

\usepackage{../custom_as}



%%Drar in tabell och figurtexter
\usepackage[margin=10 pt]{caption}
%%För att lägga in 'att göra'-noteringar i texten
\usepackage{todonotes} %\todo{...}

%%För att själv bestämma marginalerna. 
\usepackage[
%            top    = 3cm,
%            bottom = 3cm,
%            left   = 3cm, right  = 3cm
]{geometry}

%%För att ändra hur rubrikerna ska formateras
\renewcommand{\thesubsection}{\arabic{section} (\alph{subsection})}

\renewcommand{\thesubsubsection}{\arabic{section} (\alph{subsection},\,\roman{subsubsection})}

% \newcommand{\cbox}[2][cyan]
% {\mathchoice
% 	{\setlength{\fboxsep}{0pt}\colorbox{#1}{$\displaystyle#2$}}
% 	{\setlength{\fboxsep}{0pt}\colorbox{#1}{$\textstyle#2$}}
% 	{\setlength{\fboxsep}{0pt}\colorbox{#1}{$\scriptstyle#2$}}
% 	{\setlength{\fboxsep}{0pt}\colorbox{#1}{$\scriptscriptstyle#2$}}
% }
% \newcommand{\grande}{\cbox{\phantom{\frac{1}{xx}}}}


\begin{document}

%%%%%%%%%%%%%%%%% vvv Inbyggd titelsida vvv %%%%%%%%%%%%%%%%%
% \begin{titlepage}
\title{Quantum Mechanics -- PHYS\,701 \\
Assignment 3}
\author{Andréas Sundström}
\date{\today}

\maketitle

%%%%%%%%%%%%%%%%% ^^^ Inbyggd titelsida ^^^ %%%%%%%%%%%%%%%%%

%Om man vill ha en lista med vilka todo:s som finns.
%\todolist

\section{A special case of the Baker-Campbell-Hausdorf theorem}
Here we want to show that
\begin{equation}\label{eq:1_want_this}
\ee^{A+B} = \ee^A\ee^B\ee^{-\frac{1}{2}\comm{A}{B}}
\end{equation}
for two operators $A$ and $B$ wich both commute with $\comm{A}{B}$,
but not necessarily with each other.

We begin by studying
\begin{equation}
\comm{B}{\ee^{x A}}=\sum_{n=0}^\infty \frac{x^n}{n!}\comm{B}{A^n},
\end{equation}
where $x$ is just a number (\emph{not} the operator). To expand this
sum we can use the formula derived in the last problem
set\footnotemark{}:
\begin{equation}
\comm{B}{A^{m}} =\sum_{i=0}^{m-1} A^{i}\comm{B}{A}A^{m-1-i}.
\end{equation}
In this case we aslo know that $A$ commutes with $\comm{B}{A}$, which
means that we can write
\begin{equation}\label{eq:1_comm}
\comm{B}{\ee^{x A}}=\sum_{n=0}^\infty \frac{x^n}{n!}
\sum_{k=0}^{n-1}A^{n-1}\comm{B}{A}
=\sum_{n=0}^\infty \frac{x^n}{n!}
nA^{n-1}\comm{B}{A}
=x\ee^{xA}\comm{B}{A}.
\end{equation}
Here the case $n=0$ should be regraded as the whole term beeing equal
to $0$, since $k$ starts above the summation limit.
\footnotetext{I have attached a copy of the relavent derivation in Appendix~\ref{sec:comm}}

Nex up we study the function
\begin{equation}\label{eq:1_G_def}
G(x)=\ee^{xA}\ee^{xB}
=\sum_{i=0}^\infty\frac{x^iA^i}{i!}\sum_{j=0}^\infty\frac{x^jB^j}{j!}.
\end{equation}
It's derivative is
\begin{equation}
\begin{aligned}
\dv{G}{x}=&
\sum_{i=1}^\infty\frac{x^{i-1}A^i}{(i-1)!}\sum_{j=0}^\infty\frac{x^jB^j}{j!}
+\sum_{i=0}^\infty\frac{x^iA^i}{i!}\sum_{j=1}^\infty\frac{x^{j-1}B^j}{(j-1)!}\\
=&A\ee^{xA}\ee^{xB} + \ee^{xA}B\ee^{xB}.
\end{aligned}
\end{equation}
Now we use the relation \eqref{eq:1_comm} to rewrite
\begin{equation}
\ee^{xA}B = B\ee^{xA}-\comm{B}{\ee^{xA}}
=B\ee^{xA}-x\ee^{xA}\comm{B}{A}.
\end{equation}
All in all we get
\vspace{-4mm}
\begin{equation}
\dv{G}{x}=A\ee^{xA}\ee^{xB}+\qty(B\ee^{xA}-x\ee^{xA}\comm{B}{A})\ee^{xB}
=\Big(A+B+x\comm{A}{B}\Big)\overbrace{\ee^{xA}\ee^{xB}}^{=G(x)},
\end{equation}
which is a differential equation for $G$ and it results in
\begin{equation}
G(x)=\exp[x(A+B)+\frac{1}{2}x^2\comm{A}{B}].
\end{equation}
But with \eqref{eq:1_G_def} we have
\begin{equation}\label{eq:1_almost}
\ee^{xA}\ee^{xB}=\exp[x(A+B)+\frac{1}{2}x^2\comm{A}{B}]
=\ee^{x(A+B)}\ee^{\frac{1}{2}x^2\comm{A}{B}}.
\end{equation}
The last step is possible since $\comm{A}{B}$ commutes with both $A$
and $B$. By choosing $x=1$ we're almost where we want to be. 

All that's left is to move $\ee^{\frac{1}{2}x^2\comm{A}{B}}$ to the
other side. To do this we note that for any operator $\Xi$
\begin{equation}
\ee^{\Xi}\ee^{-\Xi}=\sum_{i\ge0}\sum_{j\ge0}(-1)^j\frac{\Xi^{i+j}}{i!j!}
=\sum_i\sum_{j=2n}\frac{\Xi^{i+2n}}{i!(2n)!}
-\sum_i\sum_{j=2n+1}\frac{\Xi^{i+2n+1}}{i!(2n+1)!}.
\end{equation}
Now since $i$ is still free to run over all values in both sums, we
will get two sums where they almost cancel each other out. The
exception is thet the second double sum can't produce any $\Xi^0$
term. Therefore
\begin{equation}
\ee^{\Xi}\ee^{-\Xi}=1,
\end{equation}
in the sence of $\Xi^0$ beeing the identity operator. 

With this result its now safe to ``multiply'' both sides of
\eqref{eq:1_almost} by $\ee^{-\frac{1}{2}x^2\comm{A}{B}}$, and we get
\begin{equation}
\ee^{A}\ee^{B}\ee^{-\frac{1}{2}\comm{A}{B}} = \ee^{A+B},
\end{equation}
with $x=1$.
\qed


\section{Coherent states}



\section{Harmonic oscillator and the Feynman path integral}




\section{The Feynman pathintegral and electromagnetic potentials}




\section{Hollow cylindrical shell}





\section{Spinless particle in a uniform magnetic field}





\section{Particle in a uniform magnetic field}


\appendix


\section{Excerpt from my last hand-in}\label{sec:comm}
\textit{All of the following appendix is copied from my last hand-in.}

Before we set out to prove the two commutator relations, it will be
helpfull to derive an intermediary result first. We're going to prove
the formula
\begin{equation} \label{eq:comm_power}
\comm{A}{B^n} = \sum_{i=0}^{n-1} B^{i}\comm{A}{B}B^{n-1-i}.
\end{equation}
This formula is readily derived with a proof by induction. 

To start off with, we can clearly see that \eqref{eq:comm_power} holds
in the base case of $n=1$. Then assume that it also holds for all
intergers up to some value $m$. Does it then also hold for $m+1$?
To show that it does, use the elementary commutator
indentity
\begin{equation} \label{eq:comm_A_BC}
\comm{A}{BC} = \comm{A}{B}C+B\comm{A}{C},
\end{equation}
which gives
\begin{equation}
\begin{aligned}
\comm{A}{B^{m+1}} &= \comm{A}{B^{m}B} \\
&\hspace{-3pt}\stackrel{\eqref{eq:comm_A_BC}}{=} \comm{A}{B^{m}}B + B^m\comm{A}{B}\\
&\hspace{-14pt}\stackrel{\stackrel{\text{\tiny induction}}{\text{\tiny assumption}}}{=}
\left(\sum_{i=0}^{m} B^{i}\comm{A}{B}B^{m-1-i}\right) B + B^m\comm{A}{B}\\
&=\sum_{i=0}^{m-1} B^{i}\comm{A}{B}B^{m-i} + B^m\comm{A}{B}&=\sum_{i=0}^{m} B^{i}\comm{A}{B}B^{m-i}.
\end{aligned}
\end{equation}
The last expression is just \eqref{eq:comm_power} with $n=m+1$. This
concludes the proof by induction. \qed



\end{document}





%% På svenska ska citattecknet vara samma i både början och slut.
%% Använd två apostrofer (två enkelfjongar): ''.


%% Inkludera PDF-dokument
\includepdf[pages={1-}]{filnamn.pdf} %Filnamnet får INTE innehålla 'mellanslag'!

%% Figurer inkluderade som pdf-filer
\begin{figure}\centering
\centerline{ %centrerar även större bilder
\includegraphics[width=1\textwidth]{filnamn.pdf}
}
\caption{}
\label{fig:}
\end{figure}

%% Figurer inkluderade med xfigs "Combined PDF/LaTeX"
\begin{figure}\centering
\resizebox{.8\textwidth}{!}{\input{filnamn.pdf_t}}
\caption{}
\label{fig:}
\end{figure}

%% Figurer roterade 90 grader
\begin{sidewaysfigure}\centering
\centerline{ %centrerar även större bilder
\includegraphics[width=1\textwidth]{filnamn.pdf}
}
\caption{}
\label{fig:}
\end{sidewaysfigure}


%%Om man vill lägga till något i TOC
\stepcounter{section} %Till exempel en 'section'
\addcontentsline{toc}{section}{\Alph{section}\hspace{8 pt}Labblogg} 

