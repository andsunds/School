\documentclass[11pt,letter, swedish, english
]{article}
\pdfoutput=1

\usepackage{../custom_as}
\usepackage[makeroom
]{cancel}
\usepackage{esint}
\let\oldint\int 
\renewcommand{\int}{\oldint\limits}
\graphicspath{{figures/}}

\swapcommands{\Phi}{\varPhi}
\swapcommands{\Omega}{\varOmega}
\swapcommands{\Sigma}{\varSigma}
\swapcommands{\Lambda}{\varLambda}

\swapcommands{\epsilon}{\varepsilon}


\newcommand{\enaught}{\ensuremath\varepsilon_0}

%%Drar in tabell och figurtexter
\usepackage[margin=10 pt]{caption}
%%För att lägga in 'att göra'-noteringar i texten
\usepackage{todonotes} %\todo{...}

%%För att själv bestämma marginalerna. 
\usepackage[
%            top    = 2.5cm,
%            bottom = 3cm,
%            left   = 3cm, right  = 3cm
]{geometry}

%%För att ändra hur rubrikerna ska formateras


%\renewcommand{\thefootnote}{\fnsymbol{footnote}}

%\newcommand{\Tc}{\ensuremath{T_{\text{c}}}}
\newcommand{\sign}{\ensuremath{\,\text{sign}}}

%\usepackage{tikz}


\renewcommand{\thesubsection}{\arabic{section} (\alph{subsection})}
\renewcommand{\thesubsubsection}{\arabic{section} (\alph{subsection},\,\roman{subsubsection})}


\begin{document}

%\tikzstyle{every picture}+=[remember picture]
%\tikzstyle{na} = [shape=rectangle,inner sep=0pt,text depth=0pt]



%%%%%%%%%%%%%%%%% vvv Inbyggd titelsida vvv %%%%%%%%%%%%%%%%%

\title{E\&M -- PHYS\,706 \\
Assignment 1}
\author{Andréas Sundström}
\date{\today}

\maketitle

%%%%%%%%%%%%%%%%% ^^^ Inbyggd titelsida ^^^ %%%%%%%%%%%%%%%%%

\section*{Boundary conditions}
Before we begin, I will just list the boundary conditions for the
electromagnetic field at a boundary, see Jackson (I.17)--(I.20). The
boundary conditions are
\begin{align}
&(\vb*E_1 - \vb*E_2)\cross\vu{n}_{12} = 0     \label{eq:01}\\
&(\vb*D_2 - \vb*D_1)\vdot\vu{n}_{12} = \sigma \label{eq:02}\\
&(\vb*H_1 - \vb*H_2)\cross\vu{n}_{12} = \vb*j \label{eq:03}\\
&(\vb*B_1 - \vb*B_2)\vdot\vu{n}_{12} = 0,     \label{eq:04}
\end{align}
where $\vu{n}_{12}$ is a normal to the interface pointing from region 1
to 2, $\sigma$ is the surface charge density at the interface, and
$\vu*j$ is the surface current density at the interface. These
boundary conditions will be useful in the following problems. 

\section{Three indices of refraction -- normal incidence}
\newcommand{\Ep}{{\mathcal{E}_1^+}}
\newcommand{\Epp}{{\mathcal{E}_2^+}}
\newcommand{\Eppp}{{\mathcal{E}_3^+}}
\newcommand{\Em}{{\mathcal{E}_1^-}}
\newcommand{\Emm}{{\mathcal{E}_2^-}}

In this problem we have a normally incident wave, with wavenumber
$k_0=2\pi/\lambda_0$, at an interface to a second media of finite
width $d$; after the waves has passed through the second region, it
enters a third region that is infinite. The three regions have indices
of refraction $n_1=1$, $n_2$ and $n_3$. All of this together with some 
definition of notation are shown in \figref{fig:1_geometry}.

\begin{figure}
\centering
\resizebox{.6\textwidth}{!}{\input{figures/1_geometry.pdf_t}}
\caption{Three different regions all with their own index of
  refraction. This figure also shows the notation used for the
  different fields. }
\label{fig:1_geometry}
\end{figure}

This problem will require the use of the boundary conditions
\eqref{eq:02} and \eqref{eq:03} with $\sigma=0$ and $\vb*j=0$. To use
these boundary conditions we must first establish expressions for the
fields:
\begin{equation}
\begin{aligned}
\vb*E_1^\pm=&\vu{x}\mathcal{E}_1^\pm\ee^{\pm\ii k_1z -\ii\omega t}\\
\vb*E_2^\pm=&\vu{x}\mathcal{E}_2^\pm\ee^{\pm\ii k_2z -\ii\omega t}\\
\vb*E_3^{+}=&\vu{x}\mathcal{E}_3^+\ee^{+\ii k_3z -\ii\omega t},
\end{aligned}
\end{equation}
where $k_i=n_ik_0=n_i2\pi/\lambda_0$. We see that since $\omega$
is the same for all, we can drop the factor $\ee^{-\i\omega t}$ in all
further calculations. 

We also note that all the fields are in the $\vu{x}$ direction, all
wavevectors are in the $\pm\vu{z}$ direction, and
$\vu{n}_{12}=\vu{z}$, so it will be sufficient to work with the
complex amplitudes and phases. With this observation we are ready to
write down the boundary conditions for each boundary. 

First the $z=0$ boundary, where \eqref{eq:02} gives
\begin{equation}
\Ep+\Em -(\Epp-\Emm) =0, 
\end{equation}
and \eqref{eq:03} gives
\begin{equation}
\frac{k_1}{\mu_1}\qty(\Ep-\Em) 
-\frac{k_2}{\mu_2}\qty(\Epp-\Emm) =0.
\end{equation}
In the last step we used the fact that 
$\vb*H = \vb*B/\mu = \vb*k\cross\vb*E/(\omega\mu)$. The different
signs between ``$+$'' and ``$-$'' amplitudes are due to the fact that
they travel in different directions. 

Similarly for the $z=d$ boundary, now we do however need to keep the
complex phases, due a to non-zero $z$ value, in mind. Nonetheless the
result is similar. From \eqref{eq:02} we get
\begin{equation}
\Epp\ee^{\ii k_2d}+\Emm\ee^{-\ii k_2d} -\Eppp\ee^{\ii k_3d} =0, 
\end{equation}
and from \eqref{eq:03} we get
\begin{equation}
\frac{k_2}{\mu_2}\qty(\Epp\ee^{\ii k_2d}-\Emm\ee^{-\ii k_2d}) 
-\frac{k_3}{\mu_3}\Eppp\ee^{\ii k_3d} =0.
\end{equation}

We now have four equations and four unknowns (given $\Ep$, the
unknowns are $\Em$, $\Epp$, $\Emm$, and $\Eppp$). Furthermore, the
equations are all linear, meaning that we can write them on matrix
form
\begin{equation}\label{eq:1_master_eqn}
\begin{bmatrix}
+1 & -1 & -1 & 0 \\
-K_1 & -K_2 & +K_2 &0\\
0& \ee^{\ii k_2d}& \ee^{-\ii k_2d}& -\ee^{\ii k_3d}\\
0& K_2\ee^{\ii k_2d}& -K_2\ee^{-\ii k_2d}& K_3\ee^{\ii k_3d}\\
\end{bmatrix}
\begin{bmatrix}
\Em \\ \Epp \\ \Emm \\ \Eppp
\end{bmatrix}
= \Ep
\begin{bmatrix}
-1 \\ -K_1 \\ 0 \\ 0
\end{bmatrix},
\end{equation}
where $K_i:=k_i/\mu_i$. This linear system of equations is not trivial
to solve, but the methods needed are (first year linear algebra), so I
will let \textit{Mathematica} do the grunt work and solve this system
for me. The raw output from Mathematica is
\begin{equation}\label{eq:1_Em}
\frac{\Em}{\Ep}=
\frac{k_2\mu_2(k_1\mu_3 + k_3\mu_1)(1+\ee^{2\ii d k_2}) 
+(k_1k_3\mu_2^2 + k_2^2\mu_1\mu_3)(-1+\ee^{2\ii d k_2})
}{
-k_2\mu_2(k_3\mu_1 - k_1\mu_3)(1+\ee^{2\ii dk_2})
+(k_1k_3\mu_2^2 - k_2^2\mu_1\mu_3)(-1+\ee^{2\ii dk_2})
}
\end{equation}
and 
\begin{equation}\label{eq:1_Eppp}
\frac{\Eppp}{\Ep} = 
\frac{4k_1k_2\mu_2\mu_3 \ee^{\ii d (k_2-k_3)}
}{
-k_2\mu_2(k_3\mu_1 - k_1\mu_3)(1+\ee^{2\ii d k2}) 
+(k_1k_3\mu_2^2 + k2^2\mu_1\mu_3) (-1+\ee^{2\ii d k2})
}.
\end{equation}
The other two outputs, $\Epp$ and $\Emm$, will not be relevant to the
rest of the problem, so they are omitted here. 




\subsection{Reflectivity}
In this part of the problem we want to calculate the reflectivity
$R=|\Em/Ep|^2$. By studyin \eqref{eq:1_Em}, we see that we can
multiply both the numerator and denominator by $\ee^{-\ii dk_2}/2$ to
get
\begin{equation}
\frac{\Em}{\Ep}=
\frac{k_2\mu_2(k_1\mu_3 + k_3\mu_1)\cos(dk_2) 
+\ii(k_1k_3\mu_2^2 + k_2^2\mu_1\mu_3)\sin(dk_2) 
}{
-k_2\mu_2(k_3\mu_1 - k_1\mu_3)\cos(dk_2) 
+\ii(k_1k_3\mu_2^2 - k_2^2\mu_1\mu_3)\cos(dk_2) 
}.
\end{equation}
To simplify matters a bit I will assume that $\mu_1=\mu_2=\mu_3$, so
that they all cancel. We can also replace $k_i=k_0n_i$. In the end we
get
\begin{equation}
\frac{\Em}{\Ep}=
\frac{n_2(n_1 + n_3)\cos(dk_0n_2) 
+\ii(n_1n_3 + n_2^2)\sin(dk_0n_2) 
}{
n_2(n_1 - n_3)\cos(dk_0n_2) 
+\ii(n_1n_3 - n_2^2)\sin(dk_0n_2) 
}.
\end{equation}
And finally the reflectivity is the square of the absolute value of
this, meaning that
\begin{equation}
R = \abs{\frac{\Em}{\Ep}}^2=
\frac{n_2^2(n_1 + n_3)^2\cos^2(dk_0n_2) 
+(n_1n_3 + n_2^2)^2\sin^2(dk_0n_2) 
}{
n_2^2(n_1 - n_3)^2\cos^2(dk_0n_2) 
+(n_1n_3 - n_2^2)^2\sin^2(dk_0n_2) 
},
\end{equation}
where $k_0=2\pi/\lambda_0$.
From here it is possible to go nuts with trig identities and expanding
the squares to get (via Mathematica) 
\begin{equation}
R = 
\frac{4n_1n_2^2n_3 + n_1^2\qty(n_2^2 + n_3^2) 
+n_2^2\qty(n_2^2 + n_3^2) + \qty(n_1^2 - n_2^2)\qty(n_2^2 - n_3^2) 
\cos(2dk_0n_2)
}{
-4n_1n_2^2n_3 + n_1^2\qty(n_2^2 + n_3^2) 
+n_2^2\qty(n_2^2 + n_3^2) + \qty(n_1^2 - n_2^2)\qty(n_2^2 - n_3^2) 
\cos(2dk_0n_2)
},
\end{equation}
which only has one cosine on either side of the division.

\todo[inline]{FIGURE!!!}

\subsection{Transmitivity}
Here we start off from \eqref{eq:1_Eppp}, and do almost the same
procedure as above. Skipping the details we get
\begin{equation}
\frac{\Eppp}{\Ep} = 
\frac{
2n_1n_2 \ee^{\ii dk_0n_3}
}{
n_2(n_1 - n_3)\cos(dk_0n_2) - \ii (n_2^2 - n_1n_3)\sin(dk_0n2)}.
\end{equation}
This gives a transmitivity of
\begin{equation}
T = \abs{\frac{\Eppp}{\Ep}}^2 = 
\frac{
4n_1^2n_2^2
}{
n_2^2(n_1 - n_3)^2\cos^2(dk_0n_2) + (n_2^2 - n_1n_3)^2\sin^2(dk_0n2)},
\end{equation}
where $k_0=2\pi/\lambda_0$.
Doing the same trigonometric expansion as above, we get
\begin{equation}
T = 
\frac{
4n_1^2n_2^2
}{
-4n_1n_2^2n_3 + n_1^2\qty(n_2^2 + n_3^2) 
+n_2^2\qty(n_2^2 + n_3^2) + \qty(n_1^2 - n_2^2)\qty(n_2^2 - n_3^2) 
\cos(2dk_0n_2)
}.
\end{equation}

\todo[inline]{FIGURE!!!}


\subsection{Application}








\section{Weak conductor -- normal incidence}
\newcommand{\kt}{\tilde{k}}

This time a plane monochromatic wave is normally incident on a flat
surface of a weak conductor $\zeta=\sigma/(\epsilon_2\omega)\ll1$. And
once again we will use \eqref{eq:02} and \eqref{eq:03} to get
\begin{equation}\label{eq:2_BC1}
\Ep+\Em -\Epp =0, 
\end{equation}
and 
\begin{equation}\label{eq:2_BC2}
\frac{k_1}{\mu_1}\qty(\Ep-\Em) 
-\frac{\kt_2}{\mu_2}\Epp =0.
\end{equation}
In the last equation we do not need to worry about any surface
current, because media with finite $\sigma$ only has bulk
current. Here we also have $\kt_{2} = k_2 + \ii\kappa_2$, where
$k_2$ and $\kappa_2$ are given by problem 4 in \eqref{eq:4_ktilde}.

Since $\zeta=\sigma/(\epsilon_2\omega)\ll1$, we can get some
approximate expressions. Using \eqref{eq:4_ktilde}, we can get
approximations for $k_2$ and $\kappa_2$. We begin by rewriting
\begin{equation}
\omega\sqrt{\frac{\epsilon_2\mu_2}{2}} 
=\frac{1}{\sqrt{2}}\; \frac{\omega}{c} 
\sqrt{\frac{\epsilon_2\mu_2}{\epsilon_0\mu_0}}
=\frac{1}{\sqrt{2}}k_0n_2.
\end{equation}
Now we can continue on $k_2$:
\begin{equation}
\begin{aligned}
k_2 =& \frac{1}{\sqrt{2}}k_0n_2
\qty[\sqrt{1+\zeta^2}+1]^{1/2}\\
=& \frac{1}{\sqrt{2}}k_0n_2
\qty[1+\frac{1}{2}\zeta^2+\order{\zeta^4}+1]^{1/2}\\
=& k_0n_2  +\order{\zeta^2}.
%\qty[1+\frac{1}{8}\zeta^2] +\order{\zeta^4}.
\end{aligned}
\end{equation}
Similarly for $\kappa_2$:
\begin{equation}
\begin{aligned}
\kappa_2 =& \frac{1}{\sqrt{2}}k_0n_2
\qty[\sqrt{1+\zeta^2}-1]^{1/2}\\
=& \frac{1}{\sqrt{2}}k_0n_2
\qty[\cancel{1}+\frac{1}{2}\zeta^2-\frac{1}{8}\zeta^4+
\order{\zeta^6}-\cancel{1}]^{1/2}\\
=& \frac{1}{2}k_0n_2\zeta
\qty[1-\frac{1}{4}\zeta^2+\order{\zeta^4}]^{1/2}\\
=& \frac{1}{2}k_0n_2\zeta +\order{\zeta^3}.
\end{aligned}
\end{equation}

\subsection{The reflected wave}
We are now ready to start on an expression for the refelcted wave. 
From \eqref{eq:2_BC1} we can get $\Epp=\Ep+\Em$. Substitute this into
\eqref{eq:2_BC2}, and we get
\begin{equation}
\qty(\frac{k_1}{\mu_1} - \frac{\kt_2}{\mu_2})\Ep 
=\qty(\frac{k_1}{\mu_1} + \frac{\kt_2}{\mu_2})\Em.
\end{equation}
Continuing on, assuming that $\mu_1=\mu_2$, we get
\begin{equation}
\begin{aligned}
\frac{\Em}{\Ep} 
= \frac{k_1-\kt_2}{k_1+\kt_2}
=& \frac{k_0n_1 - k_0n_2 - \ii k_0n_2\zeta/2}
{k_0n_1 + k_0n_2 + \ii k_0n_2\zeta/2}
= \frac{(n_1-n_2) - \ii n_2\zeta/2}
{(n_1+n_2) + \ii n_2\zeta/2}\\
=&\frac{(n_1-n_2)}{(n_1+n_2)}
\qty[1-\frac{\ii n_2\zeta}{2(n_1-n_2)}]
\qty[1+\frac{\ii n_2\zeta}{2(n_1+n_2)}]^{-1}\\
=&\frac{(n_1-n_2)}{(n_1+n_2)}
\qty[1-\frac{\ii n_2\zeta}{2(n_1-n_2)}]
\qty[1-\frac{\ii n_2\zeta}{2(n_1+n_2)}] +\order{\zeta^2}\\
=&\frac{n_1-n_2}{n_1+n_2}
\qty[1-\frac{\ii n_2\zeta}{2(n_1-n_2)}
-\frac{\ii n_2\zeta}{2(n_1+n_2)}] +\order{\zeta^2}\\
=&\frac{(n_1-n_2)}{(n_1+n_2)}
-\ii\frac{n_2\zeta}{2(n_1+n_2)}
\qty[1+\frac{n_1-n_2}{n_1+n_2}]+\order{\zeta^2}\\
=&\frac{n_1-n_2}{n_1+n_2}
-\ii\frac{n_1n_2\zeta}{(n_1+n_2)^2}.
\end{aligned}
\end{equation}
The intensity, to order 1 in $\zeta$, of the reflected wave (i.e. the
reflectivity) is given by
\begin{equation}
\abs{\frac{\Em}{\Ep}}^2 
= \qty(\frac{n_1-n_2}{n_1+n_2})^2
+\qty(\frac{n_1n_2\zeta}{(n_1+n_2)^2})^2
=\qty(\frac{n_1-n_2}{n_1+n_2})^2 + \order{\zeta^2}.
\end{equation}
And the phase shift of the reflected wave is
\begin{equation}
\arg\qty{\frac{\Em}{\Ep}} 
= \arctan(\frac{-n_1n_2\zeta/(n_1+n_2)^2}{(n_1+n_2)/(n_1+n_2)})
= -\frac{n_1n_2\zeta}{n_1^2-n_2^2}.
\end{equation}




\subsection{The transmitted wave}
The transmitted amplitude is given by
\begin{equation}
\frac{\Epp}{\Ep} = \frac{\Ep + \Em}{\Ep} 
= 1 + \qty[\frac{n_1-n_2}{n_1+n_2}
-\ii\frac{n_1n_2\zeta}{(n_1+n_2)^2}]
= \frac{2n_1}{n_1+n_2}
-\ii\frac{n_1n_2\zeta}{(n_1+n_2)^2},
\end{equation}
meaning that the transmitted intensity is
\begin{equation}
\frac{I_2(z=0)}{I_1} = \qty(\frac{2n_1}{n_1+n_2})^2
+\qty(\frac{n_1n_2\zeta}{(n_1+n_2)^2})^2
= \frac{4n_1^2}{(n_1+n_2)^2} + \order{\zeta^2},
\end{equation}
where $I_1$ is the incident intensity. 

The $z=0$ is to denote that that's the intensity right at the
interface (assuming that the wave travels in the positive $\vu{z}$
direction). But since the medium is conductive, there will be
losses. These are manifested through the complex $\kt$:
\begin{equation}
\vb*E_2^+(z) = \Epp \ee^{\ii\kt_2 z - \ii\omega t}
=\Epp \ee^{\ii k_2z - \ii\omega t}\,\ee^{-\kappa_2z}.
\end{equation}
That is the intensity as a function of $z$ is
\begin{equation}
\begin{aligned}
I_2(z)=&\qty(\vb*E_2^+(z))^2)=I_2(z=0)\ee^{-2\kappa_2z}
\approx I_1 \frac{4n_1^2}{(n_1+n_2)^2}\,\exp[-2z\frac{1}{2}k_0n_0\zeta]\\
=&I_1 \frac{4n_1^2}{(n_1+n_2)^2}\,
\exp[-k_0n_2z\frac{\sigma}{\omega\epsilon_2}]
=I_1 \frac{4n_1^2}{(n_1+n_2)^2}\,
\exp[-z\sigma\sqrt{\frac{\mu_2}{\epsilon_2}}]
\end{aligned}
\end{equation}





\section{Angled incidence}


\subsection{Reflectivity}

\subsection{Grazing incidence to a medium with 
refractive indec near unity}
\newcommand{\thetac}{\theta_\text{c}}
\newcommand{\phic}{\phi_\text{c}}

Here, we have grazingng incidence, where 
$\phi_1=\pi/2 - \theta\ll1$, and the medium has an index of refraction
$n_2=1-\delta$, where $\delta\ll1$. We want to find the critical angle. 

We begin with Snell's law 
\begin{equation}
n_1\sin(\theta_1) = n_2\sin(\theta_2),
\end{equation}
which for the critical angle looks like
\begin{equation}
n_1\sin(\thetac) = n_2\sin(\pi/2) = n_2,
\end{equation}
or in other words
\begin{equation}\label{eq:3b_phic}
\cos(\phic) = \frac{n_2}{n_1}.
\end{equation}

In our case the incidence is from vacuum, meaning that the RHS just
becomes $n_2$. Furthermore, we know that $n_2$ is really cose to 1,
meaning that $\phic\ll1$. So we can expand the LSH of
\eqref{eq:3b_phic} to get
\begin{equation}
1-\frac{1}{2}\phic^2 \approx n_2 = 1-\delta.
\end{equation}
This gives
\begin{equation}
\phic \approx \sqrt{2\delta}.
\end{equation}


\subsection{Approximate expression for the refelctivity}


\subsection{Plot}



\section{Dispersion relation in conducting media}
Starting off from the Maxwell equation
\begin{equation}
\curl\vb*B = \mu\vb*J + \mu\epsilon\pdv{\vb*E}{t},
\end{equation}
we can introduce Ohm's law $\vb*J=\sigma\vb*E$ and take the curl of
both sides. We now have
\begin{equation}\label{eq:4_wave-eqn}
\curl[\curl\vb*B] = 
\curl\qty[\mu\sigma\vb*E + \mu\epsilon\pdv{\vb*E}{t}]
=-\mu\sigma\pdv{\vb*B}{t} - \mu\epsilon\pdv[2]{\vb*B}{t},
\end{equation}
where the last step was due to Faraday's law 
$\curl\vb*E = -\pdv{\vb*B}{t}$ and the fact that the partial
derivative commute $\curl\pdv{\vb*E}{t} = \pdv{\curl\vb*E}{t}$. 
Next we need to use the vector identity
\begin{equation}
\curl(\curl\vb*B) = \grad{}(\cancelto{0}{\div\vb*B}) - \laplacian\vb*B.
\end{equation}
The first term is canceled by the ``no monopoles Maxwell
equation''. All together we therefore have
\begin{equation}\label{eq:4_wave-eqn}
\laplacian\vb*B
-\mu\sigma\pdv{\vb*B}{t} - \mu\epsilon\pdv[2]{\vb*B}{t} =0
\end{equation}

We now have a (damped) wave equation for the $\vb*B$ field. Assuming
harmonic solutions
\begin{equation}
\vb*B(\vb*r, t) = B_0\ee^{\ii\vb*k\vdot\vb*r - \ii\omega t}\vu{n}
\end{equation}
we can get a dispersion relation by substituting this into
\eqref{eq:4_wave-eqn}. We then get
\begin{equation}
\qty[(\ii\vb*k)^2
-\mu\sigma(-\ii\omega) - \mu\epsilon(-\ii\omega)^2
]B_0\ee^{\ii\vb*k\vdot\vb*r - \ii\omega t}\vu{n} =0.
\end{equation}
Now, the things outside the brackets are not always 0, so instead what
is inside the brackets must be 0. I.e.
\begin{equation}
0=(\ii\vb*k)^2
-\mu\sigma(-\ii\omega) - \mu\epsilon(-\ii\omega)^2
\end{equation}
or when cleaned up
\begin{equation}
k^2 = \ii\mu\sigma\omega +\mu\epsilon\omega^2.
\end{equation}
This is our dispersion relation for an Ohmic condutive media. 




$\kt=k+\ii\kappa$
\begin{equation}\label{eq:4_ktilde}
k=\omega\sqrt{\frac{\epsilon\mu}{2}}
\qty[\sqrt{1+\qty(\frac{\sigma}{\epsilon\omega})^2}+1]^{1/2}
\qcomma
\kappa=\omega\sqrt{\frac{\epsilon\mu}{2}}
\qty[\sqrt{1+\qty(\frac{\sigma}{\epsilon\omega})^2}-1]^{1/2}
\end{equation}




\end{document}




%  LocalWords:  MFT MF Ising
