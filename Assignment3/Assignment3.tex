\documentclass[11pt,letter, swedish, english
]{article}
\pdfoutput=1

\usepackage{../custom_as}

\renewcommand{\thesubsection}{\arabic{section} (\alph{subsection})}

\renewcommand{\thesubsubsection}{\arabic{section} (\alph{subsection},\,\roman{subsubsection})}


%%Drar in tabell och figurtexter
\usepackage[margin=10 pt]{caption}
%%För att lägga in 'att göra'-noteringar i texten
\usepackage{todonotes} %\todo{...}

%%För att själv bestämma marginalerna. 
\usepackage[
%            top    = 3cm,
%            bottom = 3cm,
%            left   = 3cm, right  = 3cm
]{geometry}

\DeclareMathAlphabet{\mathpzc}{OT1}{pzc}{m}{it}
\newcommand{\oh}{\ensuremath\mathpzc{o}}

\newcommand{\as}{\qcomma\text{as }}

\renewcommand{\thefootnote}{\fnsymbol{footnote}}

\begin{document}

%%%%%%%%%%%%%%%%% vvv Inbyggd titelsida vvv %%%%%%%%%%%%%%%%%
% \begin{titlepage}
\title{Asymptotic Analasys and Pertubation Theory -- AMATH\,732 \\
Assignment 3}
\author{Andréas Sundström}
\date{\today}

\maketitle

%%%%%%%%%%%%%%%%% ^^^ Inbyggd titelsida ^^^ %%%%%%%%%%%%%%%%%

%Om man vill ha en lista med vilka todo:s som finns.
%\todolist

\section{An ODE}
We have the differential equation
\begin{equation}\label{eq:1_ODE}
x^4y''-y=0
\quad\Longleftrightarrow\quad
y''-\frac{1}{x^4}y = 0.
\end{equation}

\subsection{Classifysing the suspect points 
  and trying the method of Frobenius} 

\subsubsection{Classifying the points $x=0, \infty$}
To classify the point $x=0$, we first see that $\nicefrac{1}{x^4}$
isn't analytic there and so is $x^2\cdot\nicefrac{1}{x^4}$. Thus the
point $x=0$ is an \emph{irregular singular point}.

To classify the point $x=\infty$, we nee to change variables to
\begin{equation}\label{eq:t_x}
t=\frac{1}{x}\qcomma \tilde{y}(t)=y(x),
\end{equation}
and classify the new DE at $t=0$.
The second derivative (w.r.t. $x$) becomes
\begin{equation}
\dv[2]{y}{x}=\dv{t}{x}
\dv{t}\qty[\dv{t}{x}\dv{\tilde{y}}{t}]=t^4\tilde{y}''+2t^3\tilde{y}',
\end{equation}
where $\tilde{y}'=\nicefrac{\rd\tilde{y}}{\rd{t}}$. With this we can
write (\ref{eq:1_ODE}\,a) as
\begin{equation}\label{eq:1_new_ODE}
0=\frac{1}{t^4}\qty(t^4\tilde{y}''+2t^3\tilde{y}') - \tilde{y} '
= \tilde{y}'' +\frac{2}{t}\tilde{y}' - \tilde{y}.
\end{equation}
From here we see that $t=0$ is not an ordinary point since we have the
$\nicefrac{2}{t}$ coefficient, but it is however a \emph{regular singular
point} since $t\cdot\nicefrac{2}{t}=2$ isn't singular at $t=0$
\qed

\subsubsection{Trying the method of Frobenius}
The method of Frobenius is a method for finding series solutions around
regular singular points, which utilizes the fact that there is 
at least one solution of the form $(x-x_0)^{\alpha}A(x)$ in a
neighbourhood of the regular singular point $x_0$ and where $A(x)$ is
analytic near $x_0$. When searching for a solution with Frobenius'
method we assume that
\begin{equation}
y(x)=\sum_{k=0}^\infty a_k (x-x_0)^{\alpha+k}.
\end{equation}
The second derivative of $y$ then becomes
\begin{equation}
y(x)=\sum_{k=0}^\infty a_k (\alpha+k)(\alpha+k-1) (x-x_0)^{\alpha+k-2}.
\end{equation}

In our case we going to look at $x_0=0$. Thus the DE becomes
\begin{equation}
0=x^4y''-y=
\sum_{k=0}^\infty a_k (\alpha+k)(\alpha+k-1) x^{\alpha+k+2}
-\sum_{k=0}^\infty a_k x^{\alpha+k}.
\end{equation}
Collecting like powers of $x$ yields
\begin{equation}
\begin{cases}
a_0=0\\
a_1=0\\
a_0(\alpha)(\alpha-1)-a_2=0\\
\vdots\\
a_n(\alpha+n)(\alpha+n-1)-a_{n+2}=0
\end{cases}
\end{equation}
which reduces to $a_n=0$ for all $n$.
\todo{Why dosen't it work?}


\subsection{The method of Carlini-Liouville-Green}
In the method of Carlini-Liouville-Green we let
\begin{equation}
y=\ee^{S(x)},
\end{equation}
and then we find the leading asymptotic behaviour of $S(x)$ as
$x\to x_0$. 

To use this method we calculate the second derivative
\begin{equation}
y''=\dv[2]{x}\qty[\ee^{s(x)}]
=\qty[\Big(S'(x)\Big)^2 + S''(x)]\ee^{S(x)}.
\end{equation}
This makes it possible to rewrite (\ref{eq:1_ODE}\,b) as
\begin{equation}\label{eq:1_DE_S}
\qty[S'' + (S')^2 -\frac{1}{x^4}]\ee^{S(x)}=0
\quad\Longrightarrow\quad
S'' + (S')^2 -\frac{1}{x^4}=0. 
\end{equation}

\subsubsection{First term}
The next setp is to assume that $S''\ll(S')^2$ as $x\to0$, which gives
us the asymptotic behaviour as $x\to0$:
\begin{equation}\label{eq:1_S0'}
\Big(S'(x)\Big)^2\sim \frac{1}{x^4}
\quad\Longrightarrow\quad
S'(x)= \frac{-\xi}{x^2} + C'(x)\qcomma\text{where }
|C'(x)|\ll\frac{1}{x^2} \;\text{ as }\; x\to0,
\end{equation}
and $\xi=\pm1$\footnotemark{} is the parameter which keeps track of
the two linearly independent solutions. This gives the first term in
the asymptotic expansion of $S(x)$ as $x\to0$: 
\begin{equation}\label{eq:1_S0}
%S(x)\sim \frac{\xi}{x}
%\quad\Longrightarrow\quad
S(x)=\frac{\xi}{x} + C(x).
\end{equation}
We can also see that (assuming $C''(x)=\oh(x^{-3})$)
\begin{equation}
\abs{S''(x)}\sim\frac{2}{x^3}\ll\Big(S'(x)\Big)^2\sim\frac{1}{x^4}
\end{equation}
as $x\to0$.

\footnotetext{The reason for the minus sign in the expression is
  purely esthetical, as the sign will change upon integration.}

\subsubsection{Second term}
To find the secod term in the expansion just substitute
(\ref{eq:1_S0}) into (\ref{eq:1_DE_S}\,b):
\begin{equation}
\frac{2\xi}{x^3}+C''
+(C')^2-2\frac{\xi}{x^2}C' + \qty(\frac{-\xi}{x^2})^2
-\frac{1}{x^4}=0.
\end{equation}
To simplify this we note that $\xi^2\equiv1$, which removes the two
terms of $x^{-4}$. We're left with
\begin{equation}
C''+(C')^2-\frac{2\xi}{x^2}C' + \frac{2\xi}{x^3} = 0.
\end{equation}
From here we use the fact, from \eqref{eq:1_S0'}, that
$|C'(x)|\ll\nicefrac{1}{x^2}$ as $x\to0$ which implies that
\begin{equation}
\Big(C'(x)\Big)^2\ll\abs{\frac{2\xi}{x^2}C'}.
\end{equation}
We also assume that
\begin{equation}\label{eq:1_C''}
\abs{C''(x)}\ll\abs{\frac{1}{x^3}}\as x\to0.
\end{equation}
The dominant balance, as $x\to0$, for the equation then becomes
\begin{equation}
\frac{2\xi}{x^2}C'(x) \sim \frac{2\xi}{x^3}
\quad\Longrightarrow\quad
C'(x)\sim\frac{1}{x}.
\end{equation}
This gives 
\begin{equation}
C(x)=\ln|x|+D(x),
\end{equation}
where $|D'(x)|\ll|\nicefrac{1}{x}|$ as $x\to0$. We can also confirm the
asumption in \eqref{eq:1_C''} by seeing that
\begin{equation}
C''(x)\sim -\frac{1}{x^2}\as x\to0,
\end{equation}
under the (standard) assumption that $D''(x)=\oh(x^{-2})$.

\subsubsection{Third term}
Once again we substitute the new solution into the DE for $C$. With
some simplification we get
\begin{equation}
D''+(D')^2+\frac{2}{x}D' - \frac{2\xi}{x^2}D'=0.
\end{equation}
But now the mashinery grinds to a halt. We know that 
\begin{equation}
\Big(D'(x)\Big)^2\ll\abs{\frac{D'(x)}{x}}
\end{equation}
and by assumption $|D''(x)|$ is negigable compared to the rest of the
terms. Therefore we get the dominant balance
\begin{equation}
\frac{2\xi}{x^2}D'(x) \,"{\sim}"\, 0\as x\to0,
\end{equation}
which yields
\begin{equation}
D(x)=d,
\end{equation}
constant. And we also clearly see that the assumptions made here are
consistent.

\subsubsection*{Final result}
The final result for the asymptotic expansion of $S$ is then
\begin{equation}
S(x)\sim \frac{\xi}{x}+\ln|x|+d = \pm\frac{1}{x}+\ln|x|+d\as x\to0.
\end{equation}
These are the only non-vanishing terms of $S(x)$ as $x\to0$.\qed


\subsection{Finding the actual solutions}

\todo{Still have to do this!}





\section{The Airy equation}
\newcommand{\xtoi}{\ensuremath{x\to\infty}}
\newcommand{\ttoz}{\ensuremath{t\to0^+}}
\renewcommand{\thesubsubsection}{(\roman{subsubsection})}

The Airy equation
\begin{equation}\label{eq:Airy}
y''-xy=0
\end{equation}
is a second order linear differential equation, and thus has two
linearly independent solutions. We here want to find their leading
asymptotic behaviour as \xtoi. 

\subsection*{Rewrite and classify}
To begin with we change coordinates as in \eqref{eq:t_x}, which yilds 
\begin{equation}\label{eq:Airy_t}
\tilde{y}''+\frac{2}{t}\tilde{y}'-\frac{1}{t^5}\tilde{y} = 0,
\end{equation}
analogous to how \eqref{eq:1_new_ODE} was obtained. Here we can
clearly see that the Airy equation has an irregular singular point at
$x=\infty$, or at $t=0$ for \eqref{eq:Airy_t}.

\subsection*{Use the method of Carlini-Liouville-Green}
Because the singularity at $t=0$ is irregular, we now set
\begin{equation}
\tilde{t}(t)=\ee^{S(t)}
\end{equation}
and try to find the leading behaviour of $S(t)$ as \ttoz. With this
\eqref{eq:Airy_t} becomes
\begin{equation}\label{eq:Airy_S}
S''+(S')^2+\frac{2}{t}S'-\frac{1}{t^5}=0,
\end{equation}
analogous to the proceeding in \eqref{eq:1_DE_S}. 

\subsubsection{First term}
Next we assume
\begin{equation}
S''(t)=\oh\qty([S'(t)]^2)\as\ttoz
\end{equation}
which gives
\begin{equation}
\Big(S'(t)\Big)^2+\frac{2}{t}S'(t)\sim\frac{1}{t^5}
\quad\Longrightarrow\quad
S'(t)\sim\frac{-1}{t}\mp\sqrt{\frac{1}{t^2}+\frac{1}{t^5}}
\sim\frac{-\xi}{t^{5/2}},
\end{equation}
where $\xi=\pm1$, like previously, keeps track of the two linearly
independent solutions. So
\begin{equation}\label{eq:2_S'}
S'(t)=\frac{-\xi}{t^{5/2}}+C'(t)\qcomma
\abs{C'(t)} \ll t^{-5/2} \;\text{ as }\; \ttoz,
\end{equation}
and
\begin{equation}\label{eq:2_S}
S(t)=\frac{2\xi}{3t^{3/2}}+C(t).
\end{equation}



\subsubsection{Second term}
Now we substitute \eqref{eq:2_S} into \eqref{eq:Airy_S} to get
\begin{equation}
\frac{5\xi}{2t^{7/2}}+C'' 
+\frac{\xi^2}{t^5} - \frac{2\xi}{t^{5/2}}C' + (C')^2
+\frac{2}{t}\qty(\frac{-\xi}{t^{5/2}}+C') - \frac{1}{t^5}=0,
\end{equation}
which, with $\xi^2\equiv1$, becomes
\begin{equation}\label{eq:Airy_C}
C'' + (C')^2 + \frac{2}{t}C' - \frac{2\xi}{t^{5/2}}C'
+\frac{\xi}{2t^{7/2}}=0
\end{equation}
As before, for \ttoz, we kwow from \eqref{eq:2_S'} that $C'(t)=\oh(t^{-5/2})$ and
we also asume that $C''(t)=\oh(t^{-7/2})$. The dominant balance as \ttoz
therefore becomes
\begin{equation}\label{eq:2_C'}
\frac{2\xi}{t^{5/2}}C'(t) \sim \frac{\xi}{2t^{7/2}}
\quad\Longrightarrow\quad
C'(t)=\frac{1}{4t}+D'(t)\qcomma |D'(t)|\ll t^{-1}.
\end{equation}
To obtain this we have also used the fact that
\begin{equation}
\abs{\frac{2}{t}C'(t)} \ll \abs{\frac{2\xi}{t^{5/2}}C'(t)}\as\ttoz.
\end{equation}
From \eqref{eq:2_C'} we get
\begin{equation}\label{eq:2_C}
C(t)=\frac{1}{4}\ln(t) + D(t).
\end{equation}

\subsubsection{Third term}
Once again we substitue the new term, \eqref{eq:2_C}, into the DE,
\eqref{eq:Airy_C}, and simplify to get
\begin{equation}\label{eq:Airy_D}
D''+(D')^2+\frac{5}{2t}D'-\frac{2\xi}{t^{5/2}}D'+\frac{5}{16t^2}=0.
\end{equation}
With the regular asumptions and previous knowlegde:
\begin{equation}
D''(t)=\oh\qty(t^-2)\qcomma
\Big(D'(t)\Big)^2=\oh\qty(t^{-5/2}D'(t))\qcomma
t^{-1}D'(t)=\oh\qty(t^{-5/2}D'(t)),
\end{equation}
all as \ttoz, we get the dominat balance
\begin{equation}
\frac{2\xi}{t^{5/2}}D'(t) \sim \frac{5}{16t^2}
\quad\Longrightarrow\quad
D'(t) = \frac{5\xi}{32}t^{1/2} + E'(t)\qcomma 
E'(t)=\oh\qty(t^{1/2}) \text{ as } \ttoz.
\end{equation}
Now we've come to a turning point. Since $D'(t)\to0$ as \ttoz
and\footnotemark{} 
\begin{equation}
D(t)=\frac{5\xi}{48}t^{3/2}+d+E(t)\to d\as\ttoz,
\end{equation}
where $d$ is some constant,
we have reached the point where new terms after this will vanish in
the limit  \ttoz.
\footnotetext{We're also assuming that $E(t)=\oh(t^{3/2})$ as \ttoz.}

\subsubsection{Check for consistency}
\begin{enumerate}[label=(\roman*)]
\item 
The first assumption was that $S''(t)$, which is $\order{t^{-7/2}}$
should be negligable compared to $(S'(t))^2=\order{t^{-5}}$, which is
clearly consistent with our findings.
\item 
Next we assumed that $C''(t)$, which is $\order{t^{-2}}$, should be
of order $\oh(t^{-7/2})$. This also turned out to be consistent. For
completeness, let's also mention that $C''(t)=\order{t^{-2}}$ is
negligable compared to $t^{-5/2}C'(t)=\order{t^{-7/2}}$.
\item 
Here the assumption was that $D''(t)=\oh(t^-2)$, which also isn't any
problems. 
\end{enumerate}
All big and little oh's are as \ttoz.

\subsection*{Final result}
To wrap everything up we now have the leading behaviours of the two
linearly independent solutions
\begin{equation}
\tilde{y}_\xi(t)=\ee^{S(t)}
\sim\exp[\frac{2\xi}{3t^{3/2}}+\frac{1}{4}\ln(t)+d]
=c t^{1/4} \ee^{\frac{2\xi}{3t^{3/2}}},
\end{equation}
or in terms of $x$
\begin{equation}
y_\xi(x)\sim c x^{-1/4} \ee^{\frac{2\xi}{3}x^{3/2}},
\end{equation}
where $c$ is just some new constant. With $\xi=\pm1$ we can write the
leading oder terms of the two linearly independent solutions to the
Airy equation \eqref{eq:Airy} as
\begin{equation}
\begin{cases}
y_+(x)= c x^{-1/4} \ee^{+\frac{2}{3}x^{3/2}}\\
y_-(x)= c x^{-1/4} \ee^{-\frac{2}{3}x^{3/2}}.
\end{cases}
\end{equation}
\qed


\end{document}



