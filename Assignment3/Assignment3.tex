\documentclass[11pt,letter, swedish, english
]{article}
\pdfoutput=1

\usepackage{../custom_as}

\renewcommand{\thesubsection}{\arabic{section} (\alph{subsection})}

\renewcommand{\thesubsubsection}{\arabic{section} (\alph{subsection},\,\roman{subsubsection})}


%%Drar in tabell och figurtexter
\usepackage[margin=10 pt]{caption}
%%För att lägga in 'att göra'-noteringar i texten
\usepackage{todonotes} %\todo{...}

%%För att själv bestämma marginalerna. 
\usepackage[
%            top    = 3cm,
%            bottom = 3cm,
%            left   = 3cm, right  = 3cm
]{geometry}



\begin{document}

%%%%%%%%%%%%%%%%% vvv Inbyggd titelsida vvv %%%%%%%%%%%%%%%%%
% \begin{titlepage}
\title{Asymptotic Analasys and Pertubation Theory -- AMATH\,732 \\
Assignment 3}
\author{Andréas Sundström}
\date{\today}

\maketitle

%%%%%%%%%%%%%%%%% ^^^ Inbyggd titelsida ^^^ %%%%%%%%%%%%%%%%%

%Om man vill ha en lista med vilka todo:s som finns.
%\todolist

\section{An ODE}
We have the differential equation
\begin{equation}\label{eq:1_ODE}
x^4y''-y=0
\quad\Longleftrightarrow\quad
y''-\frac{1}{x^4}y = 0.
\end{equation}

\subsection{Classifysing the suspect points 
  and trying the method of Frobenius} 

\subsubsection{Classifying the points $x=0, \infty$}
To classify the point $x=0$, we first see that $\nicefrac{1}{x^4}$
isn't analytic there and so is $x^2\cdot\nicefrac{1}{x^4}$. Thus the
point $x=0$ is an \emph{irregular singular point}.

To classify the point $x=\infty$, we nee to change variables to
\begin{equation}
t=\frac{1}{x}\qcomma \tilde{y}(t)=y(x),
\end{equation}
and classify the new DE at $t=0$.
The second derivative (w.r.t. $x$) becomes
\begin{equation}
\dv[2]{y}{x}=\dv{t}{x}
\dv{t}\qty[\dv{t}{x}\dv{\tilde{y}}{t}]=t^4\tilde{y}''+2t^3\tilde{y}',
\end{equation}
where $\tilde{y}'=\nicefrac{\rd\tilde{y}}{\rd{t}}$. With this we can
write (\ref{eq:1_ODE}\,a) as
\begin{equation}
0=\frac{1}{t^4}\qty(t^4\tilde{y}''+2t^3\tilde{y}') - \tilde{y} '
= \tilde{y}'' +\frac{2}{t}\tilde{y}' - \tilde{y}.
\end{equation}
From here we see that $t=0$ is not an ordinary point since we have the
$\nicefrac{2}{t}$ coefficient, but it is however a \emph{regular singular
point} since $t\cdot\nicefrac{2}{t}=2$ isn't singular at $t=0$


\subsubsection{Trying the method of Frobenius}
The method of Frobenius is a method for finding series solutions around
regular singular points, which utilizes the fact that there is 
at least one solution of the form $(x-x_0)^{\alpha}A(x)$ in a
neighbourhood of the regular singular point $x_0$ and where $A(x)$ is
analytic near $x_0$. When searching for a solution with Frobenius'
method we assume that
\begin{equation}
y(x)=\sum_{k=0}^\infty a_k (x-x_0)^{\alpha+k}.
\end{equation}
The second derivative of $y$ then becomes
\begin{equation}
y(x)=\sum_{k=0}^\infty a_k (\alpha+k)(\alpha+k-1) (x-x_0)^{\alpha+k-2}.
\end{equation}

In our case we going to look at $x_0=0$. Thus the DE becomes
\begin{equation}
0=x^4y''-y=
\sum_{k=0}^\infty a_k (\alpha+k)(\alpha+k-1) x^{\alpha+k+2}
-\sum_{k=0}^\infty a_k x^{\alpha+k}.
\end{equation}
Collecting like powers of $x$ yields
\begin{equation}
\begin{cases}
a_0=0\\
a_1=0\\
a_0(\alpha)(\alpha-1)-a_2=0\\
\vdots\\
a_n(\alpha+n)(\alpha+n-1)-a_{n+2}=0
\end{cases}
\end{equation}
which reduces to $a_n=0$ for all $n$.
\todo{Why dosen't it work?}


\subsection{The method of Carlini-Liouville-Green}



\subsection{Finding the actual solutions}







\section{The Airy equation}








\end{document}





%% På svenska ska citattecknet vara samma i både början och slut.
%% Använd två apostrofer (två enkelfjongar): ''.


%% Inkludera PDF-dokument
\includepdf[pages={1-}]{filnamn.pdf} %Filnamnet får INTE innehålla 'mellanslag'!

%% Figurer inkluderade som pdf-filer
\begin{figure}\centering
\centerline{ %centrerar även större bilder
\includegraphics[width=1\textwidth]{filnamn.pdf}
}
\caption{}
\label{fig:}
\end{figure}

%% Figurer inkluderade med xfigs "Combined PDF/LaTeX"
\begin{figure}\centering
\resizebox{.8\textwidth}{!}{\input{filnamn.pdf_t}}
\caption{}
\label{fig:}
\end{figure}

%% Figurer roterade 90 grader
\begin{sidewaysfigure}\centering
\centerline{ %centrerar även större bilder
\includegraphics[width=1\textwidth]{filnamn.pdf}
}
\caption{}
\label{fig:}
\end{sidewaysfigure}


%%Om man vill lägga till något i TOC
\stepcounter{section} %Till exempel en 'section'
\addcontentsline{toc}{section}{\Alph{section}\hspace{8 pt}Labblogg} 

