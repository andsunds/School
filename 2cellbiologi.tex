\chapter{Cellbiologi}

Miljontals år av evolution har lett till att det idag finns en mängd olika sorters celler med sina egna inre strukturer. Det finns allt från bakteriers tillsynes oordnade inre till djur- och växtcellers högst strukturerade innanmäte. 

Cellerna måste både kunna hålla sig vid liv och reproducera sig för att dess art ska leva vidare. Dessa två uppgifter kan vidare delas upp i deluppgifter som tilldelas olika delar av cellens beståndsdelar. Får man cellerna att samverka kan mer avancerat flercelligt liv upprätthållas men då med större krav på ordning inom och mellan varje cell.

Resterande del av detta kapitel bygger på information från boken \textit{The Cell} av G. M. \todo{ok så här?} Cooper~\cite{Cooper_TheCell2000} om inget annat anges.


\section{Cytoplasman}
I studiet av levande organismer skiljer man på celler med sitt arvsanlag samlat i en cellkärna, eukaryoter, och de utan cellkärna, prokaryoter. Bakterier tillhör de sistnämnda medan svampar, växter och djur tillhör de förstnämnda. Cellen fyller dock många fler funktioner än att bara vara förvaringsplats för arvsmassan. Och dessa egenskaper beror stark på vilken miljö den anpassats till evolutionärt. De flesta celltyper har dock något slags skyddande hölje i form av cellvägg eller cellmembran och där innanför en vätska fylld med diverse filament, partiklar och en mängd olika organeller som var och en har sina specifika arbetsuppgifter. Denna blandning kallas med ett gemensamt namn för cytoplasman. 

Organellerna varierar i storlek och koncentration och det finns allt från många mindre mitokondrier som ser till att maten vi förtär omvandlas till kroppens egna energikälla, ATP, till den stora strukturen av endoplasmatiska nätverket som bland annat syntetiserar proteiner. Förutom att bryta ner näringsämnen och från dem tillverka nya ämnen är en annan viktig process i cytoplasman transport, där en mängd olika organeller och strukturer inom cellen bidrar.


\subsection{Transport inom cellen}
\todo[inline]{Mer om passiv transport ska infogas här}

Att partiklar, från små näringsämnen till stora proteiner, kan ta sig fram genom cytoplasman spelar onekligen en viktig roll för många funktioner i cellen. Om exempelvis näringen vi stoppade i oss inte skulle nå cellernas energifabriker, mitokondrierna, tillräckligt snabbt eller komma fram i för liten skara skulle kroppen med stor sannolikhet snabbt upphöra att fungera. Detta då många av cellens funktioner är starkt beroende av det ATP som mitokondrierna\footnotemark{} producerar.
\footnotetext{Även om en mindre mängd ATP kan produceras även utan mitokondriernas hjälp bildas majoriteten av ATP:n just här \cite{Solunetti_ATP}.}

I celler talar man om två typer av transport i cytoplasman: den aktiva och den passiva transporten. Under den aktiva transporten vandrar motorprotein längs med proteintrådar och för med sig det som ska transporteras. Under den passiva transporten tillåts ämnena diffundera fritt genom cytoplasman, vilket tillskillnad från driften av motorproteinerna inte kräver energi. 
Vilket transportsystem som är dominerande beror på vilken celltyp man betraktar. I mer primitiva celltyper så som bakterier och jästceller dominerar den passiva transporten medan det i celltyper som bildar stora avancerade och sammanhängande organismer, till exempel djur och växtceller, är vanligare med en dominant aktiv transport. 
%\todo{Varför?} För att tillgodose det ökade behovet av ordning och snabb transport?

Förutom aktinfilamenten som beskrivs i avsnittet nedan finns i cellen även lite tjockare proteintrådar kallade mikrotubuli. Dessa är inte symmetriska utan har en orientering där deras ena ände kallas ''$+$'' och den andra ''$-$''. Längs dessa proteintrådar kan så kallade motorprotein vandra samtidigt som de på sin ovansida fäster tag i något annat för att transportera detta längs med strängen. Det finns en typ av motorprotein som går från ''$+$'' till ''$-$'' och en annan typ som går åt motsatt håll. Då dessa mikrotubuli oftast sitter ordnade i grupp med samma sida inåt möjliggörs den aktiva transporten inom cellerna genom att rätt sorts motorprotein tillåts binda till det som behöver transporteras in mot cellens mitt eller ut mot cellens ytterkanter. 

%Förutom att transportera organeller och membranförslutna paket kan strukturen med dessa tjockare proteintrådar med rätt sorts motorproteiner även hålla vissa membranomslutna organeller på plats. Till exempel skulle Golgiapparaten, som är med och ser till att de syntiserade proteinerna i cellen hamnar på rätt plats, splittras upp i bitar och spridas runt i cellen om den inte hölls på plats mot cellens mitt av inåtvandrande motorproteiner. Även vid anafasen, som är en av faserna under celldelning där de duplicerade kromosomerna separeras så att de två sidorna av cellen får var sin kompletta uppsättning av dem, spelar mikrotubuli och motorproteiner en viktig roll.

Alla dessa proteintrådar skulle kunna påverka cytoplasmans genomtränglighet och i nuläget råder viss oenighet gällande hur cytoplasman egentligen ter sig för partiklar som rör sig genom den. Man har länge trott att cytoplasman kan ha två distinkta faser, en kolloid vätskefas där partiklarna är väl blandade i cytoplasman och en fas där ett sammanhängande nätverk av interagerande komponenter bildats i cellen vilket får cytoplasman att anta ett mer fast eller glasliknande tillstånd.

Det finns även vissa teorier om att partikelrörelsen i celler skulle kunna påverkas av den sammanlagda effekten av motorproteinernas framryckningar. Även om dessa på liten skala ter sig väl riktade och ordnade kan den sammanlagda effekten av alla dessa transporter resultera i en stokastisk kraft som därmed påverkar partikelns rörelsemönster. 

%\todo[inline]{Källor ska infogas}


\section{Aktinfilament}

Aktinfilament, även kallat F-aktin, skapas genom att fria G-aktin binds till varandra och bildar polymerer. Denna process kan ske spontant i en aktinlösning och går då åt båda hållen så att filament både byggs upp och bryts ner. Utifrån en kort filamentbit kan därmed en längre kedja byggas upp men på grund av att G-aktinet inte är helt symmetriskt utan har en orientering kommer tillväxten att ske snabbare i ena änden än den andra. Asymmetrin för de enskilda monomererna gör att hela filamentet i sig får en orientering vilket bland annat möjliggör dess användning som transportväg för motorproteiner. Denna syntes av aktinfilament sker cirka 100 gånger snabbare inuti celler än utanför celler i laboratorium då cellen har hjälp av en mängd  katalyserande proteiner med både uppbyggande och nedbrytande egenskaper. Dessa proteiners aktivitet kan i sig regleras och fås att öka eller minska som respons på visst stimuli.

Varje G-aktin i filamenten sitter lite vridet i förhållande till sina grannar vilket leder till att filamenten får formen av en dubbelhelix med bredd på ca 7\,nm och upp till flera mikrometer långa. Dessa dubbelhelixar kan i sin tur kopplas samman av andra proteiner till mer avancerade 3-dimensionella strukturer.
\todo{källa}

Större delen av aktinfilamenten finns koncentrerade strax innanför cellmembranet där de kopplats ihop till ett nätverk som både ger cellen form och stadga samtidigt som det möjliggör transport. Detta nätverk har egenskaper liknande de som återfinns hos semisolida geler. Förutom att bilda 3-dimensionella nät kan aktinfilamenten även ordnas parallellt i mer tätpackade buntar. De buntar som ligger tätast packade ger stöd åt utstickande strukturer från cellmembranet, exempelvis microvilli. De lite mer löst packade används tillsammans med motorprotein i strukturer som har förmågan att kontrahera, något som möjliggör sista steget i celldelningen där själva cellen delas i två. 
%Ett annat exempel är kroppens alla muskler som består av strukturer av aktintrådar och motorproteinet myosin.
\todo{källa}

%Som ett sista exempel kan nämnas att aktinfilament fyller en viktig funktion när cellen i sig förflyttar sig i sin omgivning. Cellen ändrar då form genom att skjuta fram ett utskott framför sig, fäster tag och drar sig fram en bit för att sedan upprepa processen.


\section{Jästceller}
%Datan som detta arbete bygger på kommer från observationer av partikelrörelse i jästceller, så här följer en kort introduktion till jäst. 

Jäst hör till riket svampar och utgörs av encelliga organismer~\cite{SGD_yeast}.
De finns att finna på växter, i jorden men även på huden och i tarmkanalen hos varmblodiga djur. 
%Jästen kan där leva antingen i symbios med värddjuret eller som parasit och i värsta fall orsaka värddjuret skada. 
Att jästen är en encellig organism möjliggör snabb reproduktion vilket gör den smidig att arbeta med i laboratorium. Dessutom uppvisar de större likhet med djurceller än de likväl encelliga bakterierna och ger därmed ges större möjlighet till att dra paralleller till djurceller med försök på jästceller. 

\subsection{Jästcellers cytoplasma}
Att jästceller är svampar innebär att de därmed varken är djur, växter eller bakterier men delar vissa likheter med alla tre celltyper. Med sitt arvsanlag samlat i en cellkärna~\cite{SGD_yeast}, precis som djur- och växtceller, skiljer sig jästceller från bakterier där arvsanlaget ligger blandat med resten av beståndsdelarna i cytoplasman.
De har även en vakuol och stabiliserande cellvägg som växtceller men saknar växtcellens kloroplaster och kan därmed inte utföra någon fotosyntes. Att jästcellen har en cellvägg innebär att den inte är lika beroende av ett stabiliserande proteinfilamentsnätverk och därmed endast har ett rudimentärt sådant~\cite{Midtveldt_etal2016}.

\subsection{Jästcellers transport inom cellen}
Djurcellernas komplicerade nät av proteintrådar, som möjliggör en aktiv transport inom cellen,, finns inte hos jästceller~\cite{Midtveldt_etal2016} som istället får förlita sig på passiv transport, förutom vid celldelning. 
Med jästceller kan man därför undersöka om anomalier från Brownsk rörelse, som beskrivs i kommande kapitel, uppkommer även utan de stokastiska krafter med ursprung i det kollektiva bidraget från motorproteinernas framryckningar.




%Bara en liten kodsnutt som behövs när man kompilerar lokalt
%%% Local Variables: 
%%% mode: latex
%%% TeX-master: "main.tex"
%%% End: 