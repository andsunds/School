\input{template_files/packages}
\usepackage{units}
%\usepackage[T1]{fontenc}
%\usepackage[utf8]{inputenc}
\usepackage{physics}

\newcommand{\ee}{\mathrm{e}}
\newcommand{\ii}{\mathrm{i}}

\title{H2a: Binary Alloy}
\author{Andr\'eas Sundstr\"om and Linnea Hesslow}
\date{\today}

\begin{document}

\input{template_files/titlepage}

\section*{Introduction}

....

\section*{Task 1: mean field theory}
Fits: we obtained $\alpha \approx 0.494$

\begin{figure}[!ht]
\begin{center}
  \includegraphics[width=0.7\textwidth]{../figures/P_MFT} 
  \caption{.....}
  \label{fig:T1:P}
\end{center}
\end{figure}

\begin{figure}[!ht]
\begin{center}
  \includegraphics[width=0.7\textwidth]{../figures/E_MFT} 
  \caption{......}
  \label{fig:T1:E}
\end{center}
\end{figure}

\begin{figure}[!ht]
\begin{center}
  \includegraphics[width=0.7\textwidth]{../figures/C_MFT} 
  \caption{......}
  \label{fig:T1:C}
\end{center}
\end{figure}

%%%%%%%%%%%%%%%%%%%%%%%%%%%%%%%%%%%%%%%%%%%%%%%%%%%%%%%
\section*{Task 2: Ising model}
\begin{align}
E_{\rm CuZn} &= \unit[-294]{meV} \\
E_{\rm CuCu} &= \unit[-436]{meV} \\
E_{\rm ZnZn} &= \unit[-133]{meV} \\
\end{align}

Figure~\ref{fig:T2:equil} shows the equilibration at three different temperatures. We note that the energy per bond is in the range $E_{\rm CuZn} \leq E \leq (E_{\rm CuCu} + E_{\rm ZnZn})/2 = \unit[284.5]{meV}$, which it should be. 

\begin{figure}[!ht]
\begin{center}
  \includegraphics[width=0.7\textwidth]{../figures/equilibration} 
  \caption{... }
  \label{fig:T2:equil}
\end{center}
\end{figure}


\begin{figure}[!ht]
\begin{center}
  \includegraphics[width=0.7\textwidth]{../figures/U} 
  \caption{... }
  \label{fig:U}
\end{center}
\end{figure}

\begin{figure}[!ht]
\begin{center}
  \includegraphics[width=0.7\textwidth]{../figures/C} 
  \caption{... }
  \label{fig:C}
\end{center}
\end{figure}

\begin{figure}[!ht]
\begin{center}
  \includegraphics[width=0.7\textwidth]{../figures/r} 
  \caption{... }
  \label{fig:r}
\end{center}
\end{figure}

\begin{figure}[!ht]
\begin{center}
  \includegraphics[width=0.7\textwidth]{../figures/P} 
  \caption{... }
  \label{fig:P}
\end{center}
\end{figure}


\subsection{Statistical inefficiency}
Figures~\ref{fig:ns_phi} and~\ref{fig:ns_block} show the statistical inefficiency at three temperatures, calculated with the correlation function and block average respectively. 

In the case of block average, we used a moving average as the data was noisy when the block size become comparable to the total number of steps. 

\begin{figure}[!ht]
\begin{center}
  \includegraphics[width=\textwidth]{../figures/stat_inefficiency_Phi} 
  \caption{... }
  \label{fig:ns_phi}
\end{center}
\end{figure}

\begin{figure}[!ht]
\begin{center}
  \includegraphics[width=\textwidth]{../figures/stat_inefficiency_block} 
  \caption{... }
  \label{fig:ns_block}
\end{center}
\end{figure}

\begin{figure}[!ht]
\begin{center}
  \includegraphics[width=0.7\textwidth]{../figures/stat_inefficiency_both} 
  \caption{... }
  \label{fig:ns_both}
\end{center}
\end{figure}
\section*{Concluding discussion}
 ...  
\newpage

\appendix

\section{Source Code}

%\subsection{Calculating pi using matlab: \texttt{pi.m}}
%\lstinputlisting[language=matlab,numbers=left]{template_files/pi.m}

%\subsection{Calculating pi using python: \texttt{pi.py}}
%\lstinputlisting[language=python,numbers=left]{template_files/pi.py}

\subsection{Main program task 2: \texttt{main\_T2.c}}
\lstinputlisting[language=c,numbers=left]{../code/main_T2.c}


\subsection{Misc functions : \texttt{funcs.c}}
\lstinputlisting[language=c,numbers=left]{../code/funcs.c}

\section{Auxiliary }
\subsection{Makefile}
\lstinputlisting[language=bash,numbers=left]{../code/Makefile}

\section{MATLAB scripts}
\subsection{Task 1 and analysis scripts for Task 2}
\lstinputlisting[language=matlab,numbers=left]{../m_scripts/H2_analysis.m}

\subsection{Improve figure appearance: \texttt{ImproveFigureCompPhys.m}}
\lstinputlisting[language=matlab,numbers=left]{../m_scripts/ImproveFigureCompPhys.m}

\subsection{Change size of figures: \texttt{setFigureSize.m}}
\lstinputlisting[language=matlab,numbers=left]{../m_scripts/setFigureSize.m}
\end{document}

%%% Local Variables:
%%% mode: latex
%%% TeX-master: t
%%% End:
