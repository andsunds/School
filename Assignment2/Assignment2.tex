\documentclass[11pt,letter, swedish, english
]{article}
\pdfoutput=1

\usepackage{../custom_as}

%\usepackage{listings} 
%\usepackage[framed,numbered,autolinebreaks,useliterate]{../mcode}
%\lstloadlanguages{matlab} 
%\lstset{language=matlab} 

\usepackage[makeroom]{cancel}
\graphicspath{{figures/}}


\swapcommands{\Omega}{\varOmega}

%%Drar in tabell och figurtexter
\usepackage[margin=10 pt]{caption}
%%För att lägga in 'att göra'-noteringar i texten
\usepackage{todonotes} %\todo{...}

%%För att själv bestämma marginalerna. 
\usepackage[
%            top    = 2.5cm,
%            bottom = 3cm,
%            left   = 3cm, right  = 3cm
]{geometry}

%%För att ändra hur rubrikerna ska formateras


%\renewcommand{\thefootnote}{\fnsymbol{footnote}}

\renewcommand{\thesubsection}{\arabic{section} (\alph{subsection})}
\renewcommand{\thesubsubsection}{\arabic{section} (\alph{subsection},\,\roman{subsubsection})}

\newcommand{\Dx}{\ensuremath{\Delta{x}}}
\newcommand{\Dt}{\ensuremath{\Delta{t}}}

\begin{document}




%%%%%%%%%%%%%%%%% vvv Inbyggd titelsida vvv %%%%%%%%%%%%%%%%%

\title{Numerical solutions to PDE's -- AMATH\,741 \\
Assignment 2}
\author{Andréas Sundström}
\date{\today}

\maketitle

%%%%%%%%%%%%%%%%% ^^^ Inbyggd titelsida ^^^ %%%%%%%%%%%%%%%%%


\section{Consistency of a scheme for the heat equaiton}
Here we have a scheme
\begin{equation}
U_j^{n+1}=\alpha U_j^n+\frac{1-\alpha}{2}\qty(U_{j+1}^n+U_{j-1}^n)
\end{equation}
for the heat equation $u_t=\sigma u_{xx}$, and we want to find a
criteria on $\alpha$ for this scheme to be consistent.

Consistensy means that the truncation error
\begin{equation}
\tau_j^n=\order{\Dt^m, \Dx^k}
\end{equation}
where $m>0$ and $k>1$. 
If we have a scheme of the form
\begin{equation}
V^{n+1}_j=\mathcal{Q}_jV^n,
\end{equation}
where $\mathcal{Q}_j$ is some FD scheme, for a PDE of the form
$u_t+\mathcal{D}_xu=0$, where $\mathcal{D}_x$ is some differential
operator in $x$, then the truncation error can be defined via
\begin{equation}
\Dt\,\tau_j^{n}=u_j^{n+1}-\mathcal{Q}_ju^n.
\end{equation}

In our case we have
\begin{equation}
\mathcal{Q}_jV^n=\alpha V_j^n+\frac{1-\alpha}{2}\qty(V_{j+1}^n+V_{j-1}^n),
\end{equation}
so the truncation error is
\begin{equation}
\begin{aligned}
\Dt\,\tau_j^n=&u_j^{n+1}
-\qty{\alpha u_j^n+\frac{1-\alpha}{2}\qty(u_{j+1}^n+u_{j-1}^n)}\\
=& \qty[u_j^n+(u_t)_j^n+\order{\Dt^2}]\\
&-\Bigg\{\alpha u_j^n+\frac{1-\alpha}{2}
\Bigg[\qty(u_j^n+(u_x)_j^n\Dx+\frac{\Dx^2}{2}(u_{xx})_j^n)\\
&\hspace{90pt}+\qty(u_j^n-(u_x)_j^n\Dx+\frac{\Dx^2}{2}(u_{xx})_j^n)
+\order{\Dx^4}\Bigg]\Bigg\}.
\end{aligned}
\end{equation}
Now, canceling the terms that can cancel and dividing both sides by
$\Dt$ gives
\begin{equation}
\tau_j^n=(u_t)_j^n-\frac{1-\alpha}{2\Dt}\Dx^2(u_{xx})_j^n
+\order{\Dt, \frac{\Dx^4}{\Dt}}.
\end{equation}
We see that for this to be vanishing when $\Dt, \Dx\to0$, we need the
first two terms to cancel. That will only happen if
\begin{equation}
\frac{1-\alpha}{2\Dt}\Dx^2=\sigma
\quad\Longleftrightarrow\quad
\alpha=1-2\frac{\sigma\Dt}{\Dx^2}=1-2r.
\qed
\end{equation}
Also note that since $\alpha$ is kept constant, $r$ will also be
constant; and therefore $\Dt\propto\Dx^2$, meannig that
$\order{\Dx^4/\Dt}=\order{\Dx^2}$.

\section{Consistency for another scheme to solve the heat equation}
This time we have another scheme for the heat equation:
\begin{equation}
U_{j}^{n+1}=(1-2\alpha_1-2\alpha_2)U_{j}^{n}
+\alpha_1(U_{j+1}^n+U_{j-1}^n)+\alpha_2(U_{j+2}^n+U_{j-2}^n).
\end{equation}

\subsection{Consistency}
We have a very silimar case to the previous problem. The truncation
error is
\begin{equation}
\begin{aligned}
\Dt\,\tau_j^n=& \qty[u_j^n+(u_t)_j^n\Dt+\order{\Dt^2}]\\
&-\Bigg\{(1-2\alpha_1-2\alpha_2)u_j^n
\\ &\hspace{22pt}+\alpha_1\Bigg[
\qty(u_j^n+\frac{\Dx^2}{2}(u_{xx})_j^n+\frac{\Dx^4}{24}(u_{xxxx})_j^n)
\\ &\hspace{55pt}
+\qty(u_j^n+\frac{\Dx^2}{2}(u_{xx})_j^n+\frac{\Dx^4}{24}(u_{xxxx})_j^n)
+\order{\Dx^6}\Bigg]
\\ &\hspace{22pt}
+\alpha_2\Bigg[
\qty(u_j^n+\frac{4\Dx^2}{2}(u_{xx})_j^n+\frac{8\Dx^4}{24}(u_{xxxx})_j^n)
\\ &\hspace{55pt}
+\qty(u_j^n+\frac{4\Dx^2}{2}(u_{xx})_j^n+\frac{8\Dx^4}{24}(u_{xxxx})_j^n)
+\order{\Dx^6}\Bigg]
\Bigg\}.
\end{aligned}
\end{equation}
We can now cancel most of these terms and simplify:
\begin{equation}
\begin{aligned}
\Dt\,\tau_j^n=& (u_t)_j^n\Dt
-\Bigg\{\alpha_1\qty[
\qty(\Dx^2(u_{xx})_j^n+\frac{\Dx^4}{12}(u_{xxxx})_j^n)]
\\ &\hspace{55pt}
+\alpha_2\qty[
\qty(4\Dx^2(u_{xx})_j^n+\frac{2\Dx^4}{3}(u_{xxxx})_j^n)]
\Bigg\} +\order{\Dt^2, \Dx^6}\\
=& (u_t)_j^n\Dt-[\alpha_1+4\alpha_2](u_{xx})_j^n\Dx^2
+\frac{1}{12}[\alpha_1+8\alpha_2](u_{4x})_j^n\Dx^4
+\order{\Dt^2, \Dx^6}.
\end{aligned}
\end{equation}
And we once again note that for consitency, we need 
\begin{equation}\label{eq:2a_end}
(u_t)_j^n-[\alpha_1+4\alpha_2](u_{xx})_j^n\frac{\Dx^2}{\Dt}=0
\quad\Longrightarrow\quad
\alpha_1+4\alpha_2=\frac{\sigma\Dt}{\Dx^2}=r.
\qed
\end{equation}
And also again, since $\alpha_1$ and $\alpha_2$ are constant,
$\Dt\propto\Dx^2$. 

\subsection{Fourth order accuracy}
Given the condition \eqref{eq:2a_end}, we have the accuracy
\begin{equation}
\tau_j^n=\frac{1}{12}[\alpha_1+8\alpha_2](u_{4x})_j^n\Dx^2
+\order{\Dt, \Dx^4}.
\end{equation}
To get fourth order accuracy, we also need
\begin{equation}
\alpha_1+8\alpha_2=0.
\end{equation}
Both of these conditions together is a $2\times2$ system of linear
equations 
\begin{equation}
\begin{bmatrix}1&4\\1&8\end{bmatrix}
\begin{bmatrix}\alpha_1\\\alpha_2\end{bmatrix}
=\begin{bmatrix}r\\0\end{bmatrix}
\quad\Longrightarrow\quad
\begin{bmatrix}\alpha_1\\\alpha_2\end{bmatrix}
=r\begin{bmatrix*}[r]
2\\-1
\end{bmatrix*}.
\qed
\end{equation}


\section{A scheme for the wave equation}











\section{Matrix representation of a scheme for the advection equation}











\end{document}




%  LocalWords:  MFT MF Advection PDE's AMATH IC discretization MATLAB
