\documentclass[11pt,letter, swedish, english
]{article}
\pdfoutput=1

\usepackage{../custom_as}
\usepackage[makeroom
]{cancel}
\graphicspath{{figures/}}

\swapcommands{\Delta}{\varDelta}
\swapcommands{\Omega}{\varOmega}

%%Drar in tabell och figurtexter
\usepackage[margin=10 pt]{caption}
%%För att lägga in 'att göra'-noteringar i texten
\usepackage{todonotes} %\todo{...}

%%För att själv bestämma marginalerna. 
\usepackage[
%            top    = 2.5cm,
%            bottom = 3cm,
%            left   = 3cm, right  = 3cm
]{geometry}

%%För att ändra hur rubrikerna ska formateras
%\renewcommand{\thesubsection}{\arabic{section} (\roman{subsection})}
\renewcommand{\thesubsection}{\arabic{section} (\alph{subsection})}
\renewcommand{\thesubsubsection}{\arabic{section} (\alph{subsection},\,\roman{subsubsection})}

%\renewcommand{\thefootnote}{\fnsymbol{footnote}}

\newcommand{\Tc}{\ensuremath{T_{\text{c}}}}
\newcommand{\sign}{\ensuremath{\text{sign}}}

%\usepackage{tikz}

\begin{document}

%\tikzstyle{every picture}+=[remember picture]
%\tikzstyle{na} = [shape=rectangle,inner sep=0pt,text depth=0pt]



%%%%%%%%%%%%%%%%% vvv Inbyggd titelsida vvv %%%%%%%%%%%%%%%%%

\title{Statistical Physics 2 -- PHYS\,705 \\
Assignment 2}
\author{Andréas Sundström}
\date{\today}

\maketitle

%%%%%%%%%%%%%%%%% ^^^ Inbyggd titelsida ^^^ %%%%%%%%%%%%%%%%%

\section{The Ornstein-Zernike formula}
In this problem we are concerned about the Ornstein-Zernike formula 
$G(\vb*r)=r^{-(d-1)/2}\ee^{-r/\xi}$. We will start from the real space
respresentation 
\begin{equation}\label{eq:1_start}
G(\vb*r) = \frac{T}{2}\int\!\rd^dq\,\frac{\ee^{\ii\vb*q\vdot\vb*r}}{kt+bq^2},
\end{equation}
where $q^2=\vb*q\vdot\vb*q$, $t=(T-\Tc)/\Tc$, $k$ and $b$ are some
constants, and $r\gg\xi:=\sqrt{b/kt}$.

\subsection*{1~(a+b)\quad Doing the integrals}\addtocounter{subsection}{2}
Using 
\begin{equation}\label{eq:1_1/c}
\int_0^\infty\rd{u}\,\ee^{-cu}=\frac{1}{c}
\qcomma a>0,
\end{equation}
we can rewrite the target integral as
\begin{equation}
g(\vb*r)=\int_{-\infty}^\infty\!\rd^dq\,
\frac{\ee^{\ii\vb*q\vdot\vb*r}}{kt+bq^2}
\stackrel{\eqref{eq:1_1/c}}{=}
\int_{-\infty}^\infty\!\rd^dq\, \ee^{\ii\vb*q\vdot\vb*r}
\int_0^\infty\!\rd{u}\,\ee^{-u\,(kt+bq^2)}.
\end{equation}
Now, since this is physics\footnotemark{}, we can swap the order of
integration; and while we're at it we might as well change variables
to $Q=\sqrt{b/kt}\,q=\xi q$ and $U=ktu$. We thereby get
\begin{equation}
g(\vb*r)=\int_0^\infty\!\frac{\rd{U}}{kt}\,\ee^{-U}
\prod_i\int_{-\infty}^\infty\!\frac{\rd{Q}_i}{\xi}\,
 \exp{-UQ_i^2+\frac{\ii r_i}{\xi}Q_i}.
\end{equation}
Each of these $Q_i$ integrals are just regular Gaussian so
\begin{equation}
\int_{-\infty}^\infty\!\rd{Q}_i\,
 \exp{-UQ_i^2+\frac{\ii r_i}{\xi}Q_i}
=\qty(\frac{\pi}{U})^{1/2}\exp{-\frac{r_i^2}{4\xi^2U}}.
\end{equation}
and we and up with
\begin{equation}\label{eq:1b_end}
g(\vb*r)=\frac{\pi^{d/2}}{\xi^d kt}\int_0^\infty\!\rd{U}\,
U^{d/2}\exp{-U-\frac{r^2}{4\xi^2U}}.
\end{equation}

\footnotetext{I.e. we don't care, but in reality the functions are
  nice enough, so we can do it.} 


\subsection{Asymptotic behaviour}
To get to the asymptotic behaviour of $g$, we enlist the help of
\textit{Mathematica} which helps us to calculate the last remaining
integral \eqref{eq:1b_end}. And we get
\begin{equation}
g(\vb*r)=\qty(\frac{\pi}{2})^{d/2}\frac{1}{\xi^dkt} 
\qty(\frac{\xi}{r})^{-1-\frac{d}{2}} 
K_{-1-\frac{d}{2}}\qty(\frac{r}{\xi}),
\end{equation}
where $K_\mu$ is the modified Bessel function of the second kind. Next
we use the asymptotic formula 
\begin{equation}
K_\mu(z)\sim \ee^{-z}\sqrt{\frac{\pi}{2z}}
\end{equation}
to get
\begin{equation}
g(\vb*r)=\qty(\frac{\pi}{2})^{d/2}\frac{1}{\xi^dkt} 
\qty(\frac{\xi}{r})^{-1-\frac{d}{2}} 
\ee^{-r/\xi}\qty(\frac{\pi}{2})^{1/2}\qty(\frac{\xi}{r})^{1/2}
\propto
%=\qty(\frac{\pi}{2})^{(d+1)/2}\frac{1}{\xi^{(3d+1)/2}kt}\;
r^{(1-d)/2}\ee^{-r/\xi},
\end{equation}
which was what we set out to show. 



\section{MFT correlation length in the Ising model}
From class, we derived an expression for the, momentum space,
correlation function at $T>\Tc$:
\begin{equation}
G_{T>\Tc}(\vb*q)=\frac{1}{t+\frac{\kappa}{2T}q^2},
\end{equation}
for small $q$, where $t=(T-\Tc)/\Tc$ and $\kappa$ is a given
constant. From this expression we got the correlation length
\begin{equation}
\xi(t>0)=\qty[\frac{\kappa}{2Tt}]^{1/2}
=\qty[\frac{\kappa}{2(T-\Tc)}]^{1/2}.
\end{equation}
We got to this correlation length fisrst after we've done
the inverse F.T. but if we can get the momentum space correlation
function, for $t<0$, on the form
\begin{equation}
G_{T<\Tc}(\vb*q)\propto\qty[1+\xi^2q^2]^{-1},
\end{equation}
we don't have to redo the inverse F.T.

We begin with the MFT equation for the correlation function
\begin{equation}
G_{ij}=(1-M^2)\sum_l \qty[\frac{1}{T}J_{il}G_{lj}+\delta_{ij}]
\quad\Longleftrightarrow\quad
\sum_l\qty(\frac{\delta_{il}}{1-M^2}-\frac{1}{T}J_{il})G_{lj}
=\delta_{ij},
\end{equation}
where the MFT value for $M^2$ is $M^2=3|t|=-3t$.
Next step is to F.T. this using the transformation
\begin{equation}
A_{ij}=\frac{1}{N}\sum_{\vb*q} A(\vb*q)\ee^{\ii\vb*q\vdot(\vb*r_i-\vb*r_j)}.
\end{equation}
And we get
\begin{equation}\label{eq:2_MFT_FT}
\frac{1}{N}\sum_{\vb*q} \frac{G(\vb*q)}{1-M^2}
\ee^{\ii\vb*q\vdot(\vb*r_i-\vb*r_j)}
-\frac{1}{T}\sum_l\frac{1}{n^2}\sum_{\vb*q', \vb*q''}
J(\vb*q')\ee^{\ii\vb*q'\vdot(\vb*r_i-\vb*r_l)}
G(\vb*q'')\ee^{\ii\vb*q''\vdot(\vb*r_l-\vb*r_j)}
=\frac{1}{N}\sum_{\vb*q}\ee^{\ii\vb*q\vdot(\vb*r_l-\vb*r_j)}.
\end{equation}
From here, we use 
\begin{equation}
\frac{1}{N}\sum_n\ee^{-\ii \vb*r_n\vdot(\vb*q'-\vb*q'')}=\delta_{\vb*q', \vb*q''}
\end{equation}
to collapse the $l$ sum and one of the $\vb*q$ sums in the second term
of \eqref{eq:2_MFT_FT}:
\begin{equation}
\frac{1}{N}\sum_{\vb*q} 
\qty(\frac{1}{1-M^2} -\frac{1}{T}J(\vb*q))G(\vb*q)
\ee^{\ii\vb*q\vdot(\vb*r_i-\vb*r_j)}
=\frac{1}{N}\sum_{\vb*q}\ee^{\ii\vb*q\vdot(\vb*r_l-\vb*r_j)}.
\end{equation}
This should be valid for any $\vb*r_i$ and $\vb*r_j$, so we can equate
the terms:
\begin{equation}
G(\vb*q)=\frac{1}{\frac{1}{1-M^2} -\frac{1}{T}J(\vb*q)}.
\end{equation}

Now we need the long-wavelength limit for $J$, which we derived in
class to be
\begin{equation}
J(\vb*q)=J-\frac{\kappa}{2}q^2,
\end{equation}
where 
\begin{equation}
\kappa:=\frac{1}{Nd}\sum_{ij}J_{ij}\abs{\vb*r_i-\vb*r_j}^2
\end{equation}
is just a constant. 

We now have all we need
\begin{equation}
\begin{aligned}
G(\vb*q)=&\qty[\frac{1}{1-M^2}-\frac{J}{T}+\frac{\kappa}{2T}q^2]^{-1}
&\stackrel{\text{MFT}}{=}&
\qty[\frac{1}{1+3t}-\frac{\Tc}{T}+\frac{\kappa}{2T}q^2]^{-1}\\
\stackrel{|t|\ll1}{\approx}&
\qty[(1-3t)-\frac{\Tc}{T}+\frac{\kappa}{2T}q^2]^{-1}
&=\hspace{5pt}&
\qty[-3t+\frac{T-\Tc}{T}+\frac{\kappa}{2T}q^2]^{-1}\\
\stackrel{T\approx\Tc}{\approx}&
\qty[-3t+t+\frac{\kappa}{2T}q^2]^{-1}
&\propto\hspace{5pt}&
\qty[1+\frac{\kappa}{T|t|}q^2]^{-1}.
\end{aligned}
\end{equation}
We therefore have
\begin{equation}
\xi(t<0)=\qty(\frac{\kappa}{T|t|})^{1/2}
\propto |t|^{-\nu}\qcomma \nu=\frac{1}{2}
\end{equation}
and the amplitude ratio is
\begin{equation}
\frac{k_+}{k_-}=\sqrt{2}.
\end{equation}





\section{Heisenberg ferromagnet}
In this problem we have a Heisenberg ferromagnet with Hamiltonian
\begin{equation}
H=-\frac{1}{2}\sum_{ij}J_{ij}\vb*S_i\vdot\vb*S_j
=-\frac{1}{2}\sum_{ij}J_{ij}S_i^xS_j^x
-\frac{1}{2}\sum_{ij}J_{ij}S_i^yS_j^y
-\frac{1}{2}\sum_{ij}J_{ij}S_i^zS_j^z
\end{equation}
and partition function
\begin{equation}\label{eq:3_Z1}
Z=\Tr{\ee^{-H/T}}=\int_{\vb*S_i^2=1}\!\prod_i\rd{\vb*S_i}\;\ee^{-H/T},
\end{equation}
where $\vb*S_i$ are spin unit vectors ($\vb*S_i\vdot\vb*S_i=1$).

\subsection{Functional integral representation}
To find a functional integral representation of the partition
function, we will need the Hubbard-Stratonovich transformation (HST)
\begin{equation}
\exp[\frac{1}{2}\sum_{ij}A_{ij}^{-1}\gamma_i\gamma_j]
=C\int\rd\varphi_1\cdots\rd\varphi_N 
\,\exp[-\frac{1}{2}\sum_{ij}\varphi_iA_{ij}\varphi_j+\sum_i\varphi_i\gamma_i],
\end{equation}
where $C$ is some unimportant numerical constant, that will be dropped
in future calculations here. 

In our case, we will use the HST to rewrite
\begin{equation}
\begin{aligned}
\ee^{-H/T}=&\exp[\frac{1}{2T}\sum_{ij}J_{ij}S_i^xS_j^x]
\exp[\frac{1}{2T}\sum_{ij}J_{ij}S_i^yS_j^y]
\exp[\frac{1}{2T}\sum_{ij}J_{ij}S_i^zS_j^z]\\
=&\int\rd\varphi_1^x\cdots\rd\varphi_N^x
\,\exp[-\frac{1}{2T}\sum_{ij}\varphi_i^xJ_{ij}^{-1}\varphi_j^x
+\frac{1}{T}\sum_i\varphi_i^xS_i^x]\\
&\times\int\rd\varphi_1^y\cdots\rd\varphi_N^y
\,\exp[-\frac{1}{2T}\sum_{ij}\varphi_i^yJ_{ij}^{-1}\varphi_j^y
+\frac{1}{T}\sum_i\varphi_i^yS_i^y]\\
&\times\int\rd\varphi_1^z\cdots\rd\varphi_N^z
\,\exp[-\frac{1}{2T}\sum_{ij}\varphi_i^zJ_{ij}^{-1}\varphi_j^z
+\frac{1}{T}\sum_i\varphi_i^zS_i^z].
\end{aligned}
\end{equation}
Then we can write everything as one exponential under one integration:
\begin{equation}
\ee^{-H/T}=\int\rd\vb*\varphi_1\cdots\rd\vb*\varphi_N
\,\exp[-\frac{1}{2T}\sum_{ij}J_{ij}^{-1}\vb*\varphi_i\vdot\vb*\varphi_j
+\frac{1}{T}\sum_i\vb*\varphi_i\vdot\vb*S_i].
\end{equation}

The next step, to get the partition function \eqref{eq:3_Z1}, is to integrate over all
possible $\vb*S_i$'s. To do this we take a closer look at \emph{one}
of those integrals: 
\begin{equation}
\int_{\vb*S_i^2=1}\rd\vb*S_i\ee^{\vb*\varphi_i\vdot\vb*S_i/T}
=\int_0^{2\pi}\rd\phi_i\int_0^{\pi}\rd\theta_i\,\sin(\theta_i)
\ee^{|\vb*\varphi_i|\,|\vb*S_i|\,\cos(\theta_i)/T}
=4\pi\frac{\sinh(\varphi_i)/T}{\varphi_i/T},
\end{equation}
where $\varphi_i:=|\vb*\varphi_i|$.
Here we have used the trick to align the $z$~axis of the integration
with $\vb*\varphi_i$, therefore making
$\vb*\varphi_i\vdot\vb*S_i=\varphi_i\cos\theta_i$ since $|\vb*S_i|=1$.
We therefore get the full partition function
\begin{equation}
\begin{aligned}
Z=&\int\rd\vb*\varphi_1\cdots\rd\vb*\varphi_N
\exp[-\frac{1}{2T}\sum_{ij}\vb*\varphi_iJ_{ij}^{-1}\vb*\varphi_j]
\int_{\vb*S_i^2=1}\prod_i\rd\vb*S_i\ee^{\vb*\varphi_i\vdot\vb*S_i/T}\\
=&(4\pi)^N\int\rd\vb*\varphi_1\cdots\rd\vb*\varphi_N
\exp[-\frac{1}{2T}\sum_{ij}\vb*\varphi_iJ_{ij}^{-1}\vb*\varphi_j
+\sum_i\ln(\frac{\sinh(\varphi_i/T)}{\varphi_i/T})].
\end{aligned}
\end{equation}
We can trow away the prefacor, of $(4\pi)^N$, just as we did with te
prefactors from the HST. The action then becomes
\begin{equation}
S[\vb*\varphi]=\frac{1}{2T}\sum_{ij}J_{ij}^{-1}\vb*\varphi_i\vdot\vb*\varphi_j
-\sum_i\ln(\frac{\sinh(\varphi_i/T)}{\varphi_i/T}).
\end{equation}


\subsection{Continuum limit}
To expand $S[\vb*\varphi]$ in $\vb*\varphi$, all we need to do is to
expand the last term
\begin{equation}
\ln(\frac{\sinh(x)}{x})=\frac{x^2}{6}-\frac{x^4}{180}+\order{x^6}.
\end{equation}
We therefore get
\begin{equation}
\begin{aligned}
S[\vb*\varphi]=&\frac{1}{2T}\sum_{ij}J^{-1}_{ij}\vb*\varphi_i\vdot\vb*\varphi_j
-\frac{1}{6T^2}\sum_i\vb*\varphi_i\vdot\vb*\varphi_i
+\frac{1}{180T^4}\sum_i\big(\vb*\varphi_i\vdot\vb*\varphi_i\big)^2\\
=&\underbrace{\frac{1}{6T^2}\sum_{ij}\qty(3TJ^{-1}_{ij}-\delta_{ij})
\vb*\varphi_i\vdot\vb*\varphi_j}_{=:S_2[\vb*\varphi]}
+\underbrace{\frac{1}{180T^4}\sum_i
\big(\vb*\varphi_i\vdot\vb*\varphi_i\big)^2}_{=:S_4[\vb*\varphi]}
\end{aligned}
\end{equation}

\subsubsection{The quadratic term}
We need to diagonalize the quadratic term, which we will do by Fourier
transform:
\begin{equation}
\begin{aligned}
S_2[\vb*\varphi]=&\frac{1}{6T^2}\sum_{ij}\frac{1}{N^3}
\sum_{\vb*q, \vb*q', \vb*q''}\vb*\varphi(\vb*q)\vdot\vb*\varphi(\vb*q')
\qty[\frac{3T}{J(\vb*q'')}-1]
\ee^{\ii\vb*q\vdot\vb*r_i}\ee^{\ii\vb*q'\vdot\vb*r_j}
\ee^{\ii\vb*q''\vdot(\vb*r_i-\vb*r_j)}\\
=&\frac{1}{6NT^2}\sum_{\vb*q, \vb*q', \vb*q''}
\vb*\varphi(\vb*q)\vdot\vb*\varphi(\vb*q')
\qty[\frac{3T}{J(\vb*q'')}-1]
\frac{1}{N^2}\sum_{ij}\ee^{\ii\vb*r_i\vdot(\vb*q+\vb*q')}
\ee^{\ii\vb*r_j\vdot(\vb*q'-\vb*q'')}.
\end{aligned}
\end{equation}
Now, using 
\begin{equation}
\frac{1}{N}\sum_i\ee^{\ii\vb*r_i\vdot(\vb*Q-\vb*Q')}=\delta_{\vb*Q, \vb*Q'},
\end{equation}
we get
\begin{equation}
S_2[\vb*\varphi]=\frac{1}{6NT^2}\sum_{\vb*q}
\vb*\varphi(\vb*q)\vdot\vb*\varphi(-\vb*q)
\qty[\frac{3T}{J(\vb*q)}-1].
\end{equation}

From here, we are going to use the fact that we are on our way to
derive a GLW theory for this system. So we are close to $\Tc$,
meaning that we really only need to wory about the long-wavelength
limit for the coupling:
\begin{equation}
J(\vb*q)\sim J-\frac{\kappa}{2}q^2
\end{equation}
for small $q$. This means that
\begin{equation}
\frac{3T}{J(\vb*q)}-1\approx
\frac{3T}{J}\qty(1+\frac{\kappa}{2J}q^2)-1
=\frac{3T-J}{J}+\frac{3T\kappa}{2J^2}q^2.
\end{equation}
Already from here we can anticipate that $\Tc=J/3$, but let's not be
too hasty; let's just settle for calling $\tilde{t}:=(3T-J)/J$. 

We can now write
\begin{equation}
S_2[\vb*\varphi]=\frac{1}{6NT^2}\frac{3\kappa T}{2J^2}\sum_{\vb*q}
\vb*\varphi(\vb*q)\vdot\vb*\varphi(-\vb*q)
\qty[\frac{2J^2\tilde{t}}{3\kappa T}+q^2].
\end{equation}
If we set $r=2J^2\tilde{t}/(3\kappa T)$ and rescale
\begin{equation}\label{eq:3_rescale}
\qty(\frac{a^3\kappa}{TJ^2})^{1/2}\vb*\varphi\to\vb*\varphi.
\end{equation}
then we get
\begin{equation}
S_2[\vb*\varphi]=\frac{1}{4Na^3}\sum_{\vb*q}
\vb*\varphi(\vb*q)\vdot\vb*\varphi(-\vb*q)
\qty[r+q^2].
\end{equation}
We now want to change this back to real space.

To do that we first have to go to the continuum limit and use 
\begin{equation}
\frac{1}{V}\sum_{\vb*q} \to \int\frac{\rd^3q}{(2\pi)^3},
\end{equation}
where $V=Na^d$. We thus have
\begin{equation}\label{eq:3_S2_qint}
S_2[\vb*\varphi]=\frac{1}{4}\int\frac{\rd^3q}{(2\pi)^3}\,
\vb*\varphi(\vb*q)\vdot\vb*\varphi(-\vb*q)
\qty[r+q^2].
\end{equation}
We now apply the (contimuum) FT
\begin{equation}
\vb*\varphi(\vb*q)=\int\rd^3x\,
\vb*\varphi(\vb*x)\ee^{-\ii\vb*q\vdot\vb*x}
\end{equation}
to this.

The first term is simple enough
\begin{equation}
\begin{aligned}
\int\frac{\rd^3q}{(2\pi)^3}\,
\vb*\varphi(\vb*q)\vdot\vb*\varphi(-\vb*q)
=&\int\frac{\rd^3q}{(2\pi)^3}\,
\int\rd^3x\int\rd^3x'\,
\vb*\varphi(\vb*x)\vdot\vb*\varphi(\vb*x)
\ee^{-\ii\vb*q\vdot(\vb*x-\vb*x')}\\
=&\int\rd^3x\,\vb*\varphi(\vb*x)\vdot\vb*\varphi(\vb*x),
\end{aligned}
\end{equation}
since
\begin{equation}
\int\frac{\rd^3q}{(2\pi)^3}\ee^{\ii\vb*q\vdot(\vb*x-\vb*x')}
=\delta((\vb*x-\vb*x')).
\end{equation}

The other term is a bit trickier. From class we know that, for a
scalar $\psi$, we have
\begin{equation}\label{eq:3_grad^2}
\int\rd^3x \Big[\grad\psi(\vb*x)\Big]^2
=\int\frac{\rd^3q}{(2\pi)^3}\psi(\vb*q)\psi(-\vb*q)q^2,
\end{equation}
which we know through the fact that
\begin{equation}
\begin{aligned}
\int\rd{x}\qty(\pdv{\psi}{x})^2=&
\int\rd{x}\int\frac{\rd{q}}{2\pi}\psi(q)\dv{x}\ee^{\ii qx}
\int\frac{\rd{q'}}{2\pi}\psi(q')\dv{x}\ee^{\ii q'x}\\
=&\int\frac{\rd{q}}{2\pi}\int\rd{q'}\,\psi(q)\psi(q')\,(\ii q)(\ii q')
\underbrace{\int\frac{\rd{x}}{2\pi}\ee^{\ii x(q+q')}}_{=\delta(q+q')}\\
=&\int\frac{\rd{q}}{2\pi}\,\psi(q)\psi(-q)\,q^2.
\end{aligned}
\end{equation}
In this case however, all we have to do is to apply
\eqref{eq:3_grad^2} to each of hte three terms in 
$\vb*\varphi(\vb*q)\vdot\vb*\varphi(-\vb*q)
=\varphi^x(\vb*q)\varphi^x(-\vb*q)
+\varphi^y(\vb*q)\varphi^y(-\vb*q)+
\varphi^y(\vb*q)\varphi^y(-\vb*q)$. 

Eventually we end up with a final result for the quadratc term:
\begin{equation}
S_2[\vb*\varphi]=\frac{1}{4}\int\rd^3x\,
\qty{\qty[\qty(\grad\varphi^x(\vb*x))^2
+\qty(\grad\varphi^y(\vb*x))^2
+\qty(\grad\varphi^z(\vb*x))^2]
+r[\vb*\varphi(\vb*x)\vdot\vb*\varphi(\vb*x)]}
\end{equation}


\subsubsection{The quartic term}
For the quartic term, we don't have to go via the detour of FT. We can
just convert the sum to an integral. Although we do have to remember that we
rescaled $\vb*\varphi$ for the quartic term, \eqref{eq:3_rescale}. So
with this in mind we get
\begin{equation}
S_4[\vb*\varphi]=
\frac{1}{180T^4}\sum_i(\vb*\varphi_i\vdot\vb*\varphi_i)^2
\to \frac{T^2J^4}{180T^4a^6\kappa}
\sum_i(\vb*\varphi_i\vdot\vb*\varphi_i)^2.
\end{equation}
And in the continuum limit we have
\begin{equation}
a^3\sum_i\to\int\rd^3x,
\end{equation}
so 
\begin{equation}
S_4[\vb*\varphi]=\frac{J^4}{180T^2a^9\kappa}
\int\rd^3x\,\Big[\vb*\varphi(\vb*x)\vdot\vb*\varphi(\vb*x)\Big]^2
=:\frac{u}{4}
\int\rd^3x\,\Big[\vb*\varphi(\vb*x)\vdot\vb*\varphi(\vb*x)\Big]^2.
\end{equation}


\subsubsection{Final result}
The final result for the continuum behaviour of the action near
$T=\Tc$ therefore is
\begin{equation}
S[\vb*\varphi]\approx S_2[\vb*\varphi]+S_4[\vb*\varphi]=
\int\rd^3x\,\qty{
\frac{1}{4}\qty[\sum_{\alpha=x,y,z}\qty(\grad\varphi^\alpha(\vb*x))^2]
+\frac{r}{2}[\vb*\varphi(\vb*x)\vdot\vb*\varphi(\vb*x)]
+\frac{u}{4}[\vb*\varphi(\vb*x)\vdot\vb*\varphi(\vb*x)]^2},
\end{equation}
where we have also rescaled $r\to2r$.

From any general Landau theory we know that $r$ has to be the
parameter that changes value at $\Tc$. This means that $\tilde{t}$
really was the reduced temperature and that $\Tc=J/3$.





\subsection{Mean field critical exponents}





\end{document}




%  LocalWords:  MFT MF Ising
