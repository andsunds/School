\documentclass[11pt,letter, swedish, english
]{article}
\pdfoutput=1

\usepackage{../custom_as}



%%Drar in tabell och figurtexter
\usepackage[margin=10 pt]{caption}
%%För att lägga in 'att göra'-noteringar i texten
\usepackage{todonotes} %\todo{...}

%%För att själv bestämma marginalerna. 
\usepackage[
%            top    = 3cm,
%            bottom = 3cm,
%            left   = 3cm, right  = 3cm
]{geometry}


\swapcommands{\Pi}{\varPi}



\begin{document}

%%%%%%%%%%%%%%%%% vvv Inbyggd titelsida vvv %%%%%%%%%%%%%%%%%
% \begin{titlepage}
\title{Asymptotic Analasys and Pertubation Theory -- AMATH\,732 \\
Assignment 2}
\author{Andréas Sundström}
\date{\today}

\maketitle

%%%%%%%%%%%%%%%%% ^^^ Inbyggd titelsida ^^^ %%%%%%%%%%%%%%%%%

%Om man vill ha en lista med vilka todo:s som finns.
%\todolist

\section{Dimensional analysis from Lin\,\&\,Segel}
\setcounter{subsection}{4}
\renewcommand{\thesubsection}{L\&S 6.2: \arabic{subsection}}
\subsection{Ship propeller}
A ship popeller's thurst $T$ depends on it's radius $a$, the number of
revolutions per minute $n$, the velocity $V$, the acceleration of
gravity $g$, the water's density $\rho$ and kinematic viscosity $\nu$.
We're tasked to use formal dimensinal analysis to show that
\begin{equation}
\frac{T}{\rho a^2 V^2} = 
\varphi\qty(\frac{an}{V}, \frac{aV}{\nu}, \frac{ag}{V^2}).
\end{equation}

To begin with, we list all the parameters' dimensions:
\begin{equation}
\begin{aligned}
[T]=[LMT^{-2}]\qcomma [a]=[L]&\qcomma [n]=[T^{-1}]\qcomma [V]=[LT^{-1}]\\
[g]=[LT^{-2}]\qcomma [\nu]=&[L^2T^{-1}]\qcomma  [\rho]=[ML^{-3}].
\end{aligned}
\end{equation}
With the dimensions of all the parameters listed we can begin
creating dimension less parameters
\begin{equation}
\Pi_i = a^{A_i}n^{B_i} V^{C_i} g^{D_i} \nu^{E_i}\rho^{F_i} T^{G_i} \qcomma
[\Pi_i] = [1].
\end{equation}
This is no more than a linear system of equations, one equation for
each fundamental dimension,
\begin{equation}\label{eq:1_dim_matrix}
\begin{pmatrix}
0&0&0&0&0&1&1\\
1&0&1&1&2&-3&1\\
0&-1&-1&-2&-1&0&-2
\end{pmatrix}
\begin{pmatrix}
A_i\\B_i\\C_i\\D_i\\E_i\\F_i\\G_i
\end{pmatrix}
=\begin{pmatrix}
0\\0\\0
\end{pmatrix}.
\end{equation}
Here the rows represents the dimensions of mass, length and time respectivley.
We have three equations and seven unknowns, so we have four free
parameters. 

Since we were given a set relation to prove it's easy to confirm that
the relation satisfies \eqref{eq:1_dim_matrix}, but if we were to find
a relationship on our own we would now need to choose which of the
parameters to be free. A good choice would be to let $G_i=\Gamma_i$ be free and
only have the value $1$ once and the rest of the time be $0$. The
other free parameters can be chosen more freely; in lack of a beter
choice, pick $A_i=\alpha_i$, $B_i=\beta_i$ and $C_i=\gamma_i$ to be
the free parameters. 
This gives us the equations
\begin{equation}
\begin{cases}
F_i=-\Gamma_i\\
D_i+2E_i-3F_i=-\alpha_i-\gamma_i-\Gamma_i\\
2D_i+E_i=-\beta_i-\gamma_i-2\Gamma_i,
\end{cases}
\Longleftrightarrow
\begin{cases}
F_i=-\Gamma_i\\
D_i+2E_i=-\alpha_i-\gamma_i-4\Gamma_i\\
2D_i+E_i=-\beta_i-\gamma_i-2\Gamma_i.
\end{cases}
\end{equation}

From here it's just a matter of choosing values for the free
parameters. One such choice would be $\Gamma_i=(1, 0, 0, 0)$, 
$\alpha_i=(-2, 1, 1, 1)$, $\beta_i=(0, 1, 0, 0)$ and
$\gamma_i=(-2, -1, 1, -2)$, giving us $D_i=(0, 0, 0, 1)$, 
$E_i=(0, 0, -1, 0)$ and $F_i=-\Gamma_i$.
The set of Buckingham $\Pi$
parameters now becomes
\begin{equation}
\Pi_1=\frac{T}{\rho a^2V^2}\qcomma \Pi_2=\frac{an}{V}\qcomma
\Pi_3=\frac{aV}{\nu}\qcomma \Pi_4=\frac{ag}{V^2}.
\end{equation}
\qed

\subsection{Diffusion and Brownian motion}
Here we're supposed to find that
\begin{equation}
D\propto\frac{k\mathcal{T}}{a\mu},
\end{equation}
where $D$ is the diffusion constant, $k$ is the Boltzmann constant,
$\mathcal{T}$ is the temperature, $a$ the radius of the paricles, and $\mu$ the
viscosity. The dimensions are
\begin{equation}
[D]=[L^2T^{-1}]\qcomma [k]=[ML^2T^{-2}\Theta^{-1}]\qcomma 
[\mathcal{T}]=[\Theta]\qcomma 
[a]=[L]\qcomma [\mu]=[ML^{-1}T^{-1}].
\end{equation}

Instead of doing the formal approach, as above, it's some times easier
to go ahead and eliminate one dimension after the other without
having to first set up all the equations. For instance we must have the
combination $k\mathcal{T}$ to eliminate the temperature dimension; then mass can
be eliminated by dividing with $\mu$. By now, only the time and length
dimensions are left, and the only parameters left are $D$ and $a$. 

To eliminate time, we see that
\begin{equation}
\qty[\frac{k\mathcal{T}}{\mu}] = \qty[L^3T^{-1}],
\end{equation}
which means that dividing by $D$ would eliminate the time
dimension. And finally we need to divinde by $a$ to get rid of the
length dimension, leaving us with the only Buckingham $\Pi$ parameter
\begin{equation}
\Pi=\frac{k\mathcal{T}}{Da\mu} = ``\varphi(1)\text{''}.
\end{equation}
Or in other words
\begin{equation}
D=\text{const.}\times\frac{k\mathcal{T}}{a\mu}.
\end{equation}
\qed



\section{Scaling from Lin\,\&\,Segel}
\setcounter{subsection}{1}
\renewcommand{\thesubsection}{L\&S 6.3: \arabic{subsection}}
\subsection{Scaling of an exponential function}
Here we're looking at the scale of the function
\begin{equation}
u^*(x^*)=A\ee^{-ax^*}\qcomma x\in I=[b, \infty)
\end{equation}
satisfying some first order ODE. The scaled quantities are $u=u^*/U$
and $x=x^*/L$. 

By following procedures to ensure that the scaled quantities shoud be
$\mathcal{O}_M(1)$, we get
\begin{equation}
U=\max_I\abs{u^*}=A\ee^{-ab},
\end{equation}
and 
\begin{equation}
\frac{U}{L}=\max_I\abs{\dv{u^*}{x^*}}=aA\ee^{-ab} \Longrightarrow
L=\frac{1}{a}.
\end{equation}

We see that the lengthscale $L$ is independent of $b$. This is because
the lengthscale should reflect the change in $x^*$ needed to change
$u^*$ by aome significant amount. And since the function is
exponential, $u^*$ will change by a factor of $\ee^{-1}$ for every
step of $\nicefrac{1}{a}$ that $x^*$ takes, regardless of where $x^*$
had it's origin. 

\subsection{Scaling with multiple independent variables}
Here we have some first order PDE with two indepedent variables $x^*$
and $y^*$, and one dependent variable $u^*$. The PDE governs the
behaviour in some domain $\mathcal{D}$. The scaled terms are
$u=u^*/U$, $x=x^*/L_x$ and $y=y^*/L_y$.

The scaling should stil be so that the scaled quantities are
$\mathcal{O}_M(1)$. Wherefore we still get
\begin{equation}
U=\max_\mathcal{D}\abs{u^*},
\end{equation}
and
\begin{equation}
\frac{U}{L_x}=\max_\mathcal{D}\abs{\pdv{u^*}{x^*}}\qcomma
\frac{U}{L_y}=\max_\mathcal{D}\abs{\pdv{u^*}{y^*}}.
\end{equation}


\subsection{Scaling with multiple dependent variables}
This is basically the same as above, but with two dependent variables
$u^*$ and $v^*$. With the same condition thet the scaled parameters
should be $\mathcal{O}_M(1)$, we get
\begin{equation}
U=\max_\mathcal{D}\abs{u^*} \qcomma
V=\max_\mathcal{D}\abs{v^*},
\end{equation}
and 
\begin{equation}
L_x=\max\qty{
U\qty(\max_\mathcal{D}\abs{\pdv{u^*}{x^*}})^{-1},\;
V\qty(\max_\mathcal{D}\abs{\pdv{v^*}{x^*}})^{-1}
}
\end{equation}
and $L_y$ is analogous to $L_x$.





\renewcommand{\thesubsection}{\arabic{section} (\alph{subsection})}
\renewcommand{\thesubsubsection}{\arabic{section} (\alph{subsection},\,\roman{subsubsection})}

\section{The Pythagorean theorem from dimensinal analysis}


\section{The projectile problem with air resitance}


\section{A piano wire}


\section{Orbital stability }


\section{Some asymtotic sequences and expansions}


\section{The exponential integral}


\section{More of the exponential integral}







\end{document}





%% På svenska ska citattecknet vara samma i både början och slut.
%% Använd två apostrofer (två enkelfjongar): ''.


%% Inkludera PDF-dokument
\includepdf[pages={1-}]{filnamn.pdf} %Filnamnet får INTE innehålla 'mellanslag'!

%% Figurer inkluderade som pdf-filer
\begin{figure}\centering
\centerline{ %centrerar även större bilder
\includegraphics[width=1\textwidth]{filnamn.pdf}
}
\caption{}
\label{fig:}
\end{figure}

%% Figurer inkluderade med xfigs "Combined PDF/LaTeX"
\begin{figure}\centering
\resizebox{.8\textwidth}{!}{\input{filnamn.pdf_t}}
\caption{}
\label{fig:}
\end{figure}

%% Figurer roterade 90 grader
\begin{sidewaysfigure}\centering
\centerline{ %centrerar även större bilder
\includegraphics[width=1\textwidth]{filnamn.pdf}
}
\caption{}
\label{fig:}
\end{sidewaysfigure}


%%Om man vill lägga till något i TOC
\stepcounter{section} %Till exempel en 'section'
\addcontentsline{toc}{section}{\Alph{section}\hspace{8 pt}Labblogg} 

