\documentclass[11pt,letter, swedish, english
]{article}
\pdfoutput=1

\usepackage{../custom_as}

\swapcommands{\Omega}{\varOmega}

%%Drar in tabell och figurtexter
\usepackage[margin=10 pt]{caption}
%%För att lägga in 'att göra'-noteringar i texten
\usepackage{todonotes} %\todo{...}

%%För att själv bestämma marginalerna. 
\usepackage[
%            top    = 3cm,
%            bottom = 3cm,
%            left   = 3cm, right  = 3cm
]{geometry}

%%För att ändra hur rubrikerna ska formateras
\renewcommand{\thesubsection}{\arabic{section} (\alph{subsection})}

\renewcommand{\thesubsubsection}{\arabic{section} (\alph{subsection},\,\roman{subsubsection})}

% \newcommand{\cbox}[2][cyan]
% {\mathchoice
% 	{\setlength{\fboxsep}{0pt}\colorbox{#1}{$\displaystyle#2$}}
% 	{\setlength{\fboxsep}{0pt}\colorbox{#1}{$\textstyle#2$}}
% 	{\setlength{\fboxsep}{0pt}\colorbox{#1}{$\scriptstyle#2$}}
% 	{\setlength{\fboxsep}{0pt}\colorbox{#1}{$\scriptscriptstyle#2$}}
% }
% \newcommand{\grande}{\cbox{\phantom{\frac{1}{xx}}}}

\renewcommand{\thefootnote}{\fnsymbol{footnote}}

\begin{document}

%%%%%%%%%%%%%%%%% vvv Inbyggd titelsida vvv %%%%%%%%%%%%%%%%%
% \begin{titlepage}
\title{Statistical Physics -- PHYS\,704 \\
Assignment 2}
\author{Andréas Sundström}
\date{\today}

\maketitle

%%%%%%%%%%%%%%%%% ^^^ Inbyggd titelsida ^^^ %%%%%%%%%%%%%%%%%

%Om man vill ha en lista med vilka todo:s som finns.
%\todolist

\section{Compressibility and heatcapacity of ideal Fermi gases}
\subsection{Isothermal and adiabatic compressibility}
In this problem we want to show that
\begin{equation}\label{eq:1_kappa}
\kappa_T = k_T:=\frac{1}{nT}\frac{f_{1/2}(z)}{f_{3/2}(z)}
\quad\text{ and }\quad
\kappa_S = k_S:=\frac{3}{5nT}\frac{f_{3/2}(z)}{f_{5/2}(z)}.
\end{equation}


\subsubsection{Isothermal compressibility}
We begin by maniplulating the supposed expression for the isothermal
compressibility
\begin{equation}\label{eq:1_want_this_T}
k_T=\frac{1}{nT}\frac{f_{1/2}(z)}{f_{3/2}(z)}
= \frac{V}{NT}\frac{f_{1/2}(z)}{f_{3/2}(z)}
= \frac{1}{P}\frac{f_{1/2}(z)f_{5/2}(z)}{[f_{3/2}(z)]^2},
\end{equation}
where we have used the thermodynamic relation for Fermi gases
\begin{equation}\label{eq:1_PVNT}
\frac{PV}{NT}=\frac{f_{5/2}(z)}{f_{3/2}(z)}.
\end{equation}

To find the compressibility we just use the definition
\begin{equation}
\kappa_T:=-\frac{1}{V}\qty(\pdv{V}{P})_T
= -\frac{1}{V} \qty(\pdv{P} \frac{NT}{P}\frac{f_{5/2}(z)}{f_{3/2}(z)})_T,
\end{equation}
where we once again used \eqref{eq:1_PVNT} to express $V$ in the
derivative. Now, since $T$ is constant we get
\begin{equation}\label{eq:1_k_T_2}
\kappa_T
= -\frac{NT}{V} \qty[
\frac{-1}{P^2}\frac{f_{5/2}(z)}{f_{3/2}(z)}
+\frac{1}{P}\qty(\pdv{P}
\frac{f_{5/2}(z)}{f_{3/2}(z)}
)_T].
\end{equation}

Now we have to evauate the derivative of the special functions. To
begin with we note that
\begin{equation}
\qty(\pdv{f_\nu(z)}{\mu})_T=\qty(\pdv{z}{\mu})_T\pdv{f_\nu(z)}{z}
= \frac{1}{T}z\pdv{f_\nu(z)}{z}=\frac{1}{T}f_{\nu-1}(z).
\end{equation}
This follows from the fact that $z=\ee^{\mu/T}$ and the relation
\begin{equation}
z\pdv{f_\nu(z)}{z} = f_{\nu-1}(z).
\end{equation}
Next we have to evaluate
\begin{equation}
\qty(\pdv{\mu}{P})_T.
\end{equation}
To do this we remind ourselves that
\begin{equation}
PV=-\Omega=\frac{g_sVT}{\lambda^3}f_{5/2}(z)
\end{equation}
for Fermi gases. It's also a good idea to remind oursevles that
$\lambda$ only depends on $T$ and other constants, so in this case
$\lambda$ can be regardes as a constant. We now have
\begin{equation}
\qty(\pdv{P}{\mu})_T 
=\qty(\pdv{\mu} \frac{g_sT}{\lambda^3}f_{5/2}(z))_T 
=\frac{g_sT}{\lambda^3} \,\frac{1}{T}f_{3/2}(z) 
=\frac{g_s}{\lambda^3} f_{3/2}(z).
\end{equation}
The last piece needed is the fact that
\begin{equation}
n=\frac{N}{V}=-\frac{1}{V}\qty(\pdv{\Omega}{\mu})_T
=\frac{g_s}{\lambda^3} f_{3/2}(z)=\qty(\pdv{P}{\mu})_T.
\end{equation}

Finally we can continue from \eqref{eq:1_k_T_2} to get
\begin{equation}
\begin{aligned}
\kappa_T &= \frac{NT}{VP} \qty[
\frac{1}{P}\frac{f_{5/2}(z)}{f_{3/2}(z)}
-\qty(\pdv{\mu}{P})_T\qty(\pdv{\mu}\frac{f_{5/2}(z)}{f_{3/2}(z)})_T
]\\
&= \frac{nT}{P} \qty[
\frac{1}{P}\frac{f_{5/2}(z)}{f_{3/2}(z)}
-\frac{1}{n}\qty(\frac{1}{T}\frac{f_{3/2}(z)}{f_{3/2}(z)}
-\frac{1}{T}\frac{f_{5/2}(z)f_{1/2}(z)}{[f_{3/2}(z)]^2}
)]\\
&= \frac{nT}{P} \qty[
\frac{1}{P}\frac{f_{5/2}(z)}{f_{3/2}(z)}
-\frac{1}{nT}
+\frac{1}{nT}\frac{f_{5/2}(z)f_{1/2}(z)}{[f_{3/2}(z)]^2}
].
\end{aligned}
\end{equation}
From here we once agian use \eqref{eq:1_PVNT} to get
\begin{equation}
\frac{1}{P}=\frac{1}{nT}\frac{f_{3/2}(z)}{f_{5/2}(z)}.
\end{equation}
This gives
\begin{equation}
\kappa_T = \frac{nT}{P} \qty[
\frac{1}{nT}
-\frac{1}{nT}
+\frac{1}{nT}\frac{f_{5/2}(z)f_{1/2}(z)}{[f_{3/2}(z)]^2}
]
=\frac{1}{P}\frac{f_{5/2}(z)f_{1/2}(z)}{[f_{3/2}(z)]^2},
\end{equation}
which is exactly what we had in \eqref{eq:1_want_this_T}.
\qed


\subsubsection{Adiabatic compressibility}
As before, we begin with rewriting the RHS of the supposed expression
for the adiabatic compressibility
\begin{equation}\label{eq:1_want_this_S}
k_S=\frac{3}{5nT}\frac{f_{3/2}(z)}{f_{5/2}(z)}
=\frac{3}{5P}\frac{f_{5/2}(z)}{f_{3/2}(z)}\frac{f_{3/2}(z)}{f_{5/2}(z)}
=\frac{3}{5P},
\end{equation}
where we once again used \eqref{eq:1_PVNT} to rewite $nT$. Next we
need to remind ourselves about some more important relations:
\begin{equation}\label{eq:dE}
\rd{E}=T\rd{S}-P\rd{V}+\mu\rd{N}
\end{equation}
and
\begin{equation}\label{eq:1_Omega}
PV=-\Omega=\frac{2}{3}E.
\end{equation}
Of these relations only the last equality (relating $\Omega$ to $E$)
is spesific for Fermi gases, the rest are general thermodynaic
identities. 

From now we can focus more on the compressibility
\begin{equation}
\kappa_S=-\frac{1}{V}\qty(\pdv{V}{P})_S
=-\frac{1}{V\qty(\pdv{P}{V})_S}
\end{equation}
From the last expression we note that \eqref{eq:1_Omega} allows us to
write
\begin{equation}
\qty(\pdv{P}{V})_S=\frac{2}{3}\qty(\pdv{(\nicefrac{E}{V})}{V})_S
=\frac{2}{3V}\qty(\pdv{E}{V})_S-\frac{2E}{3}\frac{1}{V^2}.
\end{equation}
Now by invoking \eqref{eq:dE} we see that
\begin{equation}
\qty(\pdv{E}{V})_S=-P,
\end{equation}
we have also assumed $N$ to be constant which is resonable. And by
using \eqref{eq:1_Omega} once more to rewite $E=2PV/3$ we get
\begin{equation}
\qty(\pdv{P}{V})_S=-\frac{2P}{3V}-\frac{PV}{V^2}=-\frac{5P}{3V}.
\end{equation}
This means that we now have
\begin{equation}
\kappa_S=-\frac{1}{V\qty(\pdv{P}{V})_S}=\frac{3}{5P},
\end{equation}
which is exactly what we were looking for in
\eqref{eq:1_want_this_S}.\footnotemark{}
\qed

\footnotetext{This was a very neat way to derive $\kappa_S$. One could
  wonder if this approach is applicable to $\kappa_T$ as well. But
  unfortunatly the analogous approach yields an expression wich
  contains $(\pdv*{\mu}{V})_T$, which is not redily evaluated.} 


\subsubsection{Low temperature limit}
% Here we're going to show that 
% \begin{equation}\label{eq:1_lowT_T}
% \kappa_T = \frac{1}{nT}\frac{f_{1/2}(z)}{f_{3/2}(z)}
% \approx \frac{3}{2n\epsilon_F}
% \qty[1-\frac{\pi^2}{12}\qty(\frac{T}{\epsilon_F})^2]
% \end{equation}
% and
% \begin{equation}\label{eq:1_lowT_S}
% \kappa_T = \frac{3}{5nT}\frac{f_{3/2}(z)}{f_{5/2}(z)}
% \approx \frac{3}{2n\epsilon_F}
% \qty[1-\frac{5\pi^2}{12}\qty(\frac{T}{\epsilon_F})^2]
% \end{equation}
% in the low temperature limit. 

Here we're going to find the low temperature limit expressions for
$\kappa_T$ and $\kappa_S$.
To be in the limit of low temperature means that
\begin{equation}
T\ll\epsilon_F\approx\mu 
\quad\Longrightarrow\quad
z=\ee^{\mu/T}\approx \ee^{\epsilon_F/T}\gg1.
\end{equation}
We therefore need to expand $f_\nu(z)$ using the Sommerfeld
expansion\footnotemark{} 
\begin{equation}\label{eq:1_Sommerfeld}
\begin{aligned}
f_\nu(z)=&\frac{\ln^\nu(z)}{\Gamma(\nu+1)}\qty[
1+\nu(\nu{-}1)\frac{\pi^2}{6}\frac{1}{\ln^2(z)}
+\nu(\nu{-}1)(\nu{-}2)(\nu{-}3)\frac{7\pi^4}{360}\frac{1}{\ln^4(z)}
+\ldots]\\
=&\frac{\qty(\nicefrac{\mu}{T})^\nu}{\Gamma(\nu+1)}\qty[
1+\nu(\nu{-}1)\frac{\pi^2}{6}\qty(\frac{T}{\mu})^2
+\nu(\nu{-}1)(\nu{-}2)(\nu{-}3)\frac{7\pi^4}{360}\qty(\frac{T}{\mu})^4
+\ldots]
\end{aligned}
\end{equation}
This expansion can be found in for instance Pathria\,\&\,Beale,
\textit{Statistical Mechanics}, ed. 3.

\footnotetext{I'm including the $\ln^{-4}(z)$ term since it will be
  used in the next problem. Here however, we're only interested in the
  expansion up to $\ln^{-2}(z)$.}

Now we have to start working:
\begin{equation}\label{eq:1_1/2}
f_{1/2}(z)\approx\frac{1}{\Gamma(\nicefrac{3}{2})}\qty(\frac{\mu}{T})^{1/2}
\qty[
1-\frac{1}{4}\frac{\pi^2}{6}\qty(\frac{T}{\mu})^2]
=\frac{2}{\sqrt{\pi}}\qty(\frac{\mu}{T})^{1/2}
\qty[
1-\frac{\pi^2}{24}\qty(\frac{T}{\mu})^2],
\end{equation}
\begin{equation}\label{eq:1_3/2}
f_{3/2}(z)\approx\frac{1}{\Gamma(\nicefrac{5}{2})}\qty(\frac{\mu}{T})^{3/2}
\qty[
1+\frac{3}{4}\frac{\pi^2}{6}\qty(\frac{T}{\mu})^2]
=\frac{4}{3\sqrt{\pi}}\qty(\frac{\mu}{T})^{3/2}
\qty[1+\frac{\pi^2}{8}\qty(\frac{T}{\mu})^2],
\end{equation}
and 
\begin{equation}\label{eq:1_5/2}
f_{5/2}(z)\approx\frac{1}{\Gamma(\nicefrac{7}{2})}\qty(\frac{\mu}{T})^{5/2}
\qty[
1+\frac{15}{4}\frac{\pi^2}{6}\qty(\frac{T}{\mu})^2]
=\frac{8}{15\sqrt{\pi}}\qty(\frac{\mu}{T})^{5/2}
\qty[1+\frac{5\pi^2}{8}\qty(\frac{T}{\mu})^2].
\end{equation}
After this we also need to recognize that $\mu$ is dependent of
$T$, and $\mu$ is also only comeing in as $T/\mu$. It's therefor a
good idea to use \eqref{eq:2_mu} to expand
\begin{equation}\label{eq:1_T/mu}
\frac{T}{\mu}\approx T \frac{1}{\epsilon_F}\qty[ 1
-\frac{\pi^2}{12}\qty(\frac{T}{\epsilon_F})^2]^{-1}
\approx\tau\qty[1
+\frac{\pi^2}{12}\tau^2]
\end{equation}
where $\tau=T/\epsilon_F$. We only needed to expand to
$\order{\tau^2}$ since we're only going that far in the rest of the
expansions. 

Then we Taylor expand the quotients in \eqref{eq:1_kappa},
%{eq:1_lowT_T} and \eqref{eq:1_lowT_S}, 
which is possible since $T\ll\mu$. 
For isothermal compressibility this looks like
\begin{equation}
\begin{aligned}
\kappa_T\approx& \frac{1}{nT} 
\frac{2}{\sqrt{\pi}}\qty(\frac{\mu}{T})^{1/2}
\qty(\frac{4}{3\sqrt{\pi}}\qty(\frac{\mu}{T})^{3/2})^{-1}
\qty[
1-\frac{\pi^2}{24}\qty(\frac{T}{\mu})^2]
\qty[
1+\frac{\pi^2}{8}\qty(\frac{T}{\mu})^2]^{-1}\\
\approx&
 \frac{3}{2nT}\frac{T}{\mu}
\qty[
1-\frac{\pi^2}{24}\qty(\frac{T}{\mu})^2]
\qty[
1-\frac{\pi^2}{8}\qty(\frac{T}{\mu})^2]\\
\end{aligned}
\end{equation}
From here we use \eqref{eq:1_T/mu}. But note that only the leading
factor $T/\mu$ needs to be expanded, since we're only interesed in
expanding to $\order{\tau^2}$ in each bracket. Thus
\begin{equation}
\begin{aligned}
\kappa_T\approx& \frac{3}{2nT}
\tau\qty[1 +\frac{\pi^2}{12}\tau^2]
\qty[1 -\frac{\pi^2}{24}\tau^2]
\qty[1 -\frac{\pi^2}{8}\tau^2]\\
\approx& \frac{3}{2nT}
\frac{T}{\epsilon_F}\qty[1 +
\qty(\frac{\pi^2}{12}-\frac{\pi^2}{24}-\frac{\pi^2}{8})
\tau^2] 
&=\frac{3}{2n\epsilon_F}
\qty[1 - \frac{\pi^2}{12}\qty(\frac{T}{\epsilon_F})^2]
\end{aligned}
\end{equation}

The adiabatic compressibility is calculated in an analogous way:
\begin{equation}
\begin{aligned}
\kappa_S\approx& \frac{3}{5nT} 
\frac{4}{3\sqrt{\pi}}\qty(\frac{\mu}{T})^{3/2}
\qty(\frac{8}{15\sqrt{\pi}}\qty(\frac{\mu}{T})^{5/2})^{-1}
\qty[
1+\frac{\pi^2}{8}\qty(\frac{T}{\mu})^2]
\qty[
1+\frac{5\pi^2}{8}\qty(\frac{T}{\mu})^2]^{-1}\\
\approx&
 \frac{3}{5nT}\frac{5}{2}\tau\qty[1 +\frac{\pi^2}{12}\tau^2]
\qty[1+\frac{\pi^2}{8}\tau^2]
\qty[1-\frac{5\pi^2}{8}\tau^2]\\
\approx&\frac{3}{2n\epsilon_F}
\qty[1 - \frac{5\pi^2}{12}\qty(\frac{T}{\epsilon_F})^2].
\end{aligned}
\end{equation}
And that's the two expressions for the compressibilities in the low
temperature limit.
\qed

\subsection{Relationship between isobaric and isochoric heatcapacity}

\subsubsection{General thermodynamic relationship between $C_P$ and $C_V$}
In general we have
\begin{equation}
C_V=T\qty(\pdv{S}{T})_V\qcomma\text{and}\quad
C_P=T\qty(\pdv{S}{T})_P.
\end{equation}

To evaluate these in a bit more detail we begin by studying
$\rd{S}$. By the State postulate any property of the system can be
described by two independent variables (assuming no particle exchange
with the environment). We can therefore write
\begin{equation}
\rd{S}=\qty(\pdv{S}{T})_V\rd{T} + \qty(\pdv{S}{V})_T\rd{V}.
\end{equation}
But we are also free to rewite $\rd{V}$ in terms of $\rd{T}$ and
$\rd{P}$:
\begin{equation}
\rd{V}=\qty(\pdv{V}{T})_P\rd{T} + \qty(\pdv{V}{P})_T\rd{P}.
\end{equation}
With this $\rd{S}$ becomes
\begin{equation}
\begin{aligned}
\rd{S} =& \qty(\pdv{S}{T})_V\rd{T} + \qty(\pdv{S}{V})_T
\qty[\qty(\pdv{V}{T})_P\rd{T} + \qty(\pdv{V}{P})_T\rd{P}]\\
=&\qty[\qty(\pdv{S}{T})_V+\qty(\pdv{S}{V})_T\qty(\pdv{V}{T})_P]\rd{T}
+ \qty(\pdv{V}{P})_T\rd{P}.
\end{aligned}
\end{equation}
Now we can write 
\begin{equation}
C_P=T\qty(\pdv{S}{T})_P
=T\qty[\qty(\pdv{S}{T})_V+\qty(\pdv{S}{V})_T\qty(\pdv{V}{T})_P]
=C_V+T\qty(\pdv{S}{V})_T\qty(\pdv{V}{T})_P.
\end{equation}

To continue from here we need to use some Maxwell relations. There are
two of interest to us in the moment:
\begin{equation}
\qty(\pdv{S}{V})_T=-\pdv[2]{F}{T}{V}=\qty(\pdv{P}{T})_V
\end{equation}
and
\begin{equation}
-\qty(\pdv{S}{P})_T=\pdv[2]{\varPhi}{T}{P}=\qty(\pdv{V}{T})_P,
\end{equation}
where $F$ and $\varPhi$ are the Helmholzt and Gibbs free energy
respectively. 

With these Maxwell relations we can write
\begin{equation}
C_P-C_V=T\qty(\pdv{P}{T})_V\qty(\pdv{V}{T})_P
=TV\kappa_T\qty(\pdv{P}{T})_V^2.
\end{equation}
The last equality comes from
\begin{equation}
V\kappa_T\qty(\pdv{P}{T})_V=-\qty(\pdv{V}{P})_T\qty(\pdv{S}{V})_T
=-\qty(\pdv{S}{P})_T=\qty(\pdv{V}{T})_P.
\end{equation}

\subsubsection{In the specific cast of Fermi gases}
Here we want to study 
\begin{equation}\label{eq:1bi_1}
\frac{C_P-C_V}{C_V} 
= \frac{TV\kappa_T\qty(\pdv{P}{T})_V^2}{\qty(\pdv{E}{T})_V}
= \frac{TV\kappa_T\qty(\pdv{P}{E})_V^2\qty(\pdv{E}{T})_V^2}{\qty(\pdv{E}{T})_V}
= TV\kappa_TC_V\qty(\pdv{P}{E})_V^2.
\end{equation}
Here we used the fact that $C_V=(\pdv*{E}{T})_V$. To continue we
note that \eqref{eq:1_Omega} implies that
\begin{equation}
\qty(\pdv{E}{P})_V=\frac{3}{2}V.
\end{equation}
No we can write \eqref{eq:1bi_1} as
\begin{equation}
\frac{C_P-C_V}{C_V} 
= TV\kappa_TC_V\frac{4}{9V^2}
=\frac{4C_VT}{9V}\frac{1}{nT}\frac{f_{1/2}(z)}{f_{3/2}(z)}
=\frac{4C_V}{9N}\frac{f_{1/2}(z)}{f_{3/2}(z)}.
\end{equation}
Here we used (\ref{eq:1_kappa}\,a) to rewrite $\kappa_T$.

To get a first order approximation of the value of this in the low
temperature limit we will use \eqref{eq:1_1/2}, \eqref{eq:1_3/2} and
\eqref{eq:CV_Fermi}. To a first order approximation 
\begin{equation}
\frac{C_P-C_V}{C_V} =\frac{4C_V}{9N}\frac{f_{1/2}(z)}{f_{3/2}(z)} 
\approx \frac{4}{9} \frac{\pi^2}{2}\frac{T}{\epsilon_F}
\frac{2}{\sqrt{\pi}}\qty(\frac{\epsilon_F}{T})^{1/2}
\qty(\frac{4}{3\sqrt{\pi}}\qty(\frac{\epsilon_F}{T})^{3/2})^{-1}
=\frac{\pi^2}{3}\qty(\frac{T}{\epsilon_F})^2.
\end{equation}
\qed





\section{Cemical potential and particle energy of1 an ideal
  Fermi gas at low temperatures}
\renewcommand{\thesubsection}{\arabic{section} (\roman{subsection})}
\subsection{Chemical potential}
To calculate an approximation to $\mu$ at low temperatures
$\ln(z)=\mu/T\gg1$, we use the Sommerfeld expansion in
\eqref{eq:1_Sommerfeld}. 
Then to get an equation for $\ln(z)$ we can use
\begin{equation}
n=\frac{g_s}{\lambda^3}f_{3/2}(z)
\approx\frac{g_s}{\lambda^3}
\frac{4}{3\sqrt{\pi}}\ln^{3/2}(z)
\qty[1+\frac{\pi^2}{8}\ln^{-2}(z) 
+\frac{7\pi^4}{640}\ln^{-4}(z) ]
\end{equation}
or in other words
\begin{equation}
\zeta\approx\qty(\frac{3\sqrt{\pi}\lambda^3}{4g_s})^{2/3}
\qty[1+\frac{\pi^2}{8}\zeta^{-2}
+\frac{7\pi^4}{640}\zeta^{-4} ]^{-2/3},
\end{equation}
where $\zeta=\ln(z)=\mu/T$.
We're going to deal with these two factors separetly.
First off is the easy bit. The first factor is
\begin{equation}
\qty(\frac{3\sqrt{\pi}\lambda^3}{4g_s})^{2/3}
=\lambda^2\qty(\frac{3\sqrt{\pi}}{4g_s})^{2/3}
=\frac{1}{T}\frac{\hbar^2}{2m}4\pi\qty(\frac{3\sqrt{\pi}}{4g_s})^{2/3}
=\frac{1}{T}\frac{\hbar^2}{2m}\qty(\frac{6\pi^2}{g_s})^{2/3}
=\frac{\epsilon_F}{T}=\frac{1}{\tau}.
\end{equation}
Then there's the othe factor which has to be Taylor expanded to the
second order in $\ln^{-2}(z)\ll1$. To do this we enlist some help in
form of \emph{Mathematica}\footnotemark{}, which yields
\begin{equation}
\qty[1 + \frac{\pi^2}{8}\zeta^{-2}
 + \frac{7\pi^4}{640}\zeta^{-4}]^{-2/3} 
=\qty[1 - \frac{\pi^2}{12}\zeta^{-2}
 + \frac{\pi^4}{720}\zeta^{-4} + \order{\zeta^{-6}}].
\end{equation}
We now have every thing we need to start approximating $\zeta$.
Unfortunately what we have is something in the line of a fifth degree
polynolial in $\zeta$ to solve: 
\begin{equation}\label{eq:2_fixed-point}
\zeta\approx \tau^{-1}\qty[1 - \frac{\pi^2}{12}\zeta^{-2}
 + \frac{\pi^4}{720}\zeta^{-4}].
\end{equation}
This can not and should not be solved
analytically. Instead we're going to use a method similar to
fixed-point iteration, but for analytical expressions.
\footnotetext{Lord and Saviour in our times of need and despair!}

The first iteration is just the zeroth power approximation:
\begin{equation}
\zeta_0=\tau^{-1}.
\end{equation}
Next iteration is using the fixed-point function
\eqref{eq:2_fixed-point} on $\zeta_0$ to get
\begin{equation}
\zeta_1=\tau^{-1}\qty[1 - \frac{\pi^2}{12}\zeta_0^{-2}
 + \frac{\pi^4}{720}\zeta_0^{-4} ]
=\tau^{-1}\qty[1 - \frac{\pi^2}{12}\tau^{2}
 + \frac{\pi^4}{720}\tau^{4} ].
\end{equation}
One more iteration:
\begin{equation}
\begin{aligned}
\zeta_2=&\tau^{-1}\qty[1 - \frac{\pi^2}{12}\zeta_1^{-2}
 + \frac{\pi^4}{720}\zeta_1^{-4} ]\\
=& \tau^{-1}\qty[1 
 - \frac{\pi^2}{12}\tau^{2}
\qty(1 - \frac{\pi^2}{12}\tau^{2}
 + \frac{\pi^4}{720}\tau^{4})^{-2}
 + \frac{\pi^4}{720}\tau^{4} 
\qty(1 - \frac{\pi^2}{12}\tau^{2}
 + \frac{\pi^4}{720}\tau^{4})^{-4}
].
\end{aligned}
\end{equation}
We once again turn our faith to \emph{Mathematica} and get
\begin{equation}
\zeta_2=\tau^{-1}\qty[1 - \frac{\pi^2}{12}\tau^{2}
 - \frac{\pi^4}{80}\tau^{4} + \order{\tau^6}].
\end{equation}
From here it's no use in continuing. Since our fixed-point function
only had terms up to $\order{\zeta^{-4}}$, we can not get any higher
orders than $\tau^4$.

We can therefore conclude that
\begin{equation}\label{eq:2_mu}
\mu=T\ln(z)\approx T\zeta_2= \epsilon_F \qty[1
-\frac{\pi^2}{12}\qty(\frac{T}{\epsilon_F})^2
-\frac{\pi^4}{80}\qty(\frac{T}{\epsilon_F})^4].
\end{equation}
\qed

\subsection{Mean particle energy}
To calculate an approximation to the mean particle energy $E/N$, we
once again have to use the Sommerfeld expansion
\eqref{eq:1_Sommerfeld}. Then we use the relation
\begin{equation}
\frac{E}{N}=-\frac{3\Omega}{2N}=\frac{3T}{2}\frac{f_{5/2}(z)}{f_{3/2}(z)}.
\end{equation}
Fortunately we've already calculated $\ln(z)=\zeta\approx\zeta_2$, so all
that left do here is to expand the Fermi functions and then
Taylor expand the quotient.

Let's begin:
\begin{equation}
%\begin{aligned}
\frac{f_{5/2}(z)}{f_{3/2}(z)}\approx
\overbrace{\frac{8\zeta_2^{5/2}}{15\sqrt{\pi}}
\qty(\frac{4\zeta_2^{3/2}}{3\sqrt{\pi}})^{-1}}^{\nicefrac{2\zeta_2}{5}}
\qty[1+\frac{5\pi^2}{8}\zeta_2^{-2}
-\frac{7\pi^4}{384}\zeta_2^{-4} ]
\times \qty[1+\frac{\pi^2}{8}\zeta_2^{-2} 
+\frac{7\pi^4}{640}\zeta_2^{-4} ]^{-1}
%\end{aligned}
\end{equation}
There's no need to even pretend to want deal with this by hand.
Just go directly to \emph{Mathematica}, which yields
\begin{equation}
\frac{f_{5/2}(z)}{f_{3/2}(z)}\approx \frac{2}{5\tau}
\qty[1+\frac{5\pi^2}{12}\tau^2-\frac{\pi^4}{16}\tau^4].
\end{equation}
For reference tha comman in use were \texttt{Series}.

Now we have
\begin{equation}
\frac{E}{N}\approx\frac{3T}{2}\frac{2}{5\tau}
\qty[1+\frac{5\pi^2}{12}\tau^2-\frac{\pi^4}{16}\tau^4]
=\frac{3\epsilon_F}{5}\qty[1
+\frac{5\pi^2}{12}\qty(\frac{T}{\epsilon_F})^2
-\frac{\pi^4}{16}\qty(\frac{T}{\epsilon_F})^4].
\end{equation}
This can be used to get an expression for the heatcapacity of the
system
\begin{equation}\label{eq:CV_Fermi}
\frac{C_V}{N}=\frac{1}{N}\qty(\pdv{E}{T})_V
=\;\frac{\pi^2}{2}\frac{T}{\epsilon_F}
 - \frac{3\pi^2}{20}\qty(\frac{T}{\epsilon_F})^3.
\end{equation}
\qed



\section{Speed of sound in an idela Bose gas}
\renewcommand{\thesubsection}{\arabic{section} (\alph{subsection})}


\section{Regarding ideal Bose gases }






\end{document}





%% På svenska ska citattecknet vara samma i både början och slut.
%% Använd två apostrofer (två enkelfjongar): ''.


%% Inkludera PDF-dokument
\includepdf[pages={1-}]{filnamn.pdf} %Filnamnet får INTE innehålla 'mellanslag'!

%% Figurer inkluderade som pdf-filer
\begin{figure}\centering
\centerline{ %centrerar även större bilder
\includegraphics[width=1\textwidth]{filnamn.pdf}
}
\caption{}
\label{fig:}
\end{figure}

%% Figurer inkluderade med xfigs "Combined PDF/LaTeX"
\begin{figure}\centering
\resizebox{.8\textwidth}{!}{\input{filnamn.pdf_t}}
\caption{}
\label{fig:}
\end{figure}

%% Figurer roterade 90 grader
\begin{sidewaysfigure}\centering
\centerline{ %centrerar även större bilder
\includegraphics[width=1\textwidth]{filnamn.pdf}
}
\caption{}
\label{fig:}
\end{sidewaysfigure}


%%Om man vill lägga till något i TOC
\stepcounter{section} %Till exempel en 'section'
\addcontentsline{toc}{section}{\Alph{section}\hspace{8 pt}Labblogg} 

