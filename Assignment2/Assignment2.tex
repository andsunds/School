\documentclass[11pt,letter, swedish, english
]{article}
\pdfoutput=1

\usepackage{../custom_as}
\usepackage[makeroom
]{cancel}
\graphicspath{{figures/}}

\swapcommands{\Delta}{\varDelta}
\swapcommands{\Omega}{\varOmega}

%%Drar in tabell och figurtexter
\usepackage[margin=10 pt]{caption}
%%För att lägga in 'att göra'-noteringar i texten
\usepackage{todonotes} %\todo{...}

%%För att själv bestämma marginalerna. 
\usepackage[
%            top    = 2.5cm,
%            bottom = 3cm,
%            left   = 3cm, right  = 3cm
]{geometry}

%%För att ändra hur rubrikerna ska formateras
%\renewcommand{\thesubsection}{\arabic{section} (\roman{subsection})}
\renewcommand{\thesubsection}{\arabic{section} (\alph{subsection})}
\renewcommand{\thesubsubsection}{\arabic{section} (\alph{subsection},\,\roman{subsubsection})}

%\renewcommand{\thefootnote}{\fnsymbol{footnote}}

\newcommand{\Tc}{\ensuremath{T_{\text{c}}}}
\newcommand{\sign}{\ensuremath{\text{sign}}}

%\usepackage{tikz}

\begin{document}

%\tikzstyle{every picture}+=[remember picture]
%\tikzstyle{na} = [shape=rectangle,inner sep=0pt,text depth=0pt]



%%%%%%%%%%%%%%%%% vvv Inbyggd titelsida vvv %%%%%%%%%%%%%%%%%

\title{Statistical Physics 2 -- PHYS\,705 \\
Assignment 2}
\author{Andréas Sundström}
\date{\today}

\maketitle

%%%%%%%%%%%%%%%%% ^^^ Inbyggd titelsida ^^^ %%%%%%%%%%%%%%%%%

\section{The Ornstein-Zernike formula}
In this problem we are concerned about the Ornstein-Zernike formula 
$G(\vb*r)=r^{-(d-1)/2}\ee^{-r/\xi}$. We will start from the real space
respresentation 
\begin{equation}\label{eq:1_start}
G(\vb*r) = \frac{T}{2}\int\!\rd^dq\,\frac{\ee^{\ii\vb*q\vdot\vb*r}}{kt+bq^2},
\end{equation}
where $q^2=\vb*q\vdot\vb*q$, $t=(T-\Tc)/\Tc$, $k$ and $b$ are some
constants, and $r\gg\xi:=\sqrt{b/kt}$.

\subsection*{1~(a+b)\quad Doing the integrals}\addtocounter{subsection}{2}
Using 
\begin{equation}\label{eq:1_1/c}
\int_0^\infty\rd{u}\,\ee^{-cu}=\frac{1}{c}
\qcomma a>0,
\end{equation}
we can rewrite the target integral as
\begin{equation}
g(\vb*r)=\int_{-\infty}^\infty\!\rd^dq\,
\frac{\ee^{\ii\vb*q\vdot\vb*r}}{kt+bq^2}
\stackrel{\eqref{eq:1_1/c}}{=}
\int_{-\infty}^\infty\!\rd^dq\, \ee^{\ii\vb*q\vdot\vb*r}
\int_0^\infty\!\rd{u}\,\ee^{-u\,(kt+bq^2)}.
\end{equation}
Now, since this is physics\footnotemark{}, we can swap the order of
integration; and while we're at it we might as well change variables
to $Q=\sqrt{b/kt}\,q=\xi q$ and $U=ktu$. We thereby get
\begin{equation}
g(\vb*r)=\int_0^\infty\!\frac{\rd{U}}{kt}\,\ee^{-U}
\prod_i\int_{-\infty}^\infty\!\frac{\rd{Q}_i}{\xi}\,
 \exp{-UQ_i^2+\frac{\ii r_i}{\xi}Q_i}.
\end{equation}
Each of these $Q_i$ integrals are just regular Gaussian so
\begin{equation}
\int_{-\infty}^\infty\!\rd{Q}_i\,
 \exp{-UQ_i^2+\frac{\ii r_i}{\xi}Q_i}
=\qty(\frac{\pi}{U})^{1/2}\exp{-\frac{r_i^2}{4\xi^2U}}.
\end{equation}
and we and up with
\begin{equation}\label{eq:1b_end}
g(\vb*r)=\frac{\pi^{d/2}}{\xi^d kt}\int_0^\infty\!\rd{U}\,
U^{d/2}\exp{-U-\frac{r^2}{4\xi^2U}}.
\end{equation}

\footnotetext{I.e. we don't care, but in reality the functions are
  nice enough, so we can do it.} 


\subsection{Asymptotic behaviour}
To get to the asymptotic behaviour of $g$, we enlist the help of
\textit{Mathematica} which helps us to calculate the last remaining
integral \eqref{eq:1b_end}. And we get
\begin{equation}
g(\vb*r)=\qty(\frac{\pi}{2})^{d/2}\frac{1}{\xi^dkt} 
\qty(\frac{\xi}{r})^{-1-\frac{d}{2}} 
K_{-1-\frac{d}{2}}\qty(\frac{r}{\xi}),
\end{equation}
where $K_\mu$ is the modified Bessel function of the second kind. Next
we use the asymptotic formula 
\begin{equation}
K_\mu(z)\sim \ee^{-z}\sqrt{\frac{\pi}{2z}}
\end{equation}
to get
\begin{equation}
g(\vb*r)=\qty(\frac{\pi}{2})^{d/2}\frac{1}{\xi^dkt} 
\qty(\frac{\xi}{r})^{-1-\frac{d}{2}} 
\ee^{-r/\xi}\qty(\frac{\pi}{2})^{1/2}\qty(\frac{\xi}{r})^{1/2}
\propto
%=\qty(\frac{\pi}{2})^{(d+1)/2}\frac{1}{\xi^{(3d+1)/2}kt}\;
r^{(1-d)/2}\ee^{-r/\xi},
\end{equation}
which was what we set out to show. 



\section{MFT correlation length in the Ising model}
From class, we derived an expression for the, momentum space,
correlation function at $T>\Tc$:
\begin{equation}
G_{T>\Tc}(\vb*q)=\frac{1}{t+\frac{\kappa}{2T}q^2},
\end{equation}
for small $q$, where $t=(T-\Tc)/\Tc$ and $\kappa$ is a given
constant. From this expression we got the correlation length
\begin{equation}
\xi(t>0)=\qty[\frac{\kappa}{2Tt}]^{1/2}
=\qty[\frac{\kappa}{2(T-\Tc)}]^{1/2}.
\end{equation}
We got to this correlation length fisrst after we've done
the inverse F.T. but if we can get the momentum space correlation
function, for $t<0$, on the form
\begin{equation}
G_{T<\Tc}(\vb*q)\propto\qty[1+\xi^2q^2]^{-1},
\end{equation}
we don't have to redo the inverse F.T.

We begin with the MFT equation for the correlation function
\begin{equation}
G_{ij}=(1-M^2)\sum_l \qty[\frac{1}{T}J_{il}G_{lj}+\delta_{ij}]
\quad\Longleftrightarrow\quad
\sum_l\qty(\frac{\delta_{il}}{1-M^2}-\frac{1}{T}J_{il})G_{lj}
=\delta_{ij},
\end{equation}
where the MFT value for $M^2$ is $M^2=3|t|=-3t$.
Next step is to F.T. this using the transformation
\begin{equation}
A_{ij}=\frac{1}{N}\sum_{\vb*q} A(\vb*q)\ee^{\ii\vb*q\vdot(\vb*r_i-\vb*r_j)}.
\end{equation}
And we get
\begin{equation}\label{eq:2_MFT_FT}
\frac{1}{N}\sum_{\vb*q} \frac{G(\vb*q)}{1-M^2}
\ee^{\ii\vb*q\vdot(\vb*r_i-\vb*r_j)}
-\frac{1}{T}\sum_l\frac{1}{n^2}\sum_{\vb*q', \vb*q''}
J(\vb*q')\ee^{\ii\vb*q'\vdot(\vb*r_i-\vb*r_l)}
G(\vb*q'')\ee^{\ii\vb*q''\vdot(\vb*r_l-\vb*r_j)}
=\frac{1}{N}\sum_{\vb*q}\ee^{\ii\vb*q\vdot(\vb*r_l-\vb*r_j)}.
\end{equation}
From here, we use 
\begin{equation}
\frac{1}{N}\sum_n\ee^{-\ii \vb*r_n\vdot(\vb*q'-\vb*q'')}=\delta_{\vb*q', \vb*q''}
\end{equation}
to collapse the $l$ sum and one of the $\vb*q$ sums in the second term
of \eqref{eq:2_MFT_FT}:
\begin{equation}
\frac{1}{N}\sum_{\vb*q} 
\qty(\frac{1}{1-M^2} -\frac{1}{T}J(\vb*q))G(\vb*q)
\ee^{\ii\vb*q\vdot(\vb*r_i-\vb*r_j)}
=\frac{1}{N}\sum_{\vb*q}\ee^{\ii\vb*q\vdot(\vb*r_l-\vb*r_j)}.
\end{equation}
This should be valid for any $\vb*r_i$ and $\vb*r_j$, so we can equate
the terms:
\begin{equation}
G(\vb*q)=\frac{1}{\frac{1}{1-M^2} -\frac{1}{T}J(\vb*q)}.
\end{equation}

Now we need the long-wavelength limit for $J$, which we derived in
class to be
\begin{equation}
J(\vb*q)=J-\frac{\kappa}{2}q^2,
\end{equation}
where 
\begin{equation}
\kappa:=\frac{1}{Nd}\sum_{ij}J_{ij}\abs{\vb*r_i-\vb*r_j}^2
\end{equation}
is just a constant. 

We now have all we need
\begin{equation}
\begin{aligned}
G(\vb*q)=&\qty[\frac{1}{1-M^2}-\frac{J}{T}+\frac{\kappa}{2T}q^2]^{-1}
&\stackrel{\text{MFT}}{=}&
\qty[\frac{1}{1+3t}-\frac{\Tc}{T}+\frac{\kappa}{2T}q^2]^{-1}\\
\stackrel{|t|\ll1}{\approx}&
\qty[(1-3t)-\frac{\Tc}{T}+\frac{\kappa}{2T}q^2]^{-1}
&=\hspace{5pt}&
\qty[-3t+\frac{T-\Tc}{T}+\frac{\kappa}{2T}q^2]^{-1}\\
\stackrel{T\approx\Tc}{\approx}&
\qty[-3t+t+\frac{\kappa}{2T}q^2]^{-1}
&\propto\hspace{5pt}&
\qty[1+\frac{\kappa}{T|t|}q^2]^{-1}.
\end{aligned}
\end{equation}
We therefore have
\begin{equation}
\xi(t<0)=\qty(\frac{\kappa}{T|t|})^{1/2}
\propto |t|^{-\nu}\qcomma \nu=\frac{1}{2}
\end{equation}
and the amplitude ratio is
\begin{equation}
\frac{k_+}{k_-}=\sqrt{2}.
\end{equation}





\section{Heisenberg ferromagnet}
In this problem we have a Heisenberg ferromagnet with Hamiltonian
\begin{equation}
H=-\frac{1}{2}\sum_{ij}J_{ij}\vb*S_i\vdot\vb*S_j
=-\frac{1}{2}\sum_{ij}J_{ij}S_i^xS_j^x
-\frac{1}{2}\sum_{ij}J_{ij}S_i^yS_j^y
-\frac{1}{2}\sum_{ij}J_{ij}S_i^zS_j^z
\end{equation}
and partition function
\begin{equation}\label{eq:3_Z1}
Z=\Tr{\ee^{-H/T}}=\int_{\vb*S_i^2=1}\!\prod_i\rd{\vb*S_i}\;\ee^{-H/T},
\end{equation}
where $\vb*S_i$ are spin unit vectors ($\vb*S_i\vdot\vb*S_i=1$).

\subsection{Functional integral representation}
To find a functional integral representation of the partition
function, we will need the Hubbard-Stratonovich transformation (HST)
\begin{equation}
\exp[\frac{1}{2}\sum_{ij}A_{ij}^{-1}\gamma_i\gamma_j]
=C\int\rd\varphi_1\cdots\rd\varphi_N 
\,\exp[-\frac{1}{2}\sum_{ij}\varphi_iA_{ij}\varphi_j+\sum_i\varphi_i\gamma_i],
\end{equation}
where $C$ is some unimportant numerical constant, that will be dropped
in future calculations here. 

In our case, we will use the HST to rewrite
\begin{equation}
\begin{aligned}
\ee^{-H/T}=&\exp[\frac{1}{2T}\sum_{ij}J_{ij}S_i^xS_j^x]
\exp[\frac{1}{2T}\sum_{ij}J_{ij}S_i^yS_j^y]
\exp[\frac{1}{2T}\sum_{ij}J_{ij}S_i^zS_j^z]\\
=&\int\rd\varphi_1^x\cdots\rd\varphi_N^x
\,\exp[-\frac{1}{2T}\sum_{ij}\varphi_i^xJ_{ij}^{-1}\varphi_j^x
+\frac{1}{T}\sum_i\varphi_i^xS_i^x]\\
&\times\int\rd\varphi_1^y\cdots\rd\varphi_N^y
\,\exp[-\frac{1}{2T}\sum_{ij}\varphi_i^yJ_{ij}^{-1}\varphi_j^y
+\frac{1}{T}\sum_i\varphi_i^yS_i^y]\\
&\times\int\rd\varphi_1^z\cdots\rd\varphi_N^z
\,\exp[-\frac{1}{2T}\sum_{ij}\varphi_i^zJ_{ij}^{-1}\varphi_j^z
+\frac{1}{T}\sum_i\varphi_i^zS_i^z].
\end{aligned}
\end{equation}
Then we can write everything as one exponential under one integration:
\begin{equation}
\ee^{-H/T}=\int\rd\vb*\varphi_1\cdots\rd\vb*\varphi_N
\,\exp[-\frac{1}{2T}\sum_{ij}J_{ij}^{-1}\vb*\varphi_i\vdot\vb*\varphi_j
+\frac{1}{T}\sum_i\vb*\varphi_i\vdot\vb*S_i].
\end{equation}

The next step, to get the partition function \eqref{eq:3_Z1}, is to integrate over all
possible $\vb*S_i$'s. To do this we take a closer look at \emph{one}
of those integrals: 
\begin{equation}
\int_{\vb*S_i^2=1}\rd\vb*S_i\ee^{\vb*\varphi_i\vdot\vb*S_i/T}
=\int_0^{2\pi}\rd\phi_i\int_0^{\pi}\rd\theta_i\,\sin(\theta_i)
\ee^{|\vb*\varphi_i|\,|\vb*S_i|\,\cos(\theta_i)/T}
=4\pi\frac{\sinh(\varphi_i)/T}{\varphi_i/T},
\end{equation}
where $\varphi_i:=|\vb*\varphi_i|$.
Here we have used the trick to align the $z$~axis of the integration
with $\vb*\varphi_i$, therefore making
$\vb*\varphi_i\vdot\vb*S_i=\varphi_i\cos\theta_i$ since $|\vb*S_i|=1$.
We therefore get the full partition function
\begin{equation}
\begin{aligned}
Z=&\int\rd\vb*\varphi_1\cdots\rd\vb*\varphi_N
\exp[-\frac{1}{2T}\sum_{ij}\vb*\varphi_iJ_{ij}^{-1}\vb*\varphi_j]
\int_{\vb*S_i^2=1}\prod_i\rd\vb*S_i\ee^{\vb*\varphi_i\vdot\vb*S_i/T}\\
=&(4\pi)^N\int\rd\vb*\varphi_1\cdots\rd\vb*\varphi_N
\exp[-\frac{1}{2T}\sum_{ij}\vb*\varphi_iJ_{ij}^{-1}\vb*\varphi_j
+\sum_i\ln(\frac{\sinh(\varphi_i/T)}{\varphi_i/T})].
\end{aligned}
\end{equation}
We can trow away the prefacor, of $(4\pi)^N$, just as we did with te
prefactors from the HST. The action then becomes
\begin{equation}
S[\vb*\varphi]=\frac{1}{2T}\sum_{ij}J_{ij}^{-1}\vb*\varphi_i\vdot\vb*\varphi_j
-\sum_i\ln(\frac{\sinh(\varphi_i/T)}{\varphi_i/T}).
\end{equation}


\subsection{Continuum limit}
To expand $S[\vb*\varphi]$ in $\vb*\varphi$, all we need to do is to
expand the last term
\begin{equation}
\ln(\frac{\sinh(x)}{x})=\frac{x^2}{6}-\frac{x^4}{180}+\order{x^6}.
\end{equation}
We therefore get
\begin{equation}
\begin{aligned}
S[\vb*\varphi]=&\frac{1}{2T}\sum_{ij}J^{-1}_{ij}\vb*\varphi_i\vdot\vb*\varphi_j
-\frac{1}{6T^2}\sum_i\vb*\varphi_i\vdot\vb*\varphi_i
+\frac{1}{180T^4}\sum_i\big(\vb*\varphi_i\vdot\vb*\varphi_i\big)^2\\
=&\frac{1}{6T^2}\sum_{ij}\qty(3TJ^{-1}_{ij}-\delta_{ij})\vb*\varphi_i\vdot\vb*\varphi_j
+\frac{1}{180T^4}\sum_i\big(\vb*\varphi_i\vdot\vb*\varphi_i\big)^2.
\end{aligned}
\end{equation}


\subsection{Mean field critical exponents}





\end{document}




%  LocalWords:  MFT MF Ising
