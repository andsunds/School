\documentclass[11pt,letter, swedish, english
]{article}
\pdfoutput=1

\usepackage{../custom_as}

%\usepackage{listings} 
%\usepackage[framed,numbered,autolinebreaks,useliterate]{../mcode}
%\lstloadlanguages{matlab} 
%\lstset{language=matlab} 

\usepackage[makeroom]{cancel}
\graphicspath{{figures/}}


\swapcommands{\Omega}{\varOmega}

%%Drar in tabell och figurtexter
\usepackage[margin=10 pt]{caption}
%%För att lägga in 'att göra'-noteringar i texten
\usepackage{todonotes} %\todo{...}

%%För att själv bestämma marginalerna. 
\usepackage[
%            top    = 2.5cm,
%            bottom = 3cm,
%            left   = 3cm, right  = 3cm
]{geometry}

%%För att ändra hur rubrikerna ska formateras


%\renewcommand{\thefootnote}{\fnsymbol{footnote}}

\renewcommand{\thesubsection}{\arabic{section} (\alph{subsection})}
\renewcommand{\thesubsubsection}{\arabic{section} (\alph{subsection},\,\roman{subsubsection})}

\newcommand{\Dx}{\ensuremath{\Delta{x}}}
\newcommand{\Dt}{\ensuremath{\Delta{t}}}

\begin{document}




%%%%%%%%%%%%%%%%% vvv Inbyggd titelsida vvv %%%%%%%%%%%%%%%%%

\title{Numerical solutions to PDE's -- AMATH\,741 \\
Assignment 2}
\author{Andréas Sundström}
\date{\today}

\maketitle

%%%%%%%%%%%%%%%%% ^^^ Inbyggd titelsida ^^^ %%%%%%%%%%%%%%%%%


\section{Consistency of a scheme for the heat equaiton}
Here we have a scheme
\begin{equation}
U_j^{n+1}=\alpha U_j^n+\frac{1-\alpha}{2}\qty(U_{j+1}^n+U_{j-1}^n)
\end{equation}
for the heat equation $u_t=\sigma u_{xx}$, and we want to find a
criteria on $\alpha$ for this scheme to be consistent.

Consistensy means that the truncation error
\begin{equation}
\tau_j^n=\order{\Dt^m, \Dx^k}
\end{equation}
where $m>0$ and $k>1$. 
If we have a scheme of the form
\begin{equation}
V^{n+1}_j=\mathcal{Q}_jV^n,
\end{equation}
where $\mathcal{Q}_j$ is some FD scheme, for a PDE of the form
$u_t+\mathcal{D}_xu=0$, where $\mathcal{D}_x$ is some differential
operator in $x$, then the truncation error can be defined via
\begin{equation}
\Dt\,\tau_j^{n}=u_j^{n+1}-\mathcal{Q}_ju^n.
\end{equation}

In our case we have
\begin{equation}
\mathcal{Q}_jV^n=\alpha V_j^n+\frac{1-\alpha}{2}\qty(V_{j+1}^n+V_{j-1}^n),
\end{equation}
so the truncation error is found by Taylor expanding:
\begin{equation}
\begin{aligned}
\Dt\,\tau_j^n=&u_j^{n+1}
-\qty{\alpha u_j^n+\frac{1-\alpha}{2}\qty(u_{j+1}^n+u_{j-1}^n)}\\
=& \qty[u_j^n+(u_t)_j^n+\order{\Dt^2}]\\
&-\Bigg\{\alpha u_j^n+\frac{1-\alpha}{2}
\Bigg[\qty(u_j^n+(u_x)_j^n\Dx+\frac{\Dx^2}{2}(u_{xx})_j^n)\\
&\hspace{90pt}+\qty(u_j^n-(u_x)_j^n\Dx+\frac{\Dx^2}{2}(u_{xx})_j^n)
+\order{\Dx^4}\Bigg]\Bigg\}.
\end{aligned}
\end{equation}
We see that by symmetry, all odd power terms in the $x$ expansion must
vanish; whereby  it is justified to have $\order{\Dx^4}$. 

Now, canceling the terms that can cancel and dividing both sides by
$\Dt$ gives
\begin{equation}
\tau_j^n=(u_t)_j^n-\frac{1-\alpha}{2\Dt}\Dx^2(u_{xx})_j^n
+\order{\Dt, \frac{\Dx^4}{\Dt}}.
\end{equation}
We see that for this to be vanishing when $\Dt, \Dx\to0$, we need the
first two terms to cancel. That will only happen if
\begin{equation}
\frac{1-\alpha}{2\Dt}\Dx^2=\sigma
\quad\Longleftrightarrow\quad
\alpha=1-2\frac{\sigma\Dt}{\Dx^2}=1-2r.
\qed
\end{equation}
Also note that since $\alpha$ is kept constant, $r$ will also be
constant; and therefore $\Dt\propto\Dx^2$, meannig that
$\order{\Dx^4/\Dt}=\order{\Dx^2}$.

\section{Consistency for another scheme to solve the heat equation}
This time we have another scheme for the heat equation:
\begin{equation}
U_{j}^{n+1}=(1-2\alpha_1-2\alpha_2)U_{j}^{n}
+\alpha_1(U_{j+1}^n+U_{j-1}^n)+\alpha_2(U_{j+2}^n+U_{j-2}^n).
\end{equation}

\subsection{Consistency}
We have a very silimar case to the previous problem. The truncation
error is
\begin{equation}
\begin{aligned}
\Dt\,\tau_j^n=& \qty[u_j^n+(u_t)_j^n\Dt+\frac{1}{2}(u_{tt})_j^n\Dt^2+\order{\Dt^3}]\\
&-\Bigg\{(1-2\alpha_1-2\alpha_2)u_j^n
\\ &\hspace{22pt}+\alpha_1\Bigg[
\qty(u_j^n+\frac{\Dx^2}{2}(u_{xx})_j^n+\frac{\Dx^4}{24}(u_{xxxx})_j^n)
\\ &\hspace{55pt}
+\qty(u_j^n+\frac{\Dx^2}{2}(u_{xx})_j^n+\frac{\Dx^4}{24}(u_{xxxx})_j^n)
+\order{\Dx^6}\Bigg]
\\ &\hspace{22pt}
+\alpha_2\Bigg[
\qty(u_j^n+\frac{4\Dx^2}{2}(u_{xx})_j^n+\frac{16\Dx^4}{24}(u_{xxxx})_j^n)
\\ &\hspace{55pt}
+\qty(u_j^n+\frac{4\Dx^2}{2}(u_{xx})_j^n+\frac{16\Dx^4}{24}(u_{xxxx})_j^n)
+\order{\Dx^6}\Bigg]
\Bigg\}.
\end{aligned}
\end{equation}
As before, we see that all odd power terms in the $x$ expansion must
vanish; so $\order{\Dx^6}$ is used. 

We can now cancel most of these terms and simplify:
\begin{equation}
\begin{aligned}
\Dt\,\tau_j^n=& (u_t)_j^n\Dt+\frac{\Dt^2}{2}(u_{tt})_j^n
-\Bigg\{\alpha_1\qty[
\qty(\Dx^2(u_{xx})_j^n+\frac{\Dx^4}{12}(u_{4x})_j^n)]
\\ &\hspace{110pt}
+\alpha_2\qty[
\qty(4\Dx^2(u_{xx})_j^n+\frac{16\Dx^4}{12}(u_{4x})_j^n)]
\Bigg\} +\order{\Dt^3, \Dx^6}\\
=& (u_t)_j^n\Dt -[\alpha_1+4\alpha_2](u_{xx})_j^n\Dx^2\\
&+\frac{\Dt^2}{2}(u_{tt})_j^n-\frac{1}{12}[\alpha_1+16\alpha_2](u_{4x})_j^n\Dx^4
+\order{\Dt^3, \Dx^6}.
\end{aligned}
\end{equation}
And we once again note that for consitency, we need 
\begin{equation}\label{eq:2a_end}
(u_t)_j^n-[\alpha_1+4\alpha_2](u_{xx})_j^n\frac{\Dx^2}{\Dt}=0
\quad\Longrightarrow\quad
\alpha_1+4\alpha_2=\frac{\sigma\Dt}{\Dx^2}=r.
\qed
\end{equation}
And also again, since $\alpha_1$ and $\alpha_2$ are constant,
$\Dt\propto\Dx^2$. 

\subsection{Fourth order accuracy}
%For this part of the problem we need to expand one more term of the 

Given the condition \eqref{eq:2a_end}, we have the truncation error
\begin{equation}
\tau_j^n=\frac{\Dt}{2}(u_{tt})_j^n
-\frac{1}{12}[\alpha_1+16\alpha_2](u_{4x})_j^n\frac{\Dx^4}{\Dt}
+\order{\Dt^2, \Dx^4}.
\end{equation}
Since $\Dt\propto\Dx^2$, we can write $\order{\Dt^2,
  \Dx^4}=\order{\Dx^4}$, and all we need to do now is to make the
fisrt terms cancel.

For this we employ a small trick:
\begin{equation}
u_t=\sigma u_{xx}
\quad\Longrightarrow\quad
u_{tt}=\pdv{t}u_t=\sigma\pdv{t}u_{xx}
=\sigma\pdv[2]{x}u_t=\sigma^2u_{4x}.
\end{equation}
So to get fourth order accuracy, we need
\begin{equation}
\frac{\sigma^2\Dt}{2}-\frac{\alpha_1+16\alpha_2}{12}\frac{\Dx^4}{\Dt}=0,
\end{equation}
i.e.
\begin{equation}
\alpha_1+16\alpha_2=\frac{6\sigma^2\Dt^2}{\Dx^4}=6r^2.
\end{equation}


Both of these conditions together is a $2\times2$ system of linear
equations 
\begin{equation}
\begin{bmatrix}1&4\\1&16\end{bmatrix}
\begin{bmatrix}\alpha_1\\\alpha_2\end{bmatrix}
=r\begin{bmatrix}1\\6r\end{bmatrix}
\quad\Longrightarrow\quad
\begin{bmatrix}\alpha_1\\\alpha_2\end{bmatrix}
=\frac{r}{12}\begin{bmatrix*}[c]
16-24r\\6r-1
\end{bmatrix*}.
\qed
\end{equation}


\section{A scheme for the wave equation}
This time we're dealing with a scheme
\begin{equation}\label{eq:3_start}
\mathcal{P}_j^nU=\frac{U_j^{n+1}-2U_j^{n}+U_j^{n-1}}{\Dt^2}
-\frac{U_{j+1}^{n}-2U_j^{n}+U_{j-1}^{n}}{\Dx^2}=0
\end{equation}
for the wave equation $u_{tt}-u_{xx}=0$.

\subsection{Truncation error}
This time, since the scheme is written on a difference-quotient form,
the truncation error is just
\begin{equation}
\tau_j^n=\mathcal{P}_j^nu
=\frac{u_j^{n+1}-2u_j^{n}+u_j^{n-1}}{\Dt^2}
-\frac{u_{j+1}^{n}-2u_j^{n}+u_{j-1}^{n}}{\Dx^2}.
\end{equation}
Both of htese terms have a very similar form and subsequently, their
Taylor expansions will also be similar. Let's therefore look at one of
them:
\begin{equation}
\begin{aligned}
u_{j+1}^{n}-2u_j^{n}+u_{j-1}^{n}=&
\qty[u_j^n+\frac{1}{2}(u_{xx})_j^n\Dx^2]-2u_j^n
+\qty[u_j^n+\frac{1}{2}(u_{xx})_j^n\Dx^2]+\order{\Dx^4}\\
=&(u_{xx})_j^n\Dx^2+\order{\Dx^4}.
\end{aligned}
\end{equation}
And as in the two previous problems, the odd power terms cancel each
others from the $j+1$ and the $j-1$ expansion. Then divinding through
by $\Dx^2$, we get $(u_{xx})_j^n+\order{\Dx^2}$.
By symmetry the other term will be $(u_{tt})_j^n+\order{\Dt^2}$.

We can therefore write the truncation error as
\begin{equation}
\tau_j^n=
\overbrace{(u_{tt})_j^n-(u_{xx})_j^n}^{=0\ \text{by definig PDE}}
+\order{\Dt^2, \Dx^2}=\order{\Dt^2, \Dx^2}.
\end{equation}

\subsection{von Neumann approach}
In the von Neumann approach we will use the Fourier transform
\begin{equation}\label{eq:3_FT}
U(t_n,x_j)=\sum_{k=0}^{J-1} A_k^n\exp(\ii\frac{2\pi k}{J}j)
=:\sum_{k=0}^{J-1} A_k^nw_j^k.
\end{equation}
We also see that
\begin{equation}
w_{j+s}^k=\exp(\ii\frac{2\pi k}{J}s)w_j^k.
\end{equation}

We begin by rewriting \eqref{eq:3_start} as
\begin{equation}
U_j^{n+1}=\qty(1-v^{-2})2U_j^n+v^{-2}\qty[U_{j+1}^n+U_{j-1}^n]
-U_j^{n-1},
\end{equation}
where $v=\Dx/\Dt$. Now we can apply the FT \eqref{eq:3_FT} to this
expression
\begin{equation}
\begin{aligned}
\sum_kA_k^{n+1}w_j^k=&\sum_k \qty{\qty[
2(1-v^{-2})+v^{-2}\qty(\ee^{\ii 2\pi k/J}+\ee^{-\ii 2\pi k/J})]A_k^n
-A_k^{n-1} }w_j^k\\
=&\sum_k \qty{2v^{-2}\qty[
v^{2}-1+\cos(\frac{2\pi k}{J})]A_k^n
-A_k^{n-1} }w_j^k.
\end{aligned}
\end{equation}
Equation this for all $j$, we get
\begin{equation}\label{eq:3_FTcoef}
A_k^{n+1}=2\lambda A_k^n-A_k^{n-1},
\end{equation}
where $\lambda=v^{-2}[v^{2}-1+\cos(2\pi k/J)]$.

To calculate a possible amplification factor we assume that
$A_k^n=C_k\,(M_k)^n$, meaning that \eqref{eq:3_FTcoef} becomes
\begin{equation}
C_k\,(M_k)^n=C_k\,(M_k)^{n-1}\qty[2\lambda M_k-1]
\end{equation}
or
\begin{equation}
M_k^2-2\lambda M_k+1=0
\quad\Longrightarrow\quad
M_k^\pm=\lambda\pm\sqrt{\lambda^2-1}.
\end{equation}
I  general we will have Fourier coefficients of the form
\begin{equation}
A_k^n=C_k^+\,(M_k^+)^n+C_k^-\,(M_k^-)^n.
\end{equation}
For this to be stable we want \emph{both} $|M_k^+|\le1$ and
$|M_k^-|\le1$. 

The conditions for this to occur are non-trivial, but we know that 
\begin{equation}
M_k^+M_k^-=\qty(\lambda+\sqrt{\lambda^2-1})
\qty(\lambda-\sqrt{\lambda^2-1})
=1.
\end{equation}
This means that if $M_k^\pm\in\R$, one of them must have magnitude
greater than 1. That is $M_k^\pm$ \emph{cannot} be real for this
scheme to be stable. If, on the other hand, $M_k^\pm$ were complex, we
see that $M_k^\pm$ are a pair of komplex conjugates lying on the
complex unit circle. This gives us the condition that
\begin{equation}\label{eq:3_lambdacond}
v^{-2}\abs{v^{2}-1+\cos(2\pi k/J)}=|\lambda|\le1.
\end{equation}

To study this condition closer, we begin by using the trigonometric
identity
\begin{equation}
\cos(2\alpha)=1-2\sin^2(\alpha)
\end{equation}
to rewrite \eqref{eq:3_lambdacond} as
\begin{equation}
\abs{1-2v^{-2}\sin^2(2\pi k/J)}\le1.
\end{equation}
It is clear that this will only be satisfied when
\begin{equation}
\frac{\Dt^2}{\Dx^2}=v^{-2}\le1
\quad\Longleftrightarrow\quad
\Dt\le\Dx,
\end{equation}
(assuming both $\Dt$ and $\Dx$ to be positive).

\subsection{The merits of this scheme}
As far as I can tell this scheme isn't too bad. The truncation error
is $\order{\Dt^2,\Dx^2}$, which becomes just $\order{\Dx^2}$ given the
stability condition $\Dt\le\Dx$ meaning that $\Dt=\order{\Dx}$. Also,
when the stability condition is met, we have $|M_k^\pm|=1$, meaning
that there is no numerical dissipation in the scheme. 

What I'm saying is that this scheme lacks many of the major drawbacks
of other schemes discussed in class. 




\section{Matrix representation of a scheme for the advection equation}
\newcommand{\odelta}[1]{{\mathring{\delta}^{J}_{#1}}}
In this problem we have the implicit scheme
\begin{equation}
\frac{U_j^{n+1}-U_j^n}{\Dt}
+a\frac{U_{j+1}^{n+1}-U_{j-1}^{n+1}}{2\Dx}=0.
\end{equation}
for the advection equation $u_t+au_x=0$. The PDE is applied
together with periodic boundary conditions.
We want to find the matrix representation of this scheme.

We begin by rearranging the scheme to look like
\begin{equation}\label{eq:4_SLE1}
\nu U_j^{n+1}+U_{j+1}^{n+1}-U_{j-1}^{n+1}=\nu U_j^n,
\end{equation}
where $\nu=\Dx/(a\Dt)$. 
Since we have \emph{periodic boundary conditions}\footnotemark{},
$U_0=U_J$ and $U_{J+1}=U_1$. With this in mind we clearly see that
\eqref{eq:4_SLE1} is a system of linear equation which can be
represented, using the Einstein sum convention, as
\begin{equation}
\qty(-\odelta{i,j-1}+\nu\odelta{i, j}+\odelta{i, j+1})U^{n+1}_i=\nu U_j^n,
\end{equation}
where $\odelta{}$ is the ``periodic Kronecker delta'' of period $J$,
defined as:
\begin{equation}
\mathring\delta^{P}_{k,l}:=
\begin{cases}
1\quad&\text{if}\ k\equiv l\mod P,\\
0&\text{otherwise}.
\end{cases}
\end{equation}
Written out in ``full'' matrix notation we get
\begin{equation}
\begin{bmatrix*}[r]
\nu&+1&0&\cdots&0&0&-1\\
-1&\nu&+1&0&\cdots&0&0\\
0&-1&\nu&+1&0&\cdots&0\\
\vdots&&&\ddots&&&\vdots\\
0&\cdots&0&-1&\nu&+1&0\\
0&0&\cdots&0&-1&\nu&+1\\
+1&0&0&\cdots&0&-1&\nu\\
\end{bmatrix*}
\begin{bmatrix*}
%\\|\\ \\U^{n+1}\\ \\|\\
{}\\|\;\\{}\\U^{n+1}\\{}\\|\;\\{}
\end{bmatrix*}
=\nu
\begin{bmatrix*}
{}\\|\\{}\\U^{n}\\{}\\|\\{}
\end{bmatrix*},
\end{equation}
where $\nu=\Dx/(a\Dt)$. In words, the matrix is a tridiagonal matrix
with the extra addition of a $\pm1$ in the at the anti-diagonal
corners. 


\footnotetext{With the ``main'' idices being 
$j\in\{1, 2, \ldots, J\}$.} 





\end{document}




%  LocalWords:  MFT MF Advection PDE's AMATH IC discretization MATLAB
