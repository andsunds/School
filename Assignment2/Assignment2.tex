\documentclass[11pt,letter, swedish, english
]{article}
\pdfoutput=1

\usepackage{../custom_as}
\graphicspath{{figures/}}
\swapcommands{\Omega}{\varOmega}

%%Drar in tabell och figurtexter
\usepackage[margin=10 pt]{caption}
%%För att lägga in 'att göra'-noteringar i texten
\usepackage{todonotes} %\todo{...}

%%För att själv bestämma marginalerna. 
\usepackage[
%            top    = 3cm,
%            bottom = 3cm,
%            left   = 3cm, right  = 3cm
]{geometry}

%%För att ändra hur rubrikerna ska formateras
\renewcommand{\thesubsection}{\arabic{section} (\alph{subsection})}

\renewcommand{\thesubsubsection}{\arabic{section} (\alph{subsection},\,\roman{subsubsection})}

% \newcommand{\cbox}[2][cyan]
% {\mathchoice
% 	{\setlength{\fboxsep}{0pt}\colorbox{#1}{$\displaystyle#2$}}
% 	{\setlength{\fboxsep}{0pt}\colorbox{#1}{$\textstyle#2$}}
% 	{\setlength{\fboxsep}{0pt}\colorbox{#1}{$\scriptstyle#2$}}
% 	{\setlength{\fboxsep}{0pt}\colorbox{#1}{$\scriptscriptstyle#2$}}
% }
% \newcommand{\grande}{\cbox{\phantom{\frac{1}{xx}}}}

\renewcommand{\thefootnote}{\fnsymbol{footnote}}

\begin{document}

%%%%%%%%%%%%%%%%% vvv Inbyggd titelsida vvv %%%%%%%%%%%%%%%%%
% \begin{titlepage}
\title{Statistical Physics -- PHYS\,704 \\
Assignment 2}
\author{Andréas Sundström}
\date{\today}

\maketitle

%%%%%%%%%%%%%%%%% ^^^ Inbyggd titelsida ^^^ %%%%%%%%%%%%%%%%%

%Om man vill ha en lista med vilka todo:s som finns.
%\todolist

\section{Compressibility and heatcapacity of ideal Fermi gases}
\subsection{Isothermal and adiabatic compressibility}
In this problem we want to show that
\begin{equation}\label{eq:1_kappa}
\kappa_T = k_T:=\frac{1}{nT}\frac{f_{1/2}(z)}{f_{3/2}(z)}
\quad\text{ and }\quad
\kappa_S = k_S:=\frac{3}{5nT}\frac{f_{3/2}(z)}{f_{5/2}(z)}.
\end{equation}


\subsubsection{Isothermal compressibility}
We begin by maniplulating the supposed expression for the isothermal
compressibility
\begin{equation}\label{eq:1_want_this_T}
k_T=\frac{1}{nT}\frac{f_{1/2}(z)}{f_{3/2}(z)}
= \frac{V}{NT}\frac{f_{1/2}(z)}{f_{3/2}(z)}
= \frac{1}{P}\frac{f_{1/2}(z)f_{5/2}(z)}{[f_{3/2}(z)]^2},
\end{equation}
where we have used the thermodynamic relation for Fermi gases
\begin{equation}\label{eq:1_PVNT}
\frac{PV}{NT}=\frac{f_{5/2}(z)}{f_{3/2}(z)}.
\end{equation}

To find the compressibility we just use the definition
\begin{equation}
\kappa_T:=-\frac{1}{V}\qty(\pdv{V}{P})_T
= -\frac{1}{V} \qty(\pdv{P} \frac{NT}{P}\frac{f_{5/2}(z)}{f_{3/2}(z)})_T,
\end{equation}
where we once again used \eqref{eq:1_PVNT} to express $V$ in the
derivative. Now, since $T$ is constant we get
\begin{equation}\label{eq:1_k_T_2}
\kappa_T
= -\frac{NT}{V} \qty[
\frac{-1}{P^2}\frac{f_{5/2}(z)}{f_{3/2}(z)}
+\frac{1}{P}\qty(\pdv{P}
\frac{f_{5/2}(z)}{f_{3/2}(z)}
)_T].
\end{equation}

Now we have to evauate the derivative of the special functions. To
begin with we note that
\begin{equation}
\qty(\pdv{f_\nu(z)}{\mu})_T=\qty(\pdv{z}{\mu})_T\pdv{f_\nu(z)}{z}
= \frac{1}{T}z\pdv{f_\nu(z)}{z}=\frac{1}{T}f_{\nu-1}(z).
\end{equation}
This follows from the fact that $z=\ee^{\mu/T}$ and the relation
\begin{equation}
z\pdv{f_\nu(z)}{z} = f_{\nu-1}(z).
\end{equation}
Next we have to evaluate
\begin{equation}
\qty(\pdv{\mu}{P})_T.
\end{equation}
To do this we remind ourselves that
\begin{equation}
PV=-\Omega=\frac{g_sVT}{\lambda^3}f_{5/2}(z)
\end{equation}
for Fermi gases. It's also a good idea to remind oursevles that
$\lambda$ only depends on $T$ and other constants, so in this case
$\lambda$ can be regardes as a constant. We now have
\begin{equation}
\qty(\pdv{P}{\mu})_T 
=\qty(\pdv{\mu} \frac{g_sT}{\lambda^3}f_{5/2}(z))_T 
=\frac{g_sT}{\lambda^3} \,\frac{1}{T}f_{3/2}(z) 
=\frac{g_s}{\lambda^3} f_{3/2}(z).
\end{equation}
The last piece needed is the fact that
\begin{equation}
n=\frac{N}{V}=-\frac{1}{V}\qty(\pdv{\Omega}{\mu})_T
=\frac{g_s}{\lambda^3} f_{3/2}(z)=\qty(\pdv{P}{\mu})_T.
\end{equation}

Finally we can continue from \eqref{eq:1_k_T_2} to get
\begin{equation}
\begin{aligned}
\kappa_T &= \frac{NT}{VP} \qty[
\frac{1}{P}\frac{f_{5/2}(z)}{f_{3/2}(z)}
-\qty(\pdv{\mu}{P})_T\qty(\pdv{\mu}\frac{f_{5/2}(z)}{f_{3/2}(z)})_T
]\\
&= \frac{nT}{P} \qty[
\frac{1}{P}\frac{f_{5/2}(z)}{f_{3/2}(z)}
-\frac{1}{n}\qty(\frac{1}{T}\frac{f_{3/2}(z)}{f_{3/2}(z)}
-\frac{1}{T}\frac{f_{5/2}(z)f_{1/2}(z)}{[f_{3/2}(z)]^2}
)]\\
&= \frac{nT}{P} \qty[
\frac{1}{P}\frac{f_{5/2}(z)}{f_{3/2}(z)}
-\frac{1}{nT}
+\frac{1}{nT}\frac{f_{5/2}(z)f_{1/2}(z)}{[f_{3/2}(z)]^2}
].
\end{aligned}
\end{equation}
From here we once agian use \eqref{eq:1_PVNT} to get
\begin{equation}
\frac{1}{P}=\frac{1}{nT}\frac{f_{3/2}(z)}{f_{5/2}(z)}.
\end{equation}
This gives
\begin{equation}
\kappa_T = \frac{nT}{P} \qty[
\frac{1}{nT}
-\frac{1}{nT}
+\frac{1}{nT}\frac{f_{5/2}(z)f_{1/2}(z)}{[f_{3/2}(z)]^2}
]
=\frac{1}{P}\frac{f_{5/2}(z)f_{1/2}(z)}{[f_{3/2}(z)]^2},
\end{equation}
which is exactly what we had in \eqref{eq:1_want_this_T}.
\qed


\subsubsection{Adiabatic compressibility}
As before, we begin with rewriting the RHS of the supposed expression
for the adiabatic compressibility
\begin{equation}\label{eq:1_want_this_S}
k_S=\frac{3}{5nT}\frac{f_{3/2}(z)}{f_{5/2}(z)}
=\frac{3}{5P}\frac{f_{5/2}(z)}{f_{3/2}(z)}\frac{f_{3/2}(z)}{f_{5/2}(z)}
=\frac{3}{5P},
\end{equation}
where we once again used \eqref{eq:1_PVNT} to rewite $nT$. Next we
need to remind ourselves about some more important relations:
\begin{equation}\label{eq:dE}
\rd{E}=T\rd{S}-P\rd{V}+\mu\rd{N}
\end{equation}
and
\begin{equation}\label{eq:1_Omega}
PV=-\Omega=\frac{2}{3}E.
\end{equation}
Of these relations only the last equality (relating $\Omega$ to $E$)
is spesific for Fermi gases, the rest are general thermodynaic
identities. 

From now we can focus more on the compressibility
\begin{equation}
\kappa_S=-\frac{1}{V}\qty(\pdv{V}{P})_S
=-\frac{1}{V\qty(\pdv{P}{V})_S}
\end{equation}
From the last expression we note that \eqref{eq:1_Omega} allows us to
write
\begin{equation}
\qty(\pdv{P}{V})_S=\frac{2}{3}\qty(\pdv{(\nicefrac{E}{V})}{V})_S
=\frac{2}{3V}\qty(\pdv{E}{V})_S-\frac{2E}{3}\frac{1}{V^2}.
\end{equation}
Now by invoking \eqref{eq:dE} we see that
\begin{equation}
\qty(\pdv{E}{V})_S=-P,
\end{equation}
we have also assumed $N$ to be constant which is resonable. And by
using \eqref{eq:1_Omega} once more to rewite $E=2PV/3$ we get
\begin{equation}
\qty(\pdv{P}{V})_S=-\frac{2P}{3V}-\frac{PV}{V^2}=-\frac{5P}{3V}.
\end{equation}
This means that we now have
\begin{equation}
\kappa_S=-\frac{1}{V\qty(\pdv{P}{V})_S}=\frac{3}{5P},
\end{equation}
which is exactly what we were looking for in
\eqref{eq:1_want_this_S}.\footnotemark{}
\qed

\footnotetext{This was a very neat way to derive $\kappa_S$. One could
  wonder if this approach is applicable to $\kappa_T$ as well. But
  unfortunatly the analogous approach yields an expression wich
  contains $(\pdv*{\mu}{V})_T$, which is not redily evaluated.} 


\subsubsection{Low temperature limit}
% Here we're going to show that 
% \begin{equation}\label{eq:1_lowT_T}
% \kappa_T = \frac{1}{nT}\frac{f_{1/2}(z)}{f_{3/2}(z)}
% \approx \frac{3}{2n\epsilon_F}
% \qty[1-\frac{\pi^2}{12}\qty(\frac{T}{\epsilon_F})^2]
% \end{equation}
% and
% \begin{equation}\label{eq:1_lowT_S}
% \kappa_T = \frac{3}{5nT}\frac{f_{3/2}(z)}{f_{5/2}(z)}
% \approx \frac{3}{2n\epsilon_F}
% \qty[1-\frac{5\pi^2}{12}\qty(\frac{T}{\epsilon_F})^2]
% \end{equation}
% in the low temperature limit. 

Here we're going to find the low temperature limit expressions for
$\kappa_T$ and $\kappa_S$.
To be in the limit of low temperature means that
\begin{equation}
T\ll\epsilon_F\approx\mu 
\quad\Longrightarrow\quad
z=\ee^{\mu/T}\approx \ee^{\epsilon_F/T}\gg1.
\end{equation}
We therefore need to expand $f_\nu(z)$ using the Sommerfeld
expansion\footnotemark{} 
\begin{equation}\label{eq:1_Sommerfeld}
\begin{aligned}
f_\nu(z)=&\frac{\ln^\nu(z)}{\Gamma(\nu+1)}\qty[
1+\nu(\nu{-}1)\frac{\pi^2}{6}\frac{1}{\ln^2(z)}
+\nu(\nu{-}1)(\nu{-}2)(\nu{-}3)\frac{7\pi^4}{360}\frac{1}{\ln^4(z)}
+\ldots]\\
=&\frac{\qty(\nicefrac{\mu}{T})^\nu}{\Gamma(\nu+1)}\qty[
1+\nu(\nu{-}1)\frac{\pi^2}{6}\qty(\frac{T}{\mu})^2
+\nu(\nu{-}1)(\nu{-}2)(\nu{-}3)\frac{7\pi^4}{360}\qty(\frac{T}{\mu})^4
+\ldots]
\end{aligned}
\end{equation}
This expansion can be found in for instance Pathria\,\&\,Beale,
\textit{Statistical Mechanics}, ed. 3.

\footnotetext{I'm including the $\ln^{-4}(z)$ term since it will be
  used in the next problem. Here however, we're only interested in the
  expansion up to $\ln^{-2}(z)$.}

Now we have to start working:
\begin{equation}\label{eq:1_1/2}
f_{1/2}(z)\approx\frac{1}{\Gamma(\nicefrac{3}{2})}\qty(\frac{\mu}{T})^{1/2}
\qty[
1-\frac{1}{4}\frac{\pi^2}{6}\qty(\frac{T}{\mu})^2]
=\frac{2}{\sqrt{\pi}}\qty(\frac{\mu}{T})^{1/2}
\qty[
1-\frac{\pi^2}{24}\qty(\frac{T}{\mu})^2],
\end{equation}
\begin{equation}\label{eq:1_3/2}
f_{3/2}(z)\approx\frac{1}{\Gamma(\nicefrac{5}{2})}\qty(\frac{\mu}{T})^{3/2}
\qty[
1+\frac{3}{4}\frac{\pi^2}{6}\qty(\frac{T}{\mu})^2]
=\frac{4}{3\sqrt{\pi}}\qty(\frac{\mu}{T})^{3/2}
\qty[1+\frac{\pi^2}{8}\qty(\frac{T}{\mu})^2],
\end{equation}
and 
\begin{equation}\label{eq:1_5/2}
f_{5/2}(z)\approx\frac{1}{\Gamma(\nicefrac{7}{2})}\qty(\frac{\mu}{T})^{5/2}
\qty[
1+\frac{15}{4}\frac{\pi^2}{6}\qty(\frac{T}{\mu})^2]
=\frac{8}{15\sqrt{\pi}}\qty(\frac{\mu}{T})^{5/2}
\qty[1+\frac{5\pi^2}{8}\qty(\frac{T}{\mu})^2].
\end{equation}
After this we also need to recognize that $\mu$ is dependent of
$T$, and $\mu$ is also only comeing in as $T/\mu$. It's therefor a
good idea to use \eqref{eq:2_mu} to expand
\begin{equation}\label{eq:1_T/mu}
\frac{T}{\mu}\approx T \frac{1}{\epsilon_F}\qty[ 1
-\frac{\pi^2}{12}\qty(\frac{T}{\epsilon_F})^2]^{-1}
\approx\tau\qty[1
+\frac{\pi^2}{12}\tau^2]
\end{equation}
where $\tau=T/\epsilon_F$. We only needed to expand to
$\order{\tau^2}$ since we're only going that far in the rest of the
expansions. 

Then we Taylor expand the quotients in \eqref{eq:1_kappa},
%{eq:1_lowT_T} and \eqref{eq:1_lowT_S}, 
which is possible since $T\ll\mu$. 
For isothermal compressibility this looks like
\begin{equation}
\begin{aligned}
\kappa_T\approx& \frac{1}{nT} 
\frac{2}{\sqrt{\pi}}\qty(\frac{\mu}{T})^{1/2}
\qty(\frac{4}{3\sqrt{\pi}}\qty(\frac{\mu}{T})^{3/2})^{-1}
\qty[
1-\frac{\pi^2}{24}\qty(\frac{T}{\mu})^2]
\qty[
1+\frac{\pi^2}{8}\qty(\frac{T}{\mu})^2]^{-1}\\
\approx&
 \frac{3}{2nT}\frac{T}{\mu}
\qty[
1-\frac{\pi^2}{24}\qty(\frac{T}{\mu})^2]
\qty[
1-\frac{\pi^2}{8}\qty(\frac{T}{\mu})^2]\\
\end{aligned}
\end{equation}
From here we use \eqref{eq:1_T/mu}. But note that only the leading
factor $T/\mu$ needs to be expanded, since we're only interesed in
expanding to $\order{\tau^2}$ in each bracket. Thus
\begin{equation}
\begin{aligned}
\kappa_T\approx& \frac{3}{2nT}
\tau\qty[1 +\frac{\pi^2}{12}\tau^2]
\qty[1 -\frac{\pi^2}{24}\tau^2]
\qty[1 -\frac{\pi^2}{8}\tau^2]\\
\approx& \frac{3}{2nT}
\frac{T}{\epsilon_F}\qty[1 +
\qty(\frac{\pi^2}{12}-\frac{\pi^2}{24}-\frac{\pi^2}{8})
\tau^2] 
&=\frac{3}{2n\epsilon_F}
\qty[1 - \frac{\pi^2}{12}\qty(\frac{T}{\epsilon_F})^2]
\end{aligned}
\end{equation}

The adiabatic compressibility is calculated in an analogous way:
\begin{equation}
\begin{aligned}
\kappa_S\approx& \frac{3}{5nT} 
\frac{4}{3\sqrt{\pi}}\qty(\frac{\mu}{T})^{3/2}
\qty(\frac{8}{15\sqrt{\pi}}\qty(\frac{\mu}{T})^{5/2})^{-1}
\qty[
1+\frac{\pi^2}{8}\qty(\frac{T}{\mu})^2]
\qty[
1+\frac{5\pi^2}{8}\qty(\frac{T}{\mu})^2]^{-1}\\
\approx&
 \frac{3}{5nT}\frac{5}{2}\tau\qty[1 +\frac{\pi^2}{12}\tau^2]
\qty[1+\frac{\pi^2}{8}\tau^2]
\qty[1-\frac{5\pi^2}{8}\tau^2]\\
\approx&\frac{3}{2n\epsilon_F}
\qty[1 - \frac{5\pi^2}{12}\qty(\frac{T}{\epsilon_F})^2].
\end{aligned}
\end{equation}
And that's the two expressions for the compressibilities in the low
temperature limit.
\qed

\subsection{Relationship between isobaric and isochoric heatcapacity}

\subsubsection{General thermodynamic relationship between $C_P$ and $C_V$}
In general we have
\begin{equation}
C_V=T\qty(\pdv{S}{T})_V\qcomma\text{and}\quad
C_P=T\qty(\pdv{S}{T})_P.
\end{equation}

To evaluate these in a bit more detail we begin by studying
$\rd{S}$. By the State postulate any property of the system can be
described by two independent variables (assuming no particle exchange
with the environment). We can therefore write
\begin{equation}
\rd{S}=\qty(\pdv{S}{T})_V\rd{T} + \qty(\pdv{S}{V})_T\rd{V}.
\end{equation}
But we are also free to rewite $\rd{V}$ in terms of $\rd{T}$ and
$\rd{P}$:
\begin{equation}
\rd{V}=\qty(\pdv{V}{T})_P\rd{T} + \qty(\pdv{V}{P})_T\rd{P}.
\end{equation}
With this $\rd{S}$ becomes
\begin{equation}
\begin{aligned}
\rd{S} =& \qty(\pdv{S}{T})_V\rd{T} + \qty(\pdv{S}{V})_T
\qty[\qty(\pdv{V}{T})_P\rd{T} + \qty(\pdv{V}{P})_T\rd{P}]\\
=&\qty[\qty(\pdv{S}{T})_V+\qty(\pdv{S}{V})_T\qty(\pdv{V}{T})_P]\rd{T}
+ \qty(\pdv{V}{P})_T\rd{P}.
\end{aligned}
\end{equation}
Now we can write 
\begin{equation}
C_P=T\qty(\pdv{S}{T})_P
=T\qty[\qty(\pdv{S}{T})_V+\qty(\pdv{S}{V})_T\qty(\pdv{V}{T})_P]
=C_V+T\qty(\pdv{S}{V})_T\qty(\pdv{V}{T})_P.
\end{equation}

To continue from here we need to use some Maxwell relations. There are
two of interest to us in the moment:
\begin{equation}
\qty(\pdv{S}{V})_T=-\pdv[2]{F}{T}{V}=\qty(\pdv{P}{T})_V
\end{equation}
and
\begin{equation}
-\qty(\pdv{S}{P})_T=\pdv[2]{\varPhi}{T}{P}=\qty(\pdv{V}{T})_P,
\end{equation}
where $F$ and $\varPhi$ are the Helmholzt and Gibbs free energy
respectively. 

With these Maxwell relations we can write
\begin{equation}
C_P-C_V=T\qty(\pdv{P}{T})_V\qty(\pdv{V}{T})_P
=TV\kappa_T\qty(\pdv{P}{T})_V^2.
\end{equation}
The last equality comes from
\begin{equation}
V\kappa_T\qty(\pdv{P}{T})_V=-\qty(\pdv{V}{P})_T\qty(\pdv{S}{V})_T
=-\qty(\pdv{S}{P})_T=\qty(\pdv{V}{T})_P.
\end{equation}

\subsubsection{In the specific cast of Fermi gases}
Here we want to study 
\begin{equation}\label{eq:1bi_1}
\frac{C_P-C_V}{C_V} 
= \frac{TV\kappa_T\qty(\pdv{P}{T})_V^2}{\qty(\pdv{E}{T})_V}
= \frac{TV\kappa_T\qty(\pdv{P}{E})_V^2\qty(\pdv{E}{T})_V^2}{\qty(\pdv{E}{T})_V}
= TV\kappa_TC_V\qty(\pdv{P}{E})_V^2.
\end{equation}
Here we used the fact that $C_V=(\pdv*{E}{T})_V$. To continue we
note that \eqref{eq:1_Omega} implies that
\begin{equation}
\qty(\pdv{E}{P})_V=\frac{3}{2}V.
\end{equation}
No we can write \eqref{eq:1bi_1} as
\begin{equation}
\frac{C_P-C_V}{C_V} 
= TV\kappa_TC_V\frac{4}{9V^2}
=\frac{4C_VT}{9V}\frac{1}{nT}\frac{f_{1/2}(z)}{f_{3/2}(z)}
=\frac{4C_V}{9N}\frac{f_{1/2}(z)}{f_{3/2}(z)}.
\end{equation}
Here we used (\ref{eq:1_kappa}\,a) to rewrite $\kappa_T$.

To get a first order approximation of the value of this in the low
temperature limit we will use \eqref{eq:1_1/2}, \eqref{eq:1_3/2} and
\eqref{eq:CV_Fermi}. To a first order approximation 
\begin{equation}
\frac{C_P-C_V}{C_V} =\frac{4C_V}{9N}\frac{f_{1/2}(z)}{f_{3/2}(z)} 
\approx \frac{4}{9} \frac{\pi^2}{2}\frac{T}{\epsilon_F}
\frac{2}{\sqrt{\pi}}\qty(\frac{\epsilon_F}{T})^{1/2}
\qty(\frac{4}{3\sqrt{\pi}}\qty(\frac{\epsilon_F}{T})^{3/2})^{-1}
=\frac{\pi^2}{3}\qty(\frac{T}{\epsilon_F})^2.
\end{equation}
\qed





\section{Cemical potential and particle energy of an ideal
  Fermi gas at low temperatures}
\renewcommand{\thesubsection}{\arabic{section} (\roman{subsection})}
\subsection{Chemical potential}
To calculate an approximation to $\mu$ at low temperatures
$\ln(z)=\mu/T\gg1$, we use the Sommerfeld expansion in
\eqref{eq:1_Sommerfeld}. 
Then to get an equation for $\ln(z)$ we can use
\begin{equation}
n=\frac{g_s}{\lambda^3}f_{3/2}(z)
\approx\frac{g_s}{\lambda^3}
\frac{4}{3\sqrt{\pi}}\ln^{3/2}(z)
\qty[1+\frac{\pi^2}{8}\ln^{-2}(z) 
+\frac{7\pi^4}{640}\ln^{-4}(z) ]
\end{equation}
or in other words
\begin{equation}
\zeta\approx\qty(\frac{3\sqrt{\pi}\lambda^3}{4g_s})^{2/3}
\qty[1+\frac{\pi^2}{8}\zeta^{-2}
+\frac{7\pi^4}{640}\zeta^{-4} ]^{-2/3},
\end{equation}
where $\zeta=\ln(z)=\mu/T$.
We're going to deal with these two factors separetly.
First off is the easy bit. The first factor is
\begin{equation}
\qty(\frac{3\sqrt{\pi}\lambda^3}{4g_s})^{2/3}
=\lambda^2\qty(\frac{3\sqrt{\pi}}{4g_s})^{2/3}
=\frac{1}{T}\frac{\hbar^2}{2m}4\pi\qty(\frac{3\sqrt{\pi}}{4g_s})^{2/3}
=\frac{1}{T}\frac{\hbar^2}{2m}\qty(\frac{6\pi^2}{g_s})^{2/3}
=\frac{\epsilon_F}{T}=\frac{1}{\tau}.
\end{equation}
Then there's the othe factor which has to be Taylor expanded to the
second order in $\ln^{-2}(z)\ll1$. To do this we enlist some help in
form of \emph{Mathematica}\footnotemark{}, which yields
\begin{equation}
\qty[1 + \frac{\pi^2}{8}\zeta^{-2}
 + \frac{7\pi^4}{640}\zeta^{-4}]^{-2/3} 
=\qty[1 - \frac{\pi^2}{12}\zeta^{-2}
 + \frac{\pi^4}{720}\zeta^{-4} + \order{\zeta^{-6}}].
\end{equation}
We now have every thing we need to start approximating $\zeta$.
Unfortunately what we have is something in the line of a fifth degree
polynolial in $\zeta$ to solve: 
\begin{equation}\label{eq:2_fixed-point}
\zeta\approx \tau^{-1}\qty[1 - \frac{\pi^2}{12}\zeta^{-2}
 + \frac{\pi^4}{720}\zeta^{-4}].
\end{equation}
This can not and should not be solved
analytically. Instead we're going to use a method similar to
fixed-point iteration, but for analytical expressions.
\footnotetext{Lord and Saviour in our times of need and despair!}

The first iteration is just the zeroth power approximation:
\begin{equation}
\zeta_0=\tau^{-1}.
\end{equation}
Next iteration is using the fixed-point function
\eqref{eq:2_fixed-point} on $\zeta_0$ to get
\begin{equation}
\zeta_1=\tau^{-1}\qty[1 - \frac{\pi^2}{12}\zeta_0^{-2}
 + \frac{\pi^4}{720}\zeta_0^{-4} ]
=\tau^{-1}\qty[1 - \frac{\pi^2}{12}\tau^{2}
 + \frac{\pi^4}{720}\tau^{4} ].
\end{equation}
One more iteration:
\begin{equation}
\begin{aligned}
\zeta_2=&\tau^{-1}\qty[1 - \frac{\pi^2}{12}\zeta_1^{-2}
 + \frac{\pi^4}{720}\zeta_1^{-4} ]\\
=& \tau^{-1}\qty[1 
 - \frac{\pi^2}{12}\tau^{2}
\qty(1 - \frac{\pi^2}{12}\tau^{2}
 + \frac{\pi^4}{720}\tau^{4})^{-2}
 + \frac{\pi^4}{720}\tau^{4} 
\qty(1 - \frac{\pi^2}{12}\tau^{2}
 + \frac{\pi^4}{720}\tau^{4})^{-4}
].
\end{aligned}
\end{equation}
We once again turn our faith to \emph{Mathematica} and get
\begin{equation}
\zeta_2=\tau^{-1}\qty[1 - \frac{\pi^2}{12}\tau^{2}
 - \frac{\pi^4}{80}\tau^{4} + \order{\tau^6}].
\end{equation}
From here it's no use in continuing. Since our fixed-point function
only had terms up to $\order{\zeta^{-4}}$, we can not get any higher
orders than $\tau^4$.

We can therefore conclude that
\begin{equation}\label{eq:2_mu}
\mu=T\ln(z)\approx T\zeta_2= \epsilon_F \qty[1
-\frac{\pi^2}{12}\qty(\frac{T}{\epsilon_F})^2
-\frac{\pi^4}{80}\qty(\frac{T}{\epsilon_F})^4].
\end{equation}
\qed

\subsection{Mean particle energy}
To calculate an approximation to the mean particle energy $E/N$, we
once again have to use the Sommerfeld expansion
\eqref{eq:1_Sommerfeld}. Then we use the relation
\begin{equation}
\frac{E}{N}=-\frac{3\Omega}{2N}=\frac{3T}{2}\frac{f_{5/2}(z)}{f_{3/2}(z)}.
\end{equation}
Fortunately we've already calculated $\ln(z)=\zeta\approx\zeta_2$, so all
that left do here is to expand the Fermi functions and then
Taylor expand the quotient.

Let's begin:
\begin{equation}
%\begin{aligned}
\frac{f_{5/2}(z)}{f_{3/2}(z)}\approx
\overbrace{\frac{8\zeta_2^{5/2}}{15\sqrt{\pi}}
\qty(\frac{4\zeta_2^{3/2}}{3\sqrt{\pi}})^{-1}}^{\nicefrac{2\zeta_2}{5}}
\qty[1+\frac{5\pi^2}{8}\zeta_2^{-2}
-\frac{7\pi^4}{384}\zeta_2^{-4} ]
\times \qty[1+\frac{\pi^2}{8}\zeta_2^{-2} 
+\frac{7\pi^4}{640}\zeta_2^{-4} ]^{-1}
%\end{aligned}
\end{equation}
There's no need to even pretend to want deal with this by hand.
Just go directly to \emph{Mathematica}, which yields
\begin{equation}
\frac{f_{5/2}(z)}{f_{3/2}(z)}\approx \frac{2}{5\tau}
\qty[1+\frac{5\pi^2}{12}\tau^2-\frac{\pi^4}{16}\tau^4].
\end{equation}
For reference tha comman in use were \texttt{Series}.

Now we have
\begin{equation}
\frac{E}{N}\approx\frac{3T}{2}\frac{2}{5\tau}
\qty[1+\frac{5\pi^2}{12}\tau^2-\frac{\pi^4}{16}\tau^4]
=\frac{3\epsilon_F}{5}\qty[1
+\frac{5\pi^2}{12}\qty(\frac{T}{\epsilon_F})^2
-\frac{\pi^4}{16}\qty(\frac{T}{\epsilon_F})^4].
\end{equation}
This can be used to get an expression for the heatcapacity of the
system
\begin{equation}\label{eq:CV_Fermi}
\frac{C_V}{N}=\frac{1}{N}\qty(\pdv{E}{T})_V
=\;\frac{\pi^2}{2}\frac{T}{\epsilon_F}
 - \frac{3\pi^2}{20}\qty(\frac{T}{\epsilon_F})^3.
\end{equation}
\qed


\renewcommand{\thesubsection}{\arabic{section} (\alph{subsection})}
\section{Speed of sound in an idela Bose gas}
The speed of sound is given by
\begin{equation}
v_s:=\sqrt{\qty(\pdv{P}{\rho})_S}
\end{equation}
where $\rho=mn=mN/V$ is the mass density of the gas (assuming a gas of
identical particles). Here we will also asume that the temperature is
above the critical temperature for Bose-Einstein condensation, but not
necessarily in the high temperature limit.

To calculate the speed of sound we begin by noting that 
\begin{equation}\label{eq:3_PVNT}
\frac{PV}{NT}=\frac{P}{nT}=\frac{g_{5/2}(z)}{g_{3/2}(z)},
\end{equation}
hold analogous to Fermi gases. Meaning that
\begin{equation}\label{eq:3_dn/dP}
\qty(\pdv{\rho}{P})_S=m\qty(\pdv{n}{P})_S
=m\qty(\pdv{P} \frac{P}{T}\frac{g_{3/2}(z)}{g_{5/2}(z)})_S.
\end{equation}



Then we need some way of expressing the entropy, to know what is
constant when differentiating. Here we can use the relation
\begin{equation}\label{eq:3_S}
S=\frac{E-\Omega-\mu N}{T}=\frac{E-\Omega}{T}-\ln(z)N.
\end{equation}
From here we also need to use
\begin{equation}\label{eq:3_PV}
PV=-\Omega=\frac{2}{3}E,
\end{equation}
which also holds for Bose gases. This yields
\vspace{-4mm}\begin{equation}
S=N\qty[\frac{5}{2}\frac{PV}{NT}-\ln(z)]=
N\overbrace{\qty[\frac{5}{2}\frac{g_{5/2}(z)}{g_{3/2}(z)}-\ln(z)]}^{=:\varXi(z)},
\end{equation}
which is very interesting, since this means that if $S$ is constant
then $z$ must also be constant\footnotemark{}. This can be seen by
differentiating (the constant) $S$ with respect to some variable $x$: 
\begin{equation}
0=\qty(\pdv{S}{x})_S=N\bigg(\overbrace{\pdv{\varXi}{z}}^{\neq0}\bigg)_S
\qty(\pdv{z}{x})_S.
\end{equation}
The only remaining option is that $z$ must also be constant when $S$
is constant.
\footnotetext{As always we assume $N$ to be constant.}

With this new result we can evaluate \eqref{eq:3_dn/dP} further:
\begin{equation}\label{eq:3_drho/dP}
\qty(\pdv{\rho}{P})_S=\qty(\pdv{\rho}{P})_z
=m\frac{g_{3/2}(z)}{g_{5/2}(z)}
\qty[\frac{1}{T}-\frac{P}{T^2}\qty(\pdv{T}{P})_z].
\end{equation}
To continue from here we use \eqref{eq:3_PV} to write $P=-\Omega/V$,
which makes
\begin{equation}\label{eq:3_dP/dT}
\qty(\pdv{P}{T})_z=-\qty(\pdv{(\Omega/V)}{T})_z.
\end{equation}
Then we use the fact that
\begin{equation}\label{eq:3_Omega}
\Omega=-\frac{VT}{\lambda^3}g_{5/2}(z)
=-V\frac{T^{5/2}}{\varLambda^3}g_{5/2}(z),
\end{equation}
where $\Lambda=T^{1/2}\lambda$ is a gas specific constant (only
depends on fundamental constants and the mass of the gas particles).
Now \eqref{eq:3_dP/dT} becomes
\begin{equation}
\qty(\pdv{P}{T})_z
=+\qty(\pdv{T}\frac{T^{5/2}}{\varLambda^3}g_{5/2}(z))_z
=\frac{5}{2}\frac{T^{3/2}}{\varLambda^3}g_{5/2}(z)
=\frac{5P}{2T}.
\end{equation}
The last step is due to \eqref{eq:3_Omega} and \eqref{eq:3_PV}

Now all there is to do is to use this result in \eqref{eq:3_drho/dP}
giving
\begin{equation}
\qty(\pdv{\rho}{P})_S
=\frac{m}{T}\frac{g_{3/2}(z)}{g_{5/2}(z)}
\qty[1-\frac{P}{T}\frac{2T}{5P}]
=\frac{3m}{5T}\frac{g_{3/2}(z)}{g_{5/2}(z)}.
\end{equation}
The speed of sound is now
\begin{equation}\label{eq:3_vs_1}
v_s^2=\qty(\pdv{P}{\rho})_S
=\frac{5T}{3m}\frac{g_{5/2}(z)}{g_{3/2}(z)}.
\end{equation}
\qed

\subsubsection*{Speed of sound expressed in terms of the mean square
  speed of the gas particles}
We can continue manipulating \eqref{eq:3_vs_1} by using
\eqref{eq:3_PVNT} and \eqref{eq:3_PV}. More precisely
\begin{equation}
T\frac{g_{5/2}(z)}{g_{3/2}(z)}\stackrel{\eqref{eq:3_PVNT}}{=}
\frac{PV}{N}\stackrel{\eqref{eq:3_PV}}{=}
\frac{2E}{3N}.
\end{equation}
But $E/N$ is just the mean energy per particle which can be expressed
as
\begin{equation}
\frac{E}{N}=\ev{\epsilon}=\ev{\frac{mu^2}{2}}
=\frac{m}{2}\ev*{u^2}.
\end{equation}
In other words
\begin{equation}
v_s^2=\qty(\pdv{P}{\rho})_S
=\frac{5T}{3m}\frac{g_{5/2}(z)}{g_{3/2}(z)}
=\frac{5}{3m}\frac{2}{3}\frac{m}{2}\ev*{u^2}
=\frac{5}{9}\ev*{u^2}.
\end{equation}
\qed

\section{Regarding ideal Bose gases }
\newcommand{\V}{\ensuremath{\mathcal{V}}}
\newcommand{\Sd}{\ensuremath{\mathcal{S}_d}}
\newcommand{\D}[1]{\ensuremath{\frac{d}{#1}}}
\newcommand{\nD}[1]{\ensuremath{\nicefrac{d}{#1}}}
\newcommand{\Tc}{\ensuremath{T_{\text{c}}}}

Here we are concerned about bosons in $d$ dimensions. They are
confined to a box of (hyper) volume $\V=L^d$ their enegry momentum
relation is $\epsilon_{\vb{k}}=\hbar^2\vb{k}^2/(2m)$.

\subsection{Grand potential and particle density}
The grand potential is given by
\begin{equation}
\Omega=T\ln(Z)=g_sT\sum_{\vb{k}}
\ln(1-\ee^{-\frac{\epsilon_{\vb{k}}-\mu}{T}})
\to g_sT
\iint\frac{\rd^dq\rd^dk}{(2\pi)^d}\ln(1-\ee^{-\frac{\epsilon_{\vb{k}}-\mu}{T}}).
\end{equation}
The $q$ integral just gives us the volume $\V$ of the box. The $k$
integral must however be modified. Since $\epsilon_{\vb{k}}$ only
depends on $k=|\vb{k}|$, we can go to a spherical coordinate
system. By doing so we get
\vspace{-4mm}\begin{equation}
\int\rd^dk\,f(k)
=\overbrace{\text{``Area of unit sphere in $d$ dimensions''}}^{=:\Sd}
\times\int_0^\infty k^{d-1}\rd{k}f(k).
\end{equation}
The factor of the area of the unit shpere in $d$ dimensions comes from
the integration over the corresponding ``sperical'' angles. This area
does have a closed expression
\begin{equation}\label{eq:4_Sd}
\Sd=\frac{2\pi^{\frac{d}{2}}}{\Gamma(\nicefrac{d}{2})}.
\end{equation}

Then we need to convert to an integral over $\epsilon$ instead. Since 
\begin{equation}
\epsilon=\frac{\hbar^2k^2}{2m} 
\quad\Longrightarrow\quad
k=\sqrt{\frac{2m}{\hbar^2}} \,\epsilon^{1/2},
\end{equation}
which means that
\begin{equation}\label{eq:5_k->eps}
k^{d-1}\rd{k}=\qty(\frac{2m}{\hbar^2})^{\frac{d-1}{2}}\epsilon^{\frac{d-1}{2}}
\times\frac{1}{2}\sqrt{\frac{2m}{\hbar^2}}
\,\epsilon^{-1/2}\rd\epsilon
=\frac{1}{2}\qty(\frac{2m}{\hbar^2})^{\D{2}}
\epsilon^{\D{2}-1}\rd\epsilon
\end{equation}
Now we can continue to calculate the grand potential
\begin{equation}
\begin{aligned}
\Omega =& \frac{g_sT\V\Sd}{(2\pi)^d}
\times\frac{1}{2}\qty(\frac{2m}{\hbar^2})^{\D{2}}
\int_0^\infty\!\rd\epsilon\, 
\epsilon^{\D{2}-1} \ln(1-z\ee^{-\frac{\epsilon}{T}})\\
=&\frac{g_sT\V\Sd}{2(2\pi)^d}
\qty(\frac{2m}{\hbar^2})^{\D{2}} \qty{
\qty[\frac{\epsilon^{\D{2}}}{d/2}
\ln(1-z\ee^{-\frac{\epsilon}{T}})]_{\epsilon=0}^\infty
-\frac{2}{d}\int_0^\infty\!\rd\epsilon\, 
\epsilon^{\D{2}}\frac{\frac{z}{T}\ee^{-\frac{\epsilon}{T}}}
{1-z\ee^{-\frac{\epsilon}{T}}}.
}\\
\end{aligned}
\end{equation}
Here the integrated part goes to $0$ in the limits, so all we're left
with is the second integral. To handle that integral we set
$x=\epsilon/T$ which gives
\begin{equation}\label{eq:4_Omega_0}
\begin{aligned}
\Omega =&\qty(-\frac{2}{d})\frac{g_sT\V\Sd}{2(2\pi)^d}
\qty(\frac{2m}{\hbar^2})^{\D{2}} \qty{
\frac{T^{\D{2}+1}}{T}\int_0^\infty
\frac{\rd{x}\, x^{\D{2}}}{z^{-1}\ee^{x}-1}
}\\
=&\qty(-\frac{2}{d})\frac{g_sT^{\D{2}+1}\V\Sd}{2(2\pi)^d}
\qty(\frac{2m}{\hbar^2})^{\D{2}} \Gamma\qty(\nD{2}+1)g_{\D{2}+1}(z).
\end{aligned}
\end{equation}
The last step was just using the definition of $g_\nu(z)$. We can also
make use of the thermal wavelength $\lambda=h/\sqrt{2\pi mT}$ to
rewrite \eqref{eq:4_Omega_0} as 
\begin{equation}\label{eq:4_Omega}
\begin{aligned}
\Omega &=-\frac{g_sT\V\Sd}{d\,\lambda^d \pi^\D{2}}
\Gamma\qty(\nD{2}+1)g_{\D{2}+1}(z)\\
&\hspace{-4pt}\stackrel{\eqref{eq:4_Sd}}{=}
\frac{2g_sTL^d}{d\,\lambda^d}
\frac{\Gamma(\nD{2}+1)}{\Gamma(\nD{2})}g_{\D{2}+1}(z)\\
&=\frac{g_sTL^d}{\,\lambda^d}g_{\D{2}+1}(z)
\end{aligned}
\end{equation}
Here I've also substituted $L^d=\V$.

To calculate the number of (excited) particles we use the relation
\begin{equation}
N_\ee=-\qty(\pdv{\Omega}{\mu})_{T,\V}
=\frac{g_sT\V}{\lambda^d}
\qty(\pdv{\mu}g_{\D{2}+1}(z))_{T,\V}.
\end{equation}
This last differentiation is fairly simple to perform:
\begin{equation}
\pdv{g_{\nu}(z)}{\mu}=
\pdv{g_{\nu}(z)}{z}\pdv{z}{\mu}=
\pdv{g_{\nu}(z)}{z}\frac{1}{T}z
=\frac{1}{T}g_{\nu-1}(z).
\end{equation}
In other words
\begin{equation}\label{eq:4_ne}
n_\ee=\frac{N_\ee}{\V}
=\frac{g_s}{\lambda^d}g_{\D{2}}(z).
\end{equation}
This does not take the particles in the ground state into account
since $\Omega$ doesn't take them into account. For temperatures lower
than the critical temperature we would also need to take the condesed
particles, $N_0$, in to account if we want to study the full number of
particles: $N=N_\ee+N_0$. For temperatures above the critical
temperatures we can however neglect the particles in the ground
state, so $n_\ee=n=N/\V$ for $T>\Tc$.
\qed

\subsection{Energy of the system}
The energy of the system is given by
\begin{equation}
E=\sum_{\vb{k}} g_s \epsilon_{\vb{k}}\ev*{n_{\vb{k}}^\text{B.E.}}
\to \V\int\frac{\rd^dk}{(2\pi)^d}
\epsilon\ev*{n_{\vb{k}}^\text{B.E.}}
=\V\int_0^\infty\rd\epsilon\, g(\epsilon)
\epsilon\ev*{n_{\vb{k}}^\text{B.E.}},
\end{equation}
where $g(\epsilon)$ is the density of states\footnotemark{}. The density
of states can easily be derived with the help of \eqref{eq:5_k->eps}:
\begin{equation}
g(\epsilon)\id\epsilon=\frac{\rd^dk}{(2\pi)^d}
=\frac{\Sd k^{d-1}\id{k}}{(2\pi)^d}
=\frac{\Sd}{2(2\pi)^d}\qty(\frac{2m}{\hbar^2})^{\D{2}}
\epsilon^{\D{2}-1}\id\epsilon.
\end{equation}
With this we now have
\begin{equation}
\begin{aligned}
E=&\V\frac{\Sd}{2(2\pi)^d}\qty(\frac{2m}{\hbar^2})^{\D{2}}
\int_0^\infty\rd\epsilon\, \epsilon^{\D{2}-1}
\epsilon\,\frac{1}{\ee^{\frac{\epsilon-\mu}{T}}-1}\\
=&\frac{\V\Sd}{2(2\pi)^d}\qty(\frac{2m}{\hbar^2})^{\D{2}}
\int_0^\infty
\frac{\rd\epsilon\,\epsilon^{\D{2}-1}}{\ee^{\frac{\epsilon-\mu}{T}}-1}
\end{aligned}
\end{equation}
which after a substitution $x=\epsilon/T$ will be the exact same
expression as in the first line of \eqref{eq:4_Omega_0}, appart from a
factor $-2/d$. 
\footnotetext{For the energy, we do not need to wory about the
  particles in the ground state since they have 0 energy
  anyway. \label{ftn:E}} 

We can therefore conclude that
\begin{equation}\label{eq:4_E_Omega_PV}
E=-\frac{d}{2}\Omega=\frac{d}{2}P\V,
\end{equation}
where we used the general themodynamic identity $\Omega=-P\V$.

\qed


\subsection{Critical temperature}

For bosons $\mu<0$ or else the partition function dosen't
converge. This fact together with the expression in \eqref{eq:4_ne}
for the (excited) particle density gives a temperature, below which
the number of particles in the ground state can not be neglcted. This
is te critical temperature. 

To find the critical temperature we study the limit $\mu\to0^-$ which
means that $z\to1$. This gives us an equation for $T_c$:
\begin{equation}
\frac{N}{\V}= \Tc^{\D{2}}
\frac{g_s}{\Lambda^d} g_{\D{2}}(1),
\end{equation}
where $\Lambda=T^{1/2}\lambda=h/\sqrt{2\pi m}$. Now we just use the
definition of $g_\nu(z)$ to get
\begin{equation}
\frac{N}{\V}= \Tc^{\D{2}}
\frac{g_s}{\Lambda^d} 
\frac{1}{\Gamma(\nD{2})}
\int_0^\infty\frac{\rd{x}\, x^{\D{2}-1}}{\ee^{x}-1}.
\end{equation}
From here on, it's just a matter of recognizing the integralexpressino
of the Rieman $\zeta$ function:
\begin{equation}\label{eq:4_zeta}
\zeta(\nu)=\frac{1}{\Gamma(\nu)}\int_0^\infty
\frac{\rd{x}\,x^{\nu-1}}{\ee^{x}-1}
=g_\nu(1).
\end{equation}
The critical temperature is therefore given by
\begin{equation}
\Tc=\qty[\frac{N}{\V}
\frac{\Lambda^d}{g_s\zeta(\nD{2})}
]^{\frac{2}{d}}
=\frac{h^2}{2\pi m L^2} \qty[
\frac{N}{g_s}\,\frac{1}
{\zeta(\nD{2})}
]^{\frac{2}{d}},
\end{equation}
where we used $\Lambda=h/\sqrt{2\pi m}$, $\V=L^d$ 
to get to the last expression. \qed

It's interesting that we have the $\zeta$ function in the expression
for $\Tc$. Because, as we all know, the $\zeta(\nu)$ diverges for
$s=1$. That means that Bose-Einstein condensation would be impossible
in $d=2$ dimensions, or rather the condensation temperature would be
$0$ so B.E. condensation would be unachievable.

We have the same phenomenon for $d=1$, with no B.E. condensation. This
is however not so obvious from the Rieman $\zeta$ function. But
looking at the integral representation in \eqref{eq:4_zeta}, we see
that the integral dose not converge for $d=1$ either. 

From here on, we will therefore only regard $d\ge3$.

Other than that there's not that much more to say about the $d$
dependence of $\Tc$. By plotting this expression in for example
Mathematica, I can't see anything else to comment on. 


\subsection{Heatcapacity for $T<\Tc$}
\newcommand{\Kd}{\ensuremath{\mathcal{K}_d}}
As said before, the energy of the system is given by
\eqref{eq:4_Omega} and \eqref{eq:4_E_Omega_PV} to be
\begin{equation}
E=\frac{d}{2}\frac{g_sT\V}{\lambda^d}g_{\D{2}+1}(z)
=T^{\D{2}+1}\frac{d g_s\V}{2\Lambda^d}g_{\D{2}+1}(z)
\end{equation}
regardless of whether $T$ is above or below $\Tc$ (see
footnote~\ref{ftn:E}). Therefore the heatcapacity for $T<\Tc$ is given by
\begin{equation}\label{eq:4b_Cv}
C_\V=\qty(\pdv{E(z=0)}{T})_\V
=T^{\D{2}}\frac{d}{2}\qty(\D{2}+1)\frac{g_s\V}{\Lambda^d}\zeta(\nD{2}+1)
\qcomma T<\Tc.
\end{equation}
\qed


\subsection{Overall behaviour of $C_\V$}
\begin{figure}\centering
\resizebox{.5\textwidth}{!}{\input{figures/Cv.pdf_t}}
\caption{A sketch of the heatcapacity of a bose gas in $d$
  dimensions. There is a cusp at $T=\Tc$ where $C_\V$ reaches it's
  maximum value. }
\label{fig:Cv}
\end{figure}

To get the rest of the behaviour of the heatcapacity we need to
differentiate $E$ again, but now $z$ is no longer constanly 1. This
complicates the matter some what. And since we're olny interested in
the overall behaviour, not an exact expression, we can settle for just
studying the expansion for small  $z$.

The energy is 
\begin{equation}\label{eq:4e_E}
E=\frac{d}{2}\frac{g_sT\V}{\lambda^d}g_{\D{2}+1}(z)
=\frac{d}{2}TN\frac{g_{\D{2}+1}(z)}{g_{\D{2}}(z)}.
\end{equation}
For small values of $z$, corresponding to high
$T$, we have
\begin{equation}
g_\nu(z)=z+\frac{z^2}{2^\nu}+\frac{z^3}{3^\nu}+\ldots
\end{equation}
Which makes the Energy become
\begin{equation}
E\approx 
\frac{d}{2}NT\frac{z}{z}
= \frac{d}{2}NT
\end{equation}
And thus te heat capacity in the high temperature limit is
\begin{equation}\label{eq:4_Cv_infty}
C_\V(T\to\infty)=\frac{d}{2}N.
\end{equation}

For $T<\Tc$ we have \eqref{eq:4b_Cv}, which tells us that 
$C_\V\propto T^\D{2}$. We also get a maximum value at $T=\Tc$ of
\begin{equation}
\begin{aligned}\label{eq:4_Cv_max}
C_\V(T=\Tc)=&
T_c^{\D{2}}\frac{d}{2}\qty(\D{2}+1)\frac{g_s\V}{\Lambda^d}\zeta(\nD{2}+1)\\
=&\qty[\frac{N}{\V}\frac{\Lambda^d}{g_s\zeta(\nD{2})}]
\frac{d}{2}\qty(\D{2}+1)\frac{g_s\V}{\Lambda^d}\zeta(\nD{2}+1)\\
=&N\frac{d}{2}\qty(\D{2}+1)
\frac{\zeta(\nD{2}+1)}{\zeta(\nD{2})}>C_\V(T\to\infty).
\\
\end{aligned}
\end{equation}

Lastly ther sre some issues regarding continuity of $C_\V$, but
unfortunately I don't have much more time, so I leav that bit
out. It's not very complicated though. Just differentiate
\eqref{eq:4e_E} and note that the end result can be written as a
differnce between ratios of $g_\nu(z)$'s, and then take the limit
$z\to1$. 

Anyway the sketch of the heatcapacity is in \figref{fig:Cv}.

\subsection{Maximum of $C_\V(T)/C_\V(t\to\infty)$}
The maximum value of $C_\V$ is given by \eqref{eq:4_Cv_max} and the
high temperature limit value is given by
\eqref{eq:4_Cv_infty}. Whereby the maximum ratio is
\begin{equation}
\max\qty{\frac{C_\V(T)}{C_\V(t\to\infty)}}=
\qty(\D{2}+1)\frac{\zeta(\nD{2}+1)}{\zeta(\nD{2})}.
\end{equation}
In the specific case of $d=3$ we get 
\begin{equation}
\eval{\max\qty{\frac{C_\V(T)}{C_\V(t\to\infty)}}}_{d=3}=
\frac{5}{2}\frac{\zeta(\nicefrac{5}{2})}{\zeta(\nicefrac{3}{2})}
\approx 1.28.
\end{equation}
\qed


\end{document}





%% På svenska ska citattecknet vara samma i både början och slut.
%% Använd två apostrofer (två enkelfjongar): ''.


%% Inkludera PDF-dokument
\includepdf[pages={1-}]{filnamn.pdf} %Filnamnet får INTE innehålla 'mellanslag'!

%% Figurer inkluderade som pdf-filer
\begin{figure}\centering
\centerline{ %centrerar även större bilder
\includegraphics[width=1\textwidth]{filnamn.pdf}
}
\caption{}
\label{fig:}
\end{figure}

%% Figurer inkluderade med xfigs "Combined PDF/LaTeX"
\begin{figure}\centering
\resizebox{.8\textwidth}{!}{\input{filnamn.pdf_t}}
\caption{}
\label{fig:}
\end{figure}

%% Figurer roterade 90 grader
\begin{sidewaysfigure}\centering
\centerline{ %centrerar även större bilder
\includegraphics[width=1\textwidth]{filnamn.pdf}
}
\caption{}
\label{fig:}
\end{sidewaysfigure}


%%Om man vill lägga till något i TOC
\stepcounter{section} %Till exempel en 'section'
\addcontentsline{toc}{section}{\Alph{section}\hspace{8 pt}Labblogg} 

