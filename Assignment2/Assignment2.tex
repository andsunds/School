\documentclass[11pt,letter, swedish, english
]{article}
\pdfoutput=1

\usepackage{../custom_as}

\swapcommands{\Omega}{\varOmega}

%%Drar in tabell och figurtexter
\usepackage[margin=10 pt]{caption}
%%För att lägga in 'att göra'-noteringar i texten
\usepackage{todonotes} %\todo{...}

%%För att själv bestämma marginalerna. 
\usepackage[
%            top    = 3cm,
%            bottom = 3cm,
%            left   = 3cm, right  = 3cm
]{geometry}

%%För att ändra hur rubrikerna ska formateras
\renewcommand{\thesubsection}{\arabic{section} (\alph{subsection})}

\renewcommand{\thesubsubsection}{\arabic{section} (\alph{subsection},\,\roman{subsubsection})}

% \newcommand{\cbox}[2][cyan]
% {\mathchoice
% 	{\setlength{\fboxsep}{0pt}\colorbox{#1}{$\displaystyle#2$}}
% 	{\setlength{\fboxsep}{0pt}\colorbox{#1}{$\textstyle#2$}}
% 	{\setlength{\fboxsep}{0pt}\colorbox{#1}{$\scriptstyle#2$}}
% 	{\setlength{\fboxsep}{0pt}\colorbox{#1}{$\scriptscriptstyle#2$}}
% }
% \newcommand{\grande}{\cbox{\phantom{\frac{1}{xx}}}}

\renewcommand{\thefootnote}{\fnsymbol{footnote}}

\begin{document}

%%%%%%%%%%%%%%%%% vvv Inbyggd titelsida vvv %%%%%%%%%%%%%%%%%
% \begin{titlepage}
\title{Statistical Physics -- PHYS\,704 \\
Assignment 2}
\author{Andréas Sundström}
\date{\today}

\maketitle

%%%%%%%%%%%%%%%%% ^^^ Inbyggd titelsida ^^^ %%%%%%%%%%%%%%%%%

%Om man vill ha en lista med vilka todo:s som finns.
%\todolist

\section{Compressibility and heatcapacity of ideal Fermi gases}
\subsection{Isothermal and adiabatic compressibility}
In this problem we want to show that
\begin{equation}
\kappa_T = k_T:=\frac{1}{nT}\frac{f_{1/2}(z)}{f_{3/2}(z)}
\quad\text{ and }\quad
\kappa_S = k_S:=\frac{3}{5nT}\frac{f_{3/2}(z)}{f_{5/2}(z)}.
\end{equation}


\subsubsection{Isothermal compressibility}
We begin by maniplulating the supposed expression for the isothermal
compressibility
\begin{equation}\label{eq:1_want_this}
k_T=\frac{1}{nT}\frac{f_{1/2}(z)}{f_{3/2}(z)}
= \frac{V}{NT}\frac{f_{1/2}(z)}{f_{3/2}(z)}
= \frac{1}{P}\frac{f_{1/2}(z)f_{5/2}(z)}{[f_{3/2}(z)]^2},
\end{equation}
where we have used the thermodynamic relation for Fermi gases
\begin{equation}\label{eq:1_PVNT}
\frac{PV}{NT}=\frac{f_{5/2}(z)}{f_{3/2}(z)}.
\end{equation}

To find the compressibility we just use the definition
\begin{equation}
\kappa_T:=-\frac{1}{V}\qty(\pdv{V}{P})_T
= -\frac{1}{V} \qty(\pdv{P} \frac{NT}{P}\frac{f_{5/2}(z)}{f_{3/2}(z)})_T,
\end{equation}
where we once again used \eqref{eq:1_PVNT} to express $V$ in the
derivative. Now, since $T$ is constant we get
\begin{equation}\label{eq:1_k_T_2}
\kappa_T
= -\frac{NT}{V} \qty[
\frac{-1}{P^2}\frac{f_{5/2}(z)}{f_{3/2}(z)}
+\frac{1}{P}\qty(\pdv{P}
\frac{f_{5/2}(z)}{f_{3/2}(z)}
)_T].
\end{equation}

Now we have to evauate the derivative of the special functions. To
begin with we note that
\begin{equation}
\qty(\pdv{f_\nu(z)}{\mu})_T=\qty(\pdv{z}{\mu})_T\pdv{f_\nu(z)}{z}
= \frac{1}{T}z\pdv{f_\nu(z)}{z}=\frac{1}{T}f_{\nu-1}(z).
\end{equation}
This follows from the fact that $z=\ee^{\mu/T}$ and the relation
\begin{equation}
z\pdv{f_\nu(z)}{z} = f_{\nu-1}(z).
\end{equation}
Next we have to evaluate
\begin{equation}
\qty(\pdv{\mu}{P})_T.
\end{equation}
To do this we remind ourselves that
\begin{equation}
PV=-\Omega=\frac{g_sVT}{\lambda^3}f_{5/2}(z)
\end{equation}
for Fermi gases. In other words we now have
\begin{equation}
\qty(\pdv{P}{\mu})_T 
=\qty(\pdv{\mu} \frac{g_sT}{\lambda^3}f_{5/2}(z))_T 
=\frac{g_sT}{\lambda^3} \,\frac{1}{T}f_{3/2}(z) 
=\frac{g_s}{\lambda^3} f_{3/2}(z),
\end{equation}
since $T$ is constant. 
The last piec needed is the fact that
\begin{equation}
n=\frac{N}{V}=-\frac{1}{V}\qty(\pdv{\Omega}{\mu})_T
=\frac{g_s}{\lambda^3} f_{3/2}(z)=\qty(\pdv{P}{\mu})_T.
\end{equation}

Finally we can continue from \eqref{eq:1_k_T_2} to get
\begin{equation}
\begin{aligned}
\kappa_T &= \frac{NT}{VP} \qty[
\frac{1}{P}\frac{f_{5/2}(z)}{f_{3/2}(z)}
-\qty(\pdv{\mu}{P})_T\qty(\pdv{\mu}\frac{f_{5/2}(z)}{f_{3/2}(z)})_T
]\\
&= \frac{nT}{P} \qty[
\frac{1}{P}\frac{f_{5/2}(z)}{f_{3/2}(z)}
-\frac{1}{n}\qty(\frac{1}{T}\frac{f_{3/2}(z)}{f_{3/2}(z)}
-\frac{1}{T}\frac{f_{5/2}(z)f_{1/2}(z)}{[f_{3/2}(z)]^2}
)]\\
&= \frac{nT}{P} \qty[
\frac{1}{P}\frac{f_{5/2}(z)}{f_{3/2}(z)}
-\frac{1}{nT}
+\frac{1}{nT}\frac{f_{5/2}(z)f_{1/2}(z)}{[f_{3/2}(z)]^2}
].
\end{aligned}
\end{equation}
From here we once agian use \eqref{eq:1_PVNT} to get
\begin{equation}
\frac{1}{P}=\frac{1}{nT}\frac{f_{3/2}(z)}{f_{5/2}(z)}.
\end{equation}
This gives
\begin{equation}
\kappa_T = \frac{nT}{P} \qty[
\frac{1}{nT}
-\frac{1}{nT}
+\frac{1}{nT}\frac{f_{5/2}(z)f_{1/2}(z)}{[f_{3/2}(z)]^2}
]
=\frac{1}{P}\frac{f_{5/2}(z)f_{1/2}(z)}{[f_{3/2}(z)]^2},
\end{equation}
which is exactly what we had in \eqref{eq:1_want_this}. 
\qed


\subsubsection{Adiabatic compressibility}


\subsection{Relationship between isobaric and isochoric heatcapacity}




\section{Cemical potential of an ideal Fermi gas}



\section{Speed of sound in an idela Bose gas}



\section{Regarding ideal Bose gases }






\end{document}





%% På svenska ska citattecknet vara samma i både början och slut.
%% Använd två apostrofer (två enkelfjongar): ''.


%% Inkludera PDF-dokument
\includepdf[pages={1-}]{filnamn.pdf} %Filnamnet får INTE innehålla 'mellanslag'!

%% Figurer inkluderade som pdf-filer
\begin{figure}\centering
\centerline{ %centrerar även större bilder
\includegraphics[width=1\textwidth]{filnamn.pdf}
}
\caption{}
\label{fig:}
\end{figure}

%% Figurer inkluderade med xfigs "Combined PDF/LaTeX"
\begin{figure}\centering
\resizebox{.8\textwidth}{!}{\input{filnamn.pdf_t}}
\caption{}
\label{fig:}
\end{figure}

%% Figurer roterade 90 grader
\begin{sidewaysfigure}\centering
\centerline{ %centrerar även större bilder
\includegraphics[width=1\textwidth]{filnamn.pdf}
}
\caption{}
\label{fig:}
\end{sidewaysfigure}


%%Om man vill lägga till något i TOC
\stepcounter{section} %Till exempel en 'section'
\addcontentsline{toc}{section}{\Alph{section}\hspace{8 pt}Labblogg} 

