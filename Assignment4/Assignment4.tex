\documentclass[11pt,letter, swedish, english
]{article}
\pdfoutput=1

\usepackage{../custom_as}
\usepackage[makeroom]{cancel}


\renewcommand{\thesubsection}{\arabic{section} (\alph{subsection})}

\renewcommand{\thesubsubsection}{\arabic{section} (\alph{subsection},\,\roman{subsubsection})}


%%Drar in tabell och figurtexter
\usepackage[margin=10 pt]{caption}
%%För att lägga in 'att göra'-noteringar i texten
\usepackage{todonotes} %\todo{...}

%%För att själv bestämma marginalerna. 
\usepackage[
%            top    = 3cm,
%            bottom = 3cm,
%            left   = 3cm, right  = 3cm
]{geometry}

\DeclareMathAlphabet{\mathpzc}{OT1}{pzc}{m}{it}
\newcommand{\oh}{\ensuremath\mathpzc{o}}

\newcommand{\as}{\qcomma\text{as }}

\renewcommand{\thefootnote}{\fnsymbol{footnote}}

\begin{document}

%%%%%%%%%%%%%%%%% vvv Inbyggd titelsida vvv %%%%%%%%%%%%%%%%%
% \begin{titlepage}
\title{Asymptotic Analasys and Pertubation Theory -- AMATH\,732 \\
Assignment 4}
\author{Andréas Sundström}
\date{\today}

\maketitle

%%%%%%%%%%%%%%%%% ^^^ Inbyggd titelsida ^^^ %%%%%%%%%%%%%%%%%

%Om man vill ha en lista med vilka todo:s som finns.
%\todolist

\section{An IVP}
Here we have
\begin{equation}\label{eq:1_IVP}
y''+(1-\epsilon x)y=0;\qquad
y(0)=1,\quad y'(0)=0
\end{equation}
where $0<\epsilon\ll1$ and $0\le x<\infty$. We want to find the RPT
solution. 

As usual we begin by assuming
\begin{equation}
y(x, \epsilon)=y_0(x) + \epsilon y_1(x) + \epsilon^2 y_2(x)+\ldots,
\end{equation}
which, after collection of like powers of $\epsilon$ gives
\begin{equation}
\qty(y_0''+y) +\epsilon\qty(y_1''+y_1-xy_0) 
+\epsilon^2\qty(y_2''+y_2-xy_1) +\ldots=0.
\end{equation}

The order 1 problem is just
\begin{equation}
y_0''+y_0=0;\qquad
y_0(0)=1,\quad y_0'(0)=0,
\end{equation}
which has the solution
\begin{equation}
y_0(x)=\cos(x).
\end{equation}

The next order problem thus becomes
\begin{equation}
y_1''+y_1=xy_0=x\cos(x)
\end{equation}
with initial conditions $y_1(0)=0$ and $y_1'(0)=0$. Here we have the
homogeneous solution
\begin{equation}
y_1^{(\text{h})}(x)=A_1\cos(x)+B_1\sin(x),
\end{equation}
and the particular solution should have the form
\begin{equation}
y_1^{(\text{p})}(x)=\qty(a_1x+a_1'x^2)\cos(x)+\qty(b_1x+b_1'x^2)\sin(x).
\end{equation}
Alternatively you could use \textit{Mathematica} (which I did), or
some other computer program, to find that the particular solution is 
\begin{equation}
y_1^{(\text{p})}(x)=\frac{1}{8}\qty[
2x\cos(x)+\qty(-1+2x^2)\sin(x)].
\end{equation}
Then we invoke the initial conditions. For $y(0)=0$, we see that
$y_1^{(\text{p})}(0)=0$ so $A_1=0$. For the first derivative it's easy
to see that the only contribution from ${y_1'}^{(\text{p})}(0)$ must
come from the $x\cos(x)$ and $-sin(x)$ terms in $y_1^{(\text{p})}$,
and we get
\begin{equation}
B_1+\frac{1}{4}-\frac{1}{8}=0
\quad\Longrightarrow\quad
B_1=-\frac{1}{8}.
\end{equation}
In other words the final solution to the order $\epsilon^1$ problem is
\begin{equation}
\begin{aligned}
y_1(x)=&\frac{1}{8}\qty[2x\cos(x)+\qty(-1+2x^2)\sin(x)-\sin(x)]\\
=&\frac{1}{4}\qty[x\cos(x)+\qty(-1+x^2)\sin(x)].
\end{aligned}
\end{equation}

Already we can see that this is not going very well. The
$\order{\epsilon}$ solution, $y_1$, grows and blows up as
$x\to\infty$, whereas $y_0$ is bounded. This means that the RPT
solution is \emph{not} uniformly ordered. The exact solution to
\eqref{eq:1_IVP} will however not be bounded, since for
$x>\epsilon^{-1}$ we get exponential behavior. The problem of
non-uniformity is however that the approximation that a truncated
expansion does not hold for $x\gg\epsilon^{-1}$. 






\section{Underdamped linear damped oscillator}
Here we have the underdamped oscillator
\begin{equation}
\ddot{x}+2\epsilon\dot{x}+x=0;\qquad
x(0)=1,\quad \dot{x}(0)=0,
\end{equation}
with $0<\epsilon\ll1$.


\subsection{Introducing new time scales}
We begin by introducing two new, slow, time scales $\tau_1=\epsilon x$
and $\tau_2=\epsilon^2 x$. We now have
\begin{equation}
x(t)=f(t, \tau_1, \tau_2, \epsilon),
\end{equation}
where we treat $t$, $\tau_1$ and $\tau_2$ as independent variables. 
As such the total derivative must change:
\begin{equation}
\dv{t}=\pdv{t}{t}\pdv{t}+\pdv{\tau_1}{t}\pdv{\tau_1}
+\pdv{\tau_2}{t}\pdv{\tau_2}
=\pdv{t}+\epsilon\pdv{\tau_1}+\epsilon^2\pdv{\tau_2}. 
\end{equation}

With this information the $t$ derivatives of $x$ becomes
\begin{equation}
\dot{x}=\dv{t}f
=f_t + \epsilon f_{\tau_1}+\epsilon^2 f_{\tau_2}
\end{equation}
and
\begin{equation}
\begin{aligned}
\ddot{x}=\dv{t}\qty[\dv{t}f]
=&f_{tt}+\epsilon f_{t,\tau_1}+\epsilon^2 f_{t,\tau_2}
+\epsilon\qty(
f_{\tau_1,t}+\epsilon f_{\tau_1,\tau_1}+\epsilon^2f_{\tau_1,\tau_2})
\\ 
&+\epsilon^2\qty(
f_{\tau_2,t}+\epsilon f_{\tau_2,\tau_1}+\epsilon^2f_{\tau_2,\tau_2})
\\
=&f_{tt} + 2\epsilon f_{t,\tau_1} 
+ \epsilon^2 f_{\tau_1,\tau_1} + 2\epsilon^2 f_{t,\tau_2}
+\order{\epsilon^3}.
\end{aligned}
\end{equation}

Thus the original DE now becomes
\begin{equation}\label{eq:2_DE_f}
\qty(f_{tt} + 2\epsilon f_{t,\tau_1} 
+ \epsilon^2 f_{\tau_1,\tau_1} + 2\epsilon^2 f_{t,\tau_2})
+2\epsilon\qty(f_t + \epsilon f_{\tau_1})
+f+\order{\epsilon^3}=0.
\end{equation}
And the initial conditions are
\begin{equation}\label{eq:2_init}
f(0, 0, 0, \epsilon)=1
\end{equation}
and
\begin{equation}\label{eq:2_init_d0}
f_t(0, 0, 0, \epsilon) + \epsilon f_{\tau_1}(0, 0, 0,
\epsilon)+\epsilon^2 f_{\tau_2}(0, 0, 0, \epsilon)
=0
\end{equation}
for all $\epsilon$. 

% That it's for all $\epsilon$ means that
% \begin{equation}\label{eq:2_init_d}
% f_t(0, 0, 0, \epsilon) = f_{\tau_1}(0, 0, 0, \epsilon)
% = f_{\tau_2}(0, 0, 0, \epsilon)
% =0.
% \end{equation}



\subsection{Leading order problem}
Now we continue on to expanding
\begin{equation}
f(t, \tau_1, \tau_2, \epsilon)=
f_0(t, \tau_1, \tau_2, \epsilon) 
+\epsilon f_1(t, \tau_1, \tau_2, \epsilon) 
+\epsilon^2 f_2(t, \tau_1, \tau_2, \epsilon)+\ldots 
\end{equation}
The $\order{1}$ problem
\begin{equation}
f_{0\;t,t}+f_0=0
\end{equation}
does look a lot like the original leading order problem. But with one
important difference: this is a \emph{partial} DE, meaning that
\begin{equation}\label{eq:2_f0}
f_0=A_0(\tau_1, \tau_2)\cos(t)+B_0(\tau_1, \tau_2)\sin(t),
\end{equation}
but here $A_0$ and $B_0$ now depends on $\tau_1$ and $\tau_2$.

With the initial conditions we get:
\begin{equation}\label{eq:2b_init_A}
\eqref{eq:2_init}\quad\Longrightarrow\quad
A_0(0, 0)=1.
\end{equation}
But with \eqref{eq:2_init_d0} we run into trouble, since
with the expansion \eqref{eq:2_init_d0} becomes
\begin{equation}\label{eq:2_init_d}
\qty[f_{0\,t}+\epsilon\Big(f_{0\;\tau_1}+f_{1\;t} \Big)
+\epsilon^2\Big(f_{0\;\tau_2}+f_{1\;\tau_1}+f_{2\;t} \Big)
+\order{\epsilon^3}]_{t=\tau_1=\tau_2=0}=0.
\end{equation}
This is true for all $\epsilon$ and since $f_i$ does not depend on
$\epsilon$ we must have that all coefficients must be 0. Therefore all
we can say so far is that
\begin{equation}
f_{0\;t}(0, 0)=0, 
\end{equation}
which gives
\begin{equation}\label{eq:2b_init_B}
B_0(0, 0)=0.
\end{equation}

\subsection{First solvability condition}
For the $\order{\epsilon}$ problem \eqref{eq:2_DE_f} gives
\begin{equation}\label{eq:2c_f1_0}
f_{1\;t,t}+f_1 + 2f_{0\;t,\tau_1} +2f_{0\;t}=0
\end{equation}
or in other words
\begin{equation}\label{eq:2c_f1_1}
\begin{aligned}
f_{1\;t,t}+f_1 = -2f_{0\;t,\tau_1} -2f_{0\;t}
=&2A_0(\tau_1, \tau_2)\sin(t) - 2B_0(\tau_1, \tau_2)\cos(t)\\
&+ 2A_{0\;\tau_1}(\tau_1, \tau_2)\sin(t) 
- 2B_{0\;\tau_1}(\tau_1, \tau_2)\cos(t).
\end{aligned}
\end{equation}

To eliminate secular terms we must have
\begin{equation}
\begin{cases}
A_{0\;\tau_1}+A_0=0\\
B_{0\;\tau_1}+B_0=0,
\end{cases}
\end{equation}
which of course have the solutions
\begin{equation}
A_0(\tau_1, \tau_2)=\tilde{A}_0(\tau_2)\ee^{-\tau_1}
\end{equation}
and
\begin{equation}
B_0(\tau_1, \tau_2)=\tilde{B}_0(\tau_2)\ee^{-\tau_1}.
\end{equation}
The initial conditions for $\tilde{A}_0$ and $\tilde{B}_0$ are
\begin{equation}\label{eq:2_init_tildeA}
\eqref{eq:2b_init_A}\quad\Longrightarrow\quad
\tilde{A}_0(0)=1,
\end{equation}
and
\begin{equation}\label{eq:2_init_tildeB}
\eqref{eq:2b_init_B}\quad\Longrightarrow\quad
\tilde{B}_0(0)=0.
\end{equation}

With the secular terms eliminated we're left with
\begin{equation}
f_{1\;t,t}+f_1 =0
\end{equation}
which has the solution
\begin{equation}\label{eq:2_f1}
f_1=A_1(\tau_1, \tau_2)\cos(t)+B_1(\tau_1, \tau_2)\sin(t),
\end{equation}
much like $f_0$. The initial conditions here are
\begin{equation}
f_1(0, 0, 0)=0\quad\Longleftrightarrow\quad
A_1(0, 0)=0,
\end{equation}
and
\begin{equation}
\begin{aligned}
f_{1\;t}=-f_{0\;\tau_1}
\quad\Longleftrightarrow\quad
B_1(0, 0) =& \qty[+\ee^{-\tau_1}\qty(
\tilde{A}_0(\tau_2)\cos(t)+\tilde{B}_0(\tau_2)\sin(t)
)]_{t=\tau_1=\tau_2=0}\\
=&\tilde{A}_0(0)=1.
\end{aligned}
\end{equation}

\subsection{Second solvability condition}
Now to the $\order{\epsilon^2}$ problem. From \eqref{eq:2_DE_f} we get
\begin{equation}
f_{2\;t,t}+f_2 + 2f_{1\;t,\tau_1} +2f_{1\;t}
+f_{0\;\tau_1,\tau_1}+2f_{0\;t,\tau_2}+2f_{0\;\tau_1}
=0.
\end{equation}
This equation has a resemblance to \eqref{eq:2c_f1_0}, only with
$f_i\to f_{i+1}$ and some new terms with $f_0$'s. We also know that
$f_1$ has the same form as $f_0$, c.f. \eqref{eq:2_f0} and
\eqref{eq:2_f1}. This means that we can eliminate the secular terms
coming from the $f_1$ terms in the same way as in
\eqref{eq:2c_f1_1}. 

All that we're left with is
\begin{equation}
\begin{aligned}
f_{2\;t,t}+f_2 =& %2f_{1\;t,\tau_1} +2f_{1\;t}
-f_{0\;\tau_1,\tau_1}-2f_{0\;t,\tau_2}-2f_{0\;\tau_1}\\
=& -\ee^{-\tau_1}
\qty[\tilde{A}_0(\tau_2)\cos(t)+\tilde{B}_0(\tau_2)\sin(t)]\\
&-2\ee^{-\tau_1}
\qty[-\tilde{A}_{0\;\tau_2}(\tau_2)\sin(t)+\tilde{B}_{0\;\tau_2}(\tau_2)\cos(t)]\\
&+2\ee^{-\tau_1}
\qty[\tilde{A}_{0}(\tau_2)\cos(t)+\tilde{B}_0(\tau_2)\sin(t)].
\end{aligned}
\end{equation}
Eliminating the secular terms means that the coefficients before
$\cos(t)$ and $\sin(t)$ must be 0. That is
\begin{equation}
\begin{cases}
1\tilde{A}_{0}-2\tilde{B}_{0}'=0\\
1\tilde{B}_{0}+2\tilde{A}_{0}'=0,
\end{cases}
\end{equation}
where ``prime'' now denotes $\dv*{\tau_2}$ since $\tilde{A}_0$ and
$\tilde{B}_0$ only depends on $\tau_2$.
From this we get
\begin{equation}
\tilde{A}_{0}'=2\tilde{B}_{0}''
\end{equation}
which when substituted into the second equation gives
\begin{equation}
\tilde{B}_0+4\tilde{B}_0''=0.
\end{equation}
This DE has the solution
\begin{equation}
\tilde{B}_0=a\cos(\frac{\tau_2}{2})+b\sin(\frac{\tau_2}{2}).
\end{equation}
Now the initial condition \eqref{eq:2_init_tildeB} gives $a=0$. We
also have
\begin{equation}
\tilde{A}_{0}=2\tilde{B}_{0}'=b\cos(\frac{\tau_2}{2}).
\end{equation}
And the initial condition \eqref{eq:2_init_tildeA} gives $b=1$.

Finally the full expression for $f_0$ is
\begin{equation}
\begin{aligned}
f_0(t, \tau_1, \tau_2)=&A_0(\tau_1, \tau_2)\cos(t)+B_0(\tau_1, \tau_2)\sin(t)\\
=&\ee^{-\tau_1}
\qty[\tilde{A}_0(\tau_2)\cos(t)+\tilde{B}_0(\tau_2)\sin(t)]\\
=&\ee^{-\tau_1}
\qty[\cos(\frac{\tau_2}{2})\cos(t)+\sin(\frac{\tau_2}{2})\sin(t)]
=\ee^{-\tau_1}\cos(t-\frac{\tau_2}{2}).
\end{aligned}
\end{equation}
Or, in only terms of $t$:
\begin{equation}
x_0(t)=\ee^{-\epsilon t}\cos(\qty(1-\epsilon^2/2)t).
\end{equation}
This can be compared to the exact solution (found using
\textit{Mathematica's} \texttt{DSolve}) 
\begin{equation}
\begin{aligned}
x_\text{exact}(t)=&
\frac{\ee^{-\epsilon t}}{\sqrt{1-\epsilon^2}} \qty[
\sqrt{1-\epsilon^2}\cos(t\sqrt{1-\epsilon^2})
+\epsilon\sin(t\sqrt{1-\epsilon^2})
]\\ \approx&
\ee^{-\epsilon t}\qty[
\cos(t\qty(1-\epsilon^2/2))
].
\end{aligned}
\end{equation}
The approximation comes from discarding all $\order{\epsilon}$ terms
outside the cosine and the doing a first order Taylor expansion of
$\sqrt{1-\epsilon^2}$. 


\section{Free Duffing oscillator}
\renewcommand{\thesubsection}{\arabic{section} (\roman{subsection})}

In this problem we're going to study
\begin{equation}
x''+x=\epsilon {x}^3;\qquad
x(0)=1,\quad {x'}(0)=0,
\end{equation}
using Pritulo's technique.

\subsection{RPT}
The first step in Pritulo's method is to find the (disordered)
RPT. This is done in the usual way by expanding 
$x(t, \epsilon)=x_0(t)+\epsilon x_1(t)+\ldots$ and
substituting into the DE:
\begin{equation}
\begin{aligned}
x_0''+x_0 + \epsilon\qty(x_1''+x_1)
+\ldots
=&\epsilon\qty(x_0+\epsilon x_1+ \ldots)^3\\
=&\epsilon\qty(x_0^3+ \ldots).
\end{aligned}
\end{equation}

The $\order{1}$ problem is 
\begin{equation}
x_0''+x_0=0;\qquad
x_0(0)=1,\quad {x_0'}(0)=0,
\end{equation}
which has the solution
\begin{equation}
x_0(t)=\cos(t).
\end{equation}

Then the $\order{\epsilon}$ problem becomes
\begin{equation}
x_1''+x_1=x_0^3=\cos^3(t)=\frac{1}{4}
\Big[3\cos(t)+\cos(3t)\Big].
\end{equation}
This DE will have a particular solution of the form
\begin{equation}
x_1^\text{(p)}(t)=a_1\cos(3t)+b_1t\sin(t)
\end{equation}
and a homogeneous solution of the form
\begin{equation}
x_1^\text{(h)}(t)=A_1\cos(t)+B_1\sin(t).
\end{equation}
By using \textit{Mathematica} we get
\begin{equation}
x_1(t)=\frac{1}{32}\Big[\cos(t)-\cos(3t) \Big] + \frac{3}{8}t\sin(t).
\end{equation}

The expression for $x$ is now
\begin{equation}\label{eq:3i_x}
x(t,\epsilon)=\cos(t)+
\frac{\epsilon}{32}\Big[\cos(t)-\cos(3t)+ 12t\sin(t) \Big]
+\mathcal{O}_\text{F}\qty(\epsilon^2).
\end{equation}

\subsection{Change of variables}
Next up in Pritulo's technique is to set
\begin{equation}
t(z)=z+\epsilon t_1(z)+\ldots
\end{equation}
and setting
\begin{equation}
\tilde{x}(z)=x(t).
\end{equation}
Then we Taylor expand (assumed to work) \eqref{eq:3i_x} around $z$:
\begin{equation}
\begin{aligned}
\tilde{x}(z,\epsilon)=&\qty{\cos(z)-\sin(z)\,\epsilon t_1
+\order*{\epsilon^2}}\\
&+\frac{\epsilon}{32}\Big[\cos(z)-\cos(3z)+ 12t\sin(z)
+\order*{\epsilon} \Big]
+\mathcal{O}_\text{F}\qty(\epsilon^2)\\
=&\cos(z)
+\frac{\epsilon}{32}\Big[\cos(z)-\cos(3z)+ (12t-32t_1)\sin(z)
\Big]
+\mathcal{O}_\text{F}\qty(\epsilon^2).
\end{aligned}
\end{equation}
Now to eliminate the singularity as $t\to\infty$, we much choose
\begin{equation}
t_1(z)=\frac{3}{8}z.
\end{equation}
Giving
\begin{equation}
\tilde{x}(z,\epsilon)=\cos(z)
+\frac{\epsilon}{32}\Big[\cos(z)-\cos(3z)\Big]
+\mathcal{O}_\text{F}\qty(\epsilon^2).
\end{equation}



\section{Another oscillator}
\renewcommand{\thesubsection}{\arabic{section} (\alph{subsection})}
Here we are going to study
\begin{equation}
y''+y=\epsilon y^2;\qquad
y(0)=\alpha,\quad y'(0)=0.
\end{equation}

\subsection{RPT}
First off is the RPT, and as usual 
$y(t,\epsilon)=y_0(t)+\epsilon y_1(t)+\epsilon^2y_2(t)+\ldots$, and
the DE becomes
\begin{equation}
\begin{aligned}
y_0''+y_0+\epsilon\qty(y_1''+y_1)+\epsilon^2\qty(y_2''+y_2)
+\ldots=&\epsilon\qty(y_0+\epsilon y_1+\ldots)^2\\
=&\epsilon\qty(y_0^2+2\epsilon y_0y_1+\ldots).
\end{aligned}
\end{equation}

The $\order{1}$ problem is 
\begin{equation}
y_0''+y_0=0;\qquad
y_0(0)=\alpha,\quad {y_0'}(0)=0,
\end{equation}
which has the solution
\begin{equation}
y_0(t)=\alpha\cos(t).
\end{equation}

Next the $\order{\epsilon}$ problem becomes
\begin{equation}
y_1''+y_1=y_0^2=\alpha^2\cos^2(t)=
\frac{\alpha^2}{2}\Big[1+\cos(2t)\Big],
\end{equation}
which doesn't have any secular terms. With
\textit{Mathematica}\footnotemark{} we get 
\begin{equation}
y_1(t)=\frac{\alpha^2}{2}
\qty[1-\frac{2}{3}\cos(t)-\frac{1}{3}\cos(2t)].
\end{equation}

\footnotetext{By now, I feel like we've done enough particular and
  homogeneous solutions to theses trigonometric DE's to just let Mathematica
  deal with them all the way. }

Lastly the $\order{\epsilon^2}$ problem becomes
\begin{equation}
\begin{aligned}
y_2''+y_2=&2y_0y_1=\alpha^3\cos(t)\qty[1-\frac{2}{3}\cos(t)-\frac{1}{3}\cos(2t)]
\\=&\frac{\alpha^3}{6}\qty[-2+5\cos(t)-2\cos(2t)-\cos(3t)].
\end{aligned}
\end{equation}
And with \textit{Mathematica} we get
\begin{equation}
y_2(t)=\frac{\alpha^3}{144}
\qty[-48+29\cos(t)+16\cos(2t)+3\cos(3t)+60t\sin(t)\Big.].
\end{equation}

Together we get
\begin{equation}\label{eq:4_RPT}
\begin{aligned}
y(t, \epsilon)=&\alpha\cos(t)+\epsilon\frac{\alpha^2}{2}
\qty[1-\frac{2}{3}\cos(t)-\frac{1}{3}\cos(2t)]\\
&+\epsilon^2\frac{\alpha^3}{144}
\qty[-48+29\cos(t)+16\cos(2t)+3\cos(3t)+60t\sin(t)\Big.]
+\mathcal{O}_\text{F}\qty(\epsilon^3).
\end{aligned}
\end{equation}
We also note that the solution is not uniformly ordered since we have
the $t\sin(t)$ term in $y_2$, meaning that after
$t=\order{\epsilon^{-1}}$, $\epsilon^2y_2$ would be in the same order
of magnitude as $\epsilon y_1$.


\subsection{The Poincaré-Lindstedt method}
\newcommand{\ty}{\ensuremath{\tilde{y}}}
Here we begin by introducing $\tau=\omega(\epsilon)t$, where
\begin{equation}\label{eq:4b_omega}
\omega(\epsilon)=1+\epsilon\omega_1+\epsilon^2\omega_2+\ldots,
\end{equation}
and $\ty(\tau)=y(t)$.
With this change of variables we also get
$y_{tt}=\omega^2\ty_{\tau\tau}$, and the DE becomes
\begin{equation}
\omega^2\ty_{\tau\tau} +\ty = \epsilon\ty^2.
\end{equation}
By expanding $\ty$ in powers of $\epsilon$ and using
\eqref{eq:4b_omega}, we get
\begin{equation}
\begin{aligned}
\qty(1+\epsilon\omega_1+\epsilon^2\omega_2+\ldots)^2
\qty(\ty_0''+\epsilon\ty_1''+\epsilon^2\ty_2''+\ldots)&\\
+
\qty(\ty_0+\epsilon\ty_1+\epsilon^2\ty_2+\ldots)
&=\epsilon\qty(\ty_0+\epsilon\ty_1+\ldots)^2
\end{aligned}
\end{equation}
which after collection like powers of $\epsilon$ is
\begin{equation}
\begin{aligned}
&\ty_0''+\ty_0+\epsilon\qty[\ty_1''+\ty_1+2\omega_1\ty_0'']
+\epsilon^2\qty[\ty_2''+\ty_2+2\omega_1\ty_1''+\qty(2\omega_2+\omega_1^2)\ty_0'']
+\ldots\\
=&\epsilon\qty(\ty_0^2+2\epsilon\ty_0\ty_1+\ldots).
\end{aligned}
\end{equation}

Now to the $\order{1}$ problem:
\begin{equation}
\ty_0''+\ty_0=0;\qquad
\ty_0(0)=\alpha,\quad {\ty_0'}(0)=0,
\end{equation}
which has the solution
\begin{equation}
\ty_0(\tau)=\alpha\cos(\tau).
\end{equation}
And then the $\order{\epsilon}$ problem:
\begin{equation}
\ty_1''+\ty_1=-2\omega_1\ty_0''+\ty_0^2
=+2\alpha\omega_1\cos(\tau)
+\frac{\alpha^2}{2}\Big[1+\cos(2\tau)\Big].
\end{equation}
Here we need to choose $\omega_1\equiv0$ to not get a secular
term. Then the DE for $\ty_1$ become identical to the DE for $y_1$ in
the previous part. Therefore
\begin{equation}
\ty_1(\tau)=\frac{\alpha^2}{2}
\qty[1-\frac{2}{3}\cos(\tau)-\frac{1}{3}\cos(2\tau)].
\end{equation}

Now the $\order{\epsilon^2}$ problem looks like
\begin{equation}
\begin{aligned}
\ty_2''+\ty_2=&2\ty_0\ty_1-\cancel{\omega_1\ty_1''}-\qty(2\omega_2+\cancel{\omega_1^2})\ty_0''\\
=&\alpha^3\cos(\tau)\qty[1-\frac{2}{3}\cos(\tau)-\frac{1}{3}\cos(2\tau)]+2\omega_2\cos(\tau)
\\=&\frac{\alpha^3}{6}\qty[-2+5\cos(\tau)-2\cos(2\tau)-\cos(3\tau)]+2\omega_2\cos(\tau).
\end{aligned}
\end{equation}
To remove the secular term we need
\begin{equation}
\frac{5\alpha^3}{6}+2\omega_2=0
\quad\Longrightarrow\quad
\omega_2=-\frac{5\alpha^3}{12}.
\end{equation}
And with \textit{Mathematica} we get
\begin{equation}
\ty_2(\tau)=\frac{\alpha^3}{144}
\qty[-48+29\cos(\tau)+16\cos(2\tau)+3\cos(3\tau)\Big.].
\end{equation}

To summarize we have
\begin{equation}\label{eq:4b_y}
\begin{aligned}
\ty(\tau, \epsilon)=&\alpha\cos(\tau)+\epsilon\frac{\alpha^2}{2}
\qty[1-\frac{2}{3}\cos(\tau)-\frac{1}{3}\cos(2\tau)]\\
&+\epsilon^2\frac{\alpha^3}{144}
\qty[-48+29\cos(\tau)+16\cos(2\tau)+3\cos(3\tau)\Big.]
+\mathcal{O}_\text{F}\qty(\epsilon^3),
\end{aligned}
\end{equation}
with
\begin{equation}\label{eq:4b_t}
\tau=\omega(\epsilon)t=\qty(1-\epsilon^2\frac{5\alpha^3}{12}+\order{\epsilon^3})t.
\end{equation}



\subsection{Pritulo's method}
\newcommand{\hy}{\ensuremath{\hat{y}}}
To use Pritulo's method we need the RPT solution, which we have in
\eqref{eq:4_RPT}. Then we expand
\begin{equation}
t(z)=z+\epsilon t_1(z)+\epsilon^2t_2(z)+\ldots,
\end{equation}
and set $\hy(z)=y(t)$. Then we Taylor \eqref{eq:4_RPT} expand around
$z$.  

Before we continue we can note that the $\order{\epsilon}$ term in
\eqref{eq:4_RPT} doesn't have any singularity, whereby $t_1(z)$ must be
constant. We can therefor choose $t_1=0$, simplifying the rest of the
calculations. 

We now get
\begin{equation}\label{eq:4c_y}
\begin{aligned}
\hy(z, \epsilon)=&\alpha\qty(\cos(z)-\sin(z)(\cancel{\epsilon t_1}+\epsilon^2t_2)+\order*{\epsilon^3})\\
&+\epsilon\frac{\alpha^2}{2}
\qty[1-\frac{2}{3}\cos(z)-\frac{1}{3}\cos(2z) + \order*{\epsilon^2}]\\
&+\epsilon^2\frac{\alpha^3}{144}
\qty[-48+29\cos(z)+16\cos(2z)+3\cos(3z)+60z\sin(z)+ \order*{\epsilon^2}]\\
&+\mathcal{O}_\text{F}\qty(\epsilon^3).
\end{aligned}
\end{equation}
To eliminate the (only) singular term we must set
\begin{equation}
-\alpha t_2+\frac{60\alpha^3}{144}z=0
\quad\Longrightarrow\quad
t_2=\frac{5\alpha^2}{12}.
\end{equation}
In other words 
\begin{equation}
\begin{aligned}
&t(z)=\qty(1+\epsilon^2\frac{5\alpha^2}{12})z+\order{\epsilon^3}\\
\quad\Longrightarrow\quad&
z=t\qty(1+\epsilon^2\frac{5\alpha^2}{12})^{-1}\!+\order{\epsilon^3}
=t\qty(1-\epsilon^2\frac{5\alpha^2}{12})+\order{\epsilon^3}
\end{aligned}
\end{equation}
This is (up to $\order{\epsilon^3}$) exactly the same as
\eqref{eq:4b_t}, which is good because we shouldn't get wildly
different solutions depending on technique. And the solution
$y(t,\epsilon)$ is given by \eqref{eq:4c_y} with the singular term
canceled and with $t$ as above, which again is exactly the same as 
\eqref{eq:4b_y}, up to $\mathcal{O}_F(\epsilon^3)$.



\section{Non-linear pendulum, with Pritulo's technique}
\renewcommand{\thesubsection}{\arabic{section} (\roman{subsection})}
Here we're going to find a solution to the non-linear pendulum
\begin{equation}
\theta''+\sin\theta =0;\qquad
\theta(0)=a,\quad \theta'(0)=0,
\end{equation}
using Pritulo's technique.

\subsection{RPT}
\newcommand{\TT}{\ensuremath{\tilde{\theta}}}
As before we have to begin by finding the RPT
solution. Non-dimensionalize by setting $\TT=\theta/a$ giving the DE
\begin{equation}
0=\theta''+\frac{\sin(a\TT)}{a} 
=\theta''+\qty(\TT-\frac{a^2\TT^3}{6}+\frac{a^4\TT^5}{120}+\ldots),
\end{equation}
with $\TT(0)=1$ and $\TT'(0)=0$.
Next we expand $\TT(a^2)=\TT_0+a^2\TT_1+a^4\TT_2+\ldots$, giving
\begin{equation}
\begin{aligned}
&\TT_0''+\TT_0+a^2\qty(\TT_1''+\TT_1)+a^4\qty(\TT_2''+\TT_2)+\ldots\\
=&\frac{a^2}{6}\qty(\TT_0^3+3a^2\TT_0^2\TT_1+\ldots) 
- \frac{a^4}{120}\qty(\TT_0^5+\ldots).
\end{aligned}
\end{equation}

The $\order{1}$ problem:
\begin{equation}
\TT_0''+\TT_0=0;\qquad
\TT_0(0)=1,\quad {\TT_0'}(0)=0,
\end{equation}
which has the solution
\begin{equation}\label{eq:5_TT0}
\TT_0(t)=\cos(t).
\end{equation}

Then the $\order{a^2}$ problem:
\begin{equation}
\TT_1''+\TT_1=\frac{1}{6}\TT_0^3
=\frac{1}{8}\qty(\cos(t) +\frac{1}{3}\cos(3t)).
\end{equation}
Once again \textit{Mathematica} gives
\begin{equation}\label{eq:5_TT1}
\TT_1(t)=\frac{1}{192}\Big[\cos(t) - \cos(3t) + 12t\sin(t)\Big],
\end{equation}
where we have used the initial condition $\TT_1(0)=0$ and
$\TT'_1(0)=0$. 

And the  $\order{a^4}$ problem:
\begin{equation}\label{eq:5_TT2_DE}
\TT_2''+\TT_2=\frac{1}{2}\TT_0^2\TT_1-\frac{1}{120}\TT_0^5;
\qquad \TT_2(0)=0\qcomma \TT_2'(0)=0.
\end{equation}
Using \eqref{eq:5_TT0} and \eqref{eq:5_TT1} to plug in
\eqref{eq:5_TT2_DE} into \textit{Mathematica}, I get
\begin{equation}
\TT_2(t)=\frac{1}{61440}
\qty[
17\cos(t)-20\cos(3t)+3\cos(5t)-120t^2\cos(t)-60t\sin(3t)
].
\end{equation}

To sum up we have
\begin{equation}\label{eq:5_TT}
\begin{aligned}
\TT(t)=&\phantom{+}\cos(t)\\
&+\frac{a^2}{192}\Big[\cos(t) - \cos(3t) + 12t\sin(t)\Big]\\
&+\frac{a^4}{61440}
\qty[
17\cos(t)-20\cos(3t)+3\cos(5t)-120t^2\cos(t)-60t\sin(3t)
]\\ &+ \mathcal{O}_\text{F}(a^6).
\end{aligned}
\end{equation}

\subsection{Change of variables}
\newcommand{\HT}{\ensuremath{\hat{\theta}}}
Next in Pritulo's method, we set
\begin{equation}
t(z)=z+a^2t_1(z)+a^4t_2(z)+\ldots\qcomma
\HT(z)=\TT(t),
\end{equation}
and Taylor expand \eqref{eq:5_TT} around $z$. This is more algebra
than I could possibly could keep straight and not loose something on
the way. So instead I'm going to fall back on the lifesaver of this
assignment \textit{Mathematica}. Using the function \texttt{Series}, I
get:
\begin{equation}\label{eq:5_HT}
\begin{aligned}
\HT(z)=&\phantom{+}\cos(z)\\
&+\frac{a^2}{192}\Big\{\cos(z)-\cos(3z) + [12z-192t_1]\sin(z)\Big\}\\
&+\frac{a^4}{61440}
\bigg\{
17\cos(z)-20\cos(3z)+3\cos(5z)\\
&\hspace{50pt}
+\qty[-120t^2-30720t_1^2+3840t_1z]\cos(z)\\
&\hspace{50pt}
+\qty[3520t_1-61440t_2]\sin(z)
+\qty[-60z+960t_1]\sin(3z)
\bigg\}
 &+ \mathcal{O}_\text{F}(a^6).
\end{aligned}
\end{equation}
This results in an over-determined system of equations
\begin{equation}
\begin{cases}
12z-192t_1=0\\
120z^2+30720t_1^2-3840t_1z=0\\
60z-960t_1=0\\
3520t_1-61440t_2=0
\end{cases}
\Longrightarrow
\begin{cases}
t_1=\frac{1}{16}z\\
t_1=\frac{1}{16}z\\
t_1=\frac{1}{16}z\\
t_2=\frac{3520}{61440}t_1%=\frac{11}{3720}z
\end{cases}.
\end{equation}
As we can see the different equations for $t_1$ were equivalent and we
only got
\begin{equation}
t_1=\frac{1}{16}z\qcomma
t_2=\frac{11}{3720}z
\end{equation}

To conclude we have
\begin{equation}
\begin{aligned}
\HT(z)=&\phantom{+}\cos(z)
+\frac{a^2}{192}\Big\{\cos(z)-\cos(3z) \Big\}\\
&+\frac{a^4}{61440}
\bigg\{
17\cos(z)-20\cos(3z)+3\cos(5z)
\bigg\}
 + \mathcal{O}_\text{F}(a^6),
\end{aligned}
\end{equation}
where $\TT(t)=\HT(z)$ and
\begin{equation}
t=\qty(1+\frac{a^2}{16}+\frac{11a^4}{3720})z+\order{a^6}.
\end{equation}

\section{Linear dispersive wave}
\renewcommand{\thesubsection}{\arabic{section} (\alph{subsection})}

\begin{equation}
\eta_t + c_0\eta_{x}+\beta\eta_{xxx}=0
\end{equation}

\begin{equation}
\eta(x, 0)=\eta_0(x)=A(\epsilon x)\cos(kx)
\end{equation}


\end{document}



