\documentclass[11pt,letter, swedish, english
]{article}
\pdfoutput=1

\usepackage{../custom_as}
\usepackage[makeroom
]{cancel}
\graphicspath{{figures/}}

\swapcommands{\Delta}{\varDelta}
\swapcommands{\Omega}{\varOmega}

%%Drar in tabell och figurtexter
\usepackage[margin=10 pt]{caption}
%%För att lägga in 'att göra'-noteringar i texten
\usepackage{todonotes} %\todo{...}

%%För att själv bestämma marginalerna. 
\usepackage[
%            top    = 2.5cm,
%            bottom = 3cm,
%            left   = 3cm, right  = 3cm
]{geometry}

%%För att ändra hur rubrikerna ska formateras
\renewcommand{\thesubsection}{\arabic{section} (\alph{subsection})}

\renewcommand{\thesubsubsection}{\arabic{section} (\alph{subsection},\,\roman{subsubsection})}

\renewcommand{\thefootnote}{\fnsymbol{footnote}}

\newcommand{\Tc}{\ensuremath{T_{\text{c}}}}
\newcommand{\eF}{\ensuremath{\epsilon_{\text{F}}}}
\newcommand{\wD}{\ensuremath{\omega_{\text{D}}}}

%\usepackage{tikz}

\begin{document}

%\tikzstyle{every picture}+=[remember picture]
%\tikzstyle{na} = [shape=rectangle,inner sep=0pt,text depth=0pt]



%%%%%%%%%%%%%%%%% vvv Inbyggd titelsida vvv %%%%%%%%%%%%%%%%%

\title{Statistical Physics -- PHYS\,704 \\
Assignment 4}
\author{Andréas Sundström}
\date{\today}

\maketitle

%%%%%%%%%%%%%%%%% ^^^ Inbyggd titelsida ^^^ %%%%%%%%%%%%%%%%%
\paragraph{Collaboration: } 
This homework has been done in collaboration with Patrick Ryan. The
written report is, however, my own writing. 

\section{The internal energy and heat capacity of a superconductor}
We will consider a superconductor at finite temperature $0<T<\Tc$. One
very useful equation in such cases is the BCS equation
\begin{equation}\label{eq:BCS}
\Delta=\frac{V}{U}\sum_k \frac{\Delta}{2E_k}\Big(1-2n(E_k)\Big),
\end{equation}
where
\begin{equation}
E_k=\sqrt{\xi_k^2+\Delta^2}\qcomma \xi_k=\epsilon_k-\eF,
\end{equation}
and
\begin{equation}\label{eq:nFD}
n(E)=\frac{1}{\ee^{\frac{E}{T}}+1}
\end{equation}
is the Fermi-Dirac distribution.

\subsection{The internal energy}
For a superconductor, we have the Hamiltonian
\begin{equation}
H=\sum_{k, \sigma} E_k\gamma_{k \sigma}^\dagger\gamma_{k \sigma}
+\frac{V\Delta^2}{U}+\sum_k\qty(\xi_k-E_k).
\end{equation}
The internal energy is given as
\begin{equation}\label{eq:1a_E0}
E(T)=\ev{H}=
\sum_{k, \sigma} E_k\,n(E_k)
+\frac{V\Delta^2}{U}+\sum_k\qty(\xi_k-E_k).
\end{equation}
The number-operator for the quasi-particles in the first term is
converted to a Fermi-Dirac distribution, since the quasi particles are
an ideal Fermi-gas. 

Now using the BCS equation, we can rewrite the middle
term as
\begin{equation}
\Delta\frac{U\Delta}{V} \stackrel{\eqref{eq:BCS}}{=}
\Delta\sum_k \frac{\Delta}{2E_k}\Big(1-2n(E_k)\Big)
=\sum_k \frac{\Delta^2}{2E_k}\Big(1-2n(E_k)\Big).
\end{equation}
Now it's just a matter of collecting the terms in
\eqref{eq:1a_E0}. First we recognize that the sum over different spins
is just going to give us a factor~2. And we end up with
\begin{equation}
\begin{aligned}
E(T)&=\sum_k\qty[2E_k\,n(E_k)
+\frac{\Delta^2}{2E_k}\Big(1-2n(E_k)\Big)
+\xi_k-E_k ]\\
&=\sum_k\qty[
\xi_k-\qty(E_k-\frac{\Delta^2}{2E_k})\Big(1-2n(E_k)\Big) ].
\end{aligned}
\end{equation}
\qed

\subsection{The heat capacity at low temperatures}
To get the heat capacity, we differentiate $E$ with respect to $T$:
\begin{equation}\label{eq:1b_Cv0}
\begin{aligned}
C_V=\dv{E}{T}=\sum_k\Bigg[&
\cancel{\dv{\xi_k}{T}}-\dv{E_k}{T}\Big(1-2n(E_k)\Big)
-E_k\dv{T}\Big(1-2n(E_k)\Big)\\
&+\dv{\Delta^2}{T}\frac{1}{2E_k}\Big(1-2n(E_k)\Big) 
+\Delta^2 \dv{T}\qty(\frac{1-2n(E_k)}{2E_k})
\Bigg].
\end{aligned}
\end{equation}
From here we will need 
\begin{equation}\label{eq:1b_dEk/dT}
\dv{E_k}{T}=\pdv{E_k}{(\Delta^2)}\pdv{\Delta^2}{T}
=\frac{1}{2\sqrt{\xi_k^2+\Delta^2}}\pdv{\Delta^2}{T}
=\frac{1}{2E_k}\pdv{\Delta^2}{T}.
\end{equation}
And
\begin{equation}\label{eq:1b_dn/dT}
\dv{n(E_k)}{T}=\pdv{n}{E_k}\dv{E_k}{T}+\pdv{n}{T}
=-\frac{\frac{1}{T}\ee^{\nicefrac{E_k}{T}}}
{\qty(\ee^{\nicefrac{E_k}{T}}+1)^2}\frac{1}{2E_k}\pdv{\Delta^2}{T}
-\frac{-\frac{E_k}{T^2}\ee^{\nicefrac{E_k}{T}}}
{\qty(\ee^{\nicefrac{E_k}{T}}+1)^2}.
\end{equation}

Next we take a closer look at the last term in \eqref{eq:1b_Cv0}. For
$T<\Tc$ meaning that $\Delta\neq0$, the BCS equation can be written as
\begin{equation}\label{eq:1b_BCS}
\frac{U}{V}=\sum_k\frac{1-2n(E_k)}{2E_k}.
\end{equation}
If we now differentiate both sides we get
\begin{equation}
0=\dv{T}\sum_k\frac{1-2n(E_k)}{2E_k},
\end{equation}
since the LHS of \eqref{eq:1b_BCS} is a constant. In other words, the
last term of \eqref{eq:1b_Cv0} disappears.

Now the heat capacity becomes
\begin{equation}
C_V=\sum_k\Bigg[
-\cancel{\dv{E_k}{T}\Big(1-2n(E_k)\Big)}
+E_k\dv{T}\Big(2n(E_k)\Big)
+\cancel{\dv{\Delta^2}{T}\frac{1}{2E_k}\Big(1-2n(E_k)\Big)}
\Bigg].
\end{equation}
Here the first and last term cancel each other thanks to
\eqref{eq:1b_dEk/dT}. With \eqref{eq:1b_dn/dT}, we end up with
\begin{equation}\label{eq:1b_Cv_exact}
\begin{aligned}
C_V=&\sum_k2E_k\qty[
-\frac{\frac{1}{T}\ee^{\nicefrac{E_k}{T}}}
{\qty(\ee^{\nicefrac{E_k}{T}}+1)^2} \frac{1}{2E_k}\pdv{\Delta^2}{T}
-\frac{-\frac{E_k}{T^2}\ee^{\nicefrac{E_k}{T}}}
{\qty(\ee^{\nicefrac{E_k}{T}}+1)^2} ]\\
=&\frac{1}{T}\sum_k\qty(\frac{2E_k^2}{T}-\pdv{\Delta^2}{T})
\frac{\ee^{\nicefrac{E_k}{T}}}{\qty(\ee^{\nicefrac{E_k}{T}}+1)^2}.
\end{aligned}
\end{equation}
Note that this expression is \emph{exact}, and thus holds for any $T$ (if
$T>\Tc$, then the second term is~0).

Since we're in the low temperature limit ($T\ll\Tc$), we can
approximate $\pdv*{\Delta^2}{T}\approx0$, 
%\begin{equation}
%\frac{E_k^2}{T}\gg\pdv{\Delta^2}{T},
%\end{equation}
because ${\Delta}$ is almost constant at very low temperatures. Next,
the low temperature limit also means that
\begin{equation}
\frac{\ee^{\nicefrac{E_k}{T}}}{\qty(\ee^{\nicefrac{E_k}{T}}+1)^2}
\approx
\frac{\ee^{\nicefrac{E_k}{T}}}{\qty(\ee^{\nicefrac{E_k}{T}})^2}
=\ee^{-\nicefrac{E_k}{T}}
=\exp(-\frac{\sqrt{\xi^2+\Delta^2}}{T}).
\end{equation}

The final step is to go from a sum to an integral, which we do with
the prescription
\begin{equation}\label{eq:1_sum_int}
\frac{1}{V}\sum_k \to g(\eF)\int_{-\hbar\wD}^{\hbar\wD}\!\rd\xi.
\end{equation}
This results in
\begin{equation}
\begin{aligned}
C_V\to&\frac{2V}{T^2}g(\eF)\int_{-\hbar\wD}^{\hbar\wD}\!\rd\xi\;
\qty(\xi^2+\Delta^2)\exp(-\frac{\sqrt{\xi^2+\Delta^2}}{T}).
\end{aligned}
\end{equation}
To deal with this integral we will need to use Laplace's method.

\subsubsection*{Laplace's method}
Call $f(\xi)=\xi^2+\Delta^2$ and
$\varphi(\xi)=\sqrt{\xi^2+\Delta^2}$. Then the integral in question is
\begin{equation}
I(\Delta)=\int_{-\hbar\wD}^{\hbar\wD}\!\rd\xi\;f(\xi)\exp(-\frac{\varphi(\xi)}{T}).
\end{equation}
Now, $T$ is small, which means that $\ee^{-\varphi(\xi)/T}\lll1$, and
the biggest contribution to the total integral comes from the
immediate vicinity of the minimum of $\varphi$, $\xi_0$. Everything else is
effectively killed off by the exponential. 
We can therefore Taylor expand $\varphi(\xi)$ around $\xi_0$, and
integrate that instead. 
We also don't have to worry about the value of $f(\xi)$ at any other
places than at $\xi_0$.

In our case we have $\xi_0=0$. And for the Taylor expansion we will
need the second derivative of $\varphi$ at $\xi=0$:
\begin{equation}
\varphi'(\xi)=\frac{\xi}{\sqrt{\xi^2+\Delta^2}}
\quad\Longrightarrow\quad
\varphi''(\xi)=\frac{1}{\sqrt{\xi^2+\Delta^2}}
+\xi\dv{\xi}\qty[\frac{1}{\sqrt{\xi^2+\Delta^2}}].
\end{equation}
And the Taylor expansion becomes
\begin{equation}
\varphi(\xi)\approx \varphi(0)+\bcancel{\varphi'(0)\xi}
+\varphi''(0)\frac{\xi^2}{2}
=\Delta+\frac{\xi^2}{2\Delta},
\end{equation}
assuming that $\Delta$ is real and positive.

The integral can now be written as
\begin{equation}
I(\Delta)\approx\int_{-\hbar\wD}^{\hbar\wD}\!\rd\xi\;f(0)
\exp(-\frac{\Delta}{T} - \frac{\xi^2}{2\Delta T})
=f(0)\ee^{-\nicefrac{\Delta}{T}}\int_{-\hbar\wD}^{\hbar\wD}\!\rd\xi\;
\exp(-\frac{\xi^2}{2\Delta T}).
\end{equation}
To evaluate the last step here, we note that the contribution from
$\abs{\xi}>\hbar\wD$ is negligible, wherefore we can approximate with a
Gaussian integral:
\begin{equation}
I(\Delta)\approx \Delta^2\ee^{-\nicefrac{\Delta}{T}}
\int_{-\infty}^{\infty}\!\rd\xi\;
\exp(-\frac{\xi^2}{2\Delta T})
=\Delta^2\ee^{-\nicefrac{\Delta}{T}}
\sqrt{\pi2\Delta T}.
\end{equation}

\subsubsection*{Back to the heat capacity}
The heat capacity is now 
\begin{equation}\label{eq:1b_Cv}
C_V=\frac{2V}{T^2}g(\eF)I(\Delta)
=2\sqrt{2\pi}Vg(\eF)\frac{\Delta^{5/2}}{T^{3/2}}
\ee^{-\nicefrac{\Delta}{T}}.
\end{equation}
This is only valid in the very low temperature limit $T\ll\Tc$,
though. 
\qed

\subsection{Discontinuity in the heat capacity}

Here we're going to find the discontinuity, at $T=\Tc$, in the heat
capacity. To that we begin by taking another look at
\eqref{eq:1b_Cv_exact}. Since that is an exact expression we can, in
principle, use it to calculate $C_v$ at any $T$. 

However, we're only interested in the \emph{discontinuity} of
$C_v$. So our first observation is that for $T>T_c$, we only get the
first term, but that term is \emph{continuous} at $\Tc$ since
$(E_k)=(\xi_x^2+\Delta^2)\to\xi_k^2$ continuously as
$T\to\Tc$. Therefore the discontinuity can not arise from this term.

\paragraph{The discontinuity term}
The discontinuity therefore has its origin in the second term in
\eqref{eq:1b_Cv_exact}, which only appears in superconductors. Let us
now calculate the discontinuity, call it $\widehat{C}_V$:
\begin{equation}
\widehat{C}_V(T)=-\frac{1}{T}\sum_k
\pdv{\Delta^2}{T}
\frac{\ee^{\nicefrac{E_k}{T}}}{(\ee^{\nicefrac{E_k}{T}}+1)^2}.
\end{equation}

To calculate the derivative, we utilize the fact that we're close to
$\Tc$. From the next problem we have
\begin{equation}
\Delta(T)\approx\Tc\sqrt{\frac{-t}{c}},
\end{equation}
where $t=(T-\Tc)/\Tc$ is the reduced temperature, and $c$ is some
(positive) numerical constant\footnotemark{}. Thus
\begin{equation}
\pdv{\Delta^2}{T}=\pdv{T}\qty[\Tc^2\frac{-t}{c}]=-\Tc^2\frac{1}{c\Tc}.
=-\Tc^2\frac{1}{c\Tc}=-\frac{\Tc}{c}.
\end{equation}

\footnotetext{The placing of the constant will become clearer in the
  next section. Also note that there is no circular argumentation here,
  since the results of the next problem are fully independent of what
  conclusions we may draw here.}

Now we have everything needed to evaluate the sum, which we do in
usual order by converting to an integral:
\begin{equation}
\widehat{C}_V(T) = +\frac{Vg(\eF)\Tc}{c}
\int_{-\hbar\wD}^{\hbar\wD}\frac{\rd\xi}{T}
\frac{\ee^{\nicefrac{E_k}{T}}}{(\ee^{\nicefrac{E_k}{T}}+1)^2}.
\end{equation}
Unfortunately, this integral can not be computed analytically, since
$E_k=\sqrt{\xi_k^2+\Delta^2}$. But in the limit $T\to\Tc^-$, we get
$E_k=\abs{\xi}_k$, which now becomes possible to evaluate in the limit
$\hbar\wD\to\infty$. With the exponentials, that is a reasonable
approximation. So we get
\begin{equation}
\widehat{C}_V(T\to\Tc^-) = +\frac{Vg(\eF)\Tc}{c}
\underbrace{\int_{-\infty}^{\infty}\!\rd{x}\;
\frac{\ee^{|x|}}{(\ee^{|x|}+1)^2}}_{1}
= +\frac{Vg(\eF)\Tc}{c},
\end{equation}
where $x=\xi/\Tc$.

\paragraph{Final result}
The discontinuity in the heat capacity is thus
\begin{equation}
C_V(T\to\Tc^-)-C_V(T\to\Tc^+)=\frac{Vg(\eF)\Tc}{c}>0
\end{equation}
arising only from the second term in \eqref{eq:1b_Cv_exact}. And from
the next problem we will find that $1/c\approx9.3$.

\subsection{Sketch of $C_V(T)$}
\figref{fig:1_Cv} show a sketch of $C_V$ as a function of $T$. Some
notable points are: $C_v\to0$ as $T\to0$ thanks to the exponential
suppression in \eqref{eq:1b_Cv}, and the discontinuity is such that
$C_V$ makes a jump down when passing $\Tc$ from below.


\begin{figure}
\centering
\resizebox{.8\textwidth}{!}{\input{figures/C_V_sketch.pdf_t}}
\caption{}
\label{fig:1_Cv}
\end{figure}


















\section{The Helmholtz free energy of a superconductor}
\newcommand{\xD}{x_{\text{D}}}
In this problem we are concerned about the Helmholtz free energy of a
super conductor, which is given by
\begin{equation}\label{eq:2_F}
F=-T\ln{Z}=\frac{V\Delta^2}{U} + 
\sum_k\qty[\xi_k-E_k-2T\ln(1+\ee^{-\frac{E_k}{T}})].
\end{equation}

\subsection{Taylor expansion}
Here we want to expand $F$ in terms of $\Delta^2$; therefore introduce
$\delta:=\Delta^2$. The Taylor expansion then becomes:
\begin{equation}\label{eq:2a_Taylor_F}
F=F_0+a(T)\Delta^2 + \frac{1}{2}b(t)\Delta^4+\order{\Delta^6}
=F_0+a(T)\delta + \frac{1}{2}b(t)\delta^2+\order{\delta^3}.
\end{equation}
Now all we have to do is to fins the coefficients $F_0$, $a$ and $T$. 
According to Taylor's theorem 
\begin{equation}\label{eq2a:coeff}
F_0=F(\delta=0)\qcomma
a=\eval{\dv{F}{\delta}}_{\delta=0}\qcomma
b=\eval{\dv[2]{F}{\delta}}_{\delta=0}.
\end{equation}

We begin with the constant term
\begin{equation}
F_0=-2T\sum_k\ln(1+\ee^{-\frac{\xi_k}{T}}),
\end{equation}
which is just the Helmholtz free energy for an ideal Fermi gas.

\subsubsection{First coefficient}
For $a$, we need to calculate
\begin{equation}\label{eq:2a_dF/ddelta}
\dv{F}{\delta}=\frac{V}{U}
+\sum_k\qty[-\pdv{E_k}{\delta}
-2T\qty(\frac{-\frac{1}{T}\ee^{-\nicefrac{E_k}{T}}\pdv{E_k}{\delta}}
{1+\ee^{-\nicefrac{E_k}{T}}})]
=\frac{V}{U}
+\sum_k\pdv{E_k}{\delta}\qty[-1
+2\frac{\ee^{-\nicefrac{E_k}{T}}}{1+\ee^{-\nicefrac{E_k}{T}}}].
\end{equation}
We see that we need
\begin{equation}\label{eq:2a_dEk/ddelta}
\pdv{E_k}{\delta}=\pdv{\delta}\qty[\sqrt{\xi^2+\delta}]
=\frac{1}{2\sqrt{\xi^2+\delta}}=\frac{1}{2E_k}
\end{equation}
In other words
\begin{equation}\label{eq:2a_a1}
\dv{F}{\delta}
=\frac{V}{U}
+\sum_k\frac{1}{2E_k}\qty[-1
+\frac{2}{\ee^{\nicefrac{E_k}{T}}+1}]
=\frac{V}{U}
-\sum_k\frac{1-2n(E_k)}{2E_k}.
\end{equation}
Note that the second term very much reasembles the RHS in the BCS
equation:
\begin{equation}\label{eq:2a_BCS}
1=\frac{U}{V}\sum_k\frac{1-2n(E_k)}{2E_k}.
\end{equation}
This equation is an implicit equation for the $T$ dependence of
$\Delta$ -- i.e. for any $T<\Tc$, \eqref{eq:2a_BCS} will give the value
of $\Delta$. But, for $a(T)$, we're interested in the limit $\Delta\to0$, and the
only $T$, that would still satisfy \eqref{eq:2a_BCS}, would be
$\Tc$. And we still want \eqref{eq:2a_BCS} to be valid so that we can
cancel the $V/U$ term.

On the other hand, we're interested in the value of $a$ slightly off
from $\Tc$. So we must Taylor expand 
the last term in \eqref{eq:2a_a1} around $T=\Tc$, to get the leading
order behavior of $a(T)$. For that we need the
derivative of $n(\xi_k)$ with respect to $T$, where $n$ is the
Fermi-Dirac distribution \eqref{eq:nFD}. We get
\begin{equation}
\pdv{n(E)}{T}=-\frac{-\frac{E}{T^2}\ee^{\nicefrac{E}{T}}}{\qty(\ee^{\nicefrac{E}{T}}+1)^2}
=\frac{E}{T^2}\frac{\ee^{\nicefrac{E}{T}}}{\qty(\ee^{\nicefrac{E}{T}}+1)^2}.
\end{equation}
Whereby we can expand
\begin{equation}
n(E)=n(E) %\frac{1}{\ee^{\nicefrac{E}{\Tc}}+1}
+\frac{E}{\Tc}\frac{\ee^{\nicefrac{E}{\Tc}}}
{\qty(\ee^{\nicefrac{E}{\Tc}}+1)^2}
\frac{(T-\Tc)}{\Tc} + \order{(T-\Tc)^2}.
\end{equation}
We also define the reduced temperature $t=(T-\Tc)/\Tc$.

We are now ready to calculate $a$:
\begin{equation}
\begin{aligned}
a(T)=\eval{\dv{F}{\delta}}_{\delta=0}
=&\cancel{\frac{V}{U}}
-\cancel{\sum_k\frac{1-2n(\xi_k,\;T=\Tc)}{2\xi_k}}l\\
&+\sum_k\frac{t}{2\xi_k}\frac{\xi_k}{\Tc}
\frac{2\ee^{\nicefrac{|\xi_k|}{\Tc}}}{\qty(\ee^{\nicefrac{|\xi_k|}{\Tc}}+1)^2},
\end{aligned}
\end{equation}
where we also used $E_k\to|\xi_k|$ as $\Delta\to0$.
The next step is to convert this sum to an integral, using
\eqref{eq:1_sum_int}. 
And we get
% \begin{equation}  
% \begin{aligned}\label{eq:2a_a_int}
% a-\frac{V}{U}=&
% Vg(\eF)\int_{-\hbar\wD}^{\hbar\wD}\!\rd\xi\;
% \frac{1}{2\xi}\qty[-1
% +\frac{2}{\ee^{\nicefrac{\xi}{T}}+1}]
% \\
% =&Vg(\eF)\int_{-\xD}^{\xD}\!\rd{x}\;
% \frac{1}{2x}\frac{1-\ee^x}{\ee^x+1}
% \end{aligned}
% \end{equation}
\begin{equation}
\begin{aligned}
a(T)=&Vg(\eF)\;t\int_{-\hbar\wD}^{\hbar\wD}\!\frac{\rd\xi}{\Tc}\;
\frac{\ee^{\nicefrac{|\xi_k|}{\Tc}}}{\qty(\ee^{\nicefrac{|\xi_k|}{\Tc}}+1)^2}\\
&=Vg(\eF)\;t\int_{-\xD}^{\xD}\!\rd{x}\;
\frac{\ee^{|x|}}{\qty(\ee^{|x|}+1)^2}\\
\end{aligned}
\end{equation}
where $x=\xi/\Tc$ and $\xD=\hbar\wD/\Tc$. We do know that
$\hbar\wD\gg\Tc$, so we can easily approximate the integral by taking
the limit $\xD\to\infty$, which results in
\begin{equation}\label{eq:2_a}
\begin{aligned}
a(T)\approx Vg(\eF)\;t\int_{-\infty}^{\infty}\!\rd{x}\;
\frac{\ee^{|x|}}{\qty(\ee^{|x|}+1)^2}
=Vg(\eF)\;t.
\end{aligned}
\end{equation}

%\paragraph{First term}   %% Here begins black magic
% \paragraph{The integral}
% Here we have
% \vspace{-11pt}\begin{equation}
% I_1=-\frac{1}{2}\int_{-\xD}^{\xD}\!\rd{x}\;
% \frac{1}{x}
% \frac{\ee^x-1}{\ee^x+1}
% %=-\int_{0}^{\xD}\!\rd{x}\;
% %\frac{1}{x}\frac{\ee^x-1}{\ee^x+1}
% \end{equation}
% This integral does not diverge around $x=0$. However, if we were to let
% $\xD\to\infty$, then the integral would diverge, since the integrand, $f_1(x)$,
% is asymptotically $f_1(x)\sim\pm1/x$ for $x\to\pm\infty$, or in other
% words $f_1(x)\sim1/\abs{x}$ as $\abs{x}\to\infty$. 

% But we also know that, since $T<\Tc$,
% \begin{equation}\label{eq:2_xD}
% \xD=\frac{\hbar\wD}{T}>\frac{\hbar\wD}{\Tc}
% \approx\frac{\hbar\wD}{1.14\cdot\hbar\wD\exp(-\frac{1}{Ug(\eF)})}
% \approx\exp(\frac{1}{Ug(\eF)})\gg1.
% \end{equation}
% So the main contribution to the integral comes from the two
% asymptotic tails of the integral, which is just
% \begin{equation}
% \int\rd{x}\;\frac{1}{x}=\ln(x).
% \end{equation}
% We get roughly the same contribution from both integration limits.
% In other words, we can approximate
% \begin{equation}\label{eq:2a_I1_0}
% I_1\approx -\frac{1}{2}\cdot2\ln(\xD)=-\ln(\frac{\hbar\wD}{T}).
% \end{equation}

% We are interested in solutions near $\Tc$. But we note that if we were
% to just set $T=\Tc$, then \eqref{eq:2_xD} would result in $a=0$
% because of the extra term $V/U$ in \eqref{eq:2a_a_int}. We therefore
% need to Taylor expand \eqref{eq:2a_I1_0} around $T=\Tc$:
% \begin{equation}
% I_1\approx-\ln(\frac{\hbar\wD}{T})
% =-\ln(\frac{\hbar\wD}{\Tc}) -
% \frac{\Tc}{\hbar\wD}\frac{\hbar\wD}{-\Tc^2} (T-\Tc)
% +\order{(T-\Tc)^2}
% \end{equation}



% \paragraph{Second term}
% This integral,
% \begin{equation}
% I_2=t\int_{-\xD}^{\xD}\!\rd{x}\;
% \frac{\ee^{x}}{\qty(\ee^{x}+1)^2},
% \end{equation}
% can be computed easily by doing a swap of variables $u=\ee^x$, giving
% $\rd{u}=\ee^x\rd{x}$. This results in
% \begin{equation}
% I_2=t\int_{\ee^{-\xD}}^{\ee^{\xD}}\!\rd{u}\;
% \frac{1}{(u+1)^2}
% =t\qty[-\frac{1}{u+1}]_{\ee^{-\xD}}^{\ee^{\xD}}
% =t\qty(\frac{1}{\ee^{-\xD}+1}-\frac{1}{\ee^{\xD}+1}).
% \end{equation}
% Now we already know that $\xD\gg1$, so $\ee^{\xD}\ggg1$ and
% $\ee^{-\xD}\lll1$, meaning that we can approximate
% \begin{equation}
% I_2\approx t.
% \end{equation}

% \paragraph{Result}
% We now get
% \begin{equation}\label{eq:2_a}
% a=\frac{V}{U}+Vg(\eF)I_1
% \approx\cancel{\frac{V}{U}}+
% Vg(\eF)\qty(\cancel{\frac{-1}{Ug(\eF)}}+t)
% =Vg(\eF)t,
% \end{equation}
% where $t=(T-\Tc)/\Tc$ is the reduced temperature.


\subsubsection{Second coefficient}
By the definition in \eqref{eq2a:coeff}, we nee the second derivative
of $F$ with respect to $T$. With \eqref{eq:2a_dF/ddelta}, we get
\begin{equation}
\begin{aligned}
\pdv[2]{F}{\delta}=&\pdv{\delta}\sum_k\pdv{E_k}{\delta}
\qty(-1 + \frac{2}{\ee^{\nicefrac{E_k}{T}}+1})\\
=&\sum_k\qty[\pdv[2]{E_k}{\delta}
\qty(-1 + \frac{2}{\ee^{\nicefrac{E_k}{T}}+1})
+\pdv{E_k}{\delta}\qty(
-2\frac{\frac{1}{T}\ee^{\nicefrac{E_k}{T}}}{\qty(\ee^{\nicefrac{E_k}{T}}+1)^2}
\pdv{E_k}{\delta})
].
\end{aligned}
\end{equation}
With \eqref{eq:2a_dEk/ddelta}, we have
\begin{equation}
\pdv[2]{E_k}{\delta}=-\frac{1}{2E_k^2}\pdv{E_k}{\delta}=-\frac{1}{4E_k^3}.
\end{equation}
Now
\begin{equation}
\begin{aligned}
b=\eval{\pdv[2]{F}{\delta}}_{\delta=0}
=&\sum_k\qty[
\frac{1}{4\xi_k^3}
\qty(+1 - \frac{2}{\ee^{\nicefrac{\xi_k}{T}}+1})
+\frac{-2}{(2\xi_k)^2}
\frac{\frac{1}{T}\ee^{\nicefrac{\xi_k}{T}}}{\qty(\ee^{\nicefrac{\xi_k}{T}}+1)^2}
].
\end{aligned}
\end{equation}
Once again, we convert to an integral using the prescription
\eqref{eq:1_sum_int}, and also setting $x=\xi/T$ and
$\xD=\hbar\wD/T$. This results in 
\begin{equation}
\begin{aligned}
b\to& Vg(\eF)\int_{-\xD}^{\xD}\!T\rd{x}\;
 \frac{1}{4T^3x^3}\qty[
\frac{\ee^{x}-1}{\ee^{x}+1}
-\frac{2x\ee^{x}}{\qty(\ee^{x}+1)^2}
]\\
=&\frac{Vg(\eF)}{4T^2}\int_{-\xD}^{\xD}\!\rd{x}\;
\frac{1}{x^3}
\qty[
\frac{\ee^{2x}-1-2x\ee^{x}}{\qty(\ee^{x}+1)^2}
].\\
\end{aligned}
\end{equation}
And once again, we need to take a closer look at the integral.

\paragraph{The integral}
In this case what we have is 
\begin{equation}
I_2=\int_{-\xD}^{\xD}\!\rd{x}\;
\frac{1}{x^3}
\qty[
\frac{\ee^{2x}-1-2x\ee^{x}}{\qty(\ee^{x}+1)^2}
],
\end{equation}
which looks worrying, because this might be divergent when integrating
past $x=0$. 
We therefore study the numerator near $x=0$:
\begin{equation}
\begin{aligned}
\ee^{2x}-1-2x\ee^{x}=&-1 + \qty(1+2x+\frac{1}{2}(2x)^2+\frac{1}{3!}(2x)^3)
-2x\qty(1+x +\frac{1}{2}x^2) +\order{x^4}\\
=&\cancel{-1}+\qty(\cancel{1}+\bcancel{2x}+\xcancel{2x^2}+\frac{4x^3}{3})
-\qty(\bcancel{2x}+\xcancel{2x^2} +x^3) +\order{x^4}\\
=&\frac{x^3}{3}+\order{x^4}.
\end{aligned}
\end{equation}
This means that the integrand is actually well behaved around $x=0$.

Unfortunately this integral cannot be computed analytically. We
therefore utilize the fact that the integrand, $f(x)$ is asymptotic to
$\pm1/x^3$ as $x\to\pm\infty$, or in other words $f(x)\sim1/\abs{x}^3$ as
$\abs{x}\to\infty$. Which means that since $\xD\gg1$, we can
approximate
\begin{equation}
I_2\approx\int_{-\infty}^{\infty}\!\rd{x}\;
\frac{1}{x^3}
\qty[\frac{\ee^{2x}-1-2x\ee^{x}}{\qty(\ee^{x}+1)^2}]
\approx 0.426,
\end{equation}
where the last step was just a numerical integration.

\paragraph{Result}
We want $b(T)$ in a vicinity of $\Tc$, and this time we get to keep
the constant term, which results in
\begin{equation}\label{eq:2_b}
b(T)=\frac{Vg(\eF)}{4T^2}I_2
\approx c\frac{Vg(\eF)}{\Tc^2}\approx 0.107\frac{Vg(\eF)}{\Tc^2},
\end{equation}
where we have defined $c=I_2/4$.

\subsubsection{Final expression}
Using \eqref{eq:2a_Taylor_F} together with \eqref{eq:2_a} and
\eqref{eq:2_b}, we get
\begin{equation}\label{eq:2a_Taylor_F_final}
F(\Delta)=F_0+Vg(\eF)t\Delta^2 
+ \frac{0.107}{2}\frac{Vg(\eF)}{\Tc^2}\Delta^4 + \order{\Delta^6}+\order{t^2}.
\end{equation}
\qed


\subsection{Finding the equilibrium}
To find the equilibrium value of $\delta=\Delta^2$, we need to minimize
the free energy. With the Taylor expansion \eqref{eq:2a_Taylor_F}, we
easily minimize $F$ by completeing the square
\begin{equation}\label{eq:2b_F_compl_sq}
F-F_0=\frac{b}{2}\qty[\qty(\delta+\frac{a}{b})^2-\frac{a^2}{b^2}].
\end{equation}
So the minimum occurs at $\delta=-a/b$, which in terms of $\Delta$ and
with \eqref{eq:2_a} and \eqref{eq:2_b}, gives
\begin{equation}\label{eq:2_Delta(T)}
\Delta(T)=\sqrt{-\frac{a(T)}{b(T)}}
=\sqrt{\frac{-t}{c/\Tc^2}}
=\Tc\sqrt{\frac{-t}{c}}\approx3.1\,\Tc\sqrt{\frac{\Tc-T}{\Tc}}.
\end{equation}
Note that the minus sign is there because $t<0$ when $T<\Tc$.
\qed

\subsection{Heat capacity, again}
Now we're going to calculate discontinuity of the heat capacity
again. This time we will use
\begin{equation}\label{eq:2c_Cv_F}
C_V=-T\qty(\pdv[2]{F}{T})_V.
\end{equation}
We see from \eqref{eq:2a_Taylor_F_final} that the first term $F_0$,
which is just the ideal Fermi gas free energy, will not contribute to
the discontinuity, we will therefore focus on the two other terms. 

With \eqref{eq:2_Delta(T)}, we get
\begin{equation}
\begin{aligned}
F(T)=&F_0+Vg(\eF)\qty[
\qty(\Tc\sqrt{\frac{-t}{c}})^2 t+
\frac{c}{2\Tc^2}\,
\qty(\Tc\sqrt{\frac{-t}{c}})^4
]\\
=&F_0+\frac{Vg(\eF)}{c}\Tc^2\qty[
-t^2+\frac{1}{2}t^2 ]
=F_0-\frac{Vg(\eF)}{2c}\,(T-\Tc)^2,
\end{aligned}
\end{equation}
for $T<\Tc$.
This results in
\begin{equation}
C_V=C_V^{\text{(ideal)}} - T\frac{-Vg(\eF)}{2c}2
\qcomma
T<\Tc.
\end{equation}
The extra minus sign come from the minus sign in \eqref{eq:2c_Cv_F}.
This results in
\begin{equation}
C_V(T\to\Tc^-)-C_V(T\to\Tc^+)=\frac{Vg(\eF)\Tc}{c}
\approx 9.3\,Vg(\eF)\Tc.
\end{equation}
\qed

\subsection{Sketch of the free energy}
Here we are going to sketch how the Free energy behaves as a function
of $\Delta$.
Using \eqref{eq:2b_F_compl_sq} we see that $F$ more or has a
parabola shape, as a function of $\delta=\Delta^2$. But depending on
temperature, the vertex of the parabola is going to lie on either side
of $\delta=0$. For $T<\Tc$ then $a<0$ means that the vertex is going
to lie on the positive side, and for $T>\Tc$ then $a>0$ and the vertex
is going to lie on the negative side. See \figref{fig:2_F}.


We also know that the equilibrium value of $\Delta$ is the value that
minimizes $F$. But don't let \figref{fig:2_F} deceive you,
$\delta=\Delta^2\ge0$. So the only posible values for $F$ is on the
right hand side of the graph. 
We thus see that the equilibrium value of $\Delta$ for $T>\Tc$ is
$\Delta=0$, and for $T<\Tc$ we have $\Delta>0$ according to the
expression in \eqref{eq:2_Delta(T)}.

Another point here is that it should be reasonable to expect much less
fluctuation in $\Delta$ for $T>\Tc$ than for $T<\Tc$. This is because
the at equlibirium value of $F$ 
\begin{equation}
\eval{\pdv{F}{\Delta}}_{\text{equilibrium}}\neq0\quad\text{at }\;T>\Tc.
\end{equation}
This means that even a small fluctuation in $\Delta$ would induce a
change in $F$. Meanwhile
\begin{equation}
\eval{\pdv{F}{\Delta}}_{\text{equilibrium}}=0\quad\text{at }\;T\le\Tc,
\end{equation}
which means that a small fluctuation in $\Delta$ will not change
$F$. Therefore it's reasonable to believe that the fluctuations in
$\Delta$ is \emph{much} smaller for $T>\Tc$.



\begin{figure}
\centering
\resizebox{.8\textwidth}{!}{\input{figures/F_sketch.pdf_t}}
\caption{}
\label{fig:2_F}
\end{figure}

















% \newpage
% \appendix
% \setcounter{equation}{0}
% \renewcommand{\theequation}{A\arabic{equation}}

% \section{Exerpt from assigngment 2, problem 2 (ii)}
% %\label{sec:ass_2_Cv}
% To calculate an approximation to the mean particle energy $E/N$, we
% once again have to use the Sommerfeld expansion. Then we use the relation
% \begin{equation}
% \frac{E}{N}=-\frac{3\Omega}{2N}=\frac{3T}{2}\frac{f_{5/2}(z)}{f_{3/2}(z)}.
% \end{equation}
% Fortunately we've already calculated $\ln(z)=\zeta\approx\zeta_2$, so all
% that left do here is to expand the Fermi functions and then
% Taylor expand the quotient.

% Let's begin:
% \begin{equation}
% %\begin{aligned}
% \frac{f_{5/2}(z)}{f_{3/2}(z)}\approx
% \overbrace{\frac{8\zeta_2^{5/2}}{15\sqrt{\pi}}
% \qty(\frac{4\zeta_2^{3/2}}{3\sqrt{\pi}})^{-1}}^{\nicefrac{2\zeta_2}{5}}
% \qty[1+\frac{5\pi^2}{8}\zeta_2^{-2}
% -\frac{7\pi^4}{384}\zeta_2^{-4} ]
% \times \qty[1+\frac{\pi^2}{8}\zeta_2^{-2} 
% +\frac{7\pi^4}{640}\zeta_2^{-4} ]^{-1}
% %\end{aligned}
% \end{equation}
% There's no need to even pretend to want deal with this by hand.
% Just go directly to \emph{Mathematica}, which yields
% \begin{equation}
% \frac{f_{5/2}(z)}{f_{3/2}(z)}\approx \frac{2}{5\tau}
% \qty[1+\frac{5\pi^2}{12}\tau^2-\frac{\pi^4}{16}\tau^4].
% \end{equation}
% For reference, the command in use were \texttt{Series}.

% Now we have
% \begin{equation}
% \frac{E}{N}\approx\frac{3T}{2}\frac{2}{5\tau}
% \qty[1+\frac{5\pi^2}{12}\tau^2-\frac{\pi^4}{16}\tau^4]
% =\frac{3\epsilon_F}{5}\qty[1
% +\frac{5\pi^2}{12}\qty(\frac{T}{\epsilon_F})^2
% -\frac{\pi^4}{16}\qty(\frac{T}{\epsilon_F})^4].
% \end{equation}
% This can be used to get an expression for the heat capacity of the
% system
% \begin{equation}\label{eq:CV_Fermi}
% \frac{C_V}{N}=\frac{1}{N}\qty(\pdv{E}{T})_V
% =\;\frac{\pi^2}{2}\frac{T}{\epsilon_F}
%  - \frac{3\pi^2}{20}\qty(\frac{T}{\epsilon_F})^3.
% \end{equation}



\end{document}


%  LocalWords:  Pathria  idealities bosonic Bogoliubov Beale BCS
%  LocalWords:  Laplace's
