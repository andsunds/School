\documentclass[11pt,letter, swedish, english
]{article}
\pdfoutput=1

\usepackage{../custom_as}
\usepackage[makeroom
]{cancel}
\graphicspath{{figures/}}

\swapcommands{\Delta}{\varDelta}
\swapcommands{\Omega}{\varOmega}

%%Drar in tabell och figurtexter
\usepackage[margin=10 pt]{caption}
%%För att lägga in 'att göra'-noteringar i texten
\usepackage{todonotes} %\todo{...}

%%För att själv bestämma marginalerna. 
\usepackage[
%            top    = 2.5cm,
%            bottom = 3cm,
%            left   = 3cm, right  = 3cm
]{geometry}

%%För att ändra hur rubrikerna ska formateras
%\renewcommand{\thesubsection}{\arabic{section} (\roman{subsection})}
\renewcommand{\thesubsection}{\arabic{section} (\alph{subsection})}
\renewcommand{\thesubsubsection}{\arabic{section} (\alph{subsection},\,\roman{subsubsection})}

%\renewcommand{\thefootnote}{\fnsymbol{footnote}}

\newcommand{\Tc}{\ensuremath{T_{\text{c}}}}
\newcommand{\sign}{\ensuremath{\text{sign}}}

%\usepackage{tikz}

\begin{document}

%\tikzstyle{every picture}+=[remember picture]
%\tikzstyle{na} = [shape=rectangle,inner sep=0pt,text depth=0pt]



%%%%%%%%%%%%%%%%% vvv Inbyggd titelsida vvv %%%%%%%%%%%%%%%%%

\title{Statistical Physics 2 -- PHYS\,705 \\
Assignment 4}
\author{Andréas Sundström}
\date{\today}

\maketitle

%%%%%%%%%%%%%%%%% ^^^ Inbyggd titelsida ^^^ %%%%%%%%%%%%%%%%%

\section{Gaussian expectationvalue of fast fields}
In this problem we want to prove the equality
\begin{equation}\label{eq:1_want}
\ev{\phi_>(\vb*k_1)\phi_>(\vb*k_2)}
= G_0(\vb*k_1) (2\pi)^d \delta(\vb*k_1+\vb*k_2),
\end{equation}
where $G_0(\vb*k) = (k^2+r)^{-1}$. Our starting ponint is
\begin{equation}\label{eq:1_start}
\ev{\phi_>(\vb*k_1)\phi_>(\vb*k_2)}
:= \frac{1}{Z_{0>}}\! \int\!\mathcal{D}\phi_> 
\qty[\phi_>(\vb*k_1)\phi_>^*(-\vb*k_2)
\exp(\!{-}\frac{1}{2V}\sum_{\vb*q}G_0^{-1}(\vb*q)\,
%\phi_{\!>}\!(\vb*q) \phi_{\!>}^*\!(\vb*q)
\qty|\phi_{>}\!(\vb*q)|^2
)],
\end{equation}
where $Z_{0>} = \int\mathcal{D}\phi_>\,
\exp(\sum_{\vb*q}G_0^{-1}\!(\vb*q)\,
\qty|\phi_{\!>}\!(\vb*q)|^2/2V)$.


To do this, we will use a formula from class, based on Wick's
theorem. For $N$ real variables $x_n$ and a non-singular, symmetric
matrix $\mathsf{M}$, we write
\begin{equation}\label{eq:1_ev1}
\begin{aligned}
\ev{x_{i}x_{j}} =& 
\frac{\int\rd^N\!x\,\qty[x_{i}x_{j}
\exp(-\frac{1}{2}\vb*x^\mathsfrm{T}\mathsf{M}\vb*x)]}
{\int\rd^N\!x\, \exp(-\frac{1}{2}\vb*x^\mathsfrm{T}\mathsf{M}\vb*x)}\\
=&\frac{\eval{\pdv[2]{}{J_i}{J_j}
\qty[\int\rd^N\!x\, 
\exp(-\frac{1}{2}\vb*x^\mathsfrm{T}\mathsf{M}\vb*x
+\vb*J^\mathsfrm{T}\vb*x)]}_{J=0}}
{\int\rd^N\!x\, \exp(-\frac{1}{2}\vb*x^\mathsfrm{T}\mathsf{M}\vb*x)},
\end{aligned}
\end{equation}
where $\vb*J\in\R^N$ is just a help variable. Now rewrite, using the
Hubbard–Stratonovich transformation, the integral in the numerator as
\begin{equation}
\begin{aligned}
I[\mathsf{M}, \vb*J] =& \int\rd^N\!x\, 
\exp(-\frac{1}{2}\vb*x^\mathsfrm{T}\mathsf{M}\vb*x
+\vb*J^\mathsfrm{T}\vb*x) \\
=& I[\mathsf{M}, 0]\,\exp(\frac{1}{2}\vb*J^\mathsfrm{T}\mathsf{M}^{-1}\vb*J).
\end{aligned}
\end{equation}
This means that we can rewrite \eqref{eq:1_ev1} as
\begin{equation}\label{eq:1_ev2}
\begin{aligned}
\ev{x_{i}x_{j}} =&
\frac{\eval{\pdv[2]{}{J_i}{J_j}
\qty[I[\mathsf{M}, \vb*J]]}_{J=0}}
{I[\mathsf{M}, 0]}\\
=&\eval{\pdv[2]{}{J_i}{J_j}
\qty[\exp(\frac{1}{2}\vb*J^\mathsfrm{T}\mathsf{M}^{-1}\vb*J)]}_{J=0}.
\end{aligned}
\end{equation}

This is all good and well, but we have a small problem in that
$\phi_>$ is a \emph{complex} variable. We are however saved by the
fact that $\mathsf{M} = \mathsf{G}_0^{-1}/V$ is diagonal, which means
that we still can write
\begin{equation}
\sum_{\vb*q} \frac{G_0^{-1}(\vb*q)}{V}\qty|\phi_{\!>}\!(\vb*q)|^2
=\sum_{\vb*q} \phi_{>}^*\!(\vb*q)M(\vb*q)\phi_{>}\!(\vb*q) 
= \vb*\phi_{>}^\dagger\mathsf{M}\vb*\phi_{>}\in\R
\end{equation}
and still have $\mathsf{M}\in\R^{N\times N}$. I.e. since $\mathsf{M}$
is diagonal, we do still end up with a real value despite $\vb*\phi_>$
being complex. And now to keep the term $\vb*J^\dagger\vb*\phi_>$ real,
corresponding to $\vb*J^\mathsfrm{T}\vb*x$ in \eqref{eq:1_ev1}, 
we have to make sure that
\begin{equation}
J^*(\vb*k)\phi_>(\vb*k)\in\R, \quad\forall\vb*k;
\end{equation}
i.e. $J(\vb*k)$ and $\phi_>(\vb*k)$ have to have the same complex phase:
\begin{equation}\label{eq:1_J_restr}
\phi_>(\vb*k) = |\phi_>(\vb*k)|\ee^{\ii\alpha}
\quad\Longrightarrow\quad
J(\vb*k) = |J(\vb*k)|\ee^{\ii\alpha}.
\end{equation}
Another way of looking at these conditions for $\vb*J$, is that we have
changed variables to $x(\vb*k)=|\phi_>(\vb*k)|$ to keep $\vb*J$
real. And then reintroducing the complex phase in the derivatives.

We begin the final steps in this problem by noting a few important
corollaries of \eqref{eq:1_J_restr}. That is
\begin{enumerate}[label=(\roman*)]
\item Since $\phi_>(-\vb*k)=\phi_>^*(\vb*k)$, so must
$J(-\vb*k)=J^*(\vb*k)$.
\item $\displaystyle \phi_>(\vb*k_1) \ee^{\vb*J^\dagger\vb*\phi_>}
=\phi_>(\vb*k_1) \exp(\sum_{\vb*q}J^*(\vb*q)\phi_>(\vb*q))
=\pdv{J^*(\vb*k_1)}\qty[\ee^{\vb*J^\dagger\vb*\phi_>}]$.
\item $\displaystyle \phi_>^*(-\vb*k_2) \ee^{\vb*J^\dagger\vb*\phi_>}
=\phi_>^*(-\vb*k_2) \exp(\sum_{\vb*q}J(-\vb*q)\phi_>(\vb*q))
=\pdv{J(-\vb*k_2)}\qty[\ee^{\vb*J^\dagger\vb*\phi_>}]$.
\end{enumerate}
Now, using \eqref{eq:1_ev2}, with (ii) and (iii), on
\eqref{eq:1_start} yields 
\begin{equation}
\begin{aligned}
\ev{\phi_>(\vb*k_1)\phi_>(\vb*k_2)} 
=& \eval{\pdv[2]{}{J^*(\vb*k_1)}{J(-\vb*k_2)}
\qty[\exp(\frac{1}{2}\vb*J^\dagger\mathsf{M}^{-1}\vb*J)]}_{J=0}\\
=&\eval{\pdv[2]{}{J^*(\vb*k_1)}{J(-\vb*k_2)}
\qty[\exp(\frac{V}{2}\sum_{\vb*q} J^*(\vb*q)G_0(\vb*q)J(\vb*q))]}_{J=0}.
\end{aligned}
\end{equation}
Here $M(\vb*k)=G_0^{-1}(\vb*k)/V$. 

We now have two cases. The first one being $\vb*k_1\neq-\vb*k_2$, then
we obviously get
\begin{equation}
\begin{aligned}
\ev{\phi_>(\vb*k_1)\phi_>(\vb*k_2)} 
= \frac{V}{2}\eval{\Big[
2G_0(\vb*k_1)J(\vb*k_1) + 2J^*(-\vb*k_2)G_0(-\vb*k_2)
\Big]\exp(\ldots)}_{J=0} = 0.
\end{aligned}
\end{equation}
Each term has a seemily additional factor 2 here, but that is due to
fact that for each derivative, we have 2 contributions: one from e.g.
$J(\vb*k_1)$ and one from $J^*(-\vb*k_2)$.
But in the second case, when $\vb*k_1=-\vb*k_2=:\vb*k$, we get
\begin{equation}
\begin{aligned}
\ev{\phi_>(\vb*k_1)\phi_>(\vb*k_2)} 
=&\eval{\pdv[2]{}{J^*(\vb*k)}{J(\vb*k)}
\qty[\exp(\frac{V}{2} \sum_{\vb*q}
J^*(\vb*q)G_0(\vb*q)J(\vb*q))]}_{J=0}\\
=&\eval{\pdv{J^*(\vb*k)}\qty[\frac{V}{2} 
2J^*(\vb*k)G_0(\vb*k)]\exp(\frac{V}{2} \sum_{\vb*q}
J^*(\vb*q)G_0(\vb*q)J(\vb*q))
}_{J=0}\\
=&\eval{\qty[VG_0(\vb*k) + \frac{V}{2} 
2J(\vb*k)G_0(\vb*k)]\exp(\ldots)
}_{J=0}\\
&=VG_0(\vb*k)
\end{aligned}
\end{equation}

We can write these two cases together as
\begin{equation}
\ev{\phi_>(\vb*k_1)\phi_>(\vb*k_2)} 
=G_0(\vb*k_1) \,V\delta_{\vb*k_1, -\vb*k_2},
\end{equation}
which is just the discrete version of \eqref{eq:1_want}. To get to
\eqref{eq:1_want}, we just have to use the thermo\-dynamic limit
$V\delta_{\vb*k_1, -\vb*k_2} \to (2\pi)^d\delta(\vb*k_1+\vb*k_2)$,
which arises from 
\begin{equation}
V\sum_{\vb*k} \to \int\frac{\rd^dk}{(2\pi)^d}.
\end{equation}








\section{Results from RG recursion relations}





\section{Fixedpoints from RG recursion relations}



\end{document}




%  LocalWords:  MFT MF Ising Ornstein Zernike Stratonovich GLW
%  LocalWords:  rescale quartic rescaled
