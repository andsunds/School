\documentclass[11pt,letter, swedish, english
]{article}
\pdfoutput=1

\usepackage{../custom_as}
\usepackage{cancel}
\graphicspath{ {figures/} }

%%Drar in tabell och figurtexter
\usepackage[margin=10 pt]{caption}
%%För att lägga in 'att göra'-noteringar i texten
\usepackage{todonotes} %\todo{...}

%%För att själv bestämma marginalerna. 
\usepackage[
%            top    = 3cm,
%            bottom = 3cm,
%            left   = 3cm, right  = 3cm
]{geometry}

%%För att ändra hur rubrikerna ska formateras
\renewcommand{\thesubsection}{\arabic{section} (\alph{subsection})}

\renewcommand{\thesubsubsection}{\arabic{section} (\alph{subsection},\,\roman{subsubsection})}



\swapcommands{\varPhi}{\Phi}
\swapcommands{\varPi}{\Pi}
\swapcommands{\varOmega}{\Omega}



\begin{document}

%%%%%%%%%%%%%%%%% vvv Inbyggd titelsida vvv %%%%%%%%%%%%%%%%%
% \begin{titlepage}
\title{Quantum Mechanics -- PHYS\,701 \\
Assignment 4}
\author{Andréas Sundström}
\date{\today}

\maketitle

%%%%%%%%%%%%%%%%% ^^^ Inbyggd titelsida ^^^ %%%%%%%%%%%%%%%%%

%Om man vill ha en lista med vilka todo:s som finns.
%\todolist

\section{Time-dependent perturbation of harmonic oscillator}
\newcommand{\I}{\text{I}}
%\newcommand{\S}{\text{S}}
We have a simple harmonic oscillator in it's ground state
\begin{equation}
\ket{i;\,t}=\ket{0}\qcomma
\text{for}\ t\le0.
\end{equation}
At $t=0$ a potential
\begin{equation}
V(t)=F_0x\cos\omega t
\end{equation}
is turned on. Now we want to approximate $\ev{x}$ as a function of time.

We begin by expressing the stateket in the interaction picture
\begin{equation}
\ket{i;\,t}_\I = \sum_n \ket{n}\!\!\mel*{n}{U_\I(t)}{i},
=\sum_n c_n(t)\ket{n}
\end{equation}
where in our case $i=0$. 

We now need to find $c_n(t)$, but we can only find an approximation, 
$c_n=c_n^{(0)}+c_n^{(1)}+\ldots$, based on the Dyson series for
$c_n$. This approximation begins with
\begin{equation}
c_n^{(0)}(t)=\delta_{n,i}
\end{equation}
and
\begin{equation}
\begin{aligned}
c_n^{(1)}(t)=&-\frac{\ii}{\hbar}
\int_0^t\!\rd{t'}\, \mel{n}{V_\I(t')}{i}\\
=&-\frac{\ii}{\hbar}
\int_0^t\!\rd{t'} \,\ee^{\ii\omega_{n,i}t'}\mel{n}{V(t')}{i},
\end{aligned}
\end{equation}
where $\omega_{n,i}=(E_n-E_i)/\hbar$. See Sakurai \& Napolitano,
\textit{Modern Quantum Mechanics}, ed.2, equation (5.7.17).

In our case we have $i=0$ and we get
\begin{equation}\label{eq:1_cn1_1}
\begin{aligned}
c_n^{(1)}(t)=&-\frac{\ii}{\hbar}
\int_0^t\!\rd{t'} \,\ee^{\ii\omega_{n,0}t'}
\mel{n}{F_0x\cos\omega t'}{0}\\
=&-\frac{\ii F_0}{\hbar}
\int_0^t\!\rd{t'} \,\ee^{\ii n\omega_{0}t'}
\cos(\omega t')\mel{n}{x}{0}.
\end{aligned}
\end{equation}
Using
\begin{equation}\label{eq:1_mel_x}
\mel{n}{x}{m}=sqrt{\frac{\hbar}{2m\omega_0}}
\Big[ \sqrt{m+1}\delta_{n,\,n+1}+\sqrt{m}\delta_{n,\,n-1}
\Big]
\end{equation}
we can write \eqref{eq:1_cn1_1} as
\begin{equation}\label{eq:1_cn1_2}
\begin{aligned}
c_n^{(1)}(t)=&-\frac{\ii F_0}{\hbar}\sqrt{\frac{\hbar}{2m\omega_0}}
\int_0^t\!\rd{t'} \,\ee^{\ii n\omega_{0}t'}
\cos(\omega t')\Big[ 
\sqrt{1}\delta_{n,\,1}+\cancel{\sqrt{0}\delta_{n,\,-1}}\Big]\\
&=-\frac{\ii F_0 \delta_{n,\,1}}{\sqrt{2\hbar m\omega_0}}
\int_0^t\!\rd{t'} \,\ee^{\ii n\omega_{0}t'}
\frac{1}{2}\qty(\ee^{\ii \omega t'}+\ee^{-\ii \omega t'}).
\end{aligned}
\end{equation}
We see now that the only non-zero coefficient here is $c_1^{(1)}$, so
we continue by looking at
\begin{equation}\label{eq:1_cn1_3}
\begin{aligned}
c_1^{(1)}(t)&=-\frac{\ii F_0 }{2\sqrt{2\hbar m\omega_0}}
\int_0^t\!\rd{t'} \,
\qty(\ee^{\ii(\omega+\omega_0)t'}+\ee^{-\ii(\omega-\omega_0)t'})\\
&=-\frac{\ii F_0}{\sqrt{2\hbar m\omega_0}}\frac{1}{2}
\qty[\frac{\ee^{\ii(\omega+\omega_0)t'}}{\ii(\omega+\omega_0)}
-\frac{\ee^{-\ii(\omega-\omega_0)t'}}{\ii(\omega-\omega_0)}]_0^t\\
&=-\frac{\ii F_0}{\sqrt{2\hbar m\omega_0}}\frac{1}{2}
\Bigg\{\ee^{\ii(\omega+\omega_0)t/2}\frac{\ee^{\ii(\omega+\omega_0)t/2}
-\ee^{-\ii(\omega+\omega_0)t/2}}{\ii(\omega+\omega_0)}
\\ & \hspace{80pt}
-\ee^{-\ii(\omega-\omega_0)t/2}\frac{\ee^{-\ii(\omega-\omega_0)t/2}
-\ee^{+\ii(\omega-\omega_0)t/2}}{\ii(\omega-\omega_0)}
\Bigg\}\\
&=-\frac{\ii F_0}{\sqrt{2\hbar m\omega_0}}
\qty{
\exp[\ii\frac{\omega+\omega_0}{2}t]\frac{\sin(\frac{\omega+\omega_0}{2}t)}
{\omega+\omega_0}
+\exp[-\ii\frac{\omega-\omega_0}{2}t]\frac{\sin(\frac{\omega-\omega_0}{2}t)}
{\omega-\omega_0}
}.
\end{aligned}
\end{equation}

Next is the expectation value
\begin{equation}\label{eq:1_ev_x}
\ev{x}=\bra{i;\,t}x\ket{i;\,t}.
\end{equation}
To calculate this we need to go back to the Schrödinger picture:
\begin{equation}
\ket{i;\,t}=\ee^{-\ii H_0t/\hbar}\ket{i;\,t}_\I
=\ee^{-\ii\omega_0t/2}\qty[
(1+0)\ket{0} + \ee^{-\ii\omega_0t}\qty(0+c_1^{(1)})\ket{1}].
\end{equation}
Now \eqref{eq:1_ev_x} becomes
\begin{equation}
\begin{aligned}
\ev{x}=&
\qty[\bra{0} + \ee^{+\ii\omega_0t}(c_1^{(1)})^*\bra{1}]
\cancel{\ee^{-+\ii\omega_0t/2}}
\quad x \quad
\cancel{\ee^{-\ii\omega_0t/2}}\qty[
\ket{0} + \ee^{-\ii\omega_0t}c_1^{(1)}\ket{1}]\\
=&\mel{0}{x}{0}+\abs{c_1^{(1)}}\mel{1}{x}{1}
+\ee^{+\ii\omega_0t}(c_1^{(1)})^*\mel{1}{x}{0}
+\ee^{-\ii\omega_0t}c_1^{(1)}\mel{0}{x}{1}.
\end{aligned}
\end{equation}
With the help of \eqref{eq:1_mel_x} this simplifies to
\begin{equation}
\begin{aligned}
\ev{x}=&\sqrt{\frac{\hbar}{2m\omega_0}}
\qty[\ee^{+\ii\omega_0t}(c_1^{(1)})^*
+\ee^{-\ii\omega_0t}c_1^{(1)}]\\
=&\sqrt{\frac{\hbar}{2m\omega_0}}
\frac{ F_0}{\sqrt{2\hbar m\omega_0}}\\
&\times\Bigg\{
\ee^{+\ii\omega_0t}(+\ii)\qty[
\exp(-\ii\frac{\omega{+}\omega_0}{2}t)\frac{\sin(\frac{\omega+\omega_0}{2}t)}
{\omega+\omega_0}
+\exp(+\ii\frac{\omega{-}\omega_0}{2}t)\frac{\sin(\frac{\omega-\omega_0}{2}t)}
{\omega-\omega_0}
] \\ & \hspace{8.5pt}
+\ee^{-\ii\omega_0t}(-\ii)\qty[
\exp(+\ii\frac{\omega{+}\omega_0}{2}t)\frac{\sin(\frac{\omega+\omega_0}{2}t)}
{\omega+\omega_0}
+\exp(-\ii\frac{\omega{-}\omega_0}{2}t)\frac{\sin(\frac{\omega-\omega_0}{2}t)}
{\omega-\omega_0}
]
\Bigg\}.
\end{aligned}
\end{equation}
No we collect the coefficients of the sines:
\begin{equation}
\begin{aligned}
\ev{x}=&\frac{\ii F_0}{2 m\omega_0}\Bigg\{
\frac{\sin(\frac{\omega+\omega_0}{2}t)}
{\omega+\omega_0}
\qty[
\ee^{+\ii\omega_0t}\exp(-\ii\frac{\omega+\omega_0}{2}t)-
\ee^{-\ii\omega_0t}\exp(+\ii\frac{\omega+\omega_0}{2}t)
] \\ & \hspace{25pt}
+\frac{\sin(\frac{\omega-\omega_0}{2}t)}
{\omega-\omega_0}
\qty[\ee^{+\ii\omega_0t}\exp(+\ii\frac{\omega-\omega_0}{2}t)
-\ee^{-\ii\omega_0t}\exp(-\ii\frac{\omega-\omega_0}{2}t)
]
\Bigg\}\\
=&\frac{F_0}{m\omega_0}\frac{-1}{2\ii}\Bigg\{
\frac{\sin(\frac{\omega+\omega_0}{2}t)}
{\omega+\omega_0}
\qty[
\exp(-\ii\frac{\omega-\omega_0}{2}t)-
\exp(+\ii\frac{\omega-\omega_0}{2}t)
] \\ & \hspace{35pt}
+\frac{\sin(\frac{\omega-\omega_0}{2}t)}
{\omega-\omega_0}
\qty[\exp(+\ii\frac{\omega+\omega_0}{2}t)
-\exp(-\ii\frac{\omega+\omega_0}{2}t)
]
\Bigg\}\\
=&\frac{F_0}{m\omega_0}
\Bigg\{
\frac{\sin(\frac{\omega+\omega_0}{2}t)}
{\omega+\omega_0}
\qty[ \sin(\frac{\omega-\omega_0}{2}t) ] 
-\frac{\sin(\frac{\omega-\omega_0}{2}t)}
{\omega-\omega_0}
\qty[\sin(\frac{\omega+\omega_0}{2}t)
] \Bigg\}.\\
%=&\frac{F_0}{m\omega_0}
%\sin(\frac{\omega+\omega_0}{2}t)
%\sin(\frac{\omega-\omega_0}{2}t)
%\frac{2\omega_0}{\omega_0^2-\omega^2}
\end{aligned}
\end{equation}
By factoring out the common factors and writing the fractions in a
common denominator, we end up with
\begin{equation}\label{eq:1_final}
\begin{aligned}
\ev{x}=&\frac{F_0}{m\omega_0}
\sin(\frac{\omega+\omega_0}{2}t)
\sin(\frac{\omega-\omega_0}{2}t)
\frac{-2\omega_0}{\omega^2-\omega_0^2}\\
=&\frac{-2F_0}{m\qty(\omega^2-\omega_0^2)}
\sin(\frac{\omega+\omega_0}{2}t)
\sin(\frac{\omega-\omega_0}{2}t).
\end{aligned}
\end{equation}
\qed

\subsection*{Special case}
In the special case of $\omega=\pm\omega_0$, we run in to trouble. The
limit for $\ev{x}$ in \eqref{eq:1_final} does exists, but then we
would get a factor of $t$ instead of a sine. The coefficient
$c^{(1)}_1(t)$ therefore starts growing with time. A more direct way
of finding this ``unboundedness'' of $c^{(1)}_1(t)$, is to look at the
first integral in \eqref{eq:1_cn1_3}; with $\omega=\pm\omega_0$ one of
the terms in the integrand will be constant. 

This is not good, on account of the perturbation series expansion
becoming disordered. I.e. the next order term will have a $t^2$
growth, so for large $t$'s a truncated series will not be any good
approximation.\footnotemark{} 
\footnotetext{However the calculations are still valid, and if we were
  to only regard (a \emph{very} limited set of) cases where
  $t\omega_0\ll1$, formula \eqref{eq:1_final} is still valid and could
  be used even in the limit $\omega\to\pm\omega_0$. }



\section{Photoelectric effect on harmonic oscillator atom}
\newcommand{\kf}{k_{\text{f}}}
\newcommand{\vkf}{\vb{k}_{\text{f}}}
\newcommand{\vq}{\vb{q}}
The differential cross section for ejection of a photoelectron is
\begin{equation}
\dv{\sigma}{\Omega}=\frac{\alpha \kf L^3}{2\pi\hbar m_\ee\omega}
\abs{\mel**{\vkf}{\ee^{\ii\frac{\omega}{c}\vu*{n}\vdot\vb*{r}}
\vu*{\epsilon}\vdot\vb{p}}{i}}^2 
\end{equation}
according to Sakurai \& Napolitano,
\textit{Modern Quantum Mechanics}, ed.2, equation (5.8.32). In our
case we have an initial stat that is the ground state of a 3D
isotropic harmonic oscillator. The wave-function is
\begin{equation}
\psi_{000}(\vb{r})=\qty(\frac{m\omega_0}{2\pi\hbar})^{3/4}
\ee^{-\frac{m\omega_0}{2\hbar}(x^2+y^2+z^2)}
=\qty(\frac{\beta}{\pi})^{3/4}\ee^{-\frac{\beta}{2}r^2}.
\end{equation}

We begin by evaluating the matrix element (call it $A$ for brevity)
\begin{equation}
\begin{aligned}
A=&\mel**{\vkf}{\ee^{\ii\frac{\omega}{c}\vu*{n}\vdot\vb*{r}}
\vu*{\epsilon}\vdot\vb{p}}{i}\\
=&\int\!\rd^3r'\,
\ip{\vkf}{\vb{r}'}
\mel**{\vb{r}'}{\ee^{\ii\frac{\omega}{c}\vu*{n}\vdot\vb*{r}}
\vu*{\epsilon}\vdot\vb{p}}{\psi_{000}}.
\end{aligned}
\end{equation}
The box-normalized wave-function for the free electron is
\begin{equation}
\ip{\vb{r}'}{\vkf}=\frac{1}{L^{3/2}}\ee^{\ii\vkf\vdot\vb{r}'}.
\end{equation}
We therefor get
\begin{equation}
\begin{aligned}
A=&\frac{1}{L^{3/2}}\int\!\rd^3r'\,
\ee^{-\ii\vkf\vdot\vb{r}'}
\ee^{\ii\frac{\omega}{c}\vu*{n}\vdot\vb*{r}'}
\vu*{\epsilon}\vdot(-\ii\hbar\grad)\psi_{000}(\vb{r}')\\
=&-\frac{\ii\hbar}{L^{3/2}} \qty(\frac{\beta}{\pi})^{3/4}
\vu*{\epsilon}\vdot\int\!\rd^3r'\,
\ee^{-\ii(\vkf-\frac{\omega}{c}\vu*{n})\vdot\vb{r}'}
\grad\ee^{-\frac{\beta}{2}{r'}^2}.
\end{aligned}
\end{equation}
We can integrate by part to get
\begin{equation}
\begin{aligned}
A=&\frac{-\ii\hbar}{L^{3/2}} \qty(\frac{\beta}{\pi})^{3/4}
\!\vu*{\epsilon}\vdot\qty{
\cancel{\qty[\ee^{-\ii(\vkf-\frac{\omega}{c}\vu*{n})\vdot\vb{r}'}
\ee^{-\frac{\beta}{2}r'}]_{x,y,z=-\infty}^\infty}
\!\!-\int\!\rd^3r'\,
\grad\qty(\ee^{-\ii(\vkf-\frac{\omega}{c}\vu*{n})\vdot\vb{r}'})
\ee^{-\frac{\beta}{2}{r'}^2}}\\
=&\frac{\ii\hbar}{L^{3/2}} \qty(\frac{\beta}{\pi})^{3/4}
\vu*{\epsilon}\vdot\int\!\rd^3r'\,
-\ii\qty(\vkf-\frac{\omega}{c}\vu*{n})
\ee^{-\ii(\vkf-\frac{\omega}{c}\vu*{n})\vdot\vb{r}'}
\ee^{-\frac{\beta}{2}{r'}^2}\\
=&\frac{\hbar}{L^{3/2}} \qty(\frac{\beta}{\pi})^{3/4}
\vu*{\epsilon}\vdot\vkf\int\!\rd^3r'\,
\ee^{-\ii(\vkf-\frac{\omega}{c}\vu*{n})\vdot\vb{r}'}
\ee^{-\frac{\beta}{2}{r'}^2}\\
\end{aligned}
\end{equation}

Now call $(\vkf-\frac{\omega}{c}\vu*{n})=:\vq$, and call the remaining
integral $I(\vq)$.
To integrate this, we will use spherical coordinate with the
``z-axis'' in the direction of $\vq$. Then we get
\begin{equation}
\begin{aligned}
I(\vq):=&\int\!\rd^3r'\,
\ee^{-\ii\vq\vdot\vb{r}'}
\ee^{-\frac{\beta}{2}r'}\\
=&\int_0^{2\pi}\!\rd\phi\int_0^\pi\!\rd\theta\,\sin\theta
\int_0^\infty\!\rd{r}\,r^2\ee^{\ii q r\cos\theta}\ee^{-\frac{\beta}{2}r^2}.
\end{aligned}
\end{equation}
We can begin by evaluating the $\theta$-integral by setting
$x=\cos\theta$, giving $\rd{x}=-\sin\theta\id\theta$:
\begin{equation}\label{eq:2_theta_int}
\int_0^\pi\!\rd\theta\,\sin\theta\,\ee^{q r\cos\theta}
=-\int_{+1}^{-1}\!\rd{x}\,\ee^{\ii q rx}
=2\frac{\sin(q\,r)}{q\,r}.
\end{equation}
We now have
\begin{equation}
\begin{aligned}
I(\vq)=&\frac{4\pi}{q} \int_0^\infty\!\rd{r}\,
\overbrace{r\sin(q\,r)}^{\text{even}}
\ee^{-\frac{\beta}{2}r^2}=\frac{2\pi}{q} \int_{-\infty}^\infty\!\rd{r}\,
r\sin(q\,r)\ee^{-\frac{\beta}{2}r^2}\\
=&\frac{2\pi}{q} \Im\qty{
\int_{-\infty}^\infty\!\rd{r}\,
r\ee^{\ii q\,r}\ee^{-\frac{\beta}{2}r^2}}\\
=&\frac{2\pi}{q} \Im\qty{
\int_{-\infty}^\infty\!\rd{r}\,
r\exp[-\frac{\beta}{2}\qty(r-\ii\frac{q}{\beta})^2
-\frac{q^2}{2\beta}] }\\
=&\frac{2\pi}{q}\exp[-\frac{q^2}{2\beta}]
\Im\qty{\int_{-\infty}^\infty\!\rd{r}\,
\qty(r-\ii\frac{q}{\beta}+\ii\frac{q}{\beta})
\exp[-\frac{\beta}{2}\qty(r-\ii\frac{q}{\beta})^2]}.
\end{aligned}
\end{equation}
We can now set $\xi=r-\ii q/\beta$ to get
\begin{equation}\label{eq:2_comlex_integral}
\begin{aligned}
I(\vq)=&\frac{2\pi}{q}\exp[-\frac{q^2}{2\beta}]
\Im\qty{\int_{-\infty+\ii q/\beta}^{\infty+\ii q/\beta}\!\rd{\xi}\,
\qty(\cancel{\xi}+\ii\frac{q}{\beta})
\exp[-\frac{\beta}{2}\,\xi^2]},
\end{aligned}
\end{equation}
and the $\xi$ multiplying the exponential is canceled due to the fact
that we're integrating over an even interval and $\xi$ is odd while
the exponential still is even. 


The integration path is a line parallel to and offseted from the real
axis. This offset turns out to not affect the value of the
integral, see appendix~\ref{sec:offset}. So we end up getting a
regular Gaussian integral:
\begin{equation}
\begin{aligned}
I(\vq)=&\frac{2\pi}{q}\exp[-\frac{q^2}{2\beta}]
\Im\qty{\ii\frac{q}{\beta}
\int_{-\infty}^{\infty}\!\rd{\xi}\,
\exp[-\frac{\beta}{2}\,\xi^2]}\\
=&\frac{2\pi}{q}\exp[-\frac{q^2}{2\beta}]
\Im\qty{\ii\frac{q}{\beta}
\sqrt{\frac{\pi}{\beta/2}}}\\
=&\qty(\frac{2\pi}{\beta})^{3/2}\exp[-\frac{q^2}{2\beta}].
\end{aligned}
\end{equation}

Now we are ready to find the differential cross section
\begin{equation}\label{eq:2_almost}
\begin{aligned}
\dv{\sigma}{\Omega}=&\frac{\alpha \kf L^3}{2\pi\hbar m_\ee\omega}
\abs{A}^2=\frac{\alpha \kf L^3}{2\pi\hbar m_\ee\omega}
\abs{\frac{\hbar}{L^{3/2}} \qty(\frac{\beta}{\pi})^{3/4}
\vu*{\epsilon}\vdot\vkf\,
\qty(\frac{2\pi}{\beta})^{3/2}\!\exp[-\frac{q^2}{2\beta}]}^2\\
=&\frac{\alpha \kf L^3}{2\pi\hbar m_\ee\omega}
\abs{\frac{\hbar}{L^{3/2}} 
\vu*{\epsilon}\vdot\vkf\,
\qty(\frac{4\pi}{\beta})^{3/4}\!\exp[-\frac{q^2}{2\beta}]}^2\\
=&\frac{\alpha\kf}{2\pi m_\ee\omega}
\hbar\qty(\frac{4\pi\hbar}{m_\ee\omega_0})^{3/2}
\exp[-\frac{q^2\hbar}{m_\ee\omega_0}](\vu*{\epsilon}\vdot\vkf)^2.
\end{aligned}
\end{equation}
From here, we use equation (5.8.37):
\begin{equation}
\begin{aligned}
(\vu*{\epsilon}\vdot\vkf)^2=&\kf^2\sin^2\theta\,\cos^2\phi\\
q^2=&\kf^2+\qty(\frac{\omega}{c})^2-2\kf\frac{\omega}{c}\cos\theta
\end{aligned}
\end{equation}
from Sakurai \& Napolitano, \textit{Modern Quantum Mechanics}, ed. 2,
to evaluate the expression in \eqref{eq:2_almost}. We get
\begin{equation}
\begin{aligned}
\dv{\sigma}{\Omega}=&\frac{8\alpha\kf}{2\pi m_\ee\omega}
\frac{\pi\hbar^2}{m_\ee\omega_0}
\sqrt{\frac{\pi\hbar}{m_\ee\omega_0}}
\exp[-\frac{\hbar}{m_\ee\omega_0}
\qty(\kf^2+\qty(\frac{\omega}{c})^2-2\kf\frac{\omega}{c}\cos\theta)^2]\\
&\times \kf^2\sin^2\theta\,\cos^2\phi\\
=&\frac{4\alpha\kf^3\hbar^2}{m_\ee^2\omega\omega_0}
\sqrt{\frac{\pi\hbar}{m_\ee\omega_0}}
\exp[-\frac{\hbar}{m_\ee\omega_0}
\qty(\kf^2+\qty(\frac{\omega}{c})^2)]\\
&\times\sin^2\theta\,\cos^2\phi\;
\exp[\frac{2\hbar\kf\omega}{m_\ee\omega_0}\cos\theta].
\end{aligned}
\end{equation}
\qed


\section{Momentum wave-function for hydrogen-like atom}
\newcommand{\vp}{\vb{p}}
\newcommand{\vk}{\vb{k}}
\newcommand{\Za}{\frac{Z}{a_0}}
To find the probability of finding the electron having a certain
momentum, we need to find the momentum representation of the
wave-function. The momentum wave-function is
\begin{equation}\label{eq:3_start}
\hat{\phi}(\vp)=\ip{\vp}{\phi}
=\int\!\rd^3r'\, \ip{\vp}{\vb{r}'}\ip{\vb{r}'}{\phi},
%=\int\!\rd^3r'\,
%\frac{\ee^{-\ii(\vp\vdot\vb{r}')/\hbar}}{(2\pi\hbar)^{3/2}}
%\phi(\vb{r}').
\end{equation}
where
\begin{equation}
\ip{\vp}{\vb{r}'}=\Big(\ip{\vb{r}'}{\vp}\Big)^*
=\frac{\ee^{-\ii(\vp\vdot\vb{r}')/\hbar}}{(2\pi\hbar)^{3/2}}
=\frac{\ee^{-\ii\vk\vdot\vb{r}'}}{(2\pi\hbar)^{3/2}}
\end{equation}
in three dimensions. Next is the position wave-function:
\begin{equation}
\phi(\vb{r}')=\ip{\vb{r}'}{\phi}=\qty(\Za)^{3/2}\ee^{-\Za r'}
=\zeta^{3/2}\ee^{-\zeta r'}
\end{equation}
for the ground state of a hydrogen-like atom, where $\zeta=Z/a_0$.
We can now write \eqref{eq:3_start} as
\begin{equation}
\hat{\phi}(\vp)=\qty(\frac{\zeta}{2\pi\hbar})^{3/2}
\int\!\rd^3r'\,
\ee^{-\ii\vk\vdot\vb{r}'}
\ee^{-\zeta r'}.
\end{equation}
Call the remaining integral $I(\vk)$.

To evaluate $I(\vk)$, we will use spherical coordinates with the
``z-axis'' aligned along $\vk$. We now get
\begin{equation}
\begin{aligned}
I(\vk):=&\int\!\rd^3r'\,\ee^{-\ii\vk\vdot\vb{r}'} \ee^{-\zeta r'}\\
=&\int_0^{2\pi}\!\rd\phi\int_0^\pi\!\rd\theta\,\sin\theta
\int_0^\infty\!\rd{r}\,r^2\ee^{-\ii kr\cos\theta}\ee^{-\zeta r}.
\end{aligned}
\end{equation}
We can evaluate the $\theta$-integral in the same way as in
\eqref{eq:2_theta_int}, to get
\begin{equation}\label{eq:3_sloped_path}
\begin{aligned}
I(\vk)=&\frac{4\pi}{k}
\int_0^\infty\!\rd{r}\,r\sin(kr)\ee^{-\zeta r}\\
=&\frac{4\pi}{k}\Im\qty{
\int_0^\infty\!\rd{r}\,r\ee^{\ii kr}\ee^{-\zeta r}
} =\frac{4\pi}{k}\Im\qty{
\int_0^\infty\!\rd{r}\,r\ee^{-(\zeta-\ii k) r}
}.
\end{aligned}
\end{equation}
Now, it turns out that although the factor in the exponential is
complex, the integral can be evaluated as if the factor were real, see
appendix~\ref{sec:sloped_path}. We thus get
\begin{equation}
\begin{aligned}
I(\vk)=&\frac{4\pi}{k}
\Im\qty{\frac{1}{(\zeta-\ii k)^2}}
=\frac{4\pi}{k}
\Im\qty{\frac{1}{\zeta^2-k^2 -2\ii\zeta k}}\\
=&\frac{4\pi}{k}
\Im\qty{\frac{\zeta^2-k^2 +2\ii\zeta k}
{\qty(\zeta^2-k^2)^2 +4\zeta^2k^2}}\\
=&\frac{4\pi}{k}
\frac{2\zeta k}
{\qty(\zeta^4-2\zeta^2k^2+k^4) +4\zeta^2k^2}
=\frac{8\pi\zeta}{\qty(\zeta^2+k^2)^2}.
\end{aligned}
\end{equation}

Thus the momentum wave-function now is
\begin{equation}
\hat{\phi}(\vp)=\qty(\frac{\zeta}{2\pi\hbar})^{3/2}
\frac{8\pi\zeta}{\qty(\zeta^2+k^2)^2}
=\frac{8\pi}{(2\pi\hbar)^{3/2}}\qty(\Za)^{5/2}
\frac{1}{\qty(Z^2/a_0^2+k^2)^2},
\end{equation}
and the probability of finding the electron with momentum $\vp$ is
\begin{equation}
\abs{\hat{\phi}(\vp)}^2\rd^3\!p=
\frac{64\pi^2}{(2\pi\hbar)^{3}}\qty(\Za)^{5}
\frac{\rd^3\!p}{\qty(Z^2/a_0^2+k^2)^4},
\end{equation}
where $k=\abs{\vb{p}}/\hbar$.
\qed

\newpage
\appendix
\renewcommand{\thesubsection}{\Alph{section}.\arabic{subsection}}
\section{Complex integrals}
In this appendix I will present more details regarding the complex
integrals in problem 2 and 3. 

\subsection{Offseted integration path}\label{sec:offset}
Here we will show that the  imaginary offset in the integral in
\eqref{eq:2_comlex_integral} gives the same value as if there was no
offset. To begin with we will only study the partial integral from
$-R+\ii q$ to $R+\ii q$. Next we can use the fact that the
integrand, $f(z)=a\ee^{-bz^2}$, is analytic for all finite $z$, which
allows us to deform the integration path into the shape in
\figref{fig:int_path_2}.

Now we want to show that, as $R\to\infty$, the contribution from the
vertical integrations will disappear. The absolute value of the
left vertical integrations is given as
\begin{equation}
\abs{\int_{q}^0\!\rd{y}\,a\ee^{-b(-R+\ii y)^2}}
=\abs{\int_{0}^q\!\rd{y}\,a\ee^{-b(R^2+2\ii Ry - y^2)}}
%\le\int_{0}^q\!\rd{y}\abs{a\ee^{-b(R^2+2\ii Ry - y^2)}}
\le\abs{a}\ee^{-b(R^2-Q)}\to0\;\text{as}\;R\to\infty
\end{equation}
for some constant value $Q$ originating from the absolute integration
of $\ee^{-b(-y^2)}$. The contribution from the right vertical integral
follows similarly. 

Also note that even if the depiction in \figref{fig:int_path_2}
suggests that the offset is into the upper half-plane ($q>0$), the argument
above also holds for offsets into the lower half-plane ($q<0$).

We can now conclude that, as $R\to\infty$, the offseted integral will
only have contribution from the path along the real axis. Thus the
integral in \eqref{eq:2_comlex_integral} will have the value as a
regular Gaussian integral.

\begin{figure}\centering
\resizebox{.5\textwidth}{!}{\input{figures/integration_path_2.pdf_t}}
\caption{Deformation of the integration path in
  \eqref{eq:2_comlex_integral}. The contribution from the vertical
  paths will disappear in the limit $R\to\infty$.}
\label{fig:int_path_2}
\end{figure}


\subsection{Integration along an angled path} \label{sec:sloped_path}
To evaluate the integral in \eqref{eq:3_sloped_path}, we first set
$z=(\zeta-\ii k)r$, then $r\rd{r}=(\zeta-\ii k)^{-2}z\rd{z}$. Thus
\begin{equation}
\int_0^\infty\!\rd{r}\,r\ee^{-(\zeta-\ii k) r}
=\frac{1}{(\zeta-\ii k)^{2}}
\oldint_{\gamma}\!\rd{z}\, z\ee^{-z},
\end{equation}
where $\gamma$ is an angled path making an angle $\varTheta=\arg\{\zeta-\ii k\}$.

\begin{figure}\centering
\resizebox{.4\textwidth}{!}{\input{figures/closed_path.pdf_t}}
\caption{A closed path tracing out a circle segment with the two
  radii connecting the segment to the origin. Here
  $\varGamma=\gamma_R\cup C_R\cup\gamma_R^{(\R)}$ denotes the full closed path.}
\label{fig:closed_path}
\end{figure}

Call $f(z):=z\ee^{-z}$. We have
\begin{equation}
\oldint_{\gamma}f(z)\id{z}
=\lim_{R\to\infty}\oldint_{\gamma_R}f(z)\id{z}
=\lim_{R\to\infty}\qty[
\oint_{\varGamma}f(z)\id{z}
-\oldint_{C_R}f(z)\id{z}
-\oldint_{\gamma_R^{(\R)}}f(z)\id{z}
].
\end{equation}

Now by Cauchy's integral theorem
\begin{equation}
\oint_{\varGamma}f(z)\id{z}=0
\end{equation}
since $f$ is analytic. Next
\begin{equation}\label{eq:A2_CR}
\abs{\oldint_{C_R}f(z)\id{z}}
=\abs{\int_\varTheta^0R\ee^{\ii\theta}\exp[-R\ee^{\ii\theta}]
\;\ii R\ee^{\ii\theta}\rd{\theta}}
\le R^2\exp[-R\cos\Theta]\to0\;\text{as}\ R\to\infty.
\end{equation}
Therefore all that we are left with is
\begin{equation}
\oldint_{\gamma}f(z)\id{z}
=\lim_{R\to\infty}-\!\oldint_{\gamma_R^{(\R)}}f(z)\id{z}
=\int_0^\infty x\ee^{-x}\id{x}=1,
\end{equation}
and
\begin{equation}
\int_0^\infty\!\rd{r}\,r\ee^{-(\zeta-\ii k) r}
=\frac{1}{(\zeta-\ii k)^{2}}
\oldint_{\gamma}\!\rd{z}\, z\ee^{-z}
=\frac{1}{(\zeta-\ii k)^{2}}
\end{equation}

\paragraph{Note: }
This argument relies on the fact that $\zeta=Z/a_0>0$, so that the
limit in \eqref{eq:A2_CR} actually is $0$. Otherwise, there is no
other limitations, i.e. $k$ can be both positive and negative. 

\end{document}





%% På svenska ska citattecknet vara samma i både början och slut.
%% Använd två apostrofer (två enkelfjongar): ''.


%% Inkludera PDF-dokument
\includepdf[pages={1-}]{filnamn.pdf} %Filnamnet får INTE innehålla 'mellanslag'!

%% Figurer inkluderade som pdf-filer
\begin{figure}\centering
\centerline{ %centrerar även större bilder
\includegraphics[width=1\textwidth]{filnamn.pdf}
}
\caption{}
\label{fig:}
\end{figure}

%% Figurer inkluderade med xfigs "Combined PDF/LaTeX"
\begin{figure}\centering
\resizebox{.8\textwidth}{!}{\input{filnamn.pdf_t}}
\caption{}
\label{fig:}
\end{figure}

%% Figurer roterade 90 grader
\begin{sidewaysfigure}\centering
\centerline{ %centrerar även större bilder
\includegraphics[width=1\textwidth]{filnamn.pdf}
}
\caption{}
\label{fig:}
\end{sidewaysfigure}


%%Om man vill lägga till något i TOC
\stepcounter{section} %Till exempel en 'section'
\addcontentsline{toc}{section}{\Alph{section}\hspace{8 pt}Labblogg} 


%  LocalWords:  stateket Sakurai Napolitano photoelectron
