\documentclass[11pt,letter, swedish, english
]{article}
\pdfoutput=1

\usepackage{../custom_as}
\usepackage{cancel}
\graphicspath{ {figures/} }

%%Drar in tabell och figurtexter
\usepackage[margin=10 pt]{caption}
%%För att lägga in 'att göra'-noteringar i texten
\usepackage{todonotes} %\todo{...}

%%För att själv bestämma marginalerna. 
\usepackage[
%            top    = 3cm,
%            bottom = 3cm,
%            left   = 3cm, right  = 3cm
]{geometry}

%%För att ändra hur rubrikerna ska formateras
\renewcommand{\thesubsection}{\arabic{section} (\alph{subsection})}

\renewcommand{\thesubsubsection}{\arabic{section} (\alph{subsection},\,\roman{subsubsection})}



\swapcommands{\varPhi}{\Phi}
\swapcommands{\varPi}{\Pi}
\swapcommands{\varOmega}{\Omega}



\begin{document}

%%%%%%%%%%%%%%%%% vvv Inbyggd titelsida vvv %%%%%%%%%%%%%%%%%
% \begin{titlepage}
\title{Quantum Mechanics -- PHYS\,701 \\
Assignment 4}
\author{Andréas Sundström}
\date{\today}

\maketitle

%%%%%%%%%%%%%%%%% ^^^ Inbyggd titelsida ^^^ %%%%%%%%%%%%%%%%%

%Om man vill ha en lista med vilka todo:s som finns.
%\todolist

\section{Time-dependent perturbation of harmonic oscillator}
\newcommand{\I}{\text{I}}
%\newcommand{\S}{\text{S}}
We have a simple harmonic oscillator in it's groundstate
\begin{equation}
\ket{i;\,t}=\ket{0}\qcomma
\text{for}\ t\le0.
\end{equation}
At $t=0$ a potential
\begin{equation}
V(t)=F_0x\cos\omega t
\end{equation}
is turned on. Now we want to appoximate $\ev{x}$ as a funtion of time.

We begin by expressing the stateket in the interaction picture
\begin{equation}
\ket{i;\,t}_\I = \sum_n \ket{n}\!\!\mel*{n}{U_\I(t)}{i},
=\sum_n c_n(t)\ket{n}
\end{equation}
where in our case $i=0$. 

We now need to find $c_n(t)$, but we can only find an approximation, 
$c_n=c_n^{(0)}+c_n^{(1)}+\ldots$, based on the Dyson series for
$c_n$. This approximation begins with
\begin{equation}
c_n^{(0)}(t)=\delta_{n,i}
\end{equation}
and
\begin{equation}
\begin{aligned}
c_n^{(1)}(t)=&-\frac{\ii}{\hbar}
\int_0^t\!\rd{t'}\, \mel{n}{V_\I(t')}{i}\\
=&-\frac{\ii}{\hbar}
\int_0^t\!\rd{t'} \,\ee^{\ii\omega_{n,i}t'}\mel{n}{V(t')}{i},
\end{aligned}
\end{equation}
where $\omega_{n,i}=(E_n-E_i)/\hbar$. See Sakurai \& Napolitano,
\textit{Modern Quantum Mechanics}, ed.2, equation (5.7.17).

In our case we have $i=0$ and we get
\begin{equation}\label{eq:1_cn1_1}
\begin{aligned}
c_n^{(1)}(t)=&-\frac{\ii}{\hbar}
\int_0^t\!\rd{t'} \,\ee^{\ii\omega_{n,0}t'}
\mel{n}{F_0x\cos\omega t'}{0}\\
=&-\frac{\ii F_0}{\hbar}
\int_0^t\!\rd{t'} \,\ee^{\ii n\omega_{0}t'}
\cos(\omega t')\mel{n}{x}{0}.
\end{aligned}
\end{equation}
Using
\begin{equation}\label{eq:1_mel_x}
\mel{n}{x}{m}=sqrt{\frac{\hbar}{2m\omega_0}}
\Big[ \sqrt{m+1}\delta_{n,\,n+1}+\sqrt{m}\delta_{n,\,n-1}
\Big]
\end{equation}
we can write \eqref{eq:1_cn1_1} as
\begin{equation}\label{eq:1_cn1_2}
\begin{aligned}
c_n^{(1)}(t)=&-\frac{\ii F_0}{\hbar}\sqrt{\frac{\hbar}{2m\omega_0}}
\int_0^t\!\rd{t'} \,\ee^{\ii n\omega_{0}t'}
\cos(\omega t')\Big[ 
\sqrt{1}\delta_{n,\,1}+\cancel{\sqrt{0}\delta_{n,\,-1}}\Big]\\
&=-\frac{\ii F_0 \delta_{n,\,1}}{\sqrt{2\hbar m\omega_0}}
\int_0^t\!\rd{t'} \,\ee^{\ii n\omega_{0}t'}
\frac{1}{2}\qty(\ee^{\ii \omega t'}+\ee^{-\ii \omega t'}).
\end{aligned}
\end{equation}
We see no that the only non-zero coefficient here is $c_1^{(1)}$, so
we continue by looking at
\begin{equation}\label{eq:1_cn1_3}
\begin{aligned}
c_1^{(1)}(t)&=-\frac{\ii F_0 }{2\sqrt{2\hbar m\omega_0}}
\int_0^t\!\rd{t'} \,
\qty(\ee^{\ii(\omega+\omega_0)t'}+\ee^{-\ii(\omega-\omega_0)t'})\\
&=-\frac{\ii F_0}{\sqrt{2\hbar m\omega_0}}\frac{1}{2}
\qty[\frac{\ee^{\ii(\omega+\omega_0)t'}}{\ii(\omega+\omega_0)}
-\frac{\ee^{-\ii(\omega-\omega_0)t'}}{\ii(\omega-\omega_0)}]_0^t\\
&=-\frac{\ii F_0}{\sqrt{2\hbar m\omega_0}}\frac{1}{2}
\Bigg\{\ee^{\ii(\omega+\omega_0)t/2}\frac{\ee^{\ii(\omega+\omega_0)t/2}
-\ee^{-\ii(\omega+\omega_0)t/2}}{\ii(\omega+\omega_0)}
\\ & \hspace{80pt}
-\ee^{-\ii(\omega-\omega_0)t/2}\frac{\ee^{-\ii(\omega-\omega_0)t/2}
-\ee^{+\ii(\omega-\omega_0)t/2}}{\ii(\omega-\omega_0)}
\Bigg\}\\
&=-\frac{\ii F_0}{\sqrt{2\hbar m\omega_0}}
\qty{
\exp[\ii\frac{\omega+\omega_0}{2}t]\frac{\sin(\frac{\omega+\omega_0}{2}t)}
{\omega+\omega_0}
+\exp[-\ii\frac{\omega-\omega_0}{2}t]\frac{\sin(\frac{\omega-\omega_0}{2}t)}
{\omega-\omega_0}
}.
\end{aligned}
\end{equation}

Next is the expectationvalue
\begin{equation}\label{eq:1_ev_x}
\ev{x}=\bra{i;\,t}x\ket{i;\,t}.
\end{equation}
To calculate this we need to go back to the Schrödinger picture:
\begin{equation}
\ket{i;\,t}=\ee^{-\ii H_0t/\hbar}\ket{i;\,t}_\I
=\ee^{-\ii\omega_0t/2}\qty[
(1+0)\ket{0} + \ee^{-\ii\omega_0t}\qty(0+c_1^{(1)})\ket{1}].
\end{equation}
Now \eqref{eq:1_ev_x} becomes
\begin{equation}
\begin{aligned}
\ev{x}=&
\qty[\bra{0} + \ee^{+\ii\omega_0t}(c_1^{(1)})^*\bra{1}]
\cancel{\ee^{-+\ii\omega_0t/2}}
\quad x \quad
\cancel{\ee^{-\ii\omega_0t/2}}\qty[
\ket{0} + \ee^{-\ii\omega_0t}c_1^{(1)}\ket{1}]\\
=&\mel{0}{x}{0}+\abs{c_1^{(1)}}\mel{1}{x}{1}
+\ee^{+\ii\omega_0t}(c_1^{(1)})^*\mel{1}{x}{0}
+\ee^{-\ii\omega_0t}c_1^{(1)}\mel{0}{x}{1}.
\end{aligned}
\end{equation}
With the help of \eqref{eq:1_mel_x} this simplifies to
\begin{equation}
\begin{aligned}
\ev{x}=&\sqrt{\frac{\hbar}{2m\omega_0}}
\qty[\ee^{+\ii\omega_0t}(c_1^{(1)})^*
+\ee^{-\ii\omega_0t}c_1^{(1)}]\\
=&\sqrt{\frac{\hbar}{2m\omega_0}}
\frac{ F_0}{\sqrt{2\hbar m\omega_0}}\\
&\times\Bigg\{
\ee^{+\ii\omega_0t}(+\ii)\qty[
\exp(-\ii\frac{\omega{+}\omega_0}{2}t)\frac{\sin(\frac{\omega+\omega_0}{2}t)}
{\omega+\omega_0}
+\exp(+\ii\frac{\omega{-}\omega_0}{2}t)\frac{\sin(\frac{\omega-\omega_0}{2}t)}
{\omega-\omega_0}
] \\ & \hspace{8.5pt}
+\ee^{-\ii\omega_0t}(-\ii)\qty[
\exp(+\ii\frac{\omega{+}\omega_0}{2}t)\frac{\sin(\frac{\omega+\omega_0}{2}t)}
{\omega+\omega_0}
+\exp(-\ii\frac{\omega{-}\omega_0}{2}t)\frac{\sin(\frac{\omega-\omega_0}{2}t)}
{\omega-\omega_0}
]
\Bigg\}.
\end{aligned}
\end{equation}
No we collect the coefficients of the sines:
\begin{equation}
\begin{aligned}
\ev{x}=&\frac{\ii F_0}{2 m\omega_0}\Bigg\{
\frac{\sin(\frac{\omega+\omega_0}{2}t)}
{\omega+\omega_0}
\qty[
\ee^{+\ii\omega_0t}\exp(-\ii\frac{\omega+\omega_0}{2}t)-
\ee^{-\ii\omega_0t}\exp(+\ii\frac{\omega+\omega_0}{2}t)
] \\ & \hspace{25pt}
+\frac{\sin(\frac{\omega-\omega_0}{2}t)}
{\omega-\omega_0}
\qty[\ee^{+\ii\omega_0t}\exp(+\ii\frac{\omega-\omega_0}{2}t)
-\ee^{-\ii\omega_0t}\exp(-\ii\frac{\omega-\omega_0}{2}t)
]
\Bigg\}\\
=&\frac{F_0}{m\omega_0}\frac{-1}{2\ii}\Bigg\{
\frac{\sin(\frac{\omega+\omega_0}{2}t)}
{\omega+\omega_0}
\qty[
\exp(-\ii\frac{\omega-\omega_0}{2}t)-
\exp(+\ii\frac{\omega-\omega_0}{2}t)
] \\ & \hspace{35pt}
+\frac{\sin(\frac{\omega-\omega_0}{2}t)}
{\omega-\omega_0}
\qty[\exp(+\ii\frac{\omega+\omega_0}{2}t)
-\exp(-\ii\frac{\omega+\omega_0}{2}t)
]
\Bigg\}\\
=&\frac{F_0}{m\omega_0}
\Bigg\{
\frac{\sin(\frac{\omega+\omega_0}{2}t)}
{\omega+\omega_0}
\qty[ \sin(\frac{\omega-\omega_0}{2}t) ] 
-\frac{\sin(\frac{\omega-\omega_0}{2}t)}
{\omega-\omega_0}
\qty[\sin(\frac{\omega+\omega_0}{2}t)
] \Bigg\}.\\
%=&\frac{F_0}{m\omega_0}
%\sin(\frac{\omega+\omega_0}{2}t)
%\sin(\frac{\omega-\omega_0}{2}t)
%\frac{2\omega_0}{\omega_0^2-\omega^2}
\end{aligned}
\end{equation}
By factoring out the common factors and writing the fractions in a
common denominator, we end up with
\begin{equation}
\begin{aligned}
\ev{x}=&\frac{F_0}{m\omega_0}
\sin(\frac{\omega+\omega_0}{2}t)
\sin(\frac{\omega-\omega_0}{2}t)
\frac{-2\omega_0}{\omega^2-\omega_0^2}\\
=&\frac{-2F_0}{m\qty(\omega^2-\omega_0^2)}
\sin(\frac{\omega+\omega_0}{2}t)
\sin(\frac{\omega-\omega_0}{2}t).
\end{aligned}
\end{equation}
\qed



\section{Photoelectric effect on harmonic oscillator atom}
The differeintial cross section for ejection of a photoelectron is
\begin{equation}
\dv{\sigma}{\Omega}
\end{equation}

\section{Momentum wave-function for hydrogen-like atom}





\end{document}





%% På svenska ska citattecknet vara samma i både början och slut.
%% Använd två apostrofer (två enkelfjongar): ''.


%% Inkludera PDF-dokument
\includepdf[pages={1-}]{filnamn.pdf} %Filnamnet får INTE innehålla 'mellanslag'!

%% Figurer inkluderade som pdf-filer
\begin{figure}\centering
\centerline{ %centrerar även större bilder
\includegraphics[width=1\textwidth]{filnamn.pdf}
}
\caption{}
\label{fig:}
\end{figure}

%% Figurer inkluderade med xfigs "Combined PDF/LaTeX"
\begin{figure}\centering
\resizebox{.8\textwidth}{!}{\input{filnamn.pdf_t}}
\caption{}
\label{fig:}
\end{figure}

%% Figurer roterade 90 grader
\begin{sidewaysfigure}\centering
\centerline{ %centrerar även större bilder
\includegraphics[width=1\textwidth]{filnamn.pdf}
}
\caption{}
\label{fig:}
\end{sidewaysfigure}


%%Om man vill lägga till något i TOC
\stepcounter{section} %Till exempel en 'section'
\addcontentsline{toc}{section}{\Alph{section}\hspace{8 pt}Labblogg} 

