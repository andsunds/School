\documentclass[11pt,letter, swedish, english
]{article}
\pdfoutput=1

\usepackage{../custom_as}

\renewcommand{\thesubsection}{\arabic{section} (\alph{subsection})}

\renewcommand{\thesubsubsection}{\arabic{section} (\alph{subsection},\,\roman{subsubsection})}


%%Drar in tabell och figurtexter
\usepackage[margin=10 pt]{caption}
%%För att lägga in 'att göra'-noteringar i texten
\usepackage{todonotes} %\todo{...}

%%För att själv bestämma marginalerna. 
\usepackage[
%            top    = 3cm,
%            bottom = 3cm,
%            left   = 3cm, right  = 3cm
]{geometry}

\DeclareMathAlphabet{\mathpzc}{OT1}{pzc}{m}{it}
\newcommand{\oh}{\ensuremath\mathpzc{o}}

\newcommand{\as}{\qcomma\text{as }}

\renewcommand{\thefootnote}{\fnsymbol{footnote}}

\begin{document}

%%%%%%%%%%%%%%%%% vvv Inbyggd titelsida vvv %%%%%%%%%%%%%%%%%
% \begin{titlepage}
\title{Asymptotic Analasys and Pertubation Theory -- AMATH\,732 \\
Assignment 4}
\author{Andréas Sundström}
\date{\today}

\maketitle

%%%%%%%%%%%%%%%%% ^^^ Inbyggd titelsida ^^^ %%%%%%%%%%%%%%%%%

%Om man vill ha en lista med vilka todo:s som finns.
%\todolist

\section{An IVP}

\begin{equation}
y''+(1+\epsilon x)y=0;\qquad
y(0)=1,\quad y'(0)=0
\end{equation}

\section{Underdamped linear damped oscillator}

\begin{equation}
\ddot{x}+2\epsilon\dot{x}+x=0;\qquad
x(0)=0,\quad \dot{x}(0)=0
\end{equation}


\subsection{Introducing new time scales}

\subsection{Leading order problem}

\subsection{First solvability condition}

\subsection{Second solvability condition}



\section{Free Duffing oscillator}

\begin{equation}
x''+x=\epsilon {x}^3+;\qquad
x(0)=0,\quad {x'}(0)=0
\end{equation}

\section{Another oscillator}

\begin{equation}
y''+y=\epsilon y^2;\qquad
y(0)=1,\quad y'(0)=0
\end{equation}

\subsection{RPT}

\subsection{The Poincaré-Lindstedt method}

\subsection{Pritulo's method}



\section{Non-linear pendulum, with Pritulo's technique}

\begin{equation}
\theta''+\sin\theta =0;\qquad
\theta(0)=a,\quad \theta'(0)=0
\end{equation}


\section{Linear dsipersive wave}


\begin{equation}
\eta_t + c_0\eta_{x}+\beta\eta_{xxx}=0
\end{equation}

\begin{equation}
\eta(x, 0)=\eta_0(x)=A(\epsilon x)\cos(kx)
\end{equation}


\end{document}



