\documentclass[11pt,letter, swedish, english
]{article}
\pdfoutput=1

\usepackage{../custom_as}
\usepackage[%makeroom
]{cancel}
\graphicspath{{figures/}}

\swapcommands{\Delta}{\varDelta}
\swapcommands{\Omega}{\varOmega}

%%Drar in tabell och figurtexter
\usepackage[margin=10 pt]{caption}
%%För att lägga in 'att göra'-noteringar i texten
\usepackage{todonotes} %\todo{...}

%%För att själv bestämma marginalerna. 
\usepackage[
%            top    = 3cm,
%            bottom = 3cm,
%            left   = 3cm, right  = 3cm
]{geometry}

%%För att ändra hur rubrikerna ska formateras
\renewcommand{\thesubsection}{\arabic{section} (\alph{subsection})}

\renewcommand{\thesubsubsection}{\arabic{section} (\alph{subsection},\,\roman{subsubsection})}

\renewcommand{\thefootnote}{\fnsymbol{footnote}}

\newcommand{\Tc}{\ensuremath{T_{\text{c}}}}
\newcommand{\eF}{\ensuremath{\epsilon_{\text{F}}}}
\newcommand{\wD}{\ensuremath{\omega_{\text{D}}}}

%\usepackage{tikz}

\begin{document}

%\tikzstyle{every picture}+=[remember picture]
%\tikzstyle{na} = [shape=rectangle,inner sep=0pt,text depth=0pt]



%%%%%%%%%%%%%%%%% vvv Inbyggd titelsida vvv %%%%%%%%%%%%%%%%%

\title{Statistical Physics -- PHYS\,704 \\
Assignment 4}
\author{Andréas Sundström}
\date{\today}

\maketitle

%%%%%%%%%%%%%%%%% ^^^ Inbyggd titelsida ^^^ %%%%%%%%%%%%%%%%%

\section{The internal energy and heat capacity of a superconductor}
We will consider a superconductor at finite temperature $0<T<\Tc$. One
very useful equation in such cases is the BCS equation
\begin{equation}\label{eq:BCS}
\Delta=\frac{V}{U}\sum_k \frac{\Delta}{2E_k}\Big(1-2n(E_k)\Big),
\end{equation}
where
\begin{equation}
E_k=\sqrt{\xi_k^2+\Delta^2}\qcomma \xi_k=\epsilon_k-\eF,
\end{equation}
and
\begin{equation}
n(E)=\frac{1}{\ee^{\frac{E}{T}}+1}
\end{equation}
is the Fermi-Dirac distribution.

\subsection{The internal energy}
For a superconductor, we have the Hamiltonian
\begin{equation}
H=\sum_{k, \sigma} E_k\gamma_{k \sigma}^\dagger\gamma_{k \sigma}
+\frac{V\Delta^2}{U}+\sum_k\qty(\xi_k-E_k).
\end{equation}
The internal energy is given as
\begin{equation}\label{eq:1a_E0}
E(T)=\ev{H}=
\sum_{k, \sigma} E_k\,n(E_k)
+\frac{V\Delta^2}{U}+\sum_k\qty(\xi_k-E_k).
\end{equation}
The numberoperator for the quasiparticles in the first term is
converted to a Fermi-Dirac distribution, since the quasi particles are
an ideal Fermi-gas. 

Now using the BCS equation, we can rewrite the middle
term as
\begin{equation}
\Delta\frac{U\Delta}{V} \stackrel{\eqref{eq:BCS}}{=}
\Delta\sum_k \frac{\Delta}{2E_k}\Big(1-2n(E_k)\Big)
=\sum_k \frac{\Delta^2}{2E_k}\Big(1-2n(E_k)\Big).
\end{equation}
Now it's just a matter of collectin the terms in
\eqref{eq:1a_E0}. First we recognize that the sum over different sipns
is just going to give us a factor~2. And we end up with
\begin{equation}
\begin{aligned}
E(T)&=\sum_k\qty[2E_k\,n(E_k)
+\frac{\Delta^2}{2E_k}\Big(1-2n(E_k)\Big)
+\xi_k-E_k ]\\
&=\sum_k\qty[
\xi_k-\qty(E_k-\frac{\Delta^2}{2E_k})\Big(1-2n(E_k)\Big) ].
\end{aligned}
\end{equation}
\qed

\subsection{The heat capacity at $T<\Tc$}
To get the heat capacity, we diffentiate $E$ with respect to $T$:
\begin{equation}\label{eq:1b_Cv0}
\begin{aligned}
C_V=\dv{E}{T}=\sum_k\Bigg[&
\cancel{\dv{\xi_k}{T}}-\dv{E_k}{T}\Big(1-2n(E_k)\Big)
-E_k\dv{T}\Big(1-2n(E_k)\Big)\\
&+\dv{\Delta^2}{T}\frac{1}{2E_k}\Big(1-2n(E_k)\Big) 
+\Delta^2 \dv{T}\qty(\frac{1-2n(E_k)}{2E_k})
\Bigg].
\end{aligned}
\end{equation}
From here we will need 
\begin{equation}\label{eq:1b_dEk/dT}
\dv{E_k}{T}=\pdv{E_k}{(\Delta^2)}\pdv{\Delta^2}{T}
=\frac{1}{2\sqrt{\xi_k^2+\Delta^2}}\pdv{\Delta^2}{T}
=\frac{1}{2E_k}\pdv{\Delta^2}{T}.
\end{equation}
And
\begin{equation}\label{eq:1b_dn/dT}
\dv{n(E_k)}{T}=\pdv{n}{E_k}\dv{E_k}{T}+\pdv{n}{T}
=-\frac{\frac{1}{T}\ee^{\nicefrac{E_k}{T}}}
{\qty(\ee^{\nicefrac{E_k}{T}}+1)^2}\frac{1}{2E_k}\pdv{\Delta^2}{T}
-\frac{-\frac{E_k}{T^2}\ee^{\nicefrac{E_k}{T}}}
{\qty(\ee^{\nicefrac{E_k}{T}}+1)^2}.
\end{equation}

Next we take a closer look at the last term in \eqref{eq:1b_Cv0}. For
$T<\Tc$ mening that $\Delta\neq0$, the BCS equation can be written as
\begin{equation}\label{eq:1b_BCS}
\frac{U}{V}=\sum_k\frac{1-2n(E_k)}{2E_k}.
\end{equation}
If we now diffentiate both sides we get
\begin{equation}
0=\dv{T}\sum_k\frac{1-2n(E_k)}{2E_k},
\end{equation}
since the LHS of \eqref{eq:1b_BCS} is a constant. In other words, the
last term of \eqref{eq:1b_Cv0} disappears.

Now th heat capacity becomes
\begin{equation}
C_V=\sum_k\Bigg[
-\cancel{\dv{E_k}{T}\Big(1-2n(E_k)\Big)}
+E_k\dv{T}\Big(2n(E_k)\Big)
+\cancel{\dv{\Delta^2}{T}\frac{1}{2E_k}\Big(1-2n(E_k)\Big)}
\Bigg].
\end{equation}
Here the first and last term cancel each other thanks to
\eqref{eq:1b_dEk/dT}. With \eqref{eq:1b_dn/dT}, we end up with
\begin{equation}
\begin{aligned}
C_V=&\sum_k2E_k\qty[
-\frac{\frac{1}{T}\ee^{\nicefrac{E_k}{T}}}
{\qty(\ee^{\nicefrac{E_k}{T}}+1)^2} \frac{1}{2E_k}\pdv{\Delta^2}{T}
-\frac{-\frac{E_k}{T^2}\ee^{\nicefrac{E_k}{T}}}
{\qty(\ee^{\nicefrac{E_k}{T}}+1)^2} ]\\
=&\frac{1}{T}\sum_k\qty(\frac{2E_k^2}{T}-\pdv{\Delta^2}{T})
\frac{\ee^{\nicefrac{E_k}{T}}}{\qty(\ee^{\nicefrac{E_k}{T}}+1)^2}.
\end{aligned}
\end{equation}
Since we're in the low temperature limit 
\begin{equation}
\frac{E_k^2}{T}\gg\pdv{\Delta^2}{T},
\end{equation}
because we know that $\pdv*{\Delta}{T}$ is bounded. Next, the low
temperature limit means that
\begin{equation}
\frac{\ee^{\nicefrac{E_k}{T}}}{\qty(\ee^{\nicefrac{E_k}{T}}+1)^2}
\approx
\frac{\ee^{\nicefrac{E_k}{T}}}{\qty(\ee^{\nicefrac{E_k}{T}})^2}
=\ee^{-\nicefrac{E_k}{T}}
=\exp(-\frac{\sqrt{\xi^2+\Delta^2}}{T}).
\end{equation}
The final step is to go from a sum to an integral, which we do with
the prescription
\begin{equation}\label{eq:1_sum_int}
\frac{1}{V}\sum_k \to g(\eF)\int_{-\hbar\wD}^{\hbar\wD}\!\rd\xi.
\end{equation}
This results in
\begin{equation}
\begin{aligned}
C_V\to&\frac{2V}{T^2}g(\eF)\int_{-\hbar\wD}^{\hbar\wD}\!\rd\xi\;
(\xi^2+\Delta^2)\exp(-\frac{\sqrt{\xi^2+\Delta^2}}{T}).
\end{aligned}
\end{equation}
To deal with this integral we will need to use Laplace's method.

\subsubsection*{Laplace's method}
Call $f(\xi)=\xi^2+\Delta^2$ and
$\varphi(\xi)=\sqrt{\xi^2+\Delta^2}$. Then the integral in question is
\begin{equation}
I(\Delta)=\int_{-\hbar\wD}^{\hbar\wD}\!\rd\xi\;f(\xi)\exp(-\frac{\varphi(\xi)}{T}).
\end{equation}
Now, $T$ is small, which means that $\ee^{-\varphi(\xi)/T}\lll1$, and
the, by far, biggest contribution to the total integral comes from the
immidiate vicinity of the minimum of $\varphi$, $\xi_0$. Everything else is
effectively killed off by the exponential. 
We can therefore Taylor expand $\varphi(\xi)$ around $\xi_0$, and
integrate that instead. 
We also don't have to wory about the value of $f(\xi)$ at any other
places than at $\xi_0$.

In our case we have $\xi_0=0$. And for the taylor expansion we will
need the second derivative of $\varphi$ at $\xi=0$:
\begin{equation}
\varphi'(\xi)=\frac{\xi}{\sqrt{\xi^2+\Delta^2}}
\quad\Longrightarrow\quad
\varphi''(\xi)=\frac{1}{\sqrt{\xi^2+\Delta^2}}
+\xi\dv{\xi}\qty[\frac{1}{\sqrt{\xi^2+\Delta^2}}].
\end{equation}
And the Taylor expansion becomes
\begin{equation}
\varphi(\xi)\approx \varphi(0)+\bcancel{\varphi'(0)\xi}
+\varphi''(0)\frac{\xi^2}{2}
=\Delta+\frac{\xi^2}{2\Delta},
\end{equation}
assuming that $\Delta$ is real and positive.

The integral can now be written as
\begin{equation}
I(\Delta)\approx\int_{-\hbar\wD}^{\hbar\wD}\!\rd\xi\;f(0)
\exp(-\frac{\Delta}{T} - \frac{\xi^2}{2\Delta T})
=f(0)\ee^{-\nicefrac{\Delta}{T}}\int_{-\hbar\wD}^{\hbar\wD}\!\rd\xi\;
\exp(-\frac{\xi^2}{2\Delta T}).
\end{equation}
To evaluate the last step here, we note that the contribution from
$\abs{\xi}>\hbar\wD$ is negligable, wherefore we can approximate with a
Gassian integral:
\begin{equation}
I(\Delta)\approx \Delta^2\ee^{-\nicefrac{\Delta}{T}}
\int_{-\infty}^{\infty}\!\rd\xi\;
\exp(-\frac{\xi^2}{2\Delta T})
=\Delta^2\ee^{-\nicefrac{\Delta}{T}}
\sqrt{\pi2\Delta T}.
\end{equation}

\subsubsection*{Back to the heat capacity}
The heat capacity is now \todo{Factor 2 off?}
\begin{equation}\label{eq:1b_Cv}
C_V=\frac{2V}{T^2}g(\eF)I(\Delta)
=2\sqrt{2\pi}Vg(\eF)\frac{\Delta^{5/2}}{T^{3/2}}\ee^{-\nicefrac{\Delta}{T}}
=\frac{2Vm^{3/2}\sqrt{\eF}}{\sqrt{\pi}\hbar^3}
\frac{\Delta^{5/2}}{T^{3/2}}\ee^{-\nicefrac{\Delta}{T}}.
\end{equation}
\qed

\subsection{Discontinuity in the heat capacity}

Here we're going to find the discontinuity, at $T=\Tc$, in the heat
capacity. To to that we first need to find the heat capacity for
$T>\Tc$. 

\subsubsection{Heat capacity above the critical temperature}
For $T>\Tc$, we have $\Delta=0$, whereby
\begin{equation}
E(T>\Tc)=2\sum_k \xi_kn(\xi_k).
\end{equation}
But this is just the energy of an ideal Fermi gas (with spinn
$1/2$). 

From assignment~2 (see \eqref{eq:CV_Fermi} of
appendix~\ref{sec:ass_2_Cv}), we know that the heat capacity in the
low temperature limit is given by
\begin{equation}
C_V(\Tc<T\ll\eF)=\frac{\pi^2N}{2}\frac{T}{\eF}
-\frac{3\pi^2N}{20}\qty(\frac{T}{\eF})^3+\order{T^4}
\approx\frac{\pi^2N}{2}\frac{T}{\eF}.
\end{equation}
And in the high temperature limit we know that
\begin{equation}
C_V(T\gg\eF)=\frac{3N}{2},
\end{equation}
approaches the classical limt.

The value of $C_V$ when aproaching $\Tc$ from above subsequently
becomes
\begin{equation}
C_V(T\to\Tc^+)=\frac{\pi^2N}{2}\frac{\Tc}{\eF}
%-\frac{3\pi^2N}{20}\qty(\frac{\Tc}{\eF})^3
\end{equation}

\subsubsection{Heat capacity below the critical temperature}
We've already found $C_V$ for $T<\Tc$, but we still need to calculate
the value as $T$ approaches $\Tc$ from below. Using \eqref{eq:1b_Cv},
we have
\begin{equation}
C_V(T<\Tc)\propto
\frac{\Delta^{5/2}}{T^{3/2}}\ee^{-\nicefrac{\Delta}{T}}
\to 0 \quad
\text{as}\; T\to\Tc^-
\end{equation}
because $\Delta\to0$ as $T\to\Tc$.\todo{Is this really right???}

\subsubsection{Final result}
The discontinuity in the heat capacity is thus
\begin{equation}
C_V(T\to\Tc^+)-C_V(T\to\Tc^-)=\frac{\pi^2N}{2}\frac{\Tc}{\eF},
\end{equation}
arising from the fact that $C_V(T\to\Tc^-)=0$.

\subsection{Sketch of $C_V(T)$}












\section{The Helmholtz free energy of a superconductor}


















\newpage
\appendix
\setcounter{equation}{0}
\renewcommand{\theequation}{A\arabic{equation}}

\section{Exerpt from assigngment 2, problem 2 (ii)}
\label{sec:ass_2_Cv}
To calculate an approximation to the mean particle energy $E/N$, we
once again have to use the Sommerfeld expansion. Then we use the relation
\begin{equation}
\frac{E}{N}=-\frac{3\Omega}{2N}=\frac{3T}{2}\frac{f_{5/2}(z)}{f_{3/2}(z)}.
\end{equation}
Fortunately we've already calculated $\ln(z)=\zeta\approx\zeta_2$, so all
that left do here is to expand the Fermi functions and then
Taylor expand the quotient.

Let's begin:
\begin{equation}
%\begin{aligned}
\frac{f_{5/2}(z)}{f_{3/2}(z)}\approx
\overbrace{\frac{8\zeta_2^{5/2}}{15\sqrt{\pi}}
\qty(\frac{4\zeta_2^{3/2}}{3\sqrt{\pi}})^{-1}}^{\nicefrac{2\zeta_2}{5}}
\qty[1+\frac{5\pi^2}{8}\zeta_2^{-2}
-\frac{7\pi^4}{384}\zeta_2^{-4} ]
\times \qty[1+\frac{\pi^2}{8}\zeta_2^{-2} 
+\frac{7\pi^4}{640}\zeta_2^{-4} ]^{-1}
%\end{aligned}
\end{equation}
There's no need to even pretend to want deal with this by hand.
Just go directly to \emph{Mathematica}, which yields
\begin{equation}
\frac{f_{5/2}(z)}{f_{3/2}(z)}\approx \frac{2}{5\tau}
\qty[1+\frac{5\pi^2}{12}\tau^2-\frac{\pi^4}{16}\tau^4].
\end{equation}
For reference, the command in use were \texttt{Series}.

Now we have
\begin{equation}
\frac{E}{N}\approx\frac{3T}{2}\frac{2}{5\tau}
\qty[1+\frac{5\pi^2}{12}\tau^2-\frac{\pi^4}{16}\tau^4]
=\frac{3\epsilon_F}{5}\qty[1
+\frac{5\pi^2}{12}\qty(\frac{T}{\epsilon_F})^2
-\frac{\pi^4}{16}\qty(\frac{T}{\epsilon_F})^4].
\end{equation}
This can be used to get an expression for the heat capacity of the
system
\begin{equation}\label{eq:CV_Fermi}
\frac{C_V}{N}=\frac{1}{N}\qty(\pdv{E}{T})_V
=\;\frac{\pi^2}{2}\frac{T}{\epsilon_F}
 - \frac{3\pi^2}{20}\qty(\frac{T}{\epsilon_F})^3.
\end{equation}



\end{document}


%  LocalWords:  Pathria  idealities bosonic Bogoliubov Beale
