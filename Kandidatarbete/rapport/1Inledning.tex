\chapter{Inledning}


%\section{Bakgrund, syfte och begränsningar}

%\paragraph{Bakgrund}
Transport inuti celler är en av grundstenarna för cellers överlevnad. Ett exempel på en livsnödvändig intracellulär transport är hur ATP, molekylen som driver \emph{alla} energikrävande biologiska processer, kan ta sig från mitokondrien till resten av cellen. Detta är ett fall av passiv transport, där molekylerna eller partiklarna förflyttas genom att slumpvis diffundera. Det är därför viktigt att kunna beskriva transport inuti celler, för att förstå cellers inre processer.

I cellens inre finns trådliknande strukturer, uppbyggda av proteinfilament, som ger både stadga och möjliggör en aktiv transport inom cellen. Den aktiva transporten kompletterar den passiva genom att kunna transportera över länge sträckor och styra vart transporten går. Eftersom den aktiva transporten går längs med proteinfilament är det även viktigt att kunna förstå deras dynamik.


Som en första approximation skulle partiklars rörelse i cytoplasman kunna beskrivas med klassisk brownsk rörelse. Studier av diffusion i celler \cite{Hofling&Franosch2013,Dix_Crowdingeffects2008,Gou_etal2014,Parry_etal2014} har dock visat på avvikelser från denna teori. Partiklarna diffunderar långsammare än förväntat inuti cellen. Ytterligare skillnad i partikelrörelser har tidigare observerats beroende på om cellen befinner sig i dvala eller i sitt normala metabola tillstånd \cite{Parry_etal2014,Midtveldt_etal2016}. En alltäckande teori för vad som kan förklara dessa observationer finns i dagsläget inte; ämnet utgör därför ett aktuellt forskningsområde. Rörelsens stokastiska natur och cellens avancerade inre struktur är troligtvis en förklaring till att det har varit svårt att finna en bra modell för rörelsen. 

På samma sätt är även strängdynamiken outforskad. Förståelsen här är till och med mer ofullständig än för partiklar\cite{Koster_etal2005,Koster_etal2007,Koster_etal2008}. Här finns heller inte en lika självklar utgångspunkt, varifrån man kan börja modellera rörelserna. Istället för att utgå från en enkel modell får man här direkt försöka beskriva de observationer som görs.


Det finns i nuläget några modeller för både partikel- och strängrörelser. Bland annat ''fractional Brownian motion'' \cite{Mandelbrot_fBm1968}, fBm,
%(sv. )
och ''continuous time random ralk'' \cite{Hofling&Franosch2013}, CTRW, 
%(sv. tidskontinuerlig slumpvandring) 
 för partikelrörelse; 
samt ''worm-like chain'' \cite{Milstein2013}, WLC, 
%(sv. masklik kedja) 
 för strängrörelse. %Alla dessa modeller presenteras senare i den här rapporten.
%\paragraph{Syfte} 

I den här studien har partikelrörelser inuti jästceller och strängdynamiken för aktinfilament undersökts med olika statistiska metoder. Observationerna från datan jämförs med förutsägelser från bland annat modellerna ovan för att avgöra deras förmåga att kunna beskriva rörelsen.

Rapporten börjar med en översikt av den stokastiska teoribakgrund som används i de efterföljande kapitlen. 
Här presenteras huvuddelen av den stokastiska och statistiska analys som behövs för studierna av partiklar och strängar. Därefter följer två kapitel där undersökningarna som gjorts i den här studien presenteras.

Partikarna undersöks bland annat med hjälp av deras ''mean squared displacement'', MSD, (sv. medel av den kvadrerade förflyttningen) och asfärisitet, den senare används för att bedöma isotropin i deras rörelse.
Undersökningarna av MSD:n börjar med att bekräfta att partikelrörelsen är subdiffusiv\cite{Hofling&Franosch2013}. Alltså att partiklarna rör sig fundamentalt långsammare än i vanlig brownsk rörelse. Vidare erhålls också en tydlig skillnad mellan aktiva och passiva celler, vilket tyder på att cytoplasman förändras när celler går i dvala. 
I isotropiundersökningarna erhålls en viss skillnad mellan partiklar i olika cellfaser. Det verkar heller inte finnas några märkbara anisotropier inuti cellerna som påverkar partikelrörelsen. 

För strängarna studeras huvudsakligen deras tangentkorrelation och en uppdelning av rörelsen i egenmoder. Tangentkorrelationen är ett mått på hur mycket och snabbt strängen ändrar form. Där erhålls att tangentkorrelationen för strängar begränsade av mikrokanaler blir mer beständig än för de som fluktuerar fritt. För egenmoderna undersöks två sätt att dela upp strängarnas svängningar i oberoende moder. Detta kan ses som en analogi till svängningar hos exempelvis en gitarrsträng; där finns det oberoende svängningsmoder med olika frekvenser. Här erhålls ett avtagande samband mellan relaxationstid och vågtal.

Att avbilda små partiklar och filament innebär dock stora svårigheter. Detta leder till att den datan som studeras måste betraktas med försiktighet. För studien krävs alltså en viss mängd databehandling. Det mesta av databehandlingen har redan gjorts av de forskargrupper som tog fram datan~\cite{Midtveldt_etal2016,Koster_etal2005,Koster_etal2007,Koster_etal2008}. Detta utgörs främst av en brusuppskattning i partikeldatan och en mjuk anpassning till strängdatan.
%Kommenterade bort denna mening sålänge för at t 
%Men även i den här rapporten presenteras en liten del databehandling.

Eftersom alla undersökningar i denna studie är statistiska, behövs ett stort statistiskt underlag för att kunna dra några säkra slutsatser. Så på grund av de stora svårigheterna att avbilda partiklar och strängar finns tyvärr inte så mycket data som man hade kunnat önska. Den här studien  kommer därför inte att kunna dra några definitiva slutsatser om vilka modeller som verkar styra respektive fenomen.
Tyvärr leder den biologiska världens nyckfullhet till att exakta upprepningar inte är möjliga. Detta visar sig bland annat i att partiklarna som undersöks är olika stora och att strängarna är olika långa. Slutsatser måste därför dras med försiktighet.

Speciellt för strängar kan inte några direkta biologiska slutsatser dras. Detta är på grund av att de var tvungna att tas ut ur cellerna för att kunna avbildas~\cite{Koster_etal2005,Koster_etal2007,Koster_etal2008}. Strängarna undersöktes alltså när de låg i en utspädd lösning av aktin inuti mikrokanaler. På grund av cellers komplexa inre struktur kan det vara svårt att dra några direkta slutsatser om strängdynamiken inuti dem från dessa undersökningar. 

Slutligen återstår även en begränsning i vilka tidsskalor som undersökts.
Exempelvis kan skillnader uppstå på olika tidsskalor. Alltså att beteenden som observeras under mycket långa tider, upp till flera minuter, kanske är helt olika de som observerats under betydligt kortare tider, i storleksordningen sekunder. I den här studien studeras partikelrörelser som varar i $\unit[10]{s}$. För strängarna varierar datan mellan att vara i cirka $\unit[90{-}300]{s}$. Härav kommer slutsatserna som dras i den här rapporten bara att gälla i respektive tidsskala.

%Bara en liten kodsnutt som behövs när man kompilerar lokalt
%%% Local Variables: 
%%% mode: latex
%%% TeX-master: "00main.tex"
%%% End: 