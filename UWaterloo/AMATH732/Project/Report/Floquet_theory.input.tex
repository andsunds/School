

\section{Introduction to Floquet Theory}
In the previous sections we showed that, in the linear approximation, a
swing with periodically varying length has solutions whose amplitude
increases or decreases exponentially in time. Although we didn’t show
this, there are also solutions that show no steady change in amplitude
with time. Which type of solutions one gets depends on the ratio of
the frequency $\omega$ of the length oscillation to the free-running
frequency $\omega_0$ of the unperturbed swing, and also on the magnitude  of the
length oscillation. As a general rule, exponential growth occurs when
the forcing entrains the oscillation of the swing in such a way that
the motions that deliver energy occur at the most effective phase of
each cycle. This is more likely to happen if the ratio $\omega/\omega_0$ is an integer
(as we showed, 2 is particularly effective), and when $\epsilon$ is larger.  

Floquet Theory is a general approach to understanding systems of
linear ODEs with periodic parametric forcing. Specifically, it applies
to systems of the form 
\begin{equation}\label{eq:floquetsys}
\dv{\vb{x}}{t}=\vb{A}(t)\vb{x}(t)
\qcomma \vb{x}(t_0)=:\vb{x}_0,
\end{equation}
where the matrix $\vb{A}(t)$ is a $T$-periodic function of time: 
\begin{equation}\label{eq:2}
\vb{A}(t+nT)=\vb{A}(t)\qcomma
n\in\Z.
\end{equation}
Every solution of \eqref{eq:floquetsys} is encompassed in its principal
fundamental matrix solution $\vb{U}(t,t_0)$, defined as the solution
to the matrix IVP  
\begin{equation}
\dv{\vb{U}(t,t_0)}{t}=\vb{A}(t)\vb{U}(t,t_0)\qcomma
\vb{U}(t_0,t_0)=\I
\end{equation}
Here $\I$ is the $n\times n$ identity matrix. Specifically, the
solution of \eqref{eq:floquetsys} is 
$\vb{x}(t)=\vb{U}(t,t_0)\vb{x}_0$.

\bigskip
\noindent
Define the monodromy matrix $\vb{B}=\vb{U}(T, 0)$. 
%The monodromy matrix turns out to be independent of ${t}_{0}$, so it can be computed as $\vb{B}={U}(T,0)$. 
The eigenvalues of $\vb{B}$, $\rho_{1},\rho_{2},{\dots},\rho_{n}$, are called the characteristic
multipliers. The determinant of $\vb{B}$ (the product of the
characteristic multipliers) is given by, 
\begin{equation}
\det(\vb{B})=\exp \left(
\int_{0}^{T}{\tr[\vb{A}(s)]\id{s}}
\right)
\end{equation}
The multipliers determine the character (e.g. stable or exponentially
growing) of solutions of \eqref{eq:floquetsys}. In particular, 
\begin{equation}
\vb{U}(t, 0)=
\vb{U}\Big(t\;\;\text{mod}\, T,\;\;0\Big)
{\vb{B}}^{\lfloor t/T\rfloor}.
\end{equation}

The characteristic exponents  
$\mu_{1},\mu_{2},{\ldots},\mu_{n}$, defined by $\ee^{\mu_{i}T}=\rho_{i}$ determine the
growth rate of solutions. In particular, if $\Re(\mu_{i})<0$,
solutions associated with it will decay exponentially; if
$\Re(\mu_{i})>0$, solutions associated with it will grow
exponentially, and if $\Re(\mu_{i})=0$, solutions 
associated with it will be stable. (Strictly speaking, the previous
statements hold only if the geometric multiplicity of each eigenvalue
of $\vb{B}$ equals its algebraic multiplicity. When this is not the case,
polynomial factors can also arise.) There will be, associated with
each characteristic exponent $\mu$, a solution of the form 
\begin{equation}
\vb{x}(t)=\ee^{\mu t}\vb{p}(t)
\end{equation}
where $\vb{p}(t)$ is a $T$-periodic vector function of time. 

A particularly interesting case arises when $\vb{A}(t)$ is a periodic
perturbation of a constant matrix that itself has 
${T}_{0}$-periodic solutions, i.e.
\begin{equation}
\vb{A}(t)={\vb{A}}_{0}+\epsilon \widetilde{{\vb{A}}}(t)
\end{equation}
with $\widetilde{{\vb{A}}}(t)$ $T${}-periodic. For instance, the varying
length swing is of this form, with
\begin{equation}
\begin{aligned}
\vb{A}_0&=
\begin{pmatrix}0&1\\-{\eta }^{2}&0\end{pmatrix}
\\
\widetilde{{\vb{A}}}(\tau)&=
\begin{pmatrix}0&0\\
0&-2\frac{\dot{{r}}(\tau)}{r(\tau)}
\end{pmatrix}
=
\begin{pmatrix}0&0\\
0&-2\pd_{\tau}\log(r(\tau))
\end{pmatrix}.
\end{aligned}
\end{equation}
${\vb{A}}_{0}$ is traceless, and since $r$ is periodic, the integral of 
$\tr(\widetilde{{\vb{A}}})$ over one cycle is  
$-2\log({\tau}_{0}+2\pi )+2\log({\tau }_{0})=0$. Thus the
monodromy matrix $\vb{B}$ for the swing has determinant 0. The character
of the solutions can be determined from $\tr(\vb{B})$.
If $\abs{\rho_{1}+\rho_{2}}=|\tr(\vb{B})|<2$, the multipliers  
$\rho_{1},\rho_{2}$ form a complex conjugate pair, each with absolute
value 1, and solutions are stable. If, however,
$\abs{\rho_{1}+\rho_{2}}=|\tr(\vb{B})|>2$, the multipliers
are real and different, and since their product is 1, one must be
greater than one. In this case, there is an exponentially growing
solution. Finally, in the border case,
$|\rho_{1}+\rho_{2}|=\tr(\vb{B})=2$, the multipliers are equal, either
both $1$ or both $-1$. The values of $\eta,\,\epsilon $ for 
which that holds define the boundaries of the Floquet tongues, within
which exponentially growing solutions occur. 


\subsubsection*{Using Floquet therory on the swing}
With Floquet theory we now have a way to say, more generally, why
we got the exponential growth in amplitude when we studied the
linearlized EOM ($\sin\phi\to\phi$) of the swing. Note however, that
Floquet theory only applies to \emph{linear} systens of ODEs. So
it can not be directly applied on \eqref{eq:eom}. This is seen in
\figref{fig:sim}, where the linearized EOM agrees well with the
exponential growth, while the original EOM does not --- at least after
the amplitude has gotten so large that $\sin(\phi)\not\approx\phi$ any
more. 


%%% Local Variables: 
%%% mode: latex
%%% TeX-master: "Report_swing_circadian"
%%% End: 
