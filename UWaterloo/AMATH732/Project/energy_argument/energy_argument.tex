\documentclass[11pt,letter, swedish, english
]{article}
\pdfoutput=1

\usepackage{../../custom_as}

\renewcommand{\thesubsection}{\arabic{section} (\alph{subsection})}

\renewcommand{\thesubsubsection}{\arabic{section} (\alph{subsection},\,\roman{subsubsection})}


%%Drar in tabell och figurtexter
\usepackage[margin=10 pt]{caption}
%%För att lägga in 'att göra'-noteringar i texten
\usepackage{todonotes} %\todo{...}

%%För att själv bestämma marginalerna. 
\usepackage[
%            top    = 3cm,
%            bottom = 3cm,
%            left   = 3cm, right  = 3cm
]{geometry}

\DeclareMathAlphabet{\mathpzc}{OT1}{pzc}{m}{it}
\newcommand{\oh}{\ensuremath\mathpzc{o}}

\newcommand{\as}{\qcomma\text{as }}

%\renewcommand{\thefootnote}{\fnsymbol{footnote}}

\begin{document}

%%%%%%%%%%%%%%%%% vvv Inbyggd titelsida vvv %%%%%%%%%%%%%%%%%
% \begin{titlepage}
\title{\vspace{-2.5cm}Energy argument}
\author{Andréas Sundström}
\date{\today,\;\; \texttt{v\,1.4}}


\maketitle

%%%%%%%%%%%%%%%%% ^^^ Inbyggd titelsida ^^^ %%%%%%%%%%%%%%%%%
Thanks for reminding me about the energy the energy. I feel so
stupid for not thinking about that before -- we're trying to find an
increasing energy after all. 

The EOM:
\begin{equation}
\mathtt{eom312}=r(\tau)\phi''(\tau)+2r'(\tau)\phi'(\tau) + \eta^2\sin\phi(\tau)=0,
\end{equation}
where $"\,'=\dv*{\tau}"$.

The non-dimensionalized energy of the swing is
\begin{equation}
E=\frac{1}{2}\vb{v}^2 - \eta^2 r \cos\phi.
\end{equation}
The velocity, $\vb{v}$, in polar coordinates is
\begin{equation}
\vb{v} = x'\vu{x}+y'\vu{y} = r'\vu{r}+r\phi'\vu{\phi}.
\end{equation}
Therefore the energy is
\begin{equation}
E=\frac{1}{2}\qty[{r'}^2 + {r}^2{\phi'}^2] - \eta^2 r \cos\phi
\end{equation}
meaning that
\begin{equation}\label{eq:dE/dtau1}
\begin{aligned}
\dv{E}{\tau}=& \qty[r'r'' + rr'{\phi'}^2+r^2\phi'\phi'']
+ \eta^2r\sin(\phi)\phi' - \eta^2r'\cos(\phi)
\\
=&r'r''+r\phi'\qty(r\phi'' + r'\phi' + \eta^2\sin\phi) 
-\eta^2r'\cos\phi
\\
=&r'r''+r\phi'\qty(\mathtt{eom312}-r'\phi') -\eta^2r'\cos\phi
\\
=&r'r'' 
- r'\underbrace{\qty(r{\phi'}^2 + \eta^2\cos\phi)
}_{\ge0,\;\;\text{as long as } \phi\le\pi/2}.
\end{aligned}
\end{equation}
To get the pumped energy we integrate this over one
period\footnotemark{}. But
\begin{equation}
\int_0^T\!\rd\tau r' r''=\qty[\frac{1}{2}{r'}^2]_0^T\approx0,
\end{equation}
meaning that we are left with:
\begin{equation}\label{eq:int_dE1}
\Delta{E}=-\int_0^T\!\rd\tau\;
r'\qty(r{\phi'}^2 + \eta^2\cos\phi).
\end{equation}

\footnotetext{This is a little iffy, since we're not dealing with
  perfectly periodic behavior, but hopefully it's manageable. }

\newpage\todo[inline]{New stuff below here. }

\subsubsection*{Physical interpretation}
This has a really nice physical interpretation. When raising yourself
on the swing you have to do work against two forces:
the centrifugal and the gravitational force. Then since there's no
dissipating of energy, all the work done by the child must go into the
motion of the swing. 
The first term in \eqref{eq:int_dE1} corresponds to the centrifugal
force and the second term to the gravitational force.\footnotemark{} 
\footnotetext{
The third term, $r'r''$, corresponds to the power needed
to accelerate the mass along the line of the swing. (We are in a
``swing-fixed'' frame of reference, which means that $r'r''$ is the
only term for acceleration of the mass. Furthermore the rotating frame
of reference is needed to talk about a \emph{centrifugal} force.) Then
the same amount of energy is given back when the mass decelerates,
resulting in no net contribution to the swing energy. 
}


Without even looking at the analysis resulting in \eqref{eq:int_dE1},
one could derive that the best strategy to pump the swing is \emph{to move
inwards when the swing is as low as possible, and outwards when the
swing is as high as possible}. Because that would correspond to the
child doing the most amount of work against the two forces, and then
letting the forces do the least amount of work.


This physical interpretation is the reason why it's possible for the
swing to gain energy by only changing the length of the pendulum, and
not having any external forces. 

\subsubsection*{Small angle approximation}
If we limit ourselves to the small angle approximation, then
$\cos(\phi)\approx1$ and 
\begin{equation}
\int_0^T\!\rd\tau\;r'\cos\phi
\approx\int_0^T\!\rd\tau\;r'=0.
\end{equation}
The pumped energy now reduces to
\begin{equation}
\Delta{E}=-\int_0^T\!\rd\tau\;
r'r{\phi'}^2.
\end{equation}
With $r=(1+\epsilon\,\delta{r})$ and $r'=\epsilon\,\delta{r'}$, we get
\begin{equation}\label{eq:int_dE2}
\Delta{E}=-\int_0^T\!\rd\tau\;r'{\phi'}^2
+\order{\epsilon^2}.
\end{equation}



\end{document}



