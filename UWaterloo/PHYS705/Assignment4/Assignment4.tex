\documentclass[11pt,letter, swedish, english
]{article}
\pdfoutput=1

\usepackage{../custom_as}
\usepackage[makeroom
]{cancel}
\graphicspath{{figures/}}

\swapcommands{\Delta}{\varDelta}
\swapcommands{\Omega}{\varOmega}

%%Drar in tabell och figurtexter
\usepackage[margin=10 pt]{caption}
%%För att lägga in 'att göra'-noteringar i texten
\usepackage{todonotes} %\todo{...}

%%För att själv bestämma marginalerna. 
\usepackage[
%            top    = 2.5cm,
%            bottom = 3cm,
%            left   = 3cm, right  = 3cm
]{geometry}

%%För att ändra hur rubrikerna ska formateras
%\renewcommand{\thesubsection}{\arabic{section} (\roman{subsection})}
\renewcommand{\thesubsection}{\arabic{section} (\alph{subsection})}
\renewcommand{\thesubsubsection}{\arabic{section} (\alph{subsection},\,\roman{subsubsection})}

%\renewcommand{\thefootnote}{\fnsymbol{footnote}}

\newcommand{\Tc}{\ensuremath{T_{\text{c}}}}
\newcommand{\sign}{\ensuremath{\text{sign}}}

%\usepackage{tikz}

\begin{document}

%\tikzstyle{every picture}+=[remember picture]
%\tikzstyle{na} = [shape=rectangle,inner sep=0pt,text depth=0pt]



%%%%%%%%%%%%%%%%% vvv Inbyggd titelsida vvv %%%%%%%%%%%%%%%%%

\title{Statistical Physics 2 -- PHYS\,705 \\
Assignment 4}
\author{Andréas Sundström}
\date{\today}

\maketitle

%%%%%%%%%%%%%%%%% ^^^ Inbyggd titelsida ^^^ %%%%%%%%%%%%%%%%%

\section{Gaussian expectation value of fast fields}
In this problem we want to prove the equality
\begin{equation}\label{eq:1_want}
\ev{\phi_>(\vb*k_1)\phi_>(\vb*k_2)}
= G_0(\vb*k_1) (2\pi)^d \delta(\vb*k_1+\vb*k_2),
\end{equation}
where $G_0(\vb*k) = (k^2+r)^{-1}$. Our starting point is
\begin{equation}\label{eq:1_start}
\ev{\phi_>(\vb*k_1)\phi_>(\vb*k_2)}
:= \frac{1}{Z_{0>}}\! \int\!\mathcal{D}\phi_> 
\qty[\phi_>(\vb*k_1)\phi_>^*(-\vb*k_2)
\exp(\!{-}\frac{1}{2V}\sum_{\vb*q}G_0^{-1}(\vb*q)\,
%\phi_{\!>}\!(\vb*q) \phi_{\!>}^*\!(\vb*q)
\qty|\phi_{>}\!(\vb*q)|^2
)],
\end{equation}
where $Z_{0>} = \int\mathcal{D}\phi_>\,
\exp(\sum_{\vb*q}G_0^{-1}\!(\vb*q)\,
\qty|\phi_{\!>}\!(\vb*q)|^2/2V)$.


To do this, we will use a formula from class, based on Wick's
theorem. For $N$ real variables $x_n$ and a non-singular, symmetric
matrix $\mathsf{M}$, we write
\begin{equation}\label{eq:1_ev1}
\begin{aligned}
\ev{x_{i}x_{j}} =& 
\frac{\int\rd^N\!x\,\qty[x_{i}x_{j}
\exp(-\frac{1}{2}\vb*x^\mathsfrm{T}\mathsf{M}\vb*x)]}
{\int\rd^N\!x\, \exp(-\frac{1}{2}\vb*x^\mathsfrm{T}\mathsf{M}\vb*x)}\\
=&\frac{\eval{\pdv[2]{}{J_i}{J_j}
\qty[\int\rd^N\!x\, 
\exp(-\frac{1}{2}\vb*x^\mathsfrm{T}\mathsf{M}\vb*x
+\vb*J^\mathsfrm{T}\vb*x)]}_{J=0}}
{\int\rd^N\!x\, \exp(-\frac{1}{2}\vb*x^\mathsfrm{T}\mathsf{M}\vb*x)},
\end{aligned}
\end{equation}
where $\vb*J\in\R^N$ is just a help variable. Now rewrite, using the
Hubbard–Stratonovich transformation, the integral in the numerator as
\begin{equation}
\begin{aligned}
I[\mathsf{M}, \vb*J] =& \int\rd^N\!x\, 
\exp(-\frac{1}{2}\vb*x^\mathsfrm{T}\mathsf{M}\vb*x
+\vb*J^\mathsfrm{T}\vb*x) \\
=& I[\mathsf{M}, 0]\,\exp(\frac{1}{2}\vb*J^\mathsfrm{T}\mathsf{M}^{-1}\vb*J).
\end{aligned}
\end{equation}
This means that we can rewrite \eqref{eq:1_ev1} as
\begin{equation}\label{eq:1_ev2}
\begin{aligned}
\ev{x_{i}x_{j}} =&
\frac{\eval{\pdv[2]{}{J_i}{J_j}
\qty[I[\mathsf{M}, \vb*J]]}_{J=0}}
{I[\mathsf{M}, 0]}\\
=&\eval{\pdv[2]{}{J_i}{J_j}
\qty[\exp(\frac{1}{2}\vb*J^\mathsfrm{T}\mathsf{M}^{-1}\vb*J)]}_{J=0}.
\end{aligned}
\end{equation}

This is all good and well, but we have a small problem in that
$\phi_>$ is a \emph{complex} variable. We are however saved by the
fact that $\mathsf{M} = \mathsf{G}_0^{-1}/V$ is diagonal, which means
that we still can write
\begin{equation}
\sum_{\vb*q} \frac{G_0^{-1}(\vb*q)}{V}\qty|\phi_{\!>}\!(\vb*q)|^2
=\sum_{\vb*q} \phi_{>}^*\!(\vb*q)M(\vb*q)\phi_{>}\!(\vb*q) 
= \vb*\phi_{>}^\dagger\mathsf{M}\vb*\phi_{>}\in\R
\end{equation}
and still have $\mathsf{M}\in\R^{N\times N}$. I.e. since $\mathsf{M}$
is diagonal, we do still end up with a real value despite $\vb*\phi_>$
being complex. And now to keep the term $\vb*J^\dagger\vb*\phi_>$ real,
corresponding to $\vb*J^\mathsfrm{T}\vb*x$ in \eqref{eq:1_ev1}, 
we have to make sure that
\begin{equation}
J^*(\vb*k)\phi_>(\vb*k)\in\R, \quad\forall\vb*k;
\end{equation}
i.e. $J(\vb*k)$ and $\phi_>(\vb*k)$ have to have the same complex phase:
\begin{equation}\label{eq:1_J_restr}
\phi_>(\vb*k) = |\phi_>(\vb*k)|\ee^{\ii\alpha}
\quad\Longrightarrow\quad
J(\vb*k) = |J(\vb*k)|\ee^{\ii\alpha}.
\end{equation}
Another way of looking at these conditions for $\vb*J$, is that we have
changed variables to $x(\vb*k)=|\phi_>(\vb*k)|$ to keep $\vb*J$
real. And then reintroducing the complex phase in the derivatives.

We begin the final steps in this problem by noting a few important
corollaries of \eqref{eq:1_J_restr}. That is
\begin{enumerate}[label=(\roman*)]
\item Since $\phi_>(-\vb*k)=\phi_>^*(\vb*k)$, so must
$J(-\vb*k)=J^*(\vb*k)$.
\item $\displaystyle \phi_>(\vb*k_1) \ee^{\vb*J^\dagger\vb*\phi_>}
=\phi_>(\vb*k_1) \exp(\sum_{\vb*q}J^*(\vb*q)\phi_>(\vb*q))
=\pdv{J^*(\vb*k_1)}\qty[\ee^{\vb*J^\dagger\vb*\phi_>}]$.
\item $\displaystyle \phi_>^*(-\vb*k_2) \ee^{\vb*J^\dagger\vb*\phi_>}
=\phi_>^*(-\vb*k_2) \exp(\sum_{\vb*q}J(-\vb*q)\phi_>(\vb*q))
=\pdv{J(-\vb*k_2)}\qty[\ee^{\vb*J^\dagger\vb*\phi_>}]$.
\end{enumerate}
Now, using \eqref{eq:1_ev2}, with (ii) and (iii), on
\eqref{eq:1_start} yields 
\begin{equation}
\begin{aligned}
\ev{\phi_>(\vb*k_1)\phi_>(\vb*k_2)} 
=& \eval{\pdv[2]{}{J^*(\vb*k_1)}{J(-\vb*k_2)}
\qty[\exp(\frac{1}{2}\vb*J^\dagger\mathsf{M}^{-1}\vb*J)]}_{J=0}\\
=&\eval{\pdv[2]{}{J^*(\vb*k_1)}{J(-\vb*k_2)}
\qty[\exp(\frac{V}{2}\sum_{\vb*q} J^*(\vb*q)G_0(\vb*q)J(\vb*q))]}_{J=0}.
\end{aligned}
\end{equation}
Here $M(\vb*k)=G_0^{-1}(\vb*k)/V$. 

We now have two cases. The first one being $\vb*k_1\neq-\vb*k_2$, then
we obviously get
\begin{equation}
\begin{aligned}
\ev{\phi_>(\vb*k_1)\phi_>(\vb*k_2)} 
= \frac{V}{2}\eval{\Big[
2G_0(\vb*k_1)J(\vb*k_1) + 2J^*(-\vb*k_2)G_0(-\vb*k_2)
\Big]\exp(\ldots)}_{J=0} = 0.
\end{aligned}
\end{equation}
Each term has a seemingly additional factor 2 here, but that is due to
fact that for each derivative, we have 2 contributions: one from e.g.
$J(\vb*k_1)$ and one from $J^*(-\vb*k_2)$.
But in the second case, when $\vb*k_1=-\vb*k_2=:\vb*k$, we get
\begin{equation}
\begin{aligned}
\ev{\phi_>(\vb*k_1)\phi_>(\vb*k_2)} 
=&\eval{\pdv[2]{}{J^*(\vb*k)}{J(\vb*k)}
\qty[\exp(\frac{V}{2} \sum_{\vb*q}
J^*(\vb*q)G_0(\vb*q)J(\vb*q))]}_{J=0}\\
=&\eval{\pdv{J^*(\vb*k)}\qty[\frac{V}{2} 
2J^*(\vb*k)G_0(\vb*k)]\exp(\frac{V}{2} \sum_{\vb*q}
J^*(\vb*q)G_0(\vb*q)J(\vb*q))
}_{J=0}\\
=&\eval{\qty[VG_0(\vb*k) + \frac{V}{2} 
2J(\vb*k)G_0(\vb*k)]\exp(\ldots)
}_{J=0}\\
&=VG_0(\vb*k)
\end{aligned}
\end{equation}

We can write these two cases together as
\begin{equation}
\ev{\phi_>(\vb*k_1)\phi_>(\vb*k_2)} 
=G_0(\vb*k_1) \,V\delta_{\vb*k_1, -\vb*k_2},
\end{equation}
which is just the discrete version of \eqref{eq:1_want}. To get to
\eqref{eq:1_want}, we just have to use the thermo\-dynamic limit
$V\delta_{\vb*k_1, -\vb*k_2} \to (2\pi)^d\delta(\vb*k_1+\vb*k_2)$,
which arises from 
\begin{equation}
V\sum_{\vb*k} \to \int\frac{\rd^dk}{(2\pi)^d}.
\end{equation}







\section{Results from RG recursion relations}
\newcommand{\TT}{\tilde{T}}
\newcommand{\hT}{\tilde{h}}
\newcommand{\fs}{f_{\text{s}}}
We are given the differential form RG relations
\begin{equation}\label{eq:2_RG}
\begin{cases}\displaystyle
\dv{T}{l} = \frac{(n-2)}{2\pi}T^2 = CT^2
\\ \displaystyle
\dv{h}{l} = 2h,
\end{cases}
\end{equation}
with the regular definition of $l$ as $b=\ee^l$. From this we want to
fins the asymptotic behavior of some different physical quantities as
$T$ or $h$ goes towards $0$.  

The methods used, will be similar in the two first cases. We will first
need to find $\TT(l)$ and $\hT(l)$.\footnotemark{} To do that, we
recognize that \eqref{eq:2_RG} are two ODE's. The first RG relation
yields:
\begin{equation}\label{eq:2_TT_ODE}
\TT^{-2}\dv{\TT}{l} = C
\quad\Longrightarrow\quad
\frac{1}{\TT(0)}-\frac{1}{\TT(l)} = Cl
\end{equation}
which results in
\begin{equation}\label{eq:2_TT(l)}
\TT(l) = \frac{\TT(0)}{1-C\TT(0)l} = \frac{T}{1-CTl}.
\end{equation}
Here $\TT(0)$ have been replaced by $T$, since $\TT(0)$ is the value
of the temperature after \emph{no} RG ``steps'' hence just the
physical temperature $T$. For $h(l)$, we have a much simpler ODE:
\begin{equation}\label{eq:2_hT(l)}
\dv{\hT}{l} = 2\hT
\quad\Longrightarrow\quad
\hT(l) = \hT(0)\ee^{2l} = h\ee^{2l}.
\end{equation}
Same thing here with $\hT(0)=h$.

\footnotetext{I have renamed the RG functions of $l$, $\TT(l)$ and
  $\hT(l)$, to clearly separate them from the physical quantities
  $T=\TT(0)$ and $h=\hT(l)$. }

The method we will be using takes its inspiration from the discrete
RG method where if we had a variable
\begin{equation}
X(T,h) = b^{y_X}X(b^{y_T}T, b^{y_h}h),
\end{equation}
we do a certain number, $n$, of RG steps and then set $b^{ny_T}T=K_T$
as a constant, which then leads us to rewrite $b^n = (K_T/T)^{1/y_T}$ so
that we can write
\begin{equation}
X(T,h) \propto (T)^{-y_X/y_T}\mathcal{X}\qty(h/(T^{y_h/y_T})).
\end{equation}

What we want to do here is similar, but instead of $b^{ny_T}T=K_T$, we
set $\TT(l_0) = \TT_0$ constant. So we want to find the inverted
relation $l_0(\TT_0)$, which is easily done using \eqref{eq:2_TT_ODE}:
\begin{equation}\label{eq:2_l(TT)}
l_0(\TT_0) = \frac{1}{C}\qty[\frac{1}{T}-\frac{1}{\TT_0}].
\end{equation}
Remember that $\TT_0$ is just some constant that we fixed, and as such
its influence in the end result should not affect the physics,
i.e. just a multiplicative constant in the end. 

Back to the example variable $X$. With $b=\ee^l$, we can write
\begin{equation}
X(T, h) = \ee^{ly_X} X(\TT(l), \hT(l)).
\end{equation}
Now we fix $\TT(l_0)=\TT_0$ and use \eqref{eq:2_l(TT)} to rewire the
pre-factor and $\hT(l_0) = \ee^{2l_0}$, which results in
\begin{equation}
X(T, h) = \exp(\frac{y_X}{C}\qty[\frac{1}{T}-\frac{1}{\TT_0}])
\mathcal{X}\qty(h\times\exp{\frac{2}{C}\qty[\frac{1}{T}-\frac{1}{\TT_0}]})
\end{equation}
and we see indeed that the impact of $\TT_0$ is just in terms of
multiplicative constants $\ee^{-y_{...}/C\TT_0}$. So, let
$\mathcal{X}$ be redefined such that these constants becomes baked
into it. We end up with
\begin{equation}\label{eq:2_X}
X(T, h) \sim \exp(\frac{y_X}{CT})
\mathcal{X}\qty(h\ee^{\frac{2}{CT}}).
\end{equation}


\subsection{Correlation length}
The scaling form of the correlation length is
\begin{equation}
\xi(T, h) = b^1 \xi(\TT(l), \hT(l))= \ee^l \xi(\TT(l), \hT(l)),
\end{equation}
which means that, with \eqref{eq:2_X} in mind, we should have
\begin{equation}
\xi(T, h)  \sim \exp(\frac{1}{CT})
\varXi\qty(h\ee^{\frac{2}{CT}}).
\end{equation}

\subsection{Singular part of the free energy}
The singular part of the free energy scales like
\begin{equation}
\fs(T, h) = \ee^{-dl}\fs(\TT(l), \hT(l)),
\end{equation}
and thus
\begin{equation}
\fs(T, h) \sim \exp(-\frac{d}{CT})
\mathcal{F}_\text{s}\qty(h\ee^{\frac{2}{CT}}).
\end{equation}
Note that this time the exponent is negative in the exponential
outside, which means that $\fs$ goes to 0 really fast as $T\to0$. 

\subsection{Susceptibility}
The susceptibility can be found by differentiating the free energy:
\begin{equation}
\chi(T, h=0) = \eval{\pdv[2]{\fs}{h}}_{h\to0}
=\eval{\exp(-\frac{d}{CT})
\mathcal{F}_\text{s}''\qty(h\ee^{\frac{2}{CT}})\ee^{2\times\frac{2}{CT}}
}_{h\to0}
\propto \exp(\frac{4-d}{CT}).
\end{equation}
We can ignore the factor 
$\eval{\mathcal{F}_\text{s}''\qty(h\ee^{\frac{2}{CT}})}_{h\to0} 
= \mathcal{F}_\text{s}''(0)$, since that is just a constant. 



\section{Fixed points from RG recursion relations}
\renewcommand{\thesubsection}{\arabic{section} (\roman{subsection})}
The LGW functional for an $n$-component field with cubic anisotropy is
\begin{equation}
S[\ev*\varphi] = \int\rd^dx \qty[
\frac{1}{2}(\grad\vb*\varphi)^2
+\frac{r}{2}\vb*\varphi^2 +\frac{u}{4}(\vb*\varphi^2)^2
+v\sum_{i=1}^n\phi_i^4 ].
\end{equation}
The RG recursion relations for this system is
\begin{equation}\label{eq:3_start}
\begin{cases}\displaystyle
\dv{u}{l} = \epsilon u - 4C\qty[(n+8)u^2 + 6uv] 
\\ \displaystyle
\dv{v}{l} = \epsilon v - 4C\qty[12uv + 9v^2]
\end{cases}
\end{equation}
We want to find the fixed points and RG flow in the $(u, v)$ plane. 

To find the fixed point we want to find
$\dv*{u^*}{l}=\dv*{v}{l}=0$. There are 4 cases where this can be solved.
\begin{enumerate}[label=(\roman*)]
\item $u^*=v^*=0$
\item $u^*\neq0$, $v^*=0$: here we see that 
$\epsilon u = 4C(n+8)u^2 \quad\Longrightarrow\quad u^* = \epsilon/[4C(n+8)]$.
\item $u^*=0$, $v^*\neq0$: this yields 
$\epsilon v = 36Cv^2 \quad\Longrightarrow\quad v^* = \epsilon/(36C)$.
\item $u, v\neq0$: this one is a bit trickier, but not very. From the
second relation in \eqref{eq:3_start}, we get 
\[ v = \frac{\epsilon-48Cu}{36C}. \]
Substituting this into the first relation in \eqref{eq:3_start} and
solving\footnotemark{} for $u$, then substituting that back into the
expression for $v$, gives
\[ u^* = \frac{\epsilon}{12Cn}\qcomma v^* = \frac{(n-4)\epsilon}{36Cn} \]
\end{enumerate}

\footnotetext{This is really just a simple quadratic equation, but to
  not risk anything, I used \textit{Mathematica}. }

Now when we have the fixed points, we need to figure out the flow
pattern. To do that we need to calculate the scaling variables and
their scaling dimensions. To do that we will use the variables
$\beta_u=\dv*{u}{l}$ and $\beta_v=\dv*{v}{l}$ and the matrix
\begin{equation}
M=
\begin{pmatrix}
\pdv{\beta_u}{u} & \pdv{\beta_u}{v} \\
\pdv{\beta_v}{u} & \pdv{\beta_v}{v} 
\end{pmatrix}
=
\begin{pmatrix}
\epsilon-4C[2(n+8)u +6v] & -24Cu \\
 -48Cv & \epsilon-4C[12u +18v]
\end{pmatrix}.
\end{equation}
With this all we need to do is to substitute in the various fixed
points (obviously using \textit{Mathematica's} \texttt{Eigenvalue} and
\texttt{Eigenvector})
\begin{enumerate}[label=(\roman*)]
\item $u^*=v^*=0$: 
\[%%%%%%%%%%%%%%%%%%%%%%%%%%%%%%%%%%%%%%%%%%%%%%%%%%
M= 
\begin{pmatrix}
\epsilon & 0 \\
0 & \epsilon
\end{pmatrix}
\quad\Longrightarrow\quad
e_1=\begin{pmatrix}
1\\0
\end{pmatrix},\;
e_2=\begin{pmatrix}
0\\1
\end{pmatrix}\qcomma
y_1=y_2=\epsilon.
\]%%%%%%%%%%%%%%%%%%%%%%%%%%%%%%%%%%%%%%%%%%%%%%%%%%
\item $v^*=0$, $u^* = \epsilon/[4C(n+8)]$:
\[%%%%%%%%%%%%%%%%%%%%%%%%%%%%%%%%%%%%%%%%%%%%%%%%%%
M=
\begin{pmatrix}
-\epsilon & -\frac{6\epsilon}{n+8}\\
0 & \frac{\epsilon(n-4)}{n+8}
\end{pmatrix}
\quad\Longrightarrow\quad
e_1=\begin{pmatrix}
1\\0
\end{pmatrix},\;
e_2=\begin{pmatrix}
-\frac{3}{2+n}\\1
\end{pmatrix}\qcomma
y_1=-\epsilon,\;y_2=-\frac{\epsilon(n-4)}{n+8}.
\]%%%%%%%%%%%%%%%%%%%%%%%%%%%%%%%%%%%%%%%%%%%%%%%%%%
\item $u^*=0$, $v^* = \epsilon/(36C)$:
\[%%%%%%%%%%%%%%%%%%%%%%%%%%%%%%%%%%%%%%%%%%%%%%%%%%
M=
\begin{pmatrix}
\frac{\epsilon}{3}   & 0 \\
-\frac{4\epsilon}{3} & -\epsilon
\end{pmatrix}
\quad\Longrightarrow\quad
e_1=\begin{pmatrix}
-1\\1
\end{pmatrix},\;
e_2=\begin{pmatrix}
0\\1
\end{pmatrix}\qcomma
y_1=\frac{\epsilon}{3},\;y_2=-\epsilon.
\]%%%%%%%%%%%%%%%%%%%%%%%%%%%%%%%%%%%%%%%%%%%%%%%%%%
\item $u^* = \epsilon/(12Cn)$, $v^* = (n-4)\epsilon/(36Cn)$:
\[%%%%%%%%%%%%%%%%%%%%%%%%%%%%%%%%%%%%%%%%%%%%%%%%%%
\begin{aligned}
M=
\begin{pmatrix}
-\frac{\epsilon(n+8)}{3n} & -\frac{2\epsilon}{n} \\
-\frac{4\epsilon(n-4)}{3n} & \frac{\epsilon(n-4)}{n}
\end{pmatrix}
\quad\Longrightarrow\quad &
e_1=\begin{pmatrix}
-1/2\\1
\end{pmatrix},\;
e_2=\begin{pmatrix}
\frac{3}{n-4}\\1
\end{pmatrix}\qcomma \\
&y_1=-\frac{\epsilon(n-4)}{3n},\;y_2=-\epsilon.
\end{aligned}
\]%%%%%%%%%%%%%%%%%%%%%%%%%%%%%%%%%%%%%%%%%%%%%%%%%%
\end{enumerate}

To draw the RG fixed points and flow diagrams we just take a look at
items (i)--(iv) above and determine the relevant positions of the fixed
point, the direction of the scaling variables $e_i$, and the
corresponding eigenvalues $y_i$. If $y_i>0$, the RG flow is away from
the fixed point along the axis $e_i$, and if $y_i<0$ then the flow is
towards the fixed point. The RG flow diagram for $n<0$ is shown in
\figref{fig:3i}, while the one for $n>0$ is in \figref{fig:3ii}.


And lastly to the relevance of $v$ at the different stable fixed
points. 
\begin{enumerate}[label=(\roman*)]
\item This fixed point is unstable in all directions for any $n$.
\item For $n<4$ this fixed point is attractive in both its scaling
directions, meaning that a small perturbation in $v$ would also be
stable; i.e. $v$ is irrelevant. Whereas for $n>4$, we see that this
fixed point is unstable in its $e_2$ direction, thus rendering $v$
relevant, since a small perturbation in $v$ would land outside (ii)'s
basin of attraction; i.e. $v$ is relevant.
\item This fixed point has its $e_2$ direction along the $v$ axis, and
the corresponding eigenvalue is negative for any $n$. This means that
$v$ is irrelevant.
\item For $n<4$ this fixed point is unstable in its $e_1$ direction,
meaning that $v$ is relevant. But for $n>4$, (iv) is attractive in
both its scaling directions, meaning that $v$ is irrelevant. 
\end{enumerate}

The fact that whether $n$ is smaller or greater than $4$, affects the
relevance of $v$ for the fixed points (ii) and (iv) is
interesting. It is also interesting to note that for $n<4$, a small
perturbation in $v$ from (iv) will flow into (ii), while the opposite
is true for $n>4$. The last part here is interesting, since (ii) has
$v^*=0$, meaning that for a system where cubical anisotropy ($v\neq0$)
is possible, the system will condense to (iv) instead of (ii);
i.e. condensing to a state with a clear non-zero value of $v$. 




\begin{figure}
\centering
\input{figures/3_nl4.pdf_t}
\caption{RG flow diagram with $n<4$. The fixed points (i)--(iv) are
  marked and numbered, also their ``scaling directions'', $e_1$ and
  $e_2$, are marked out, as dotted lines through the relevant fixed
  point, in the cases where they do not line up with any of the $u$
  or $v$ axes. }
\label{fig:3i}
\end{figure}

\begin{figure}
\centering
\input{figures/3_ng4.pdf_t}
\caption{RG flow diagram with $n>4$. The fixed points (i)--(iv) are
  marked and numbered, also their ``scaling directions'', $e_1$ and
  $e_2$, are marked out, as dotted lines through the relevant fixed
  point, in the cases where they do not line up with any of the $u$ 
  or $v$ axes. } 
\label{fig:3ii}
\end{figure}







\end{document}




%  LocalWords:  MFT MF Ising Ornstein Zernike Stratonovich GLW RG
%  LocalWords:  rescale quartic rescaled anisotropy
