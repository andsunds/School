\documentclass[11pt,letter, swedish, english
]{article}
\pdfoutput=1

\usepackage{../custom_as}
\usepackage[makeroom
]{cancel}
\graphicspath{{figures/}}

\swapcommands{\Omega}{\varOmega}
\swapcommands{\Lambda}{\varLambda}

%%Drar in tabell och figurtexter
\usepackage[margin=10 pt]{caption}
%%För att lägga in 'att göra'-noteringar i texten
\usepackage{todonotes} %\todo{...}

%%För att själv bestämma marginalerna. 
\usepackage[
%            top    = 2.5cm,
%            bottom = 3cm,
%            left   = 3cm, right  = 3cm
]{geometry}

%%För att ändra hur rubrikerna ska formateras
%\renewcommand{\thesubsection}{\arabic{section} (\roman{subsection})}
\renewcommand{\thesubsection}{\arabic{section} (\alph{subsection})}
\renewcommand{\thesubsubsection}{\arabic{section} (\alph{subsection},\,\roman{subsubsection})}

%\renewcommand{\thefootnote}{\fnsymbol{footnote}}

\newcommand{\Tc}{\ensuremath{T_{\text{c}}}}
\newcommand{\sign}{\ensuremath{\text{sign}}}

%\usepackage{tikz}

\begin{document}

%\tikzstyle{every picture}+=[remember picture]
%\tikzstyle{na} = [shape=rectangle,inner sep=0pt,text depth=0pt]



%%%%%%%%%%%%%%%%% vvv Inbyggd titelsida vvv %%%%%%%%%%%%%%%%%

\title{Statistical Physics 2 -- PHYS\,705 \\
Assignment 5}
\author{Andréas Sundström}
\date{\today}

\maketitle

%%%%%%%%%%%%%%%%% ^^^ Inbyggd titelsida ^^^ %%%%%%%%%%%%%%%%%

\section{Wavefunction renormalization}
Here we want to find the renormalization, $z$, for $\sigma$ and
$\vb*\pi$ in the non-linear $\sigma$ model. It is defined by 
$\sigma'(\vb*k') = z^{-1}\sigma_<(\vb*k)$ and
$\vb*\pi'(\vb*k') = z^{-1}\vb*\pi_<(\vb*k)$. What we want to show is
that
\begin{equation}\label{eq:1_want}
\frac{z}{b^{d}} = 1 -\frac{n-1}{2}I_1,
\end{equation}
where
\begin{equation}
I_1:= \int_{\Lambda/b}^\Lambda \frac{\rd^dq}{(2\pi)^d}G_0(\vb*q)
= \int_{\Lambda/b}^\Lambda \frac{\rd^dq}{(2\pi)^d} \frac{T}{q^2}
= \frac{\mathcal{S}_d T}{(2\pi)^d} \Lambda^{d-2} \Delta{l},
\end{equation}
where $\mathcal{S}_d$ is the area of a unit sphere in $d$ dimensions
and $\ee^{\Delta{l}} = b$.


We begin this problem by rewriting
\begin{equation}\label{eq:1_ev-sigma}
\ev{\sigma(\vb*x)}_{0>} 
= \ev{\sqrt{1-\vb*\pi^2(\vb*x)}}_{0>} 
= \ev{1-\frac{1}{2}\vb*\pi^2(\vb*x) 
+\frac{1}{8}\qty(\vb*\pi^2(\vb*x))^2
+\ldots}_{0>}.
\end{equation}
We now split $\vb*\pi = \vb*\pi_< + \vb*\pi_>$ and get
\begin{equation}
\vb*\pi^2(\vb*x) = 
\vb*\pi_<^2(\vb*x) + \vb*\pi_>^2(\vb*x) 
+2\vb*\pi_<(\vb*x)\vdot\vb*\pi_>(\vb*x).
\end{equation}
We need to evaluate the expectation value of these terms.

It is easy to realize that 
\begin{equation}
\begin{aligned}
\ev{(\pi_<^a(\vb*x))^n(\pi_>^b(\vb*x))^m}
=&\frac{1}{Z_{0>}}
\int\!\mathcal{D}\vb*\pi_>\; (\pi_<^a(\vb*x))^n(\pi_>^b(\vb*x))^m
\ee^{S[\vb*\pi_>]}\\
=& \frac{(\pi_<^a(\vb*x))^n}{Z_{0>}}
\int\!\mathcal{D}\vb*\pi_>\; (\pi_>^b(\vb*x))^m \ee^{S[\vb*\pi_>]}
=(\pi_<^a(\vb*x))^n\ev{(\pi_>^b(\vb*x))^m},
\end{aligned}
\end{equation}
where $a$ and $b$ are arbitrary indeces for any element in the
vectors. We see especially, when $b=0$, that that any slow filed on
its own will just facor out of the expectation value. 

Next we can use Wick's theorem, to get
\begin{equation}\label{eq:1_Wick's}
\ev{\pi_>^a(\vb*k_1)\pi_>^b(\vb*k_2)}_{0>}
= \delta_{a, b} G_0(\vb*k_1) \,(2\pi)^d \delta(\vb*k_1 + \vb*k_2).
\end{equation}
We can use that to calculate
\begin{equation}
\begin{aligned}
\ev{(\vb*\pi_>(\vb*x))^2}_{0>} =& \frac{1}{Z_{0>}}
\int \mathcal{D}\vb*\pi_>\; 
[\vb*\pi_>(\vb*x)\vdot\vb*\pi_>(\vb*x)] \ee^{-S[\vb*\pi_>]}\\
=& \frac{1}{Z_{0>}} \int \mathcal{D}\vb*\pi_>
\qty[\int\frac{\rd^dk_1}{(2\pi)^d}\frac{\rd^dk_2}{(2\pi)^d}
\vb*\pi_>(\vb*k_1)\vdot\vb*\pi_>(\vb*k_2) \ee^{\ii\vb*x\vdot(\vb*k_1+\vb*k_2)}]
\ee^{-S[\vb*\pi_>]} \\
=& \int\frac{\rd^dk_1}{(2\pi)^d}\frac{\rd^dk_2}{(2\pi)^d}
\ee^{\ii\vb*x\vdot(\vb*k_1+\vb*k_2)}
\underbrace{\frac{1}{Z_{0>}} \int \mathcal{D}\vb*\pi_>\; 
\vb*\pi_>(\vb*k_1)\vdot\vb*\pi_>(\vb*k_2) 
\ee^{-S[\vb*\pi]}}_{=\sum_a\ev{\pi_>^a(\vb*k_1)\pi_>^a(\vb*k_2)}_{0>}}\\[-8pt]
\stackrel{\eqref{eq:1_Wick's}}{=}& 
\sum_a \int_{\Lambda/b}^\Lambda\frac{\rd^dk}{(2\pi)^d}G_0(\vb*k)
= (n-1) I_1.
\end{aligned}
\end{equation}
The factor $(n-1)$ comes from the fact that $\vb*\pi_>$ has $(n-1)$
components. From Wick's theorem, we also have that
\begin{equation}
\ev{\pi_>^a(\vb*x)}_{0>} = 0.
\end{equation}

We are now ready to go back and tackle \eqref{eq:1_ev-sigma}. We can
group the terms as
\begin{equation}\label{eq:1_ev-sigma2}
\begin{aligned}
\ev{\sigma(\vb*x)}_{0>} 
=& \qty[1 - \frac{1}{2}\vb*\pi_<^2(\vb*x) 
+ \frac{1}{8}(\vb*\pi_<^2(\vb*x))^2 +\ldots]\\
&- \frac{1}{2}\ev{\vb*\pi_>^2(\vb*x)} 
+\ev{\text{``higher order (cross \& regular) terms with }
\vb*\pi_>^2\text{''}}\\
\sim&\sqrt{1-\vb*\pi_<^2(\vb*x)} -\frac{(n-1)}{2} I_1.
\end{aligned}
\end{equation}

% We also note that the expectation value $\ev{\sigma(\vb*x)}_{0>}$
% shouldn't be affected by an RG step, so what we have is
% \begin{equation}
% \ev{\sigma(\vb*x)}_{0>} = \ev{\sigma'(\vb*x')}_{0>} 
% = \ev{\zeta^{-1}\sigma_<(\vb*x)}_{0>} 
% = \zeta^{-1}\sigma_<(\vb*x).
% \end{equation}
% Here we used the same principle behind how $\vb*\pi_<$ could be
% factored out from the expectation value. 

We also have to use another scaling factor, $\zeta$, for
$\sigma'(\vb*x')$ than for $\sigma'(\vb*k')$. The two are however
related through a FT. If $\zeta^{-1}\sigma_<(\vb*x) =
\sigma'(\vb*x')$, then the FT would be
\begin{equation}
\zeta^{-1}
\int\frac{\rd^dk}{(2\pi)^d}\sigma_<(\vb*k)\ee^{\ii\vb*k\vdot\vb*x}
=\int\frac{\rd^dk'}{(2\pi)^d}
\sigma'(\vb*k')\ee^{\ii\vb*k'\vdot\vb*x'}
=\int\frac{b^d\rd^dk}{(2\pi)^d}
\sigma'(\vb*k')\ee^{\ii b^d\vb*k\vdot\vb*x/b^d}.
\end{equation}
This would mean that
$ z^{-1}\sigma_<(\vb*k) = \sigma'(\vb*k') 
= \frac{\zeta^{-1}}{b^d}\sigma_<(\vb*k)$, 
or in other words
\begin{equation}
\zeta = \frac{z}{b^d}.
\end{equation}

The next thing we have to do is to calculate value
$\sigma_<(\vb*x=\vb*0)$. With the normalization criteria  
\begin{equation}
1 = (\vb*\pi'(\vb*x'))^2 +(\sigma'(\vb*x'))^2
= \zeta^{-2}\qty[ (\vb*\pi_<(\vb*x))^2 +(\sigma_<(\vb*x))^2].
\end{equation}
Now, with $\vb*\pi_<(\vb*x=\vb*0)$, we see that
\begin{equation}
\sigma_<(\vb*x=\vb*0) = \zeta.
\end{equation}

% Substituting this into \eqref{eq:1_ev-sigma2}, with
% $\vb*\pi_<(\vb*x=\vb*0)$, gives 
\paragraph{End of progress}
Unfortunately I can not for the life of me figure out a way to connect
$\ev{\sigma(\vb*x=\vb*0)}_{0>}$ to $\sigma_<(\vb*x=\vb*0)$, which we
would need to finish off this problem. Judging by the end result,
\eqref{eq:1_want}, what we want to find would be that
\begin{equation}
\ev{\sigma(\vb*x=\vb*0)}_{0>} = \zeta 
= \sigma_<(\vb*x=\vb*0),
\end{equation}
or
\begin{equation}
\ev{\sigma(\vb*x=\vb*0)}_{0>} = \zeta
\overbrace{\ev{\sigma'(\vb*x'=\vb*0)}_{0>}}^{=\sigma_<(\vb*x=\vb*0)},
\end{equation}
or maybe even
\begin{equation}
\ev{\sigma_>(\vb*x=\vb*0)}_{0>} =0
\quad\Longrightarrow\quad
\ev{\sigma(\vb*x=\vb*0)}_{0>} 
= \ev{\sigma_<(\vb*x=\vb*0) + \sigma_>(\vb*x=\vb*0) }_{0>}
=\sigma_<(\vb*x=\vb*0).
\end{equation}
But none of these approaches seems to yield anything fruitful for me. 

\textcolor{gray}{
I have to go away for a week now, so this is as far as it gets for my
part. Sorry about this, but thank you for the course!
}

\section{The Kosterlitz recursion relations}
\newcommand{\TKT}{T_{\text{KT}}}

We are given the Kosterlitz recursion relations in $d=2+\epsilon$
dimensions:
\begin{align}
\label{eq:2_dT/dl}
\beta_T := \dv{T}{l} =& \epsilon T +4\pi^3y^2\\
\label{eq:2_dy/dl}
\beta_y := \dv{y}{l} =& \qty(2 -\frac{\pi}{T})y.
\end{align}


\subsection{The finite-temperature fixed point}
The fixed point, at $T>0$, is given by setting
\begin{equation}
\epsilon T +4\pi^3y^2 = 0
\quad\Longrightarrow\quad
y = \pm\sqrt{\frac{\epsilon T}{4\pi^3}} \neq0,
\end{equation}
which gives
\begin{equation}
T^* = \TKT = \frac{\pi}{2}
\end{equation}
and 
\begin{equation}
y^* = \pm\frac{\sqrt{\epsilon}}{2^{5/2},\pi}.
\end{equation}


\subsection{Scaling dimensions}
To get the scaling dimensions we need the eigenvalues of 
\begin{equation}
M = \eval{
\begin{pmatrix}
\pdv{\beta_T}{T} & \pdv{\beta_T}{y}\\
\pdv{\beta_y}{T} & \pdv{\beta_y}{y}
\end{pmatrix}
}_{T^*, y^*}
=
\begin{pmatrix}
-\epsilon & 8\pi^3 y^*\\
\frac{\pi y^*}{\TKT^2} & 2-\frac{\pi}{\TKT}
\end{pmatrix}
=
\begin{pmatrix}
-\epsilon & \pm2^{3/2}\pi^2\sqrt{\epsilon}\\
\pm \frac{\sqrt{2\epsilon}}{\pi^2} & 0
\end{pmatrix}.
\end{equation}
To get the eigenvalues we use the characteristic equation
\begin{equation}
0=
\begin{vmatrix}
(-\epsilon -\lambda) & \pm2^{3/2}\pi^2\sqrt{\epsilon}\\
\pm \frac{\sqrt{2\epsilon}}{\pi^2} & 0-\lambda
\end{vmatrix}
= \lambda^2 + \epsilon\lambda - 4\epsilon,
\end{equation}
which has the solutions
\begin{equation}
\lambda_\pm = \frac{1}{2}\qty[ -\epsilon
\pm\sqrt{\epsilon^2 + 16\epsilon}]
= \frac{1}{2}\qty[ -\epsilon
\pm4\sqrt{\epsilon}\sqrt{1+\frac{\epsilon}{16}}].
\end{equation}
To lowest order in $\epsilon$, we therefore have
\begin{equation}
\lambda_+ =  2\sqrt{\epsilon}\qcomma
\lambda_- = -2\sqrt{\epsilon}.
\end{equation}
None of the eigenvectors align with the $T$ or $y$ axes.



\subsection{Estimate of $\nu$ and $\alpha$}
We begin by doing a change of variables to
\begin{equation}
x = 1-\frac{T^*}{T} = 1-\frac{\pi}{2T}\qcomma
\gamma = 1-\frac{y^*}{y} = 1-\frac{\sqrt{\epsilon}}{\sqrt{8}\pi\,y},
\end{equation}
which when inverted becomes
\begin{equation}
T=\frac{\pi}{2}\qty[1-x]^{-1} = \frac{\pi}{2}\qty[1+x] +\order{x^2}
\qcomma
y = \ldots = \frac{\sqrt{\epsilon}}{\sqrt{8}\pi}\qty[1+\gamma] 
+\order{\gamma^2}
\end{equation}
assuming that we are close to the fixed point, $x,\gamma\ll1$.
Substitute these into the RG equations and we get
\begin{equation}\label{eq:dx/dl}
\dv{x}{l} = \qty(\dv{T}{x})^{-1}\dv{T}{l} 
= -\epsilon(1+x) + \epsilon(1+\gamma)^2
= 2\epsilon\gamma -\epsilon x +\order{x^2, \gamma^2}
\end{equation}
and
\begin{equation}\label{eq:dgamma/dl}
\dv{\gamma}{l} = \qty(\dv{y}{\gamma})^{-1}\dv{y}{l}
=2x(1+\gamma) = 2x + \order{x\gamma}.
\end{equation}

Now, if we differentiate \eqref{eq:dgamma/dl}, we get
\begin{equation}
\dv[2]{\gamma}{l} = 2\dv{x}{l} = 4\epsilon\gamma 
- \overbrace{2\epsilon x}^{\dv*{\gamma}{l}}
\end{equation}
or in other words
\begin{equation}
\gamma'' +\epsilon\gamma' -4\epsilon\gamma = 0.
\end{equation}
This is a linear homogeneous second order ODE with constant
coefficients; its solutions must be
$ %\begin{equation}
\gamma(l) = A\ee^{r_-l} + B\ee^{r_+l},
$ %\end{equation}
where $r_\pm$ satisfies
\begin{equation}
r^2 +2\epsilon r -4\epsilon = 0
\quad\Longrightarrow\quad
r_\pm = -\frac{\epsilon}{2} 
\pm\sqrt{\frac{\epsilon^2}{4} + 4\epsilon}
=\pm2\sqrt{\epsilon} + \order{\epsilon},
\end{equation}
i.e.
\begin{equation}
\gamma(l) = A\ee^{-2\sqrt{\epsilon}l} + B\ee^{+2\sqrt{\epsilon}l}.
\end{equation}

Now we can solve \eqref{eq:dx/dl} for $x$:
\begin{equation}
x' +\epsilon x = 2\epsilon\gamma 
= 2A\ee^{-2\sqrt{\epsilon}l} + 2B\ee^{+2\sqrt{\epsilon}l}.
\end{equation}
Once again linear ODE with constant coefficients, the solution is going
to be of the form
\begin{equation}
x(l) = \tilde{A}\ee^{-2\sqrt{\epsilon}l} 
+ \tilde{B}\ee^{+2\sqrt{\epsilon}l} 
+ C\ee^{-\epsilon l}.
\end{equation}
We are only going to keep the lowest orders of $\epsilon$, so we
discard the last term. Next we want to find an expression for $l(x)$,
but we are also only eventually interested in the singular part of
$\xi$; that is going to come from the negative exponent, so we also
disregard the second term. In the end we get
\begin{equation}
l(x) \sim -\frac{1}{2\sqrt{\epsilon}} \ln(x).
\end{equation}


Now for the singular part of the correlation length. We know that
\begin{equation}
\xi(t) = b\hat\xi(t) = \ee^l\hat\xi(t)
\sim \exp(-\frac{1}{2\sqrt{\epsilon}} \ln(x))\hat\xi(t),
\end{equation}
where $\hat\xi(t)$ is a scaling function. 
We also have that $x=1-\pi/(2T) = (T-\Tc)/T \sim t$, so
\begin{equation}
\xi(t) 
\sim t^{-1/\nu}\hat\xi(t)
\end{equation}
where $\nu= 2\sqrt{\epsilon} = 2$ in 3D. 
Once we have $\nu$, we can easily get 
\begin{equation}
\alpha = 2-d\nu = 2 - (2+\epsilon)2\sqrt{\epsilon}
= 2 -4\sqrt{\epsilon} +\order{\epsilon^{3/2}} 
= -2+\order{\epsilon^{3/2}}.
\end{equation}

Comparing this to the calculations we did in class for the $d=2$
XY~model, we see that the over all idea is quite similar. Find $l(x)$
and then express $\xi$ in terms of $l(x)$. Although, the means of
getting $l(x)$ differed a bit; in the 2D case we had to use a trick to
get the hyperbolic relation between $x$ and $y$, while here in 3D we
just turned the crank with the system of ODE's that thankfully were
quite easy to solve. 

Interestingly we did not get the same essential singularity in $\xi$
for $d=3$ as we did for $d=2$. 


\end{document}




%  LocalWords:  MFT MF Ising Ornstein Zernike Stratonovich GLW RG
%  LocalWords:  rescale quartic rescaled anisotropy
