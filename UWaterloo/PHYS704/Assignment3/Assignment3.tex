\documentclass[11pt,letter, swedish, english
]{article}
\pdfoutput=1

\usepackage{../custom_as}
\usepackage[%makeroom
]{cancel}
\graphicspath{{figures/}}
\swapcommands{\Omega}{\varOmega}
%\swapcommands{\Lambda}{\varLambda}

%%Drar in tabell och figurtexter
\usepackage[margin=10 pt]{caption}
%%För att lägga in 'att göra'-noteringar i texten
\usepackage{todonotes} %\todo{...}

%%För att själv bestämma marginalerna. 
\usepackage[
%            top    = 3cm,
%            bottom = 3cm,
%            left   = 3cm, right  = 3cm
]{geometry}

%%För att ändra hur rubrikerna ska formateras
\renewcommand{\thesubsection}{\arabic{section} (\alph{subsection})}

\renewcommand{\thesubsubsection}{\arabic{section} (\alph{subsection},\,\roman{subsubsection})}

\renewcommand{\thefootnote}{\fnsymbol{footnote}}



%\usepackage{tikz}

\begin{document}

%\tikzstyle{every picture}+=[remember picture]
%\tikzstyle{na} = [shape=rectangle,inner sep=0pt,text depth=0pt]



%%%%%%%%%%%%%%%%% vvv Inbyggd titelsida vvv %%%%%%%%%%%%%%%%%

\title{Statistical Physics -- PHYS\,704 \\
Assignment 3}
\author{Andréas Sundström}
\date{\today}

\maketitle

%%%%%%%%%%%%%%%%% ^^^ Inbyggd titelsida ^^^ %%%%%%%%%%%%%%%%%

\newcommand{\Tc}{\ensuremath{T_{\text{c}}}}

\section{Cusp in $C_V$ for a Bose gas}
\renewcommand{\thesubsection}{\arabic{section} (\roman{subsection})}
Here we are going to find the discontinuity in the derivative of $C_V$
for a Bose gas. To do that we will need the energy:
\begin{equation}\label{eq:1_E}
\begin{cases}
E=\frac{3}{2}NT\frac{g_{5/2}(z)}{g_{3/2}(z)}, \quad& T>\Tc\\
E=\frac{3}{2}T^{5/2}\frac{V}{\Lambda^3}\zeta(\nicefrac{5}{2}), &T<\Tc,
\end{cases}
\end{equation}
where $\Lambda=\sqrt{T}\lambda=h/\sqrt{2\pi m}$ is a constant and
\begin{equation}
\Tc=\frac{h^2}{2\pi m}\qty[\frac{N}{V\zeta(\nicefrac{3}{2})}]^{2/3}
=\Lambda^2\qty[\frac{N}{V\zeta(\nicefrac{3}{2})}]^{2/3}.
\end{equation}
Then we're also going to need this relationship:
\begin{equation}\label{eq:1_dz}
\frac{1}{z}\qty(\pdv{z}{T})_V
=-\frac{3}{2T}\frac{g_{3/2}(z)}{g_{1/2}(z)}
\end{equation}
which can be found in Pathria\,\&\,Beale, \textit{Statistical
  Mechanics}, 3 ed., equation (7.1.36).

\subsection{Below critical temperature}
For $T<\Tc$ we have $E\propto T^{5/2}$, making
\begin{equation}
C_V=\frac{15}{4}T^{3/2}\frac{V}{\Lambda^3}\zeta(\nicefrac{5}{2}).
\end{equation}
Thus
\begin{equation}
\qty(\pdv{C_V}{T})_V
=\frac{45}{8}T^{1/2}\frac{V}{\Lambda^3}\zeta(\nicefrac{5}{2}),
\end{equation}
and
\begin{equation}
\eval{\qty(\pdv{C_V}{T})_{\!V}}_{T\to\Tc^-}
=\frac{45}{8}\Tc^{1/2}\frac{V}{\Lambda^3}\zeta(\nicefrac{5}{2})
%=\frac{45}{8}\frac{V}{\Lambda^2}
%\qty[\frac{N}{V\zeta(\nicefrac{3}{2})}]^{1/3}\zeta(\nicefrac{5}{2}).
=\frac{45}{8}\frac{V\zeta(\nicefrac{5}{2})}{\Tc\lambda_\text{c}^3},
\end{equation}
where
$\lambda_\text{c}=\lambda(T=\Tc)=[V\zeta(\nicefrac{3}{2})/N]^{1/3}$. In
other words
\begin{equation}\label{eq:1_Tc-}
\eval{\qty(\pdv{C_V}{T})_{\!V}}_{T\to\Tc^-}
=\frac{45N}{8\Tc}\frac{\zeta(\nicefrac{5}{2})}{\zeta(\nicefrac{3}{2})},
\end{equation}


\subsection{Above critical temperature}
Now we have to use the first equation of \eqref{eq:1_E}, giving
\begin{equation}
\begin{aligned}
C_V=&\frac{3}{2}N\qty{\frac{g_{5/2}(z)}{g_{3/2}(z)}
+T\qty(\pdv{T}\frac{g_{5/2}(z)}{g_{3/2}(z)})_V}\\
=&\frac{3}{2}N\qty{\frac{g_{5/2}(z)}{g_{3/2}(z)}
+T\qty(\frac{1}{g_{3/2}(z)}\qty(\pdv{g_{5/2}(z)}{z}\pdv{z}{T})_V
-\frac{g_{5/2}(z)}{[g_{3/2}(z)]^2}\qty(\pdv{g_{3/2}(z)}{z}\pdv{z}{T})_V )
}.
\end{aligned}
\end{equation}
Using \eqref{eq:1_dz} and the iteration formula
\begin{equation}
z\pdv{g_\nu(z)}{z}=g_{\nu-1}(z),
\end{equation}
we arrive at
\begin{equation}
\begin{aligned}
C_V=&\frac{3}{2}N\qty{\frac{g_{5/2}(z)}{g_{3/2}(z)}
-\frac{3}{2}\qty(\cancel{\frac{g_{3/2}(z)}{g_{3/2}(z)}}\frac{g_{3/2}(z)}{g_{1/2}(z)}
-\frac{g_{5/2}(z)}{[g_{3/2}(z)]^2} \bcancel{g_{1/2}(z)} 
\frac{g_{3/2}(z)}{\bcancel{g_{1/2}(z)}}
)}\\
=&\frac{3}{2}N\qty{\frac{5}{2}\frac{g_{5/2}(z)}{g_{3/2}(z)}
-\frac{3}{2}\frac{g_{3/2}(z)}{g_{1/2}(z)}
}.
\end{aligned}
\end{equation}
Which upon differentiation and using the same rules when
differentiating the Bose functions as before becomes
\begin{equation}
\begin{aligned}
\qty(\pdv{C_V}{T})_{\!V}
=&\frac{3}{2}N\qty(-\frac{3}{2T})\left\{
\frac{5}{2}\qty(\cancel{\frac{g_{3/2}(z)}{g_{3/2}(z)}}\frac{g_{3/2}(z)}{g_{1/2}(z)}
-\frac{g_{5/2}(z)}{[g_{3/2}(z)]^2} \bcancel{g_{1/2}(z)} 
\frac{g_{3/2}(z)}{\bcancel{g_{1/2}(z)}}
)\right.\\
&\hspace{65pt}\left.
-\frac{3}{2}\qty(\cancel{\frac{g_{1/2}(z)}{g_{1/2}(z)}}\frac{g_{3/2}(z)}{g_{1/2}(z)}
-\frac{g_{3/2}(z)}{[g_{1/2}(z)]^2} g_{-1/2}(z)
\frac{g_{3/2}(z)}{g_{1/2}(z)}
)\right\}\\
=&\frac{9N}{8T}\qty{
5\frac{g_{5/2}(z)}{g_{3/2}(z)}
-2\frac{g_{3/2}(z)}{g_{1/2}(z)}
-3\frac{[g_{3/2}(z)]^2 g_{-1/2}(z)}{[g_{1/2}(z)]^3}
}
\end{aligned}
\end{equation}

Next we need to use the formula\footnote{This is valid for all
  non-integer values of $\nu$. See appendix D of Pathria\,\&\,Beale,
  \textit{Statistical Mechanics}, 3 ed.} 
\begin{equation}
g_\nu\qty(\ee^{-\alpha})\sim
\Gamma(1-\nu)\,\alpha^{\nu-1}+\zeta(\nu)\qcomma \text{ as } \alpha\to0.
\end{equation}
We see here that for $\nu>1$ the dominant term is the constant
$\zeta(\nu)$, while for $\nu<1$ the dominant term is the singular
$\Gamma(1-\nu)\,\alpha^{\nu-1}$.

We also remind ourselves that $\alpha=-\mu/T\to0^+$ as $T\to\Tc^+$. We
therefore get
\begin{equation}\label{eq:1_Tc+}
\begin{aligned}
\eval{\qty(\pdv{C_V}{T})_{\!V}}_{T\to\Tc^+}
=&\frac{9N}{8\Tc}\Bigg\{
5\frac{\zeta(\nicefrac{5}{2})}{\zeta(\nicefrac{3}{2})}
%first term
-2\cancelto{0}{\frac{\zeta(\nicefrac{3}{2})}
{\Gamma(\nicefrac{1}{2})\alpha^{{-}1/2}}}
%second term
-3\frac{[\zeta(\nicefrac{3}{2})]^2\Gamma(\nicefrac{3}{2})}
{[\Gamma(\nicefrac{1}{2})]^3}
\overbrace{\frac{\alpha^{-3/2}}{[\alpha^{-1/2}]^3}}^{=1}
\Bigg\}\\
=&\frac{9N}{8\Tc}\Bigg\{
5\frac{\zeta(\nicefrac{5}{2})}{\zeta(\nicefrac{3}{2})}
-3\frac{[\zeta(\nicefrac{3}{2})]^2}{2\pi}
\Bigg\}
\end{aligned}
\end{equation}
The value of the last term was calculated in \textit{Mathematica}.

\subsection{The discontinuity of the derivative}
Now we just use \eqref{eq:1_Tc-} and \eqref{eq:1_Tc+} to get
\begin{equation}
\begin{aligned}
\eval{\qty(\pdv{C_V}{T})_{\!V}}_{T\to\Tc^-}
-\eval{\qty(\pdv{C_V}{T})_{\!V}}_{T\to\Tc^+}
=&\frac{45N}{8\Tc}\frac{\zeta(\nicefrac{5}{2})}{\zeta(\nicefrac{3}{2})}
-\frac{9N}{8\Tc}\Bigg\{
5\frac{\zeta(\nicefrac{5}{2})}{\zeta(\nicefrac{3}{2})}
-3\frac{[\zeta(\nicefrac{3}{2})]^2}{2\pi}
\Bigg\}\\
=&\frac{27N}{8\Tc}\frac{[\zeta(\nicefrac{3}{2})]^2}{2\pi}.
\end{aligned}
\end{equation}
\qed

\section{Bose condensation in one dimension}
\newcommand{\Ne}{\ensuremath{N_{\text{e}}}}
To show that Bose condensation is impossible for weakly interacting
bosons in one dimension, we can begin as we did for the ideal Bose
gas, by studying the number of (excited) particles
\begin{equation}
\Ne=\sum_k{'}\ev{a^\dagger_ka_k}.
\end{equation}

Using the Bogoliubov transformation from class we have
\begin{equation}
\begin{aligned}
a^\dagger_ka_k=\cosh^2(\theta_k) \xi^\dagger_k\xi_k 
+\sinh^2(\theta_k)\qty[1+\xi^\dagger_{-k}\xi_{-k} ]
-\frac{1}{2}\sinh(2\theta_k) 
\qty[\xi^\dagger_{k}\xi^\dagger_{-k}+\xi_{k}\xi_{-k}].
\end{aligned}
\end{equation}
We then use
\begin{equation}
\cosh^2(x)=\frac{1+\cosh(2x)}{2}\qcomma
\sinh^2(x)=\frac{-1+\cosh(2x)}{2}
\end{equation}
to get
\begin{equation}
\begin{aligned}
a^\dagger_ka_k=&\frac{1+\cosh(2\theta_k)}{2}\xi^\dagger_k\xi_k 
+\frac{-1+\cosh(2\theta_k)}{2}\qty[1+\xi^\dagger_{-k}\xi_{-k} ]\\
&-\frac{1}{2}\sinh(2\theta_k) 
\qty[\xi^\dagger_{k}\xi^\dagger_{-k}+\xi_{k}\xi_{-k}].
\end{aligned}
\end{equation}
Next up we note that since we're taking expectation values we get
\begin{equation}
\ev{\xi^\dagger_k\xi_k }=\ev{\xi^\dagger_{-k}\xi_{-k}},
\end{equation}
and
\begin{equation}
\ev{\xi^\dagger_{k}\xi^\dagger_{-k}}=\ev{\xi_{k}\xi_{-k}}=0.
\end{equation}
The last result is due to the fact that 
\begin{equation}
\ip{m}{n}=0\qcomma m\neq n
\end{equation}
for eigenstates to the number-operator and that $\xi^\dagger$ and $\xi$
changes that state. 
We now end up getting
\begin{equation}\label{eq:2_N_sum}
\begin{aligned}
\Ne=&\sum_k{'}\ev{a^\dagger_ka_k}\\
=&\frac{1}{2}
\sum_k{'}\bigg\{\Big[1+\cosh(2\theta_k) -1+\cosh(2\theta_k)\Big]
\ev{\xi^\dagger_k\xi_k }-1+\cosh(2\theta_k)\bigg\}\\
=&\frac{1}{2}\sum_k{'}\qty{
2\cosh(2\theta_k)\qty(\ev{\xi^\dagger_k\xi_k }+\frac{1}{2})-1
}.
\end{aligned}
\end{equation}

To continue from here we will use another result from class
\begin{equation}
\cosh(2\theta_k)=\frac{\epsilon_k+nu(k)}
{\sqrt{\epsilon_k\qty[\epsilon_k+2nu(k)]}}.
\end{equation}
Then we also know that the quasi-particles are to be treated as an
ideal Bose gas whereby
\begin{equation}
\ev{\xi^\dagger_k\xi_k
}=\ev{n_k^{\text{B.E}}}=\frac{1}{\ee^{\frac{E_k-\mu N}{T}}-1}.
\end{equation}
Now we plug this into \eqref{eq:2_N_sum} and let the sum become an
integral:
\begin{equation}
\begin{aligned}
\Ne\to&
\frac{1}{2}\int \frac{\dd{k}}{2\pi}\qty{
2\frac{\epsilon_k+nu(k)}{\sqrt{\epsilon_k\qty[\epsilon_k+2nu(k)]}}
\qty(\frac{1}{\ee^{\frac{E_k-\mu N}{T}}-1}+\frac{1}{2})-1
}\\
=&C\int_0^\infty\frac{\rd\epsilon}{\sqrt{\epsilon}}\qty{
2\frac{\epsilon+nu(k)}{\sqrt{\epsilon^2+\epsilon2nu(k)}}
\qty(\frac{1}{\ee^{\frac{E_k-\mu N}{T}}-1}+\frac{1}{2})-1
},
\end{aligned}
\end{equation}
where $C$ is just some unimportant constant. The integral has this
form since we are in just 1~dimension, thus having no more factors of
$k$ when going to ``spherical'' coordinates.

We can now see that for large $\epsilon$ the integrand becomes approximately
\begin{equation}
\frac{1}{\sqrt{\epsilon}}\qty{
2\frac{\epsilon}{\sqrt{\epsilon^2}}
\qty(0+\frac{1}{2})-1}
\propto\frac{1}{\sqrt{\epsilon}}
\end{equation}
which diverges upon integration to $\infty$, for any $T>0$. This suggests that
\emph{if} there is a critical temperature, then it should be
$\Tc=0$. Even if $\Tc>0$, we would still expect $\Ne$ to be either~0
or at least finite when $T\to0$.

To further analyze this integral we set
\begin{equation}
x=\frac{\epsilon}{T}
\quad\Longrightarrow\quad
\frac{\rd\epsilon}{\sqrt{\epsilon}}
=\sqrt{T}\frac{\rd x}{\sqrt{x}}.
\end{equation}
And we get
\begin{equation}\label{eq:2_Ne_int}
\Ne=C\int_0^\infty\rd{x}\,\sqrt{\frac{T}{x}}
\qty{
2\frac{x+\frac{nu(k)}{T}}{\sqrt{x^2+x\frac{2nu(k)}{T}}}
\qty(\frac{1}{\ee^{\frac{E(x)-\mu N}{T}}-1}+\frac{1}{2})-1
}.
\end{equation}
Now let $T\to0$, then the first part of the integrand must become
\begin{equation}\label{eq:2_lim1}
\sqrt{\frac{T}{x}}
\frac{x+\frac{nu(k)}{T}}{\sqrt{x^2+x\frac{2nu(k)}{T}}}
\to \sqrt{\frac{T}{x}}
\frac{\frac{nu(k)}{T}}{\sqrt{x\frac{2nu(k)}{T}}}
=\frac{\sqrt{nu(k)}}{\sqrt{2}\,x}.
\end{equation}
For the Bose-Einstein distribution we have
% \begin{equation}
% \frac{E_k-\mu N}{T}=\frac{\epsilon}{T}\sqrt{1+2\frac{nu(k)}{\epsilon}}
% -\frac{\mu N}{T}
% =x\sqrt{1+2\frac{nu(k)}{xT}} \to \infty \qcomma \text{ as } T\to0^+,
% \end{equation}
% meaning that
\begin{equation}
\frac{1}{\ee^{\frac{E(x)-\mu N}{T}}-1}\to
\begin{cases}
0\qcomma\text{or}\\
-1
\end{cases}
\end{equation}
depending on the sign of $E(x)-\mu N$. Either way this will make
\begin{equation}
\qty(\frac{1}{\ee^{\frac{E(x)-\mu N}{T}}-1}+\frac{1}{2})\to A,
\end{equation}
where $A=\pm1/2$ is some non-zero parameter. 

We now have
\begin{equation}\label{eq:2_Ne_div}
\Ne(T\to0)=C\int_0^\infty\rd{x}\,
\Bigg\{
2\frac{\sqrt{nu(k)}}{\sqrt{2}\,x}A
-\cancelto{0}{\sqrt{\frac{T}{x}}} \phantom{00}\Bigg\}
=\pm\infty
\end{equation}
This is divergent because $1/x$ is divergent both when the integration
limits are~0 or $\infty$, and 
\begin{equation}
u(k)=\int\rd{r}\, u(r)\ee^{-\ii kr}
\to\int\rd{r}\, u(r) = u_0 
\qcomma \text{as }\; k\to0 \;\Leftrightarrow\; x\to0.
\end{equation}
So since $u_0$ is finite, \eqref{eq:2_Ne_div} must at least diverge in
the limit $x\to 0$\footnotemark{}. This contradicts the notion that
there exists a $\Tc\ge0$ for a weakly interacting bosonic gas in one
dimension. 

\footnotetext{We didn't need to be as drastic as to go all the way to
  $T\to0$ to see this divergence. Because in the limit $x\to0$, we
  would still get the result \eqref{eq:2_lim1} for a finite
  $T$. But \eqref{eq:2_lim1} is more mathematically stringent the way
  presented here, because of the $x^{-1}$ present. }

\subsection*{No interactions}
If we were to turn off the interactions, $nu(k)\to0$. Furthermore at
and below the critical temperature $\mu=0$, meaning that $z=1$. Thus 
\eqref{eq:2_Ne_int} becomes
\begin{equation}
\begin{aligned}
\Ne(T=\Tc)\to& C\int_0^\infty\rd{x}\,\sqrt{\frac{\Tc}{x}}
\qty{2\cancel{\frac{x}{\sqrt{x^2}}}
\qty(\frac{1}{z^{-1}\ee^{\frac{\epsilon}{T}}-1}+\frac{1}{2})-1}\\
=&2C\sqrt{\Tc} \int_0^\infty\rd{x}
\frac{x^{-1/2}}{\ee^{x}-1}.
\end{aligned}
\end{equation}
This integral is divergent. However, this time it's possible to save
the day by setting $\Tc=0$. This is the only way to not get an
infinite $\Ne$.

\section{Some discontinuities}
\renewcommand{\thesubsection}{\arabic{section} (\alph{subsection})}
\newcommand{\ideal}[1]{#1^\text{{(ideal)}}}
\newcommand{\vc}{v_{\text{c}}}
\newcommand{\lc}{\lambda_{\text{c}}}

To first order in $a$ we have the canonical (\emph{not} the grand
canonical) partition function 
\begin{equation}
\ln(Z)=\ln(\ideal{Z})-\frac{1}{T}\frac{2\pi a\hbar^2}{m}\frac{N^2}{V}
\qty(2-\frac{n_0^2}{N^2}).
\end{equation}
For brevity we can let $A=2\pi a\hbar^2/m$, as a constant. For
temperature above the critical, we have $n_0\lll N$, so we get
\begin{equation}
\ln(Z)=\ln(\ideal{Z})-2A\frac{N^2}{TV}\qcomma
\text{for}\;T>\Tc.
\end{equation}
For below the critical temperature we have instead $n_0/N=1-v/\vc$
\begin{equation}
\ln(Z)=\ln(\ideal{Z})
-A\frac{N^2}{TV}\qty(1+2\frac{v}{\vc}-\frac{v^2}{\vc^2})\qcomma
\text{for}\;T<\Tc.
\end{equation}
Where
\begin{equation}\label{eq:3_v_vc}
v=\frac{V}{N}\qcomma\text{and}\quad
\vc=\frac{\lambda^3}{\zeta(\nicefrac{3}{2})}
=T^{-3/2}\frac{\Lambda^3}{\zeta(\nicefrac{3}{2})},
\end{equation}
where $\Lambda=h/\sqrt(2\pi m)$.

From the partition function we can get the Helmholtz free energy
\begin{equation}
F=-T\ln(Z)=
\begin{cases}
\ideal{F}+2A\frac{N^2}{V}\qcomma &\text{for}\;T>\Tc,\\
\ideal{F}
+AN\qty(\frac{1}{v}+\frac{2}{\vc}-\frac{v}{\vc^2})
\qcomma&\text{for}\;T<\Tc.
\end{cases}
\end{equation}


\subsection{Heat capacity and bulk modulus}

\subsubsection{Heat capacity}
To get the discontinuity in the heat capacity we use
\begin{equation}
C_V=T\qty(\pdv{S}{T})_v
\end{equation}
and 
\begin{equation}
S=-\qty(\pdv{F}{T})_v
\end{equation}

Let's begin with the case $T<\Tc$:
\begin{equation}
\begin{aligned}
S=&\ideal{S}-AN\qty[\pdv{T}\qty(\frac{1}{v}+\frac{2}{\vc}-\frac{v}{\vc^2})_v]\\
=&\ideal{S}-AN\qty[2\frac{3}{2}T^{1/2}-v3T^2\frac{\zeta(\nicefrac{3}{2})}{\Lambda^3}]
\frac{\zeta(\nicefrac{3}{2})}{\Lambda^3},
\end{aligned}
\end{equation}
where we used \eqref{eq:3_v_vc} to differentiate $\vc$.

Next is
\begin{equation}
\begin{aligned}
C_V=&T\qty(\pdv{S}{T})_v=\ideal{C_V}-TAN
\frac{\zeta(\nicefrac{3}{2})}{\Lambda^3}
\qty[\pdv{T}\qty(3T^{1/2}-v3T^2\frac{\zeta(\nicefrac{3}{2})}{\Lambda^3})_v]\\
=&\ideal{C_V}-TAN
\frac{\zeta(\nicefrac{3}{2})}{\Lambda^3}
\qty[\frac{3}{2}T^{-1/2}-v6T\frac{\zeta(\nicefrac{3}{2})}{\Lambda^3}].\\
% =&\ideal{C_V}-AN
% \frac{\zeta(\nicefrac{3}{2})}{\Lambda^3}
% \qty[\frac{3}{2}T^{1/2}-v6T^2\frac{\zeta(\nicefrac{3}{2})}{\Lambda^3}]\\
\end{aligned}
\end{equation}
To get the value as $T\to\Tc$, we use
\begin{equation}
\Tc=\Lambda^2\qty[\frac{1}{v\zeta(\nicefrac{3}{2})}]^{2/3}
\end{equation}
and get
\begin{equation}\label{eq:3a_Cv_Tc-}
\begin{aligned}
C_V=&\ideal{C_V}(\Tc)-AN
\frac{\zeta(\nicefrac{3}{2})}{\Lambda^3}
\qty[\frac{3}{2}\Tc^{1/2}-v6\Tc^2\frac{\zeta(\nicefrac{3}{2})}{\Lambda^3}]\\
=&\ideal{C_V}(\Tc)-AN
\frac{\zeta(\nicefrac{3}{2})}{\Lambda^3}
\qty[\frac{3}{2}\Lambda\qty(\frac{1}{v\zeta(\nicefrac{3}{2})})^{1/3}
-v6\Lambda^4\qty(\frac{1}{v\zeta(\nicefrac{3}{2})})^{4/3}
\frac{\zeta(\nicefrac{3}{2})}{\Lambda^3}]\\
=&\ideal{C_V}(\Tc)-AN
\frac{\zeta(\nicefrac{3}{2})}{\Lambda^3}
\qty[\frac{3}{2}-6]
\Lambda\qty(\frac{1}{v\zeta(\nicefrac{3}{2})})^{1/3}.\\
\end{aligned}
\end{equation}
Now we use $\lc=[v\zeta(\nicefrac{3}{2})]^{1/3}$ to get
\begin{equation}
\begin{aligned}
C_V(T\to\Tc-)=&\ideal{C_V}(\Tc)+\frac{9}{2}AN
\frac{\zeta(\nicefrac{3}{2})}{\Lambda^2}\frac{1}{\lc}.\\
\end{aligned}
\end{equation}

For $T>\Tc$ all we have is
\begin{equation}
S=\ideal{S}-\qty(\pdv{T} 2A\frac{N^2}{V})_v
=\ideal{S},
\end{equation}
wherefore 
\begin{equation}
C_V=\ideal{C_V}
\end{equation}
for $T>\Tc$.

Next we already know that in the ideal case $C_V$ is continuous at
$\Tc$, which means that
\begin{equation}
\begin{aligned}
C_V(T\to\Tc-)-C_V(T\to\Tc+)
=&\frac{9}{2}AN\frac{\zeta(\nicefrac{3}{2})}{\Lambda^2}\frac{1}{\lc}\\
=&N\frac{9}{2\lc}\frac{2\pi a \hbar^2}{m}
\frac{\zeta(\nicefrac{3}{2})}{h^2/(2\pi m)}
=N\frac{9a}{2\lc}\zeta(\nicefrac{3}{2}).
\end{aligned}
\end{equation}
\qed

\subsubsection{Bulk modulus}
For the bulk modulus
\begin{equation}
K=-V\pdv{P}{V}=-v\pdv{P}{v},
\end{equation}
we will use
\begin{equation}
P=-\qty(\pdv{F}{V})_T
=-\frac{1}{N}\qty(\pdv{F}{v})_T.
\end{equation}

So for $T<\Tc$ we get
\begin{equation}\label{eq:3_P<Tc}
P(T<\Tc)=\ideal{P}-A\pdv{v}\qty(\frac{1}{v}+\frac{2}{\vc}-\frac{v}{\vc^2})_T
=\ideal{P}+A\qty(\frac{1}{v^2}+\frac{1}{\vc^2}),
\end{equation}
which gives
\begin{equation}
\begin{aligned}
K=&\ideal{K}-vA\pdv{v}\qty(\frac{1}{v^2}+\frac{1}{\vc^2})\\
=&\ideal{K}-vA\qty(\frac{-2}{v^3})\\
=&\ideal{K}+\frac{2\pi a\hbar^2}{m}\frac{2}{v^2}
\to \ideal{K}+\frac{4\pi a\hbar^2}{m\vc^2}
\qcomma\text{as}\; T\to\Tc^-.
\end{aligned}
\end{equation}

But in the case of $T>\Tc$ we get
\begin{equation}\label{eq:3_P>Tc}
P(T>\Tc)=\ideal{P}-\pdv{v}\qty(A\frac{2}{v})_T
=\ideal{P}+A\frac{2}{v^2}.
\end{equation}
From here it's clear that we are going to get double the effect from
non-idealities for $T>\Tc$. This is because we have a factor~2 on the non-ideal
term this time, and the non-ideal term has the same $v$-dependence as
for $T<\Tc$.

We therefore have
\begin{equation}
\begin{aligned}
K(T\to\Tc-)-K(T\to\Tc+)
=&\cancel{\ideal{K}(\Tc)}+\frac{4\pi a\hbar^2}{m\vc^2}
-\qty(\cancel{\ideal{K}(\Tc)}+2\frac{4\pi a\hbar^2}{m\vc^2})\\
=&-\frac{4\pi a\hbar^2}{m\vc^2}.
\end{aligned}
\end{equation}
\qed

\subsection{More discontinuities}
Here we're going to examine the discontinuities in the quantities
\begin{equation}
\qty(\pdv[2]{P}{T})_v \qcomma\text{and}\quad
\qty(\pdv[2]{\mu}{T})_v.
\end{equation}

\subsubsection{}
Since we already have the expressions for $P$ in \eqref{eq:3_P<Tc} and
\eqref{eq:3_P>Tc}, we can just start differentiating.

For $T<\Tc$, we remember that
$\vc=T^{-3/2}\,\frac{\Lambda^3}{\zeta(\nicefrac{3}{2})}$, and we use
\eqref{eq:3_P<Tc} to get
\begin{equation}
\begin{aligned}
\qty(\pdv{P}{T})_v=&\qty(\pdv{\ideal{P}}{T})_v
+A\pdv{T}\qty(\frac{1}{v^2}+\frac{1}{\vc^2})_v\\
=&\qty(\pdv{\ideal{P}}{T})_v
+A\pdv{T}\qty[\frac{1}{v^2}+T^3\qty(\frac{\Lambda^3}{\zeta(\nicefrac{3}{2})})^2]_v\\
=&\qty(\pdv{\ideal{P}}{T})_v
+3AT^2\qty(\frac{\Lambda^3}{\zeta(\nicefrac{3}{2})})^2.
\end{aligned}
\end{equation}
We then end up getting
\begin{equation}
\qty(\pdv[2]{P}{T})_v=\qty(\pdv[2]{\ideal{P}}{T})_v
+6AT\qty(\frac{\Lambda^3}{\zeta(\nicefrac{3}{2})})^2
\qcomma T<\Tc.
\end{equation}

For $T>\Tc$ we can see from \eqref{eq:3_P>Tc} that there is no
$\vc$-dependence in the non-ideal term. Therefore we get
\begin{equation}
\qty(\pdv[2]{P}{T})_v=\qty(\pdv[2]{\ideal{P}}{T})_v
\qcomma T>\Tc.
\end{equation}

The discontinuity is thus given by
\begin{equation}\label{eq:3_discont._P}
\eval{\qty(\pdv[2]{P}{T})_v}_{\Tc^-}
-\eval{\qty(\pdv[2]{P}{T})_v}_{\Tc^+}
=6A\Tc\qty(\frac{\Lambda^3}{\zeta(\nicefrac{3}{2})})^2.
\end{equation}
%\todo{¿Need to show cont. in ideal case?}
Since in the ideal case this quantity is continuous\footnotemark{}.

\footnotetext{This can be seen through the fact that in the ideal case
there is no contribution to $P$ from condensed particles. So pressure
change when $T$ passes $\Tc$ is smooth, since the number of condensed
particles also changes smoothly. Therefore this quantity is continuous
in the ideal case. }

\subsubsection{}
To differentiate $\mu$ we need to find it first:
\begin{equation}
\mu=\frac{F}{N}+Pv.
\end{equation}

In the case of $T<\Tc$ we have
\begin{equation}
\begin{aligned}
\mu(T<\Tc)=&\frac{\ideal{F}}{N}
+A\qty(\frac{1}{v}+\frac{2}{\vc}-\frac{v}{\vc^2})
+\ideal{P}v+A\qty(\frac{1}{v}+\frac{v}{\vc^2})\\
=&\ideal{\mu}
+2A\qty(\frac{1}{v}+\frac{1}{\vc}).
\end{aligned}
\end{equation}
Again, we use
$\vc=T^{-3/2}\,\frac{\Lambda^3}{\zeta(\nicefrac{3}{2})}$
to get
\begin{equation}
\begin{aligned}
\qty(\pdv{\mu}{T})_v=&\qty(\pdv{\ideal{\mu}}{T})_v
+2A\pdv{T}\qty(\frac{1}{v}+
T^{3/2}\,\frac{\zeta(\nicefrac{3}{2})}{\Lambda^3})_v\\
=&\qty(\pdv{\ideal{\mu}}{T})_v
+2A\frac{3}{2}
T^{1/2}\,\frac{\zeta(\nicefrac{3}{2})}{\Lambda^3},
\end{aligned}
\end{equation}
and
\begin{equation}
\qty(\pdv[2]{\mu}{T})_v
=\qty(\pdv[2]{\ideal{\mu}}{T})_v
+A\frac{3}{2}
T^{-1/2}\,\frac{\zeta(\nicefrac{3}{2})}{\Lambda^3}
\qcomma T<\Tc.
\end{equation}

For $T>\Tc$ we instead get
\begin{equation}
\mu(T>\Tc)=\frac{\ideal{F}}{N}+2A\frac{1}{v}
+\ideal{P}v+A\frac{2}{v}
=\ideal{\mu}+A\frac{4}{v}.
\end{equation}
And again, there's no $T$-dependence in the non-ideal term, whereby 
\begin{equation}
\qty(\pdv[2]{\mu}{T})_v
=\qty(\pdv[2]{\ideal{\mu}}{T})_v
\qcomma T>\Tc.
\end{equation}

The discontinuity is now given by
\begin{equation}\label{eq:3_discont._mu}
\eval{\qty(\pdv[2]{\mu}{T})_v}_{\Tc^-}
-\eval{\qty(\pdv[2]{\mu}{T})_v}_{\Tc^+}
=A\frac{3}{2}
\Tc^{-1/2}\,\frac{\zeta(\nicefrac{3}{2})}{\Lambda^3}
\end{equation}
Since, with the same argument as before, this quantity is continuous.
%\todo{¿Need to show cont. in ideal case?}

\subsubsection{Comparison to the discontinuity in the heat capacity}
We have the thermodynamic identity
\begin{equation}\label{eq:3b_Cv_identity}
C_V=VT\qty(\pdv[2]{P}{T})_v-NT\qty(\pdv[2]{\mu}{T})_v.
\end{equation}
With \eqref{eq:3_discont._P} and \eqref{eq:3_discont._mu}, the
discontinuity on the RHS is
\begin{equation}
\begin{aligned}
V\Tc\qty[6A\Tc\qty(\frac{\Lambda^3}{\zeta(\nicefrac{3}{2})})^2]
-N\Tc\qty[A\frac{3}{2}
\Tc^{-1/2}\,\frac{\zeta(\nicefrac{3}{2})}{\Lambda^3}]
=&A\frac{\zeta(\nicefrac{3}{2})}{\Lambda^3}
\qty[V6\Tc^2\frac{\zeta(\nicefrac{3}{2})}{\Lambda^3}
-N\frac{3}{2}\Tc^{1/2}]\\
=&AN\frac{\zeta(\nicefrac{3}{2})}{\Lambda^3}
\qty[v6\Tc^2\frac{\zeta(\nicefrac{3}{2})}{\Lambda^3}
-\frac{3}{2}\Tc^{1/2}].
\end{aligned}
\end{equation}
This is exactly the same as the non-ideal term in
(\ref{eq:3a_Cv_Tc-},\,a), and as we remember: only the non-ideal term
contributed to the discontinuity. Therefore the identity
\eqref{eq:3b_Cv_identity} holds, even in the discontinuity.
\qed



\section{Field operators and the Schrödinger equation}
\newcommand{\scomm}[2]{\left\langle#1,\;#2\right\rangle}
\newcommand{\Scomm}[2]{\langle#1,\;#2\rangle}
\newcommand{\commute}[1]{\overbrace{#1}^{\text{commute}}}
For a system of interacting particles we have
\begin{equation}\label{eq:4_want_this}
\begin{aligned}
H=&-\frac{\hbar^2}{2m}\int\rd^3r\,
\psi^\dagger(\vb{r})\laplacian\psi(\vb{r})\\
&+\frac{1}{2}\iint\rd^3\!r\,\rd^3\!r'\,
\psi^\dagger(\vb{r})\psi^\dagger(\vb{r}')
u(\vb{r}-\vb{r}')\psi(\vb{r}')\psi(\vb{r}),
\end{aligned}
\end{equation}
where the field operators satisfy the commutation relations
\begin{equation}\label{eq:4_comm}
\begin{aligned}
\scomm{\psi(\vb{r})}{\psi^\dagger(\vb{r}')}
%:=\psi(\vb{r})\psi^\dagger(\vb{r}')-\Theta\psi^\dagger(\vb{r}')\psi^\dagger(\vb{r})
&=\delta(\vb{r}-\vb{r}')\\
\scomm{\psi(\vb{r})}{\psi(\vb{r}')\Big.}=
\scomm{(\psi^\dagger(\vb{r})}{\psi^\dagger(\vb{r}')}&=0
\end{aligned}
\end{equation}
where $\scomm{A}{B}=AB-\Theta BA$ corresponds to a (regular)
commutator for bosons and an anti-commutator for fermions; that is
\begin{equation}
\Theta=
\begin{cases}
+1\qcomma&\text{for bosons,}\\
-1\qcomma&\text{for fermions.}
\end{cases}
\end{equation}
The second
relation means that we can commute creation and annihilation operator
with each other like so:
\begin{equation}
\psi(\vb{r})\psi(\vb{r}')=\Theta\psi(\vb{r}')\psi(\vb{r}),
\end{equation}
and analogous for $\psi^\dagger$. In other words we pick up a factor
of $\Theta$ when commuting annihilation (creation) operators with
each other.


\subsection{Commutation with the Hamiltonian}
Here we want to show that 
\begin{equation}
\comm{\psi(\vb{r}_j)}{H}
=\qty(-\frac{\hbar^2}{2m}\laplacian_j
+\int\rd^3\!r\,
\psi^\dagger(\vb{r})
u(\vb{r}-\vb{r}_j)\psi(\vb{r})
)\psi(\vb{r}_j),
\end{equation}
for bosons as well as fermions.
This we will do in steps.

\subsubsection{The first term}
We begin with the first integral of $H$. First of all we have
\begin{equation}
\comm{\psi(\vb{r}_j)}{\int\rd^3r\,
\psi^\dagger(\vb{r})\laplacian\psi(\vb{r})}=
\int\rd^3r\,
\comm{\psi(\vb{r}_j)}
{\psi^\dagger(\vb{r})\laplacian\psi(\vb{r})}
\end{equation}
since $\vb{r}_j$ can be regarded as a constant with respect to the
integration. Then we continue and commute $\psi(\vb{r})$ and
$\psi(\vb{r}_j)$ in the second term of the expanded commutator:
\begin{equation}
\begin{aligned}
\comm{\psi(\vb{r}_j)}
{\psi^\dagger(\vb{r})\laplacian\psi(\vb{r})}
=&\psi(\vb{r}_j)\psi^\dagger(\vb{r})\laplacian\psi(\vb{r})
-\psi^\dagger(\vb{r})\laplacian
\commute{\psi(\vb{r})\psi(\vb{r}_j)}\\
=&\psi(\vb{r}_j)\psi^\dagger(\vb{r})\laplacian\psi(\vb{r})
 -\Theta\psi^\dagger(\vb{r})\laplacian\psi(\vb{r}_j)\psi(\vb{r})
% =&\begin{cases}
% \psi(\vb{r}_j)\psi^\dagger(\vb{r})\laplacian\psi(\vb{r})
% -\psi^\dagger(\vb{r})\laplacian\psi(\vb{r}_j)\psi(\vb{r})
% \qcomma&\text{bosons}\\
% \psi(\vb{r}_j)\psi^\dagger(\vb{r})\laplacian\psi(\vb{r})
% +\psi^\dagger(\vb{r})\laplacian\psi(\vb{r}_j)\psi(\vb{r}),
% &\text{fermions}.
% \end{cases}
\end{aligned}
\end{equation}
Now in both cases $\laplacian$ and $\psi(\vb{r}_j)$ commutes since the
Laplacian operates on $\vb{r}$, and \emph{not} on $\vb{r}_j$. We thus have
\begin{equation}
\begin{aligned}
\comm{\psi(\vb{r}_j)}
{\psi^\dagger(\vb{r})\laplacian\psi(\vb{r})}
=&\psi(\vb{r}_j)\psi^\dagger(\vb{r})\laplacian\psi(\vb{r})
-\Theta\psi^\dagger(\vb{r})\psi(\vb{r}_j)\laplacian\psi(\vb{r})\\
% =&\begin{cases}
% \psi(\vb{r}_j)\psi^\dagger(\vb{r})\laplacian\psi(\vb{r})
% -\psi^\dagger(\vb{r})\psi(\vb{r}_j)\laplacian\psi(\vb{r})
% \qcomma&\text{bosons}\\
% \psi(\vb{r}_j)\psi^\dagger(\vb{r})\laplacian\psi(\vb{r})
% +\psi^\dagger(\vb{r})\psi(\vb{r}_j)\laplacian\psi(\vb{r}),
% &\text{fermions}.
% \end{cases}\\
=&\scomm{\psi(\vb{r}_j)}{\psi^\dagger(\vb{r})}
\laplacian\psi(\vb{r})%\\=&
\quad=\quad
\delta(\vb{r}-\vb{r}_j)\laplacian\psi(\vb{r}).
\end{aligned}
\end{equation}
We now have
\begin{equation}\label{eq:4_comm_i}
\comm{\psi(\vb{r}_j)}{\int\rd^3r\,
\psi^\dagger(\vb{r})\laplacian\psi(\vb{r})}=
\int\rd^3r\,\delta(\vb{r}-\vb{r}_j)\laplacian\psi(\vb{r})
=\laplacian_j\psi(\vb{r}_j),
\end{equation}
which agrees with the first term in \eqref{eq:4_want_this}.

\subsubsection{The second term}
Next we tackle
\begin{equation}
\begin{aligned}
&\comm{\psi(\vb{r}_j)}{\iint\rd^3\!r\,\rd^3\!r'\,
\psi^\dagger(\vb{r})\psi^\dagger(\vb{r}')
u(\vb{r}-\vb{r}')\psi(\vb{r}')\psi(\vb{r})}=\\
=\iint\rd^3\!r\,\rd^3\!r'&
\comm{\psi(\vb{r}_j)}{
\psi^\dagger(\vb{r})\psi^\dagger(\vb{r}')
u(\vb{r}-\vb{r}')\psi(\vb{r}')\psi(\vb{r})}.
\end{aligned}
\end{equation}
Just as before, we want to commute $\psi(\vb{r}_j)$ back to the
creation operators, in the second term. The only difference now is
that we have to commute $\psi(\vb{r}_j)$ past both $\psi(\vb{r})$ and
$\psi(\vb{r}')$. So for \emph{both} bosons and fermions there will be no
change of sign in the end. We get
\begin{equation}
\begin{aligned}
-\psi^\dagger(\vb{r})\psi^\dagger(\vb{r}')
u(\vb{r}-\vb{r}')\psi(\vb{r}')
\commute{\psi(\vb{r})\psi(\vb{r}_j)}
=&-\Theta\psi^\dagger(\vb{r})\psi^\dagger(\vb{r}')
u(\vb{r}-\vb{r}')
\commute{\psi(\vb{r}')\psi(\vb{r}_j)}
\psi(\vb{r})\\
=&-\cancelto{1}{\Theta^2}\psi^\dagger(\vb{r})\psi^\dagger(\vb{r}')
u(\vb{r}-\vb{r}')\psi(\vb{r}_j)\psi(\vb{r}')\psi(\vb{r}).
\end{aligned}
\end{equation}
And once again $\psi(\vb{r}_j)$ and $u(\vb{r}-\vb{r}')$ commute since
$\vb{r}_j$ is independent of both $\vb{r}$ and $\vb{r}'$. So we are
left with 
\begin{equation}
\comm{\psi(\vb{r}_j)}{
\psi^\dagger(\vb{r})\psi^\dagger(\vb{r}')
u(\vb{r}-\vb{r}')\psi(\vb{r}')\psi(\vb{r})}
=\comm{\psi(\vb{r}_j)}
{\psi^\dagger(\vb{r})\psi^\dagger(\vb{r}')}
u(\vb{r}-\vb{r}')\psi(\vb{r}')\psi(\vb{r}).
\end{equation}

We concentrate our effort on the commutator that's still left:
\begin{equation}
\begin{aligned}
\comm{\psi(\vb{r}_j)}{\psi^\dagger(\vb{r})\psi^\dagger(\vb{r}')}
=&\hspace{-20pt}\overbrace{
\comm{\psi(\vb{r}_j)}{\psi^\dagger(\vb{r})}
}^{\acomm{\psi(\vb{r}_j)}{\psi^\dagger(\vb{r})}-2\psi^\dagger(\vb{r})\psi(\vb{r}_j)}
\hspace{-20pt}\psi^\dagger(\vb{r}')
+\psi^\dagger(\vb{r})
\hspace{-20pt}\overbrace{
\comm{\psi(\vb{r}_j)}{\psi^\dagger(\vb{r}')}
}^{2\psi(\vb{r}_j)\psi^\dagger(\vb{r}')-\acomm{\psi(\vb{r}_j)}{\psi^\dagger(\vb{r}')}}
\\%\hspace{-22pt}.
=&
\begin{cases}
\comm{\psi(\vb{r}_j)}{\psi^\dagger(\vb{r})}\psi^\dagger(\vb{r}')
+\psi^\dagger(\vb{r})\comm{\psi(\vb{r}_j)}{\psi^\dagger(\vb{r}')}
,&\text{bosons}\\
\acomm{\psi(\vb{r}_j)}{\psi^\dagger(\vb{r})}\psi^\dagger(\vb{r}')
-\psi^\dagger(\vb{r})\acomm{\psi(\vb{r}_j)}{\psi^\dagger(\vb{r}')}
,\;&\text{fermions}
\end{cases}\\
=&
\begin{cases}
\delta(\vb{r}-\vb{r}_j) \psi^\dagger(\vb{r}')
+\delta(\vb{r}'-\vb{r}_j)\psi^\dagger(\vb{r})
,&\text{bosons}\\
\delta(\vb{r}-\vb{r}_j) \psi^\dagger(\vb{r}')
-\delta(\vb{r}'-\vb{r}_j)\psi^\dagger(\vb{r})
,\;&\text{fermions}
\end{cases}\\
=&
\delta(\vb{r}-\vb{r}_j) \psi^\dagger(\vb{r}')
+\Theta\delta(\vb{r}'-\vb{r}_j)\psi^\dagger(\vb{r}).
\end{aligned}
\end{equation}
In the case of bosons we already had what we needed after using the
commutator identity in the first step. For fermions however we needed
to convert the commutators to anti-commutators; this turned out to
result in almost the same expression in the end as for bosons.

%From here on it will be simpler to deal with the two cases separately.

%\paragraph{Bosons:} 
\begin{equation}
\begin{aligned}
&\iint\rd^3\!r\,\rd^3\!r'\,
\comm{\psi(\vb{r}_j)}{
\psi^\dagger(\vb{r})\psi^\dagger(\vb{r}')
u(\vb{r}-\vb{r}')\psi(\vb{r}')\psi(\vb{r})}\\
=&\iint\rd^3\!r\,\rd^3\!r'\,
\qty(
\delta(\vb{r}-\vb{r}_j) \psi^\dagger(\vb{r}')
+\Theta\delta(\vb{r}'-\vb{r}_j)\psi^\dagger(\vb{r})
)
u(\vb{r}-\vb{r}')\psi(\vb{r}')\psi(\vb{r})\\
=&\int\rd^3\!r'\,
\psi^\dagger(\vb{r}')u(\vb{r}_j-\vb{r}')\psi(\vb{r}')\psi(\vb{r}_j)
+\Theta\int\rd^3\!r\,
\psi^\dagger(\vb{r})u(\vb{r}-\vb{r}_j)
\commute{\psi(\vb{r}_j)\psi(\vb{r})}\\
=&\int\rd^3\!r'\,
\psi^\dagger(\vb{r}')u(\vb{r}_j-\vb{r}')\psi(\vb{r}')\psi(\vb{r}_j)
+\cancelto{1}{\Theta^2}\int\rd^3\!r\,
\psi^\dagger(\vb{r})u(\vb{r}-\vb{r}_j)
\psi(\vb{r})\psi(\vb{r}_j)\\
% =&2\int\rd^3\!r'\,
% \psi^\dagger(\vb{r}')u(\vb{r}_j-\vb{r}')\psi(\vb{r}')\psi(\vb{r}_j)
\end{aligned}
\end{equation}
%Now we don't need to worry about commutations any longer.
The interaction $u$ between particles must clearly be
independent of the direction of the interaction,
i.e. $u(\vb{r}_j-\vb{r}')=u(\vb{r}'-\vb{r}_j)$. This results in the
two integrals being equal, only the name of the integration variable
differs. 

Therefore
\begin{equation}\label{eq:4_comm_ii}
\begin{aligned}
&\iint\rd^3\!r\,\rd^3\!r'\,
\comm{\psi(\vb{r}_j)}{
\psi^\dagger(\vb{r})\psi^\dagger(\vb{r}')
u(\vb{r}-\vb{r}')\psi(\vb{r}')\psi(\vb{r})}\\
=&2\int\rd^3\!r\,
\psi^\dagger(\vb{r})u(\vb{r}_j-\vb{r})\psi(\vb{r})\psi(\vb{r}_j)
\end{aligned}
\end{equation}
for bosons as well as fermions.

% \paragraph{Fermions:}
% \begin{equation}
% \begin{aligned}
% &\iint\rd^3\!r\,\rd^3\!r'\,
% \comm{\psi(\vb{r}_j)}{
% \psi^\dagger(\vb{r})\psi^\dagger(\vb{r}')
% u(\vb{r}-\vb{r}')\psi(\vb{r}')\psi(\vb{r})}\\
% =&\iint\rd^3\!r\,\rd^3\!r'\,
% \qty(
% \delta(\vb{r}-\vb{r}_j) \psi^\dagger(\vb{r}')
% -\delta(\vb{r}'-\vb{r}_j)\psi^\dagger(\vb{r})
% )
% u(\vb{r}-\vb{r}')\psi(\vb{r}')\psi(\vb{r})\\
% =&\int\rd^3\!r'\,
% \psi^\dagger(\vb{r}')u(\vb{r}_j-\vb{r}')\psi(\vb{r}')\psi(\vb{r}_j)
% -\iint\rd^3\!r\,
% \psi^\dagger(\vb{r})u(\vb{r}-\vb{r}_j)
% \commute{\psi(\vb{r}_j)\psi(\vb{r})}.
% \end{aligned}
% \end{equation}
% Now, however, the commutation results in a change of sign. This change
% of sign fortunately cancels the minus before the integral, and as
% before we get
% \begin{equation}\label{eq:4_comm_ii_ferm}
% \begin{aligned}
% &\iint\rd^3\!r\,\rd^3\!r'\,
% \comm{\psi(\vb{r}_j)}{
% \psi^\dagger(\vb{r})\psi^\dagger(\vb{r}')
% u(\vb{r}-\vb{r}')\psi(\vb{r}')\psi(\vb{r})}\\
% =&2\int\rd^3\!r\,
% \psi^\dagger(\vb{r})u(\vb{r}_j-\vb{r})\psi(\vb{r})\psi(\vb{r}_j)
% \end{aligned}
% \end{equation}
% for fermions as well.

\subsubsection{Final result}
To get the full commutator we use \eqref{eq:4_comm_i},
\eqref{eq:4_comm_ii} %and \eqref{eq:4_comm_ii_ferm} 
to get the desired
result:
\begin{equation}\label{eq:4a_comm}
\comm{\psi(\vb{r}_j)}{H}
=\qty(-\frac{\hbar^2}{2m}\laplacian_j
+\int\rd^3\!r\,
\psi^\dagger(\vb{r})
u(\vb{r}-\vb{r}_j)\psi(\vb{r})
)\psi(\vb{r}_j),
\end{equation}
which is valid for both bosons and fermions.
\qed



\subsection{Equivalence with the Schrödinger equation}
Here we want to show that the equation
\begin{equation}\label{eq:4b_start}
\frac{1}{\sqrt{N!}}
\mel{0}{\psi(\vb{r}_1)\psi(\vb{r}_2)\cdots\psi(\vb{r}_N)H}{\Psi_{NE}}
=\frac{E}{\sqrt{N!}}
\mel{0}{\psi(\vb{r}_1)\psi(\vb{r}_2)\cdots\psi(\vb{r}_N)}{\Psi_{NE}}
\end{equation}
is equivalent to the Schrödinger equation:
\begin{equation}\label{eq:4b_SE}
\qty[\sum_{i=1}^N\qty(-\frac{\hbar^2}{2m}\laplacian_i 
+ \sum_{j<i}u_{ij})]\Psi_{NE}(\vb{r}_1,\,\vb{r}_2,\cdots\vb{r}_N)
=E\Psi_{NE}(\vb{r}_1,\,\vb{r}_2,\cdots\vb{r}_N).
\end{equation}
We basically want to show that the two LHS's are equal. To do this we
want to commute $H$ (and all other possible debris) in \eqref{eq:4b_start}
all the way to the left.  

For brevity let's begin by introducing some abbreviations:
\begin{equation}
\psi_n:=\psi(\vb{r}_n)
\end{equation}
and
\begin{equation}
\comm{\psi_n}{H}\stackrel{\eqref{eq:4a_comm}}{=}
\Bigg(\overbrace{-\frac{\hbar^2}{2m}\laplacian_n}^{T_n}
+\underbrace{\int\rd^3\!r\,
\psi^\dagger(\vb{r})
u(\vb{r}-\vb{r}_n)\psi(\vb{r})}_{U_n}
\Bigg)\psi_n=:(T_n+U_n)\psi_n
=:\Xi_n\psi_n
\end{equation}
Please note that $T_m$ and $\psi_n$ commute for $m\neq n$, while $U_m$
and $\psi_n$ does not commute for any $m$ and $n$, so neither does
$\Xi_m$ and $\psi_n$


\subsubsection{Commuting $H$ to the left}
Let's begin by studying
\begin{equation}
\begin{aligned}
\psi(\vb{r}_1)\cdots\psi(\vb{r}_N)H=&
\,\psi_1\cdots\psi_{N-1}
\Big(
H\psi_N+\overbrace{\comm{\psi_N}{H}}^{\Xi_N\psi_N}
\Big)\\
=&\psi_1\cdots\psi_{N-2}
\Big(
H\psi_{N-1}\psi_N+\Xi_{N-1}\psi_{N-1}\psi_N + \psi_{N-1}\Xi_N\psi_N
\Big)
\end{aligned}
\end{equation}
If we were to follow trough with commuting $H$ to the left we would
end up getting
\begin{equation}\label{eq:4b_H_to_left}
\psi_1\psi_2\cdots\psi_{N}H=
H\psi_1\psi_2\cdots\psi_{N}
+\sum_{i=1}^N\psi_1\cdots\psi_{i-1}\Xi_i\psi_i\cdots\psi_N.
\end{equation}
(For the case $i=1$ the term in the sum is just
$\Xi_1\psi_1\cdots\psi_N$, nothing in front of $\Xi_1$.)

\subsubsection{Commuting $U_i$ to the left}
We investigate the sum. Let's take a closer look at $\psi_{i-1}\Xi_i$,
or more specifically $\psi_{i-1}U_i$. To do this we will use
\eqref{eq:4_comm} in the form
\begin{equation}
\scomm{\psi(\vb{r})}{\psi^\dagger(\vb{r}')}
=\psi(\vb{r})\psi^\dagger(\vb{r}')-\Theta\psi^\dagger(\vb{r}')\psi^\dagger(\vb{r})
=\delta(\vb{r}-\vb{r}').
\end{equation}
We are also not restricted by the fact that we have $i-1$ and $i$. It
we can just as well analyze
\begin{equation}
\begin{aligned}
\psi_{j}U_i=&\int\rd^3\!r\,
\psi(\vb{r}_{j})\psi^\dagger(\vb{r})
u(\vb{r}-\vb{r}_i)\psi(\vb{r})\\
=&\int\rd^3\!r\,
\Big[\delta(\vb{r}-\vb{r}_{j}) 
+ \Theta\psi^\dagger(\vb{r})\psi(\vb{r}_{j})\Big]
u(\vb{r}-\vb{r}_i)\psi(\vb{r})\\
=&
u(\vb{r}_{j}-\vb{r}_{i})\psi(\vb{r}_{j}) 
+ \int\rd^3\!r\,
\Theta\psi^\dagger(\vb{r})\psi(\vb{r}_{j})
u(\vb{r}-\vb{r}_i)\psi(\vb{r}).
\end{aligned}
\end{equation}
For the second integral we already know that $u(\vb{r}-\vb{r}_i)$ and
$\psi(\vb{r}_{j})$ commute, and then we pick up another factor
$\Theta$ when commuting $\psi(\vb{r}_{j})$ past $\psi(\vb{r})$. We
thus end up with
\begin{equation}\label{eq:4b_U_comm}
\begin{aligned}
\psi_{j}U_i=&
u(\vb{r}_{j}-\vb{r}_{i})\psi(\vb{r}_{j}) 
+ \cancelto{1}{\Theta^2}\int\rd^3\!r\,
\psi^\dagger(\vb{r})u(\vb{r}-\vb{r}_i)\psi(\vb{r})
\psi(\vb{r}_{j})\\
=&\Big(u_{j,\,i}+U_i \Big)\psi_{j}.
\end{aligned}
\end{equation}
Note that $u_{j,\,i}$ is just a number, so it commutes with any operator. 

Now we are ready to study one full term in the sum from
\eqref{eq:4b_H_to_left}. With the rule in \eqref{eq:4b_U_comm} applied
subsequently to commute $U_i$ all the way to the right we get
\begin{equation}
\begin{aligned}
\psi_1\cdots\psi_{i-1}\Xi_i\psi_i\cdots\psi_N
=&\psi_1\cdots\psi_{i-1}(T_i+U_i)\psi_i\cdots\psi_N\\
=&T_i\psi_1\cdots\psi_{N}
+U_i\psi_1\cdots\psi_{N}
+\sum_{j=1}^{i-1}u_{j,\,i}\;\psi_1\cdots\psi_{N}\\
=&\Bigg(T_i+U_i+\sum_{j<i}u_{j,\,i}\Bigg)\psi_1\cdots\psi_{N}.
\end{aligned}
\end{equation}
And \eqref{eq:4b_H_to_left} becomes
\begin{equation}\label{eq:4b_all_to_left}
\psi_1\psi_2\cdots\psi_{N}H=
\qty[H
+\sum_{i=1}^N\Bigg(T_i+U_i+\sum_{j<i}u_{j,\,i}\Bigg)
]\psi_1\psi_2\cdots\psi_{N}
\end{equation}

\subsubsection{Applying the results}
Now all we have to do is to apply \eqref{eq:4b_all_to_left} to
\eqref{eq:4b_start}, using $T_i=-\hbar^2\laplacian_i/(2m)$, we get
\begin{equation}\label{eq:4b_almost}
\mel**{0}{\psi_1\cdots\psi_{N}H}{\Psi_{NE}}=
\frac{1}{\sqrt{N!}}\mel**{0}{
\qty[H+\sum_{i=1}^N\qty(-\frac{\hbar^2}{2m}\laplacian_i
+U_i+\sum_{j<i}u_{j,\,i})
]\psi_1\cdots\psi_{N}}{\Psi_{NE}}.
\end{equation}
The first part here is easy to handle:
\begin{equation}
\overbrace{\bra{0} H}^{0}\;
\psi_1\psi_2\cdots\psi_{N}\ket{\Psi_{NE}}
=0.
\end{equation}
Then we can see that all that separates \eqref{eq:4b_almost} from
\eqref{eq:4b_SE} is the sum over $U_i$. But again
\begin{equation}
\bra{0} U_i=\int\rd^3r\,
\overbrace{\bra{0}\psi^\dagger(\vb{r})}^{=(\psi(\vb{r})\ket{0})^\dagger=0}
u(\vb{r}-\vb{r}_i)\psi(\vb{r})   = 0
\end{equation}
corresponds to the interaction in the vacuum state. 
And we are left with
\begin{equation}%\label{eq:}
\begin{aligned}
&\frac{1}{\sqrt{N!}}\mel**{0}{\psi_1\cdots\psi_{N}H}{\Psi_{NE}}\\
=&\frac{1}{\sqrt{N!}}\mel**{0}{
\sum_{i=1}^N\qty(-\frac{\hbar^2}{2m}\laplacian_i
+\sum_{j<i}u_{j,\,i})
\psi(\vb{r}_1)\cdots\psi(\vb{r}_{N})}{\Psi_{NE}}.\\
\stackrel{\eqref{eq:4b_start}}{=}&
\frac{E}{\sqrt{N!}}
\mel{0}{\psi(\vb{r}_1)\psi(\vb{r}_2)\cdots\psi(\vb{r}_N)}{\Psi_{NE}},
\end{aligned}
\end{equation}
which corresponds to the Schrödinger equation in
\eqref{eq:4b_SE}.
\qed


\end{document}


%  LocalWords:  Pathria  idealities bosonic Bogoliubov Beale
