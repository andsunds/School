\documentclass[11pt,letter, swedish, english, twocolumn
]{article}
\pdfoutput=1

\usepackage{../custom_as}
\usepackage[makeroom
]{cancel}
\graphicspath{{figures/}}

\swapcommands{\Delta}{\varDelta}
\swapcommands{\Omega}{\varOmega}
\swapcommands{\Phi}{\varPhi}

%%Drar in tabell och figurtexter
\usepackage[margin=10 pt]{caption}
%%För att lägga in 'att göra'-noteringar i texten
\usepackage{todonotes} %\todo{...}

%%För att själv bestämma marginalerna. 
\usepackage[
            top    = 2cm,
%            bottom = 3cm,
            left   = 1.5cm, right  = 1.5cm
]{geometry}

%%För att ändra hur rubrikerna ska formateras

\newcommand{\ZZ}{\ensuremath{\mathcal{Z}}}
\newcommand{\gs}{\ensuremath{{g_{\text{s}}}}}
\newcommand{\Ne}{\ensuremath{{N_{\text{e}}}}}
\newcommand{\Tc}{\ensuremath{{T_{\text{c}}}}}
\newcommand{\lc}{\ensuremath{{\lambda_{\text{c}}}}}
\newcommand{\vc}{\ensuremath{{v_{\text{c}}}}}
\newcommand{\eF}{\ensuremath{{\epsilon_{\text{F}}}}}
\newcommand{\cs}{\ensuremath{{c_{\text{s}}}}}
\newcommand{\wD}{\ensuremath{{\omega_{\text{D}}}}}


%\usepackage{tikz}

\begin{document}

%\tikzstyle{every picture}+=[remember picture]
%\tikzstyle{na} = [shape=rectangle,inner sep=0pt,text depth=0pt]



%%%%%%%%%%%%%%%%% vvv Inbyggd titelsida vvv %%%%%%%%%%%%%%%%%

\title{Statistical Physics -- PHYS\,704 \\
Course summary}
\author{Andréas Sundström}
\date{\today}

\maketitle

%%%%%%%%%%%%%%%%% ^^^ Inbyggd titelsida ^^^ %%%%%%%%%%%%%%%%%

\section{Some Thermodynamic relations}

\begin{itemize}
\item Energy: $E$, $\rd E = T\rd S - P\rd V + \mu\rd N$.
Min. in equilibrium when $S$ and $V$ are const.
\item Helmholtz free energy: $F=E-TS$, $\rd F = -S\rd T - P\rd V + \mu\rd N$.
Min. in equil. when $T$ and $V$ are const.
\item Enthalph: $W=E+PV$, $\rd W = T\rd S + V\rd P + \mu\rd N$.
Min. in equil. when $S$ (adiabatic) and $P$ are const.
\item Gibbs free energy: $\Phi=E+PV-TS$, $\rd\Phi = -S\rd T + V\rd P + \mu\rd N$.
Min. in equil. when $T$ and $P$ are const.
\item Grand potential: $\Omega=-PV$, $\rd\Omega = -S\rd T + P\rd V + N\rd\mu$.
\end{itemize}

Derivative relations
\begin{equation}
\begin{aligned}
&+\qty(\pdv{T}{V})_S=-\qty(\pdv{P}{S})_T=+\pdv[2]{E}{S}{V}\\
&+\qty(\pdv{T}{P})_S=+\qty(\pdv{V}{S})_P=+\pdv[2]{W}{S}{P}\\
&+\qty(\pdv{S}{V})_T=+\qty(\pdv{P}{T})_V=-\pdv[2]{F}{T}{V}\\
&-\qty(\pdv{S}{P})_T=+\qty(\pdv{V}{T})_P=+\pdv[2]{\Phi}{T}{P}\\
\end{aligned}
\end{equation}


Temperature
\begin{equation}
\frac{1}{T}=\qty(\pdv{S}{E})_{V,N}
\end{equation}

Heat-capacity
\begin{equation}
\begin{aligned}
C_V&=\qty(\dv{E}{T})_V=T\qty(\pdv{S}{T})_V=-T\qty(\pdv[2]{F}{T})_V\\
C_P&=\qty(\dv{W}{T})_P=\qty(\pdv{E}{T})_P+P\qty(\pdv{V}{T})_P\\
\end{aligned}
\end{equation}

Compressibility
\begin{equation}
\kappa_X=-\frac{1}{V}\qty(\pdv{V}{P})_X
\end{equation}


\subsection{Some def. in stat. mech.}
Definition of entropy
\begin{equation}
S:=-\sum_{n,(N)} \rho_{n,(N)}\ln(\rho_{n,(N)}),
\end{equation}
where $\rho$ is the density function or distribution fuction.

Canonical distribution ($N$ constant)
\begin{equation}
\rho_n = \frac{1}{Z}\ee^{-E_n/T}\qcomma
Z=\sum_n\ee^{-E_n/T}.
\end{equation}
\begin{equation}
F=-T\ln Z
\end{equation}

Grand canonical distribution 
\begin{equation}
\rho_{n, N} = \frac{1}{\ZZ}\ee^{-(E_n-\mu N)/T}\qcomma
\ZZ=\sum_n\ee^{-(E_n-\mu N)/T}.
\end{equation}
\begin{equation}
\Omega=-T\ln \ZZ
\end{equation}

% The energy 
% \begin{equation}
% E=T^2\qty(\pdv{\ln(\ZZ)}{T})_{N, V}
% \end{equation}


\section{Theory of ideal gases}
Ideal gas means that there is no interaction between particles,
$\epsilon=\hbar^2k^2/(2m)$. 
In 3 dim.
\begin{equation}
\frac{1}{V}\sum_k\to\int\frac{\rd^3k}{(2\pi)^3}
=\int\!\rd\epsilon\,g(\epsilon),
\end{equation}
\begin{equation}
\frac{\rd^3k}{(2\pi)^3}=\rd\epsilon\,g(\epsilon)\qcomma
g(\epsilon)=\frac{m^{3/2}}{\sqrt{2}\pi\hbar^3}\sqrt{\epsilon}.
\end{equation}
See assignment 3 for other dim.


\subsection{Ideal Fermi gases}
\begin{equation}
n^{(\text{F.D.})}=\frac{1}{\ee^{(\epsilon-\mu)/T}+1}
\end{equation}

\begin{equation}
-\Omega=PV=\frac{2}{3}E=\frac{V\gs T}{\lambda^3}f_{5/2}(z)
\end{equation}
\begin{equation}
N=-\qty(\pdv{\Omega}{\mu})_{T,V}=\frac{V\gs}{\lambda^3}f_{3/2}(z)
\end{equation}
\begin{equation}
\frac{PV}{NT}=\frac{f_{5/2}(z)}{f_{3/2}(z)}
\end{equation}
Thermal wavelength $\lambda=h/\sqrt{2\pi mT}=:\Lambda/\sqrt{T}$, $\Lambda$ is a constant.

\paragraph{Fermi functions}
\begin{equation}
f_\nu(z)=\frac{1}{\Gamma(\nu)}\int_0^\infty \!\rd{x}
\frac{x^{\nu-1}}{z^{-1}\ee^x + 1}
\end{equation}
Fugacity $z=\ee^{\mu/T}$.
\begin{equation}
z\pdv{f_\nu(z)}{z} = \pdv{f_\nu(z)}{(\ln z)}=f_{\nu-1}(z)
\end{equation}

\paragraph{Fermi energy}
As $T\to0$, the ccamical potential will go to
\begin{equation}
\mu(T\to0)=:\eF=\frac{\hbar^2}{2m}\qty(\frac{6}{\gs}\pi^2n)^{2/3},
\end{equation}
where $\gs$ is the spin deganareacy, and $n=N/V$. 

In a regular metal, $\eF\sim\unit[10^4]{K}$. For $T\ll\eF$
\begin{equation}
n^{(\text{F.D.})}(\epsilon)\approx
\begin{cases}
1\qcomma&\epsilon<\eF\\
0\qcomma&\epsilon>\eF
\end{cases}
\end{equation}
and $\int_0^\infty\!\rd\epsilon\,n^{(\text{F.D.})}(\epsilon)\ldots
\to\int_0^\eF\!\rd\epsilon\ldots$

The internal energy
\begin{equation}
E(T\ll\eF)=\frac{3}{5}N\eF
\end{equation}


\subsection{Ideal Bose gases}
\begin{equation}
n^{(\text{B.E.})}=\frac{1}{\ee^{(\epsilon-\mu)/T}-1}
\end{equation}

\begin{equation}
-\Omega=PV=\frac{2}{3}E=\frac{V\gs T}{\lambda^3}g_{5/2}(z)
\end{equation}
\begin{equation}
\Ne=-\qty(\pdv{\Omega}{\mu})_{T,V}=\frac{V\gs}{\lambda^3}g_{3/2}(z)
\end{equation}
\begin{equation}
\frac{PV}{NT}=\frac{g_{5/2}(z)}{g_{3/2}(z)}
\end{equation}
Thermal wavelength $\lambda=h/\sqrt{2\pi mT}=:\Lambda/\sqrt{T}$, $\Lambda$ is a constant.

\paragraph{Bose functions}
\begin{equation}
g_\nu(z)=\frac{1}{\Gamma(\nu)}\int_0^\infty \!\rd{x}
\frac{x^{\nu-1}}{z^{-1}\ee^x - 1}
\end{equation}
For bosons $\mu\le0$, meaning that $z=\ee^{\mu/T}\le1$. 
\begin{equation}
z\pdv{g_\nu(z)}{z} = \pdv{g_\nu(z)}{(\ln z)}=g_{\nu-1}(z)
\end{equation}

At $z=1$, $g_\nu(z=1)=\zeta(\nu)$.

\paragraph{Critical temperature}
Critical teperature for ideal Bose gas (3 dim.)
\begin{equation}
\Tc=\frac{2\pi\hbar^2}{m}\qty(\frac{N}{V\gs\,\zeta(\nicefrac{3}{2})})^{3/2}
\end{equation}
Number of condensed particles
\begin{equation}
N_0=N\qty[1-\qty(\frac{T}{\Tc})^{3/2}]\qcomma
T\le\Tc.
\end{equation}

\begin{equation}
\begin{aligned}
&\lambda=\frac{h}{\sqrt{2\pi mT}}\qcomma 
\lc=\Big[v\zeta(\nicefrac{3}{2})\Big]^{1/3}\\
&v=\frac{1}{n}=\frac{V}{N},\hspace{18pt}
\vc=\frac{\lambda^3}{\zeta(\nicefrac{3}{2})}
\end{aligned}
\end{equation}


\subsubsection{Photons}
Photons are bosons with $\mu=0$.
\begin{equation}
n^{\text{(photons)}}=\frac{1}{\ee^{\hbar\omega/T}-1}.
\end{equation}
Number of photons in the interval $\omega$ to $\omega+\rd\omega$:
\begin{equation}
\rd N_\omega=\frac{V}{\pi^2c^3}\frac{\omega^2\id\omega}{\ee^{\hbar\omega/T}-1}.
\end{equation}

Energy in the same interval:
\begin{equation}
\rd E_\omega=\hbar\omega\id{N_\omega}=Vu_\omega\id\omega,
\end{equation}
where $u_\omega$ is the radiation energy density.
Total radiation energy
\begin{equation}
E=\int_0^\infty\!\rd E_\omega = \frac{4\sigma}{c}VT^4,
\end{equation}
where $\sigma=\pi^2/(60\hbar^3c^2)$


\subsubsection{Phonons}
Chain of $N$ particles connected with springs, mean distance
$a$. Hamiltonian 
\begin{equation}
H=\sum_{s=0}^{N-1}\qty[\frac{p_s^2}{2m}+\frac{\kappa}{2}\qty(q_{s+1}-q_s)^2]
\end{equation}
Also periodoc boundary conditions: $q_{s+N}=q_s$ and $p_{s+N}=p_s$.

Fourier transform (momentum space)
\begin{equation}
H=\sum_{k=0}^{N-1}\qty[\frac{1}{2m}p_{-k}p_{k}+\frac{m\omega_k^2}{2}q_{-k}q_{k}].
\end{equation}
This is a Hamiltoninan for $N$ harmonic oscillators:
\begin{equation}
H=\sum_{k=0}^{N-1}\hbar\omega_k\qty(\hat{n}_k+\frac{1}{2}).
\end{equation}

The frequency
\begin{equation}
\begin{aligned}
\omega_k=&\sqrt{\frac{2\kappa}{m}(1-\cos(ka))}\\
\stackrel{ka\ll1}{\approx}&|ka|\sqrt{\frac{\kappa}{m}}.
\end{aligned}
\end{equation}
Speed of sound
\begin{equation}
\cs=\sqrt{\frac{\kappa a^2}{m}}.
\end{equation}

For $|ka|\ll1$, 
\begin{equation}
C_V=\frac{2\pi^2}{5\hbar^3\cs^3} VT^3.
\end{equation}






\section{Second quantaization}
Main idea: use a Bogoliubov transformation to transform the
Hamiltonian to to a hamiltonian of an ideal gas of quasi-particles. 

\paragraph{Harmonic oscillator interpretation in momentum space}
Occupation number representation:
\begin{equation}
\ket{\psi}=\ket{n_{k_1}, n_{k_2},\ldots, n_{k_i},\ldots}
\end{equation}
\begin{equation}
\hat{n}_{k_i}\ket{n_{k_1}, \ldots, n_{k_i},\ldots}=
{n}_{k_i}\ket{n_{k_1}, \ldots, n_{k_i},\ldots}
\end{equation}
For bosons:
\begin{equation}
\comm{a_k}{ a^\dagger_{k'}}=\delta_{k, k'},
\end{equation}
whereas for fermions
\begin{equation}
\acomm{c_k}{ c^\dagger_{k'}}=\delta_{k, k'}.
\end{equation}

The Hamiltonian for interacting bosons
\begin{equation}
H=\overbrace{\sum_k \epsilon_k a_k^\dagger a_k}^{\text{non-interating}}
+\frac{1}{2V}\sum_{k, k', q} U(q) a_{k+q}^\dagger a_{k'-q}^\dagger 
a_{k'}a_{k}
\end{equation}
and for fermions
\begin{equation}
H=\overbrace{\sum_{k, \sigma} \epsilon_k c_{k,\sigma}^\dagger c_{k,\sigma}}^{\text{non-interating}}
+\frac{1}{2V}\sum_{k, k', q, \sigma, \sigma'} 
U(q) c_{k+q,\sigma}^\dagger c_{k'-q,\sigma}^\dagger c_{k',\sigma'}c_{k,\sigma}.
\end{equation}
The inteaction, $U(q)$, is given by
\begin{equation}
U(\vb{r}-\vb{r'})=\frac{1}{V}\sum_q U(\vb{q})\ee^{\ii\vb{q}\vdot(\vb{r}-\vb{r}')},
\end{equation}
or
\begin{equation}
U(\vb{q})=\int\rd^3rU(\vb{r})\ee^{-\ii\vb{q}\vdot\vb{r}}.
\end{equation}

We can also talk about annihilation and creation operators in real
space: $\psi^\dagger(\vb{r})$ creates a particle at position $\vb{r}$.
\begin{equation}
\int\!\rd^3r\, \phi^\dagger(\vb{r})\phi(\vb{r})=\sum_ka_k^\dagger a_k=\hat{N}
\end{equation}
These operators faoolw the same commuttation relations as the momentum
space equivalence. 


\subsection{Superfluidity in $^\vb{4}$He}
At low enough temperatures most atoms will be in the ground state. 
Our goual is to bring the Hamiltonian to a form:
\begin{equation}
H=\sum_k E_k \xi_k^\dagger\xi_k+\text{const.}
\end{equation}

Mean-field theory:
\begin{equation}
a_0a_0^\dagger \approx a_0^\dagger a_0 = \abs{a_0}= N_0
\end{equation}
meaning that
\begin{equation}
a_0=\sqrt{N_0} \ee^{\ii\phi}.
\end{equation}

\paragraph{Bogoliubov transformation}
\begin{equation}
\begin{cases}
\xi_k=u_k a_k+v_ka_{-k}^\dagger\\
\xi_k^\dagger=u_k a_k^\dagger+v_ka_{-k}\\
\end{cases}
\end{equation}
We still want the commutation relation
$\comm{\xi_k}{\xi_k^\dagger}=1$, which gives
\begin{equation}
u_k=\cosh\theta_k\qcomma
v_k\sinh\theta_k.
\end{equation}
And
\begin{equation}
\cosh2\theta_k=\frac{A_k}{\sqrt{A_k^2-B_k^2}}\qcomma
\sinh2\theta_k=\frac{B_k}{\sqrt{A_k^2-B_k^2}}\qcomma
\end{equation}
where
\begin{equation}
A_k=\epsilon_k+nU(k)\qcomma
B_k=nU(k).
\end{equation}

\paragraph{Energy of the quasi-particles}
\begin{equation}
\begin{aligned}
E_k=&A_k\cosh2\theta_k-B_k\sinh2\theta_K\\
=&\sqrt{A_k^2-B_k^2}=\sqrt{\epsilon_k[\epsilon_k+2nU(k)]}
\end{aligned}
\end{equation}
Long wavelength limit ($k\approx0$)
\begin{equation}
E_k\approx\hbar|k|\sqrt{\frac{nU(0)}{m}}
\end{equation}
same form as phonons, and $\cs=\sqrt{nU(0)/m}$. The super fluid
behaves more like a solid than a liquid. 

As long as some disturbance is slower than $\cs$, there can not be any
dissipation of energy into the super fluid. 

\paragraph{Spontaneously broken symmetry}
We can choose $\phi$ in the mean-field therory description of $a_0$
freely. And the particles curren will be
\begin{equation}
\vb{j}=\frac{\hbar n}{m} \grad\phi
\end{equation}
But if the geomery has holes, we must have a periodically varying
$\phi$, meaning that $\grad\phi=\ell2\pi$ becomes quantized.


\subsection{BCS theory of superconductivity}
The BCS (mean-field) equation:
\begin{equation}
\begin{aligned}
\Delta &= \frac{U}{2V}\sum_k \frac{\Delta}{\sqrt{\xi_k^2+\Delta^2}}
\qty[1-2n^{(\text{F.D.})}(E_k)]\\
&= \frac{U}{2V}\sum_k \frac{\Delta}{\sqrt{\xi_k^2+\Delta^2}}
\tanh(\frac{E_k}{2T}).
\end{aligned}
\end{equation}
Here $E_k=\sqrt{\xi_k^2+\Delta^2}$, and $\xi_k=(\epsilon_k-\eF)$ varies
from $-\eF$ to $\infty$.

Sums transfroms according to
\begin{equation}
\frac{1}{V}\sum_k \to g(\eF)\int_{-\hbar\wD}^{\hbar\wD}\!\rd\xi,
\end{equation}
where $\wD$ is the Debye frequency, and $\hbar\wD\ll\eF$ (typ. values
$\hbar\wD\sim\unit[10^2]{K}$, while $\eF\sim\unit[10^4]{K}$). Only the
states around $\epsilon=\eF$ ($\xi=0$) affects superconductivity. 


\subsection{Ginsburg-Landau theory of superconductivity}




\section{Second order phase transitions}

\subsection{Ising model}

\subsection{Landau theory of cont. phase transitions}


\clearpage
\appendix
\section{Special functions}
\subsection{Gamma function}
\begin{equation}
\Gamma(\nu)=\int_0^\infty x^{\nu-1}\ee^{-x}\id{x}
\end{equation}
\begin{equation}
\Gamma(n)=(n-1)!\qcomma n\in\Z^+
\end{equation}

\begin{center}
\begin{tabular}{|l||c|c|c|c|c|}\hline
$x$ & $\nicefrac{1}{2}$ & $\nicefrac{3}{2}$ & $\nicefrac{5}{2}$
& $\nicefrac{7}{2}$ & $\nicefrac{9}{2}$\\ \hline
$\Gamma(x)$ & $\sqrt{\pi}$ & $\sqrt{\pi}/2$ & $3\sqrt{\pi}/4$
& $15\sqrt{\pi}/8$ & $105\sqrt{\pi}/16$
\\ \hline
\end{tabular}
\end{center}


\subsection{Zeta function}
The Riemann zeta function
\begin{equation}
\zeta(\nu)=\sum_{n=1}^\infty \frac{1}{n^\nu}
=\int_0^\infty \frac{x^{\nu-1}}{\ee^x -1}\id{x}
\end{equation}

\begin{center}
\begin{tabular}{|l||c|c|c|}\hline
$x$ & $2$ & $4$ & $6$
\\ \hline $\zeta(x)$ & 
$\pi^2/6$ & $\pi^4/90$ & $\pi^6/945$ 
\\ \hline\hline
$x$ & $3/2$ & $5/2$ & $7/2$
\\ \hline $\zeta(x)$ & 
$2.61238$ & $1.34149$ & $1.12673$ 
\\ \hline\hline
$x$ & $3$ & $5$ & $7$
\\ \hline$\zeta(x)$ & 
$1.20206$ & $1.03693$ & $1.00835$ 
\\ \hline
\end{tabular}
\end{center}



\section{The harmonic oscillator}








\end{document}


%  LocalWords:  Pathria  idealities bosonic Bogoliubov Beale BCS
%  LocalWords:  Laplace's
