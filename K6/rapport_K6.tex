\documentclass[11pt,a4paper, english, swedish
]{article}
\pdfoutput=1

\usepackage{custom_as}
\usepackage{multirow}

\graphicspath{ {fig_and_code/} }

%%Drar in tabell och figurtexter
\usepackage[margin=10 pt]{caption}
%%För att lägga in 'att göra'-noteringar i texten
\usepackage{todonotes} %\todo{...}

%%För att själv bestämma marginalerna. 
\usepackage[
%            top    = 3cm,
%            bottom = 3cm,
%            left   = 3cm, right  = 3cm
]{geometry}

%%För att ändra hur rubrikerna ska formateras
%\renewcommand{\thesection}{...}

\newcommand{\Thalv}[2]{\ensuremath{}T_{\nicefrac{1}{2}}\left(^{#1}\text{#2}\right)}




\begin{document}


%%%%%%%%%%%%%%%%% vvv Inbyggd titelsida vvv %%%%%%%%%%%%%%%%%

\title{K6 -- neutronaktivering av silver}
\author{Andréas Sundström}
\date{\today}

\maketitle

%\begin{abstract}
%\end{abstract}
%%%%%%%%%%%%%%%%% ^^^ Inbyggd titelsida ^^^ %%%%%%%%%%%%%%%%%

%Om man vill ha en lista med vilka todo:s som finns.
%\todolist

\section{Inledning}
I den här labben studeras neutronaktivering av silver. Detta bygger på att de två naturligt förekommande silverisotoperna $^{107}\text{Ag}$ och $^{109}\text{Ag}$ kan omvandlas till $^{108}\text{Ag}$ respektive $^{110}\text{Ag}$ genom neutroninfångning.

Rapporten börjar med att studera effekten från olika neutronenergier. Detta görs genom att dels undersöka siverisotopernas tvärsnitt för termiska neutroner ($E\sim \unit[10]{meV}$), dels undersöka påverkan från olika mycke bromsning med hjälp av parafin. De termiska neutronera undersöks genom att klä in silvret i kadmium som har ett mycket stort tvärsnitt för termiska neutroner. 

Därefter undersöks de två ny silverisotopernas halveringstider. 

\section{Metod}
Silvret först behöver först och främst neutronaktiveras, men då
behöver neutronera bromsas till rätt energier för att silverkärna
effektivt ska kunna ta upp netronerna. För att bormsa neutronerna
används framför allt väte; i den här laborationen användes parafin,
som är stora kol-vätekedjor, för att bromsa neutronerna. Alltså måste
man börja med att bestämma hur mycket parafin som behövs för att
bromsa neutronerna tillräckligt men inte heller för mycket.


\begin{figure}\centering
\resizebox{.5\textwidth}{!}{\input{fig_and_code/aktivering.pdf_t}}
\caption{Olika rymdvinklar för olika avstånd från källan. För att bara mäta effekten från olika mycket parafin måste man kompensera för att el lika stor platta på större avstånd tar upp en mindre rymdvinkel. Alltså att färre antal neutroner kommer att träffa plattan som är längre bort.}
\label{fig:aktivering}
\end{figure}


\section{Resultat och Diskussion}

\begin{table}
\centering
\caption{ }
\label{tab:diffusivitet}
\centerline{
\begin{tabular}{ll|c|c|c|}\cline{3-5}
    & & \multicolumn{1}{c|}{Rådata} & \multicolumn{1}{c|}{Bakgrundskorrigerad} & \multicolumn{1}{c|}{Rymdvinkelkorrigerad} 
    \\\hline
    % Första raden av första delen
    \multicolumn{1}{|l|}{\multirow{2}{*}{Med kadmium}}
    &Pos. 3 & 1\,798& 1\,784& 20,5\,$\times 10^3$
    \\ \cline{2-5}
    % Andra raden av första delen
    \multicolumn{1}{|l|}{} 
    & Pos. 2 & \phantom{1\,}458& \phantom{1\,}444& \phantom{1}4,0\,$\times 10^3$
    \\\hline\hline
    % Första raden av sista delen
    \multicolumn{1}{|l|}{\multirow{3}{*}{Utan kadmium}} 
    &Pos. 3 & 1\,506 & 1\,492 & 17,1\,$\times 10^3$
    \\ \cline{2-5}
    % Andra raden av sista delen
    \multicolumn{1}{|l|}{} 
    & Pos. 2 & 3\,968 & 3\,954 & 18,8\,$\times 10^3$ 
    \\ \cline{2-5}
    % Tredje raden av sista delen
    \multicolumn{1}{|l|}{} 
    & Pos. 1 & 5\,940 & 5\,926 & \phantom{1}5,9\,$\times 10^3$ 
    \\\hline
\end{tabular}}
\end{table}


\begin{figure}\centering
\input{fig_and_code/dataanpassning.tex}
\caption{Uppmätt och korrigerad aktivitet som funktion av tid med
  kurvanpassningar. De båda isotopernas aktivitet kan särskiljas genom
att i princip bara den ena isotopen finns kvar vid stora tider. Så en
exponentalanpassning kan göras till datan i den högra änden av
plotten. Därefter kan den kortlivade isotopens aktivitet extraheras
genom att subtrahera den första anpassingen från datan i den vänsta
änden. De anpassade havleringstiderna blev
$\Thalv{110}{Ag}=\unit[146]{s}$ och $\Thalv{108}{Ag}=\unit[24]{s}$.}
\label{fig:data}
\end{figure}


% \newpage
% \bibliographystyle{ieeetr}
% \bibliography{referenser}%kräver en fil som heter 'referenser.bib'          




\end{document}





%% På svenska ska citattecknet vara samma i både början och slut.
%% Använd två apostrofer (två enkelfjongar): ''.


%% Inkludera PDF-dokument
\includepdf[pages={1-}]{filnamn.pdf} %Filnamnet får INTE innehålla 'mellanslag'!

%% Figurer inkluderade som pdf-filer
\begin{figure}\centering
\centerline{ %centrerar även större bilder
\includegraphics[width=1\textwidth]{filnamn.pdf}
}
\caption{}
\label{fig:}
\end{figure}

%% Figurer inkluderade med xfigs "Combined PDF/LaTeX"
\begin{figure}\centering
\resizebox{.8\textwidth}{!}{\input{filnamn.pdf_t}}
\caption{}
\label{fig:}
\end{figure}

%% Figurer roterade 90 grader
\begin{sidewaysfigure}\centering
\centerline{ %centrerar även större bilder
\includegraphics[width=1\textwidth]{filnamn.pdf}
}
\caption{}
\label{fig:}
\end{sidewaysfigure}


%%Om man vill lägga till något i TOC
\stepcounter{section} %Till exempel en 'section'
\addcontentsline{toc}{section}{\Alph{section}\hspace{8 pt}Labblogg} 

