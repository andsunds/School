\documentclass[11pt,a4paper, 
swedish, english %% Make sure to put the main language last!
]{article}
\pdfoutput=1

%% Andréas's custom package 
%% (Will work for most purposes, but is mainly focused on physics.)
\usepackage{../custom_as}

%% Figures can now be put in a folder: 
\graphicspath{ {figures/} %{some_folder_name/}
}

%% If you want to change the margins for just the captions
\usepackage[]{caption}

%% To add todo-notes in the pdf
\usepackage[%disable  %%this will hide all notes
]{todonotes} 

%% Change the margin in the documents
\usepackage[
%            top    = 3cm,              %% top margin
%            bottom = 3cm,              %% bottom margin
%            left   = 3cm, right  = 3cm %% left and right margins
]{geometry}


\newcommand{\lD}{\ensuremath{\lambda_{\text{D}}}}
\newcommand{\wc}{\ensuremath{\omega_{\text{c}}}}
\newcommand{\rL}{\ensuremath{r_{\text{L}}}}


%% If you want to chage the formating of the section headers
\renewcommand{\thesubsection}{\arabic{section}.\Alph{subsection})}



%%%%%%%%%%%%%%%%%%%%%%%%%%%%%%%%%%%%%%%%%%%%%%%%%%%%%%%%%%%%%%%%%%%%%%
\begin{document}%% v v v v v v v v v v v v v v v v v v v v v v v v v v
%%%%%%%%%%%%%%%%%%%%%%%%%%%%%%%%%%%%%%%%%%%%%%%%%%%%%%%%%%%%%%%%%%%%%%


%%%%%%%%%%%%%%%%%%%% vvv Internal title page vvv %%%%%%%%%%%%%%%%%%%%%
\title{Assignment 2 \\
{\Large Plasma Physics -- RRY085}}
\author{Andréas Sundström}
\date{\today}%{2017-09-08}

\maketitle

%%%%%%%%%%%%%%%%%%%% ^^^ Internal title page ^^^ %%%%%%%%%%%%%%%%%%%%%
%% If you want a list of all todos
%\todolist



\section{Cyclotron motion}
Here we want to calculate the cyclotron frquencies and Larmor radii
for electrons and protons in two different systems. 

The cyclotron frequency is simply given by
\begin{equation} \label{eq1:wc}
\wc = \frac{|q|B}{m}.
\end{equation}
Then the Larmor radius is simply $\rL=v_\perp/\wc$, where $v_\perp$ is
the velocity component perpendicular to the magnetic field. We do
however need some expression for $v_\perp$. For that we use the
equipartition theorem from thermodynamics stating that, in thermal
equilibrium, the thermal energy is equally distributed over all
degrees of freedom. For single particle gases this means that 
(with $k_\text{B}=1$)
\begin{equation}
\frac{m}{2} v_i^2 = \frac{1}{3} T
\quad\Longleftrightarrow\quad
v_i^2 = \frac{2 T}{3m}.
\end{equation}
Now since the perpendicular component of the velocity ``lives'' in two
spacial dimensions, we must have
\begin{equation}
v_\perp^2 = \frac{4 T}{3m}.
\end{equation}
For the Larmor radius, this means
\begin{equation}
\rL = \frac{m}{|q|B}\sqrt{\frac{4 T}{3m}}
=\sqrt{\frac{4m T}{3q^2B^2}}.
\end{equation}

Some numerical values are given in \tabref{tab1}.

\begin{table}
\begin{center}
\caption{Numerical values of $\wc$ and $\rL$ for electrons and
  protons, in Earth's ionosphere ($T_\ee=T_\text{p}=\unit[0.1]{eV}$,
  and $B=\unit[0.3]{G}=\unit[30]{\micro T}$) and in the solar corona
  wind ($T_\ee=T_\text{p}=\unit[10]{eV}$, and
  $B=\unit[10^{-5}]{G}=\unit[1]{nT}$). [Problem 1]
}\label{tab1}
\begin{tabular}{|l|c|c|c|c|}\cline{2-5}
\multicolumn{1}{c|}{}
&\multicolumn{2}{|c|}{electron} 
&\multicolumn{2}{|c|}{proton} \\ \hline
ionosphere 
&$\wc=\unit[5.3\times10^{6}]{rad/s}$ & $\rL=\unit[29]{mm}$
&$\wc=\unit[2.9\times10^{3}]{rad/s}$ & $\rL=\unit[1.2]{m}$
\\ \hline
solar corona 
&$\wc=\unit[8.8\times10^{2}]{rad/s}$ & $\rL=\unit[1.7]{km}$
&$\wc=\unit[4.8\times10^{-1}]{rad/s}$ & $\rL=\unit[74]{km}$
\\ \hline
\end{tabular}
\end{center}
\end{table}

\section{Uniforn electric and magnetic field}
Here, we have a charged particle initially at rest with the
electromagnetic field
\begin{equation}
\vb*E = E\vu{y}\qc
\vb*B = B\vu{z}
\end{equation}
acting on the particle.

The equation of motion is
\begin{equation}
m\dv{\vb*v}{t} = q\qty(\vb*E + \vb*v\cross\vb*B),
%\qc \vb*v(t=0) = \vb*0.
\end{equation}
or in component form
\begin{equation}\label{eq2:ODEs}
\begin{cases}
\dot{v}_x = \hat\omega v_y\\
\dot{v}_y = -\hat\omega v_x + a_E\\
\dot{v}_z = 0,
\end{cases}
\end{equation}
where $\hat\omega = qB/m$ is the ``signed'' cyclotron frequency 
(as opposed to $\wc=|q|B/m$), and $a_E=qE/m$ is an
acceleration due to the electric field. The initial condition is 
$v_x(0)=v_y(0)=v_z(0)=0$.

The $z$ component obviously only has the solution $v_z(t)\equiv0$. But
for the other two, let's begin by differentiating the first equation
one more time:
\begin{equation}
\ddot{v}_x = \hat\omega \dot{v}_y
=-\hat\omega^2 {v}_x + \hat\omega a_E
\quad\Longleftrightarrow\quad
\ddot{v}_x + \hat\omega^2 {v}_x = \hat\omega a_E.
\end{equation}
The RHS of the last equation is just a constant, so the general
solution to this ODE is
\begin{equation}
v_x(t) = A\cos(\hat\omega t) + B\sin(\hat\omega t) 
+ \frac{a_E}{\hat\omega},
\end{equation}
with the initial condition $v_x(0)=0$ requiring
\begin{equation}\label{eq2:vx}
v_x(t) = \frac{a_E}{\hat\omega}[1-\cos(\hat\omega t)]
 + B\sin(\hat\omega t).
\end{equation}
Note that since we went from a first order to a second order ODE we
would need a second initial condition to fully specify $v_x$ at this
point, but instead we will use the inteconnection between $v_x$ and
$v_y$ together with the initial condition on $v_y$ to solve this
problem. The second equation of \eqref{eq2:ODEs} thus becomes
\begin{equation}
\dot{v}_y = -\hat\omega v_x + a_E
=a_E\cos(\hat\omega t)
 - \hat\omega B\sin(\hat\omega t),
\end{equation}
which has the solution 
\begin{equation}\label{eq2:vy}
v_y(t) = \frac{a_E}{\hat\omega}\sin(\hat\omega t)
+B\cos(\hat\omega t) -B.
\end{equation}
The constant of integration must be $-B$ for $v_y(0)=0$ to be
satisfied. To finally deal with $B$, we substitute \eqref{eq2:vy} back
into the first equation of \eqref{eq2:ODEs} and compare it to direct
differentiation of \eqref{eq2:vx}:
\begin{equation}
\overbrace{a_E\sin(\hat\omega t) + \hat\omega B\cos(\hat\omega t)}
^{\dv*{t}:\;\eqref{eq2:vx}}
\stackrel{\eqref{eq2:ODEs}}{=}
\hat\omega v_y
\stackrel{\eqref{eq2:vy}}{=}
a_E\sin(\hat\omega t)
+\hat\omega B [\cos(\hat\omega t)-1].
\end{equation}
Here we see that the already determined coefficient for the sine term
match and the coefficient for the cosine term do as well, but on the
RHS there is a constant term which is not on the LHS. We are therefore
forced to set $B=0$. 

The particle velocity is now finally determined to be
\begin{equation}
\begin{cases}
v_x(t) = \frac{E}{B}[1-\cos(\wc t)]\\
v_y(t) = \xi\frac{E}{B}\sin(\wc t)\\
v_z(t) \equiv 0,
\end{cases}
\end{equation}
where $a_E/\hat\omega = (qe/m)/(qB/m)=E/B$, $\wc=\abs{\hat\omega}$ and
$\xi$ carries the sign of the charge (which is not necessary in the
cosine). Or in terms of position, using $r_x(0)=r_y(0)=r_z(0)=0$, we
can write
\begin{equation}
\begin{cases}
r_x(t) = \frac{E}{B}t-\frac{E}{B\wc}\sin(\wc t)\\
r_y(t) = \xi\frac{E}{B\wc}[1-\cos(\wc t)]\\
r_z(t) \equiv 0.
\end{cases}
\end{equation}

This is seen to be Larmor gyration, with $\rL=E/(B\wc)=Em/(|q|B^2)$,
and a drift of the guiding center. The drift velosity is also seen to
be the expected\footnotemark{} ``E cross B'' drift:
\begin{equation}
\vb*v_\text{drift} = \frac{\vb*E\cross\vb*B}{B^2}
=\frac{E}{B} (\vu{y}\cross\vu{z})
=\frac{E}{B} \vu{x}.
\end{equation}
\footnotetext{This whole problem could easily have been solved on a
  few lines using these formulas from the book, but I was unsure what
  you were really asking for so I did it the long way around. }


\section{Magnetic ``beach''}


\section{Adiabatic energy change}


\section{Cyclotron radiation}


\section{Dipole field}


\section{Conservation of magnetic moment}




%%%%%%%%%%%%%%%%%%%%%%%%%%%%%%%%%%%%%%%%%%%%%%%%%%%%%%%%%%%%%%%%%%%%%%
\end{document}%% ^ ^ ^ ^ ^ ^ ^ ^ ^ ^ ^ ^ ^ ^ ^ ^ ^ ^ ^ ^ ^ ^ ^ ^ ^ ^ ^
%%%%%%%%%%%%%%%%%%%%%%%%%%%%%%%%%%%%%%%%%%%%%%%%%%%%%%%%%%%%%%%%%%%%%%
%  LocalWords:  Debye
