\documentclass[11pt,a4paper, 
swedish, english %% Make sure to put the main language last!
]{article}
\pdfoutput=1

%% Andréas's custom package 
%% (Will work for most purposes, but is mainly focused on physics.)
\usepackage{../custom_as}

%% Figures can now be put in a folder: 
\graphicspath{ {figures/} %{some_folder_name/}
}

%% If you want to change the margins for just the captions
\usepackage[]{caption}

%% To add todo-notes in the pdf
\usepackage[%disable  %%this will hide all notes
]{todonotes} 

%% Change the margin in the documents
\usepackage[
%            top    = 3cm,              %% top margin
%            bottom = 3cm,              %% bottom margin
%            left   = 3cm, right  = 3cm %% left and right margins
]{geometry}


\newcommand{\lD}{\ensuremath{\lambda_{\text{D}}}}
\newcommand{\wc}{\ensuremath{\omega_{\text{c}}}}
\newcommand{\rL}{\ensuremath{r_{\text{L}}}}
\newcommand{\vL}{\ensuremath{v_{\text{L}}}}


%% If you want to chage the formating of the section headers
\renewcommand{\thesubsection}{\arabic{section}.\Alph{subsection})}



%%%%%%%%%%%%%%%%%%%%%%%%%%%%%%%%%%%%%%%%%%%%%%%%%%%%%%%%%%%%%%%%%%%%%%
\begin{document}%% v v v v v v v v v v v v v v v v v v v v v v v v v v
%%%%%%%%%%%%%%%%%%%%%%%%%%%%%%%%%%%%%%%%%%%%%%%%%%%%%%%%%%%%%%%%%%%%%%


%%%%%%%%%%%%%%%%%%%% vvv Internal title page vvv %%%%%%%%%%%%%%%%%%%%%
\title{Assignment 2 \\
{\Large Plasma Physics -- RRY085}}
\author{Andréas Sundström}
\date{\today}%{2017-09-08}

\maketitle

%%%%%%%%%%%%%%%%%%%% ^^^ Internal title page ^^^ %%%%%%%%%%%%%%%%%%%%%
%% If you want a list of all todos
%\todolist



\section{Cyclotron motion}
Here we want to calculate the cyclotron frquencies and Larmor radii
for electrons and protons in two different systems. 

The cyclotron frequency is simply given by
\begin{equation} \label{eq1:wc}
\wc = \frac{|q|B}{m}.
\end{equation}
Then the Larmor radius is simply $\rL=v_\perp/\wc$, where $v_\perp$ is
the velocity component perpendicular to the magnetic field. We do
however need some expression for $v_\perp$. For that we use the
equipartition theorem from thermodynamics stating that, in thermal
equilibrium, the thermal energy is equally distributed over all
degrees of freedom. For single particle gases this means that 
(with $k_\text{B}=1$)
\begin{equation}
\frac{m}{2} v_i^2 = \frac{1}{3} T
\quad\Longleftrightarrow\quad
v_i^2 = \frac{2 T}{3m}.
\end{equation}
Now since the perpendicular component of the velocity ``lives'' in two
spacial dimensions, we must have
\begin{equation}
v_\perp^2 = \frac{4 T}{3m}.
\end{equation}
For the Larmor radius, this means
\begin{equation}
\rL = \frac{m}{|q|B}\sqrt{\frac{4 T}{3m}}
=\sqrt{\frac{4m T}{3q^2B^2}}.
\end{equation}

Some numerical values are given in \tabref{tab1}.

\begin{table}
\begin{center}
\caption{Numerical values of $\wc$ and $\rL$ for electrons and
  protons, in Earth's ionosphere ($T_\ee=T_\text{p}=\unit[0.1]{eV}$,
  and $B=\unit[0.3]{G}=\unit[30]{\micro T}$) and in the solar corona
  wind ($T_\ee=T_\text{p}=\unit[10]{eV}$, and
  $B=\unit[10^{-5}]{G}=\unit[1]{nT}$). [Problem 1]
}\label{tab1}
\begin{tabular}{|l|c|c|c|c|}\cline{2-5}
\multicolumn{1}{c|}{}
&\multicolumn{2}{|c|}{electron} 
&\multicolumn{2}{|c|}{proton} \\ \hline
ionosphere 
&$\wc=\unit[5.3\times10^{6}]{rad/s}$ & $\rL=\unit[29]{mm}$
&$\wc=\unit[2.9\times10^{3}]{rad/s}$ & $\rL=\unit[1.2]{m}$
\\ \hline
solar corona 
&$\wc=\unit[8.8\times10^{2}]{rad/s}$ & $\rL=\unit[1.7]{km}$
&$\wc=\unit[4.8\times10^{-1}]{rad/s}$ & $\rL=\unit[74]{km}$
\\ \hline
\end{tabular}
\end{center}
\end{table}

\section{Uniforn electric and magnetic field}
Here, we have a charged particle initially at rest with the
electromagnetic field
\begin{equation}
\vb*E = E\vu{y}\qc
\vb*B = B\vu{z}
\end{equation}
acting on the particle.

The equation of motion is
\begin{equation}
m\dv{\vb*v}{t} = q\qty(\vb*E + \vb*v\cross\vb*B),
%\qc \vb*v(t=0) = \vb*0.
\end{equation}
or in component form
\begin{equation}\label{eq2:ODEs}
\begin{cases}
\dot{v}_x = \hat\omega v_y\\
\dot{v}_y = -\hat\omega v_x + a_E\\
\dot{v}_z = 0,
\end{cases}
\end{equation}
where $\hat\omega = qB/m$ is the ``signed'' cyclotron frequency 
(as opposed to $\wc=|q|B/m$), and $a_E=qE/m$ is an
acceleration due to the electric field. The initial condition is 
$v_x(0)=v_y(0)=v_z(0)=0$.

The $z$ component obviously only has the solution $v_z(t)\equiv0$. But
for the other two, let's begin by differentiating the first equation
one more time:
\begin{equation}
\ddot{v}_x = \hat\omega \dot{v}_y
=-\hat\omega^2 {v}_x + \hat\omega a_E
\quad\Longleftrightarrow\quad
\ddot{v}_x + \hat\omega^2 {v}_x = \hat\omega a_E.
\end{equation}
The RHS of the last equation is just a constant, so the general
solution to this ODE is
\begin{equation}
v_x(t) = A\cos(\hat\omega t) + B\sin(\hat\omega t) 
+ \frac{a_E}{\hat\omega},
\end{equation}
with the initial condition $v_x(0)=0$ requiring
\begin{equation}\label{eq2:vx}
v_x(t) = \frac{a_E}{\hat\omega}[1-\cos(\hat\omega t)]
 + B\sin(\hat\omega t).
\end{equation}
Note that since we went from a first order to a second order ODE we
would need a second initial condition to fully specify $v_x$ at this
point, but instead we will use the inteconnection between $v_x$ and
$v_y$ together with the initial condition on $v_y$ to solve this
problem. The second equation of \eqref{eq2:ODEs} thus becomes
\begin{equation}
\dot{v}_y = -\hat\omega v_x + a_E
=a_E\cos(\hat\omega t)
 - \hat\omega B\sin(\hat\omega t),
\end{equation}
which has the solution 
\begin{equation}\label{eq2:vy}
v_y(t) = \frac{a_E}{\hat\omega}\sin(\hat\omega t)
+B\cos(\hat\omega t) -B.
\end{equation}
The constant of integration must be $-B$ for $v_y(0)=0$ to be
satisfied. To finally deal with $B$, we substitute \eqref{eq2:vy} back
into the first equation of \eqref{eq2:ODEs} and compare it to direct
differentiation of \eqref{eq2:vx}:
\begin{equation}
\overbrace{a_E\sin(\hat\omega t) + \hat\omega B\cos(\hat\omega t)}
^{\dv*{t}:\;\eqref{eq2:vx}}
\stackrel{\eqref{eq2:ODEs}}{=}
\hat\omega v_y
\stackrel{\eqref{eq2:vy}}{=}
a_E\sin(\hat\omega t)
+\hat\omega B [\cos(\hat\omega t)-1].
\end{equation}
Here we see that the already determined coefficient for the sine term
match and the coefficient for the cosine term do as well, but on the
RHS there is a constant term which is not on the LHS. We are therefore
forced to set $B=0$. 

The particle velocity is now finally determined to be
\begin{equation}
\begin{cases}
v_x(t) = \frac{E}{B}[1-\cos(\wc t)]\\
v_y(t) = \xi\frac{E}{B}\sin(\wc t)\\
v_z(t) \equiv 0,
\end{cases}
\end{equation}
where $a_E/\hat\omega = (qe/m)/(qB/m)=E/B$, $\wc=\abs{\hat\omega}$ and
$\xi$ carries the sign of the charge (which is not necessary in the
cosine). Or in terms of position, using $r_x(0)=r_y(0)=r_z(0)=0$, we
can write
\begin{equation}
\begin{cases}
r_x(t) = \frac{E}{B}t-\frac{E}{B\wc}\sin(\wc t)\\
r_y(t) = \xi\frac{E}{B\wc}[1-\cos(\wc t)]\\
r_z(t) \equiv 0.
\end{cases}
\end{equation}

This is seen to be Larmor gyration, with $\rL=E/(B\wc)=Em/(|q|B^2)$,
and a drift of the guiding center. The drift velosity is also seen to
be the expected\footnotemark{} ``E cross B'' drift:
\begin{equation}
\vb*v_\text{drift} = \frac{\vb*E\cross\vb*B}{B^2}
=\frac{E}{B} (\vu{y}\cross\vu{z})
=\frac{E}{B} \vu{x}.
\end{equation}
\footnotetext{This whole problem could easily have been solved on a
  few lines using this formula and the initial conditions, but I was
  unsure what you were really asking for so I did it the long way
  around. } 


\section{Magnetic ``beach''}
\newcommand{\Lp}{\ensuremath{L_{\text{p}}}}
If we have a magnetic ``beach''
\begin{equation}
\vb*B = B\qty[1+\frac{x-x_0}{L_0}\Theta(x-x_0)]\vu{x},
\end{equation}
then a particle with a non-normal incidence onto this beach will be
stopped and reflected back. To find the penetration depth, \Lp, we
will use the fact that the magnetic Lorentz force is perpendicular to
the velocity and thus cannot exert any work on the particle, and the
fact that the magnetic moment $\mu=m v_\perp^2/(2B)$ is conserved.

At the penetration depth, the velocity component parallell to the
magnetic field is~0. With concervation of energy, this means that
\begin{equation}
\frac{mv_{0}^2}{2}% + \frac{mv_{0\perp}^2}{2} 
=\frac{mv_{1\perp}^2}{2}
\quad\Longleftrightarrow\quad
v_{1\perp}^2=v_{0}^2,
\end{equation}
%where $v_{0||}$ and $v_{0\perp}$ is the parallel and perpendicular velocity components 
where $v_0$ is the particle speed 
upon entry onto the beach, and $v_{1\perp}$ is the
perpendicular (and only) velocity component at the turning point. 
Now, the conservation of magnetic moment also implies that
\begin{equation}
\frac{mv_{0\perp}^2}{2B_0}=\frac{mv_{1\perp}^2}{2B_1}
\quad\Longleftrightarrow\quad
\frac{v_{1\perp}^2}{v_{0\perp}^2}=\frac{B_1}{B_0}.
\end{equation}
Togther we have
\begin{equation}
\frac{B_1}{B_0}=\frac{v_{0}^2}{v_{0\perp}^2}=\sin^{-2}\alpha.
\end{equation}
Lastly we write
\begin{equation}
B_1 = B_0\qty[1+\frac{\Lp}{L_0}],
\end{equation}
which yiels
\begin{equation}
\Lp = L_0\qty[\sin^{-2}(\alpha)-1].
\end{equation}
As expected if $v_{0\perp}=0$ there is no turning point
($\Lp\to\infty$). 


\section{Adiabatic energy change}
We want to calculate the energy change assuming that we have a system
of a large number, $N$, of non-interacting electrons, with
isotropically distributed velocities but all with speed $v_0$, moving
in a uniform magnetic field which adiabatically increases from $B_0$
to $\alpha B_0$ ($\alpha>1$).  

To find the total energy of a system, we can use the distribution
function:
\begin{equation}
U = \oldint_{\vb*x}\!\rd^3x \oldint_{\vb*v}\!\rd^3v \,
\epsilon(\vb*x, \vb*v)f(\vb*x, \vb*v).
\end{equation}
Since there is no electric potential and the electron-electron
interactions $\epsilon(\vb*x, \vb*v) = m v^2/2$. Initially all
velociteis are isotropically distributed and have magnitude $v_0$,
which means that 
$f_0(\vb*x, \vb*v) = n_0(\vb*x)\delta(v-v_0)/(4\pi v_0^2)$.
which means that the inital energy of
the system is
\begin{equation}
U_0 = \oldint_{\vb*x}\!\rd^3x \oldint_{\vb*v}\!\rd^3v \,
\frac{mv^2}{2}f_0(\vb*x, \vb*v)
=\frac{mv_0^2}{2}\oldint_{\vb*x}\!\rd^3x\,n_0(\vb*x)
=N\frac{mv_0^2}{2}
\end{equation}


To find the final energy, after the increas in magnetic field, we once
again use the fact that the magnetic moment  $\mu=mv_\perp^2/(2B)$ is
conserved, while the velocity component parallel to the field is
unchanged. This means that the new energy of a particle is 
\begin{equation}
\epsilon = \frac{mv_{1||}^2}{2}+\frac{mv_{1\perp}^2}{2}
=\frac{mv_{0||}^2}{2}+\alpha\frac{mv_{0\perp}^2}{2}.
\end{equation}
Note that this is expressed only in the original velocity components,
which means that we can use $f_0$ when calculating the energy. 
The integration over all of phase space is made significantly easier
if we align the $z$-axis of the velocity integral with the magnetic
field, which means that $v_{0||}=v_0\cos\theta$ and
$v_{0\perp}=v_0\sin\theta$ if written in polar coordinates. 
The final energy is therefore given by
\begin{equation}
\begin{aligned}
U_1 =& \frac{m}{2}\oldint_{\vb*x}\!\rd^3x \oldint_{\vb*v}\!\rd^3v \,
\qty[\cos^2\theta + \alpha\sin^2\theta] v^2 f_0(\vb*x, \vb*v)\\
=&N\frac{m}{2}
\int_0^{2\pi}\rd\phi\int_0^\pi\rd\theta\,\sin\theta
\int_0^\infty \rd{v}\, v^2\,
\qty[\alpha - (\alpha-1)\cos^2\theta] v^2\frac{\delta(v-v_0)}{4\pi v_0^2}\\
=&N\frac{mv_0^2}{2}\,\frac{1}{2}
\int_0^\pi\rd\theta\,\sin\theta
\qty[\alpha - (\alpha-1)\cos^2\theta]\\
=&N\frac{mv_0^2}{2}\,\frac{1+2\alpha}{3}
=\frac{1+2\alpha}{3}U_0.
\end{aligned}
\end{equation}
(The last integral can easily be calculated with a substitution of
variables $s=\cos\theta$, $\rd{s}=-\sin\theta\,\rd\theta$.) As a
sanity check, we see that if $\alpha\to1$ then $U_1\to U_0$ as
expected. 



\section{Cyclotron radiation}
The total radiated power due to an acceleration $a$ of a
non-relativistic charged particle is given by the Laramor formula 
\begin{equation}
P = \frac{e^2\, a^2}{6\pi \varepsilon_0 c^3},
\end{equation}
where $e$ is the charge of the paricle.

If the particle motion is Laramor gyration, then $a=\vL^2/\rL$ where
$\rL=m\vL/(eB)$. This means that the acceleration can be written as
$a=\vL eB/m$
radiated power is
\begin{equation}\label{eq5:PL}
P=\frac{e^4B^2\, \vL^2}{6\pi \varepsilon_0 c^3m^2}
=\frac{e^4B^2}{3\pi \varepsilon_0 c^3m^3}
\frac{m\vL^2}{2}
=\frac{1}{\tau}W,
\end{equation}
where $W$ is the kinetic energy of the particle, and
\begin{equation}\label{eq5:decay-time}
\tau = \frac{3\pi \varepsilon_0 c^3m^3}{e^4B^2}
\end{equation}
is some constant with the dimensions of time.

If there is no other sourse of energy the radiated $P=-\dv*{W}{t}$ has
to be the rate of change of the kinetic energy. Therefore
\eqref{eq5:PL} yields
\begin{equation}\label{eq5:ODE}
\dv{W}{t}+\frac{1}{\tau}W=0
\end{equation}
wich means that
\begin{equation}
W(t)=W(0)\ee^{-t/\tau}.
\end{equation}
We now see the physical interpretation of the constant $\tau$ from
\eqref{eq5:decay-time}; it is the decay time of the kinetic energy.

The decay time depends on the field strength. We can therefore express
the magnetic field, given a set decay time, according to
\begin{equation}
B=\sqrt{\frac{3\pi \varepsilon_0 c^3m^3}{e^4\tau}}.
\end{equation}
For instance if $\tau=\unit[1]{s}$, then $B=\unit[1.6]{T}$. This is
easily in the same order of magnitude as in tokamaks. However for  
\eqref{eq5:ODE} to be valid, there cannot be any other interactions
between paricles. E.g. almost all of the radiated energy radiated from
particles \emph{within the bulk} of the plasma will probably be
reabsorbed by another particle in the plasma. And then energy is also
added externally to the plasma to heat it up.


\section{Dipole pendulum}
The electric field from a dipole $\vb*p$, in the origin, is given by
\begin{equation}
\vb*E(\vb*r) = 
\frac{3\vu*r(\vb*p\vdot\vu*r)-\vb*p}{4\pi\varepsilon_0r^3},
\end{equation}
where $\vb*r=r\vu*r$, and $\vu*r\vdot\vu*r=1$. Separeted into
componentes the field, produced by a dipole $\vb*p=p\vu{z}$, is given
by [eqn. 4.12, J. D. Jackson, \textit{Classical Electrodynamics}, 3rd
ed., 1999]
\begin{equation}
\begin{cases}
E_r = \dfrac{2p\,\cos\theta}{4\pi\varepsilon_0 r^3}\\
E_\theta = \dfrac{p\,\sin\theta}{4\pi\varepsilon_0 r^3}\\
E_\phi=0.
\end{cases}
\end{equation}

\subsection*{Similarities with the physical penulum}
Given that that a charged particle, starting out at rest with
$\theta=\pi/2$, will follow a semicircular\footnotemark{} trajectory
($r=a$ constant) we can find a differential equation for the
angle~$\theta$. The velocity along the tradjectory is $a\dot\theta$,
so the equation of mothion will be
\begin{equation}\label{eq6:EOM1}
ma\ddot\theta = qE_\theta = \frac{qp}{4\pi\varepsilon_0 a^3}\sin\theta
\quad\Longleftrightarrow\quad
\ddot\theta - \xi\Omega^2\,\sin\theta=0,
\end{equation}
where
\begin{equation}\label{eq6:Omega}
\Omega= \sqrt{\frac{|q|p}{4\pi\varepsilon_0 a^4m}}
\end{equation}
is some constant with dimension of invers time,
and $\xi=q/|q|=\pm1$ is the sign of the charge, and the initial
conditions are $\theta(0)=\pi/2$ and $\dot\theta(0)=0$.
\footnotetext{Which I assume we don't have to prove, otherwise 
[R. S. Jones, \textit{Circular motion of charged particle in an
  elextric dipole filed}, Am. J. Phys., \textbf{63}, November] has a
very short and simple argument, and he seems to be the first one to
prove this result.} 

Note that this is \emph{exactly} the same equation of motion as for
the simple physical pendulum ($\ddot\theta + \omega^2\sin\theta=0$),
but here the initial condition is $\theta(0)=\pi/2$ and not 0 -- we
can no longer make the assumtion $\sin\theta\approx\theta$). Also,
\eqref{eq6:EOM1} has a differnt sign 
depending on $\xi$, but all that really does is to determine wheter
the pendulum swings upward or downward. 

\subsection*{Using the centripital force to find the period of oscillation}
We could also use the radial part of the elctric field and the
centripetal force to get another differntial equation:
\begin{equation}\label{eq6:EOM2}
-ma(\dot\theta)^2 = qE_r
=\frac{2qp\,\cos\theta}{4\pi\varepsilon_0 r^3}
\quad\Longleftrightarrow\quad
(\dot\theta)^2 + 2\xi\Omega^2\,\cos\theta=0,
\end{equation}
with $\xi$ and $\Omega$ as before. (The minus sign on the LHS is due
tho the fact that the centripetal force points in the $-\vu*r$
direction, while the radial elctric force is $+qE_r\vu*r$.)

Now the period of oscillation can be calculated. We can assume,
without loss of generality, that $q<0$ ($\xi=-1$); then the particle
will swing upward and the angle $\theta\in[-\pi/2, \pi/2]$. Due to
symmetry, the oscillation period, $T$, can be expressed as 4 times the
time it takes for $\theta$ to swing from $-\pi/2$ to 0:
\begin{equation}
\begin{aligned}
T=&4\int_{\theta=-\pi/2}^{\theta=0}\rd t
=4\int_{\theta=-\pi/2}^{\theta=0}\dv{t}{\theta}\id\theta
=4\int_0^{\pi/2}\frac{\rd\theta}{\dot\theta}
\end{aligned}
\end{equation}
Now with \eqref{eq6:EOM2}, we can write this as
\begin{equation}
\begin{aligned}
T=&4\int_{-\pi/2}^0\frac{\rd\theta}{\sqrt{2\Omega^2\cos\theta}}\\
\qty{\alpha=\theta+\frac{\pi}{2}}
=&\frac{2\sqrt{2}}{\Omega}\int_0^{\pi/2}\frac{\rd\alpha}{\sqrt{\sin\alpha}}
=\frac{2\sqrt{2}}{\Omega}\frac{[\Gamma(\nicefrac14)]^2}{2\sqrt{2\pi}}
=\frac{[\Gamma(\nicefrac14)]^2}{\sqrt{\pi}\Omega},
\end{aligned}
\end{equation}
with $\Omega$ as in \eqref{eq6:Omega}.
The last integral is due to [formula 3.621.7, I. S. Gradshteyn \&
I. M. Ryzhik, \textit{Table of Integrals, Series, and Products}, 8th
ed., 2015].



\section{Conservation of magnetic moment}
Assuming that we have a slowly time and space varying magnetic field
$\vb*B(\vb*r, t)$, we want to prove that
\begin{equation}
\mu:=I\pi\rL^2=\frac{|q|\wc}{2\pi}\,\pi\rL^2=
\frac{mv_\perp^2}{2B}=\frac{W_\perp}{B}.
\end{equation}
is constant. The fact that the
variations are slow can be expressed as
\begin{equation}
\rL\abs{\frac{1}{B}\pdv{B}{s}}=\epsilon_s\ll1
\qc
\frac{1}{\wc}\abs{\frac{1}{B}\pdv{B}{t}}=\epsilon_t\ll1,
\end{equation}
where $\pdv*{s}$ denotes differentiation in any direction.

We first note that $\vb*F\vdot\vb*v_\perp$ is the power exerted by the
force $\vb*F$ in the direction of $\vb*v_\perp$. We also note that
\emph{spacial} variations in $\vb*B$ will not result in any (direct)
force in the perpendicular direction. Therefore
\begin{equation}\label{eq7:Wp1}
\dv{W_\perp}{t} = q\vb*E\vdot\vb*v_\perp,
\end{equation}
where the electric field is due to Faraday's law
\begin{equation}\label{eq7:Faraday}
\pdv{B}{t} = -\curl\vb*E.
\end{equation}
The temporal variations were very slow compared to the Laramor period,
which means that we can replace $\vb*E\vdot\vb*v_\perp$ with an
averaged over one period of oscillation:
\begin{equation}\label{eq7:evp}
\vb*E\vdot\vb*v_\perp
=\ev{\vb*E\vdot\vb*v_\perp}+\order{\epsilon_t}
=\frac{\wc}{2\pi}\int_0^{\tau_\text{L}}\vb*E\vdot\vb*v_\perp\id{t}
+\order{\epsilon_t},
\end{equation}
where $\tau_\text{L}=2\pi/\wc$.
Now, since $\vb*v_\perp\id{t}=\rd\vb*r_\perp$, we can write the last
integral in \eqref{eq7:evp} as
\begin{equation}
\int_0^{\tau_\text{L}}\vb*E\vdot\vb*v_\perp\id{t}
=\oint_{\pd S}\vb*E\vdot\rd\vb*r_\perp +\order{\epsilon_t}+\order{\epsilon_s}
\end{equation}
where $\pd S$ is the (unperturbed) Laramor orbit; this ofcourse
introduces some errors 
$\order{\epsilon_t}+\order{\epsilon_s}=\order{\epsilon_t, \epsilon_s}$
due to variations of $\vb*B$ in both space and time. Now Stoke's
theorem comes in handy:
\begin{equation}
\oint_{\pd S}\vb*E\vdot\rd\vb*r_\perp 
=\int_{S}(\curl\vb*E)\vdot\rd\vb*a 
=-\int_{S}\pdv{\vb*B}{t}\vdot\rd\vb*a 
=-\pdv{t}\int_{S}\vb*B\vdot\rd\vb*a 
\end{equation}
Now assuming that the spacial variations of $\vb*B$ are small, the
last integral just becomes
\begin{equation}
-\pdv{t}\int_{S}\vb*B\vdot\rd\vb*a 
=-\pdv{t}\qty[\pi\rL^2 \vb*B\vdot\vu{n}_S]+\order{\epsilon_s}
=-\pi\rL^2 \vu{n}_S\vdot\pdv{\vb*B}{t}+\order{\epsilon_s}
\end{equation}
where $\vu{n}_S$ is the surface normal to the surface of the disc
whose perimiter is the Laramore orbit. 

We can now collect the fruit of our work and return to
\eqref{eq7:Wp1}. We note when going back trough all the steps that no
more than first order corrections were introduced on the way, meaning
that in the end
\begin{equation}
\dv{W_\perp}{t} = -\frac{q\wc}{2\pi}\pi\rL^2 \,
\vu{n}_S\vdot\pdv{\vb*B}{t}
+\order{\epsilon_s, \epsilon_t}.
\end{equation}
We note that $\vu{n}_S$ has to be either parallel or anti-paralell to
the magnetic field, and which one depends on the sign of the
charge. Luckily we have a (signed) facor $q$ in the extression, so by
thinking the Laramor orbit through, we see that we can write
\begin{equation}
\dv{W_\perp}{t} = +\frac{|q|\wc}{2\pi}\pi\rL^2 \,\pdv{B}{t}
+\order{\epsilon_s, \epsilon_t}
=\mu\pdv{B}{t} +\order{\epsilon_s, \epsilon_t}.
\end{equation}

To pound in the last nain in this coffin, we also note that we can
write
\begin{equation}
\dv{W_\perp}{t} = \dv{(\mu B)}{t}
=B\dv{\mu}{t}+\mu\dv{B}{t}.
\end{equation}
Equating these last two we see that
\begin{equation}
B\dv{\mu}{t}+\mu\dv{B}{t}=\mu\pdv{B}{t} 
+\order{\epsilon_s, \epsilon_t}
\quad\Longleftrightarrow\quad
\dv{\mu}{t}=\order{\epsilon_s, \epsilon_t},
\end{equation}
which is what we wanted to prove.

%%%%%%%%%%%%%%%%%%%%%%%%%%%%%%%%%%%%%%%%%%%%%%%%%%%%%%%%%%%%%%%%%%%%%%
\end{document}%% ^ ^ ^ ^ ^ ^ ^ ^ ^ ^ ^ ^ ^ ^ ^ ^ ^ ^ ^ ^ ^ ^ ^ ^ ^ ^ ^
%%%%%%%%%%%%%%%%%%%%%%%%%%%%%%%%%%%%%%%%%%%%%%%%%%%%%%%%%%%%%%%%%%%%%%
%  LocalWords:  Debye Laramor
