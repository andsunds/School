\begin{itemize}
    \item Lite om transport i celler
    \item Resultat från dataanalysen
    \item Slutsatser vi kan dra från analysen
\end{itemize}

%Paragrafer kan tas bort när vi är klara

%Vad som än inte vävts in i arbetet (sammanställning från planeringsrapporten)

\paragraph{Övergripande}\todo{Varför så många paragraphs? Är det inte bättre med en lite mer flytande text?}
Att partiklar kan ta sig fram genom cytoplasman spelar onekligen en viktig roll för många funktioner i cellen, exempelvis vid signaltransport när signalämnen tillåts sprida sig genom cellen. I vissa fall har hastigheten för spridning inte en betydande roll, men vid exempelvis celldelning är det viktigt att de delade cellerna får med sig en fullständig uppsättning organeller och en likvärdig blandning av cytoplasmans beståndsdelar. För detta krävs att beståndsdelarna kan förflytta sig runt och blandas.

I nuläget råder viss oenighet gällande hur cytoplasman, cellens fyllnadsmaterial, egentligen beter sig. Man har länge trott att cytoplasman kan ha två distinkta faser, en kolloid vätskefas med partiklar väl blandade i cytoplasman och en fas med komponenter i cellen som interagerar för att bilda ett sammanhängande nätverk som gör cytoplasman mer fast eller glasliknande.

Fördjupade studier av partikelrörelse i cellen skulle till exempel kunna leda till mer effektiva läkemedel. Vet man hur transporten inom cellen sker underlättar det arbetet med att ta fram specialdesignad medicin.


\paragraph{Rapportens/Arbetets ändamål}
Syftet med detta kandidatarbete var att studera rörelser orsakade av passiv transport i celler för att utifrån denna studie förhoppningsvis kunna ge en klarare bild av cytoplasmans natur. Detta ska framför allt ske genom att försöka ta fram en teoretisk förklaringsmodell för hur partiklar och filament (proteintrådar) rör sig genom cytoplasman. 
Denna modell är lämpligen en mikroskopisk beskrivning som ger upphov till en stokastisk modell. 
%Ur en stokastisk modell kan sedan en makroskopisk, statistisk beskrivning uppnås och det är med denna statistiska beskrivning som modellen kan jämföras med data. 

Den undersökta datan kommer från Max Planck Institutet i Dresden och utgörs av positionen för fluorescerande partiklar i genmodifierade jästceller. Partiklarna varierar i storlek, omkring 10--100\,nm, vilket speglas av att de har olika stark ljusintensitet i mikroskopet. 

Vidare har även data från strängars rörelse i vätskor analyserats. Tanken är att 
\todo{Jag tror inte att vi kommer att klara av att göra det.}
generalisera modellen för enskilda partiklars rörelse till att även kunna tillämpas för strängars rörelse.
%uppmätta under jämna tidsintervall


\paragraph{Vad vi gjort}
\todo{Hör ens detta till inledningen. Hur som helst inga författarinträden.}
Första steget i detta kandidatarbete inbegrep att jämföra observationer för partiklar med motsvarande förutsägelser från teorin om Brownsk rörelse. Då det är känt att diffusionen i cytoplasman påverkas av bland annat motorprotein och andra företeelser och man funnit avvikelser från klassisk Brownsk rörelse \cite{Gou_etal2014} jämfördes dessa redan funna avvikelser med den tillgängliga datan. I denna analys jämfördes även skillnader mellan cellens olika stadier av metabol aktivitet. Härefter erhölls nu en grund varifrån man kunde försöka hitta förklaringar till betendet.

Men för att över huvud taget kunna analysera datan behövdes osäkerheten i mätningarna uppskattas, vilket inte är helt problemfritt då till exempel Brownsk rörelse i sig själv är en sorts brus. Brus brukar i vanliga fall hanteras genom att undersöka någon sorts medelvärde. I det här fallet kom datan från flera olika partiklar vilket gjorde att en direkt jämförelse av en undersökt parameter inte kunde göras; istället söktes först ett samband mellan storleken på partiklarna och den parameter man sökte.

I datan fanns utöver position även en partikels intensitet i mikroskopet. Intensiteten berodde med största sannolikhet på partikelns storlek; Dock var det exakta sambandet inte helt klart vilket ger ännu en svårighet i hur datan ska analyseras. För de små partiklarna tordes intensiteten bero på volymen medan den för de större partiklarna mer borde gå mot att bero av arean. Detta då intensiteten är proportionell mot antalet ljusemitterande ämnen på partikeln som kameran ''ser''. För stora partiklar kan en del av dessa lysande ämnen döljas av andra så att kameran bara ser ljuset från den sida den är riktad mot. 
%Troligen går det dock att från intensiteterna kunna jämföra olika partiklar och på så sätt ändå kunna utnyttja den i jämförelser mellan olika partiklar. 

För att testa dessa modellers riktighet för den mätdata som finns tillgänglig jämfördes modellernas förutsägelser vad gäller bland annat autokorrelationsfunktioner samt ''mean square displacement'' mot given data. Avvikelse mellan dem ledde till att modellen antingen kunde 
\todo{Kanske lite väl starkt uttalande?}
avfärdas eller moduleras.


\paragraph{Avgränsningar}\todo{Avsnittet behöver uppdateras (i bl.a. tempus).}
Att avbilda filament innebär stora svårigheter vilket har försvårat analysen av dess egenskaper. I dagsläget finns ingen fullständig modell som beskriver filamentens rörelse. Den teoretiska modellen som tas fram under det här kandidatarbetet kommer rimligtvis tas fram under vissa antaganden som begränsar dess användningsområde. Exempelvis kan variationer som uppstår vid betraktande av rörelser under olika tidsskalor komma att leda till svårigheter, bland annat i att finna en teoretisk modell som korrekt beskriver rörelsen oberoende av tidsskala. Således anses det mer rimligt att modellera rörelsen under antagandet att modellen i första hand beskriver rörelser för en viss tidsskala, förslagsvis observationer som varar i intervallet $\unit[10]{ms}$ till $\unit[10]{s}$ vilket speglar den data som hittills analyserats av gruppen.


%%%%%%%%%%%%%%%%%Från tidigare

%Från den observerade datan ska sedan en modell försöka konstrueras. Denna modell bör rimligtvis ta sin grund i frågorna:
%\begin{itemize}
%    \item Vad skulle en eventuell avvikelse från Brownsk rörelse kunna bero på?
    %\item Är det den totala effekten av många motorprotein som arbetar oberoende av varandra som stör den annars kanske ideala slumpvandringen? 
    %\item Beter sig cytoplasman olika vid olika tillfällen exempelvis beroende på vilken partikel den interagerar med? Ska cytoplasman betraktas som en fluid eller ett elastiskt fast ämne?
    %\item Vad är det för egenskaper hos aktiva celler och celler i dvala som svarar mot de observerade skillnaderna mellan dem?
%\end{itemize}



%Bara en liten kodsnutt som behövs när man kompilerar lokalt
%%% Local Variables: 
%%% mode: latex
%%% TeX-master: "main.tex"
%%% End: 