\chapter{Inledning}


%\section{Bakgrund, syfte och begränsningar}

%\paragraph{Bakgrund}
Transport inuti celler är en av grundstenarna som behövs för att cellen ska kunna verka. Exempel på en livsnödvändig intercellulär transport är hur ATP, molekylen som driver \emph{alla} biologiska processer, ska kunna ta sig från mitokondrien till alla delar av cellen. Detta är ett fall med passiv transport, där molekylerna eller partiklarna förflyttas genom att slumpvis diffundera genom cellen. Det är därför av hög vikt att kunna förstå dessa processer inuti cellerna.\todo{Man borde säga något om strängar här också.}

Partiklars rörelse i cytoplasman kan vid en första anblick tänkas kunna beskrivas med klassisk Brownsk rörelse. Studier inom området \cite{Gou_etal2014} har dock visat på avvikelser från denna teori och partiklarna verkar istället uppvisar ett sub-diffusivt beteende inuti cellen. Ytterligare skillnad i resultat kring partikelrörelser har tidigare observerats om cellen befinner sig i dvala eller i sitt normalt metabola tillstånd. En alltäckande teori för vad som kan förklara dessa observationer finns i dagsläget inte och ämnet utgör därför ett aktuellt forskningsområde.
Rörelsens stokastiska natur och cellens avancerade inre struktur ligger troligtvis bakom svårigheten man hittills stött på när man sökt en förklaringsmodell till rörelsen. Ett flertal modeller finns dock som beskriver delar av de observerade egenskaperna, bland annat ''fractional Brownian motion'' (fBm) och ''Continous Time Random Walk'' (CTRW) för partikelrrelse och ''Worm Like Chain'' (WLC) för strängrörelse som alla presenteras senare i detta arbete.

%\paragraph{Syfte} 
Syftet med denna rapport är att studera rörelser orsakade av passiv transport i celler för att utifrån denna studie förhoppningsvis kunna ge en klarare bild av cytoplasmans natur. Detta har framför allt gjorts genom att jämföra vedertagna teoretiska förklaringsmodeller med den givna datan, både för rörelsen hos partiklar genom cytoplasman och filaments (proteintrådar) rörelse i en vätska. %Filamenten befann sig alltså inte inuti en levande cell men studien av dessa borde ändå kunna ge en någorlunda bra bild av hur en sträng i cytoplasman skulle bete sig.


%\paragraph{Avgränsningar} %från planeringsrapporten
Att avbilda små partiklar och filament innebär stora svårigheter vilket har försvårat analysen av deras egenskaper. I dagsläget finns heller ingen fullständig modell som beskriver partiklars och filamentens rörelse. %Den teoretiska modellen som undersöks i det här kandidatarbetet behöver verka under vissa antaganden som begränsar dess användningsområde. 
Exempelvis kan variationer som uppstår vid betraktande av rörelser under olika tidsskalor komma att leda till svårigheter.%, bland annat i att finna en teoretisk modell som korrekt beskriver rörelsen oberoende av tidsskala. 
Således anses det mer rimligt att modellera rörelsen under antagandet att modellen i första hand beskriver rörelser för en viss tidsskala.%, förslagsvis observationer som varar i intervallet $\unit[10]{ms}$ till $\unit[10]{s}$ vilket speglar den data som analyseras i detta arbete.

Det finns idag olika teorier om vad som påverkar partiklars och filaments rörelse i cytoplasman. Vissa försöker beskriva vad som sker i celler med aktiv transport medan andra lägger mer fokus på den passiva transporten inom cellen. Då jästceller endast har passiv transport mellan celldelning.%, och datan inhämtats under mellan fas, kommer detta arbete att fokusera på just passiv transport, det vill säga diffusion av partiklar. 



\section{Datainsamling}
Datan som behandlats i detta arbete har inte samlats in under detta arbetes gång utan tillhandahölls från andra källor. Hur denna data där samlats in och vad den beskriver presenteras mer utförligt nedan.

\subsubsection{Datan för partikelrörelse i celler}
Datan som studerats för partikelrörelser i celler kommer från Max Planck Institutet i Dresden och utgörs av mätningar av positionen för fluorescerande partiklar i jästceller. Jästcellerna hade genmodifierats till att producera fluorescerande protein som lätt bildar kluster. 
Dessa kluster brukar vara av storleksordning 10--100\,nm, vilket kan jämföras med själva cellernas storlek på omkring 1\,\micro{m}.

Data för ett hundratal partiklar från olika jästceller ingick i mätserien, både för aktiva celler och celler som försatts i dvala med sänkt metabol aktivitet. Mätningen genomfördes med 100 bilder per sekund.


\subsubsection{Datan för strängrörelse i vätska}

Datan som analyserats för strängrörelse i vätska kommer från \todo{Var kommer datan från? Fråga Daniel?}... och består av filmer av aktinfilament som tillåts röra sig i en vätska. Dessa strängar hade en längd kring 10--30\,\micro{m} och befann sig i kanaler av olika bredd. Datan hade redan behandlats något så att strängens läge gav av en uppsättning vita pixlar mot en svart bakgrund.

Mätningar hade utförts på två typer av strängar: fria strängar i breda kanaler och inneslutna strängar i smala skåror. Det fanns två filmer för vardera strängtyp. Alla fyra hade filmats med 10 bilder per sekund. Rörelsen utfördes till största del i två dimensioner då skårornas djup var litet i förhållande till skårornas och filamentens bredd.





%\section{Inspiration från planeringsrapporten}

%Fördjupade studier av partikelrörelse i cellen skulle till exempel kunna leda till mer effektiva läkemedel. Vet man hur transporten inom cellen sker underlättar det arbetet med att ta fram specialdesignad medicin.


%\paragraph{Rapportens/Arbetets ändamål}
%Ur en stokastisk modell kan sedan en makroskopisk, statistisk beskrivning uppnås och det är med denna statistiska beskrivning som modellen kan jämföras med data. 

%\paragraph{Vad vi gjort}


%Men för att över huvud taget kunna analysera datan behövdes osäkerheten i mätningarna uppskattas, vilket inte är helt problemfritt då till exempel Brownsk rörelse i sig själv är en sorts brus. Brus brukar i vanliga fall hanteras genom att undersöka någon sorts medelvärde. I det här fallet kom datan från flera olika partiklar vilket gjorde att en direkt jämförelse av en undersökt parameter inte kunde göras; istället söktes först ett samband mellan storleken på partiklarna och den parameter man sökte.

%I datan fanns utöver position även en partikels intensitet i mikroskopet. Intensiteten berodde med största sannolikhet på partikelns storlek; Dock var det exakta sambandet inte helt klart vilket ger ännu en svårighet i hur datan ska analyseras. För de små partiklarna tordes intensiteten bero på volymen medan den för de större partiklarna mer borde gå mot att bero av arean. Detta då intensiteten är proportionell mot antalet ljusemitterande ämnen på partikeln som kameran ''ser''. För stora partiklar kan en del av dessa lysande ämnen döljas av andra så att kameran bara ser ljuset från den sida den är riktad mot. 
%Troligen går det dock att från intensiteterna kunna jämföra olika partiklar och på så sätt ändå kunna utnyttja den i jämförelser mellan olika partiklar. 


%Bara en liten kodsnutt som behövs när man kompilerar lokalt
%%% Local Variables: 
%%% mode: latex
%%% TeX-master: "main.tex"
%%% End: 