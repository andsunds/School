\documentclass[11pt,a4paper, 
english, swedish %% Make sure to put the main language last!
]{article}
\pdfoutput=1

%% Andréas's custom package 
%% (Will work for most purposes, but is mainly focused on physics.)
\usepackage{../custom_as}

%% Figures can now be put in a folder: 
\graphicspath{ {figurer/} %{some_folder_name/}
}

%% If you want to change the margins for just the captions
\usepackage[margin=10 pt]{caption}

%% To add todo-notes in the pdf
\usepackage[%disable  %%this will hide all notes
]{todonotes} 

%% Change the margin in the documents
\usepackage[
%            top    = 3cm,              %% top margin
%            bottom = 3cm,              %% bottom margin
%            left   = 3cm, right  = 3cm %% left and right margins
]{geometry}


\newcommand{\lD}{\ensuremath{\lambda_{\text{D}}}}


%% If you want to chage the formating of the section headers
\renewcommand{\thesubsection}{\arabic{section}.\Alph{subsection})}



%%%%%%%%%%%%%%%%%%%%%%%%%%%%%%%%%%%%%%%%%%%%%%%%%%%%%%%%%%%%%%%%%%%%%%
\begin{document}%% v v v v v v v v v v v v v v v v v v v v v v v v v v
%%%%%%%%%%%%%%%%%%%%%%%%%%%%%%%%%%%%%%%%%%%%%%%%%%%%%%%%%%%%%%%%%%%%%%


%%%%%%%%%%%%%%%%%%%% vvv Internal title page vvv %%%%%%%%%%%%%%%%%%%%%
\title{Assignment 1 \\
{\Large Plasma Physics -- RRY085}}
\author{Andréas Sundström}
\date{2017-09-08}

\maketitle

%%%%%%%%%%%%%%%%%%%% ^^^ Internal title page ^^^ %%%%%%%%%%%%%%%%%%%%%
%% If you want a list of all todos
%\todolist



\section{Debye shielding}
To find the potential from a point charge in a plasma, we have to
solve the Poisson equation inside the plasma:
\begin{equation}\label{eq1:start}
-\laplacian \varphi = 
\frac{1}{\varepsilon_0}\qty(q_\ii n_\ii+ q_\ee n_\ee )
+ \frac{Q}{\varepsilon_0} \delta(r),
\end{equation}
where the point charge $Q$ has been set in the origin. The problem is
assumed to be isotropic, so the Laplacian can be written as
\begin{equation}
\laplacian\varphi = \frac{1}{r}\dv[2]{u}{r},
\end{equation}
where $u(r)=r\varphi(r)$.
The next step is to assume that the particles are Boltzmann
distributed ($k_\text{B}=1$):
\begin{equation}
n=n_0\exp(-\frac{q\varphi}{T})
= n_0\qty[1-\frac{q\varphi}{T} 
+\order{\qty(\frac{q\varphi}{T})^2}].
\end{equation}
The charge of the electron is $q_\ee=-e$, and we also assume that the
charge of the ions is $q_\ii=+e$. 
We can now write \eqref{eq1:start}, for $r>0$ ($\delta(r)=0$), as
\begin{equation}
\dv[2]{u}{r} =
-\frac{n_0e^2r}{\varepsilon_0}\qty[\qty(\frac{1}{T_\ii}+\frac{1}{T_\ee})\varphi
+\order{\qty(\frac{q\varphi}{T_\ii})^2} 
+\order{\qty(\frac{q\varphi}{T_\ee})^2}].
\end{equation}
Dropping higher the order terms and factoring in the $r$ into the
bracket yields
\begin{equation}\label{eq1:ODE}
\dv[2]{u}{r} =-\frac{n_0e^2}{\varepsilon_0}
\qty(\frac{1}{T_\ii}+\frac{1}{T_\ee})u
=-\lD^{-2}u.
\end{equation}
where we have defined
\begin{equation}\label{eq1:Debye}
\lD := \qty[\frac{n_0e^2}{\varepsilon_0}
\qty(\frac{1}{T_\ii}+\frac{1}{T_\ee})]^{-1/2}
=\sqrt{\frac{\varepsilon_0 k_\text{B}T_\ii T_\ee}
{n_0e^2(T_\ii + T_\ee)}}.
\end{equation}
(The Boltzmann constant has also been inserted in the last step.)
Now \eqref{eq1:ODE} has the general solution
%\begin{equation}
$u(r) = A\ee^{-r/\lD} + B\ee^{+r/\lD}$
%\end{equation}
or 
\begin{equation}
\varphi(r) = \frac{A}{r}\ee^{-r/\lD} + \frac{B}{r}\ee^{+r/\lD},
\end{equation}
but we also have the boundary conditions
\begin{equation}
\begin{cases}
\varphi(r)\to0\qc &\text{ as } r\to\infty\\
\varphi(r)\sim \varphi_{\text{free}}(r) 
= \frac{Q}{4\pi\varepsilon_0r} \qc&\text{ as } r\to0.
\end{cases}
\end{equation}
The first one forces $B=0$ while the second yields 
$A=Q/(4\pi\varepsilon_0)$, which means that we have 
\begin{equation}
\varphi(r)= \frac{Q}{4\pi\varepsilon_0r}\ee^{-r/\lD},
\end{equation}
with $\lD$ as defined in \eqref{eq1:Debye}.

\section{Debye sphere}
In this problem we are concerned about collective behavior and how it
can be used to show that the Debye sphere has to contain many
particles ($n_\ee\lD^3\gg1$). 
The collective behavior, i.e. that long-range effects dominate,
results in the fact that most particle collisions results in small
scattering angles.

We begin by taking a look at the length scale $L$ on which a large
scattering angle, typically $90^\circ$, is to be expected. We assume
that the scattered particles have ``thermal'' kinetic energy:
\begin{equation}
\frac{p_\text{th}^2}{m} \sim T,
\end{equation}
where $T$ is the temperature and $p_\text{th}$ is the momentum of
those particles. Next, for large scattering angles, the change
in momentum is roughly the same as the initial momentum of the
particle:
\begin{equation}
\delta p \sim p_\text{th}.
\end{equation}
Furthermore,  we also assume that the interaction is mediated via a
regular Coulomb force acting on the said length scale:
\begin{equation}
F\sim\frac{e^2}{4\pi\varepsilon_0L^2}.
\end{equation}
And finally, we say that the interaction time is roughly 
$\tau \sim L/v_\text{th} = Lm/p_\text{th}$. 

We can now write down the two expressions for the change in momentum
\begin{equation}
p_\text{th} \sim \delta p \sim 
\frac{e^2}{4\pi\varepsilon_0L^2}\,\frac{Lm}{p_\text{th}}
\quad\Longleftrightarrow\quad
L\sim \frac{e^2}{4\pi\varepsilon_0}\,\frac{m}{p_\text{th}^2}
\sim \frac{e^2T}{4\pi\varepsilon_0}.
\end{equation}
In terms of Debye lengths,
\begin{equation}
\lD = \sqrt{\frac{\varepsilon_0 T}{n_\ee e^2}},
\end{equation}
we can express this large scattering angle length scale as
\begin{equation}
L\sim\frac{1}{n_\ee\lD^2}.
\end{equation}

For collective behavior, we must then require that most interactions
only result in \emph{small} angle scattering. This means that the mean
distance between e.g. electrons $\bar{d}$ has to be much \emph{larger}
that $L$. In terms of electron density, we can write
\begin{equation}
n_\ee^{-1/3}\sim \bar{d} \gg L
\end{equation}
or
\begin{equation}
n_\ee\sim \bar{d}^{-3}\ll L^{-3}
\sim\qty(n_\ee\lD^2)^3
\quad\Longleftrightarrow\quad
n_\ee\lD^3\gg1,
\end{equation}
which is just what we wanted to show. 

\section{Polarized atoms}
In this problem we want to study collisions between an electron and a
neutral atom. The collisions are made possible through polarization of
the neutral atom.

Consider an electron on position $r_0\vu{x}$, then the electric field,
at the origin, due to this electron, will be
\begin{equation}
\vb*E_\ee = \frac{e}{4\pi\varepsilon_0 r_0^2}\vu{x}.
\end{equation}
On the other hand, the dipole field at the electron due to the
polarized atom is
\begin{equation}
\vb*E_\text{d}=\frac{1}{4\pi\varepsilon_0}\,
\frac{3(\vb*p\vdot\vu{x})\vu{x}-\vb*p}{r_0^3},
\end{equation}
where $\vb*p$ is the polarization of the atom.
Assuming that the polarization of the atom is given by
\begin{equation}
\vb*p = \alpha\vb*E_\ee
=\frac{\alpha e}{4\pi\varepsilon_0 r_0^2}\vu{x}
\end{equation}
the force on the electron will then be
\begin{equation}
\vb*F=-e\vb*E_\text{d}=-\frac{e \vu{x}}{4\pi\varepsilon_0}\,
\frac{2\alpha e/(4\pi\varepsilon_0r_0^2)}{r_0^3}
=-\frac{2\alpha e^2 \vu{x}}{(4\pi\varepsilon_0)^2r_0^5}.
\end{equation}
This is much weaker interaction ($r_0^{-5}$) on long range, compared
to the regular Coulomb interaction ($r_0^{-2}$).


%%%%%%%%%%%%%%%%%%%%%%%%%%%%%%%%%%%%%%%%%%%%%%%%%%%%%%%%%%%%%%%%%%%%%%
\end{document}%% ^ ^ ^ ^ ^ ^ ^ ^ ^ ^ ^ ^ ^ ^ ^ ^ ^ ^ ^ ^ ^ ^ ^ ^ ^ ^ ^
%%%%%%%%%%%%%%%%%%%%%%%%%%%%%%%%%%%%%%%%%%%%%%%%%%%%%%%%%%%%%%%%%%%%%%
%  LocalWords:  Debye
