
Bakgrund
\begin{itemize}
    \item WLC model, ev annan modell (Måns)
    \item Protein-filament
    \item Korrelationer, tangent, tid, rum
    \item Egenmoder
\end{itemize}

Resultat som kan tas med
\begin{itemize}
    \item Uppdelningen i egenmoder, olika relaxationstid
    \item Dispersionsrelation?
    \item Skillnad mellan confined och unconfined
    
\end{itemize}


\section{Teori}

\subsection{Proteinfilament}

Aktinfilament består av glubulära aktinprotein formade som bollar vilka kopplats ihop till en lång kedja.

\subsection{Modeller för strängrörelser}

\subsubsection{Worm Like Chain-model}

\subsubsection{Månsmodell}


\subsection{Egenomder}


\subsection{Lösning arv stokastisk differentialekvation}



\section{Tillhandahållen data}

Datan som analyserats i denna andra del av arbetet kommer från \todo{Var kommer datan från?}... och består av filmer av aktinfilament som tilåts röra sig i en vätska. Dessa strängar hade en storlek kring 10--30\,\micro{m} och befann sig i kanaler av olika bredd. Datan hade redan behandlats något så att strängens läge kunde beskrivas genom lägena för ett antal punker på strängen vilka angivna som pixlar i ett koordinatsystem. Genom att bildförstoringen och kamerans pixelstorlek var känd kunde de olika strängarnas längd beräknas. Mätningarna hade utförst på två ''fria'' strängar (strängar i breda skåror) och två strängar instängda i smala skåror, alla fyra filmade med 10 bilder per sekund. Rörelsen utfördes till största del i två dimensioner då skårornas djup var litet i förhållande till skårornas och filamentens bredd.

\section{Resultat}



\section{Diskussion och slutsats}





%Bara en liten kodsnutt som behövs när man kompilerar lokalt
%%% Local Variables: 
%%% mode: latex
%%% TeX-master: "main.tex"
%%% End: 