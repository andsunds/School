\documentclass[11pt,a4paper, notitlepage, german, english, swedish]{report}
\pdfoutput=1

\usepackage{custom_as}

\graphicspath{ {bilder/} } %Gör så att man kan lägga alla bilder i en egen katalog

%%Drar in tabell och figurtexter
\usepackage[margin=10 pt]{caption}
%%För att lägga in 'att göra'-noteringar i texten
\usepackage{todonotes} %\todo{...}

%%För att själv bestämma marginalerna. 
\usepackage[
%            top    = 3cm,
%            bottom = 3cm,
%            left   = 3cm, right  = 3cm
]{geometry}

%egna kommandon för statistik
\newcommand{\VAR}[1]{\text{var}\left[#1\right]}
\newcommand{\COV}[2]{\text{cov}\left[#1,\;#2\right]}
\newcommand{\CORR}[2]{\text{corr}\left[#1,\;#2\right]}



\begin{document}

%Alla inledande sidor finns i 'titlepages.tex'.
\renewcommand{\thefootnote}{\fnsymbol{footnote}}

%kortkommandon för mailaddresserna
\newcommand{\andsunds}{andsunds@student.chalmers.se}
\newcommand{\rigon}{rigon@student.chalmers.se}



\pagenumbering{roman} %%Romersk sidnumrering i början
\begin{titlepage}
\newgeometry{top=3cm, bottom=2cm}

\newcommand{\HRule}{\rule{\linewidth}{0.5mm}} % Defines a new command for the horizontal lines, change thickness here

\center % Center everything on the page
 
%------------------------------------------------------------------------------------
%	HEADING SECTIONS
%------------------------------------------------------------------------------------

\textsc{\huge Chalmers tekniska högskola}\\[1.5cm] % Name of university/college
\textsc{\Large Rapport, Experimentell fysik 2}\\[0.2cm] % Major heading such as course name
\textsc{\large Termodynamik -- Uppgift 3 }\\[0.5cm] % Minor heading such as course title

%------------------------------------------------------------------------------------
%	TITLE SECTION
%------------------------------------------------------------------------------------

\HRule \\[0.4cm]
{ \LARGE \bfseries 
Studier av kvicksilveratomens atomära emissionsspektra samt absorptionsspektra av laserfärgämnena Rhodamin B och Kumarin 307
}\\[0.4cm] % Title of  document
\HRule \\[1.5cm]
 
%------------------------------------------------------------------------------------
%	AUTHOR SECTION
%------------------------------------------------------------------------------------

\begin{minipage}{0.4\textwidth}
\begin{flushleft} \large
\emph{Författare:}\\
Andréas Sundström\footnotemark{} \\
Rigon Demisai\footnotemark{} 
\end{flushleft}
\end{minipage}
~
\begin{minipage}{0.4\textwidth}
\begin{flushright} \large
\emph{Labassistent:} \\
Martin Wersäll
\end{flushright}
\end{minipage}\\[3cm]

\setcounter{footnote}{0}
\stepcounter{footnote}
  \footnotetext{\href{mailto:\andsunds}{\texttt{\andsunds}}}
\stepcounter{footnote}
  \footnotetext{\href{mailto:\rigon}{\texttt{\rigon}}}



%------------------------------------------------------------------------------------
%	DATE SECTION
%------------------------------------------------------------------------------------
% Följer ISO-standarden för tidsintervall:
% https://en.wikipedia.org/wiki/ISO_8601#Time_intervals
% "Double hyphen" också ok istället för '/'. -- i LaTeX är dock lite på gränsen
{ \large
\begin{tabular}{rc}
    Laboration utförd: & 2015-12-11/15 \\[0.1cm]
    Rapport inlämnad: & \today
\end{tabular}\\[1cm]
}

%------------------------------------------------------------------------------------
%	LOGO SECTION
%------------------------------------------------------------------------------------

\includegraphics[height=5cm]{logo.pdf} % Include a department/university logo
 
%------------------------------------------------------------------------------------

\vfill % Fill the rest of the page with whitespace

\end{titlepage}
\restoregeometry


\setcounter{page}{2}%detta är ANDRA (2) sidan

\renewcommand{\abstractname}{Sammandrag}
\begin{abstract}
Vi har utfört en spektroskopisk studie av kvicksilveratomens atomära emissionsspektrum ur vilket vi kartlagt atomens energinivåer baserat på vår spektrometri. Vi har också studerat absorption i lösningar av två laserfärgämnen vid namn Rhodamin B och Kumarin 307. Mätningarna har utförts med en Spex 270M spektrometer och datainsamlingen har gjorts i LabView. Kvicksilvrets emissionsspektra är taget i intervallet 365 till 984 nm, där vi detekterat totalt 23 signifikanta emissionstoppar. Detta jämförs med NIST data där vi har 4 av 5 överlappningar med NIST ''persistent lines'' och 7 av 21 överlappningar med NIST ''strong lines''. Absorptionsspektra för Rhodamin B och Kumarin 307 visar breda absorptionsband vilket är kännetecknande för flourescerande ämnen som består av stora organiska molekyler.
\end{abstract}

\renewcommand{\abstractname}{Abstract}
\begin{abstract}

We have conducted a spectroscopic study of the emission spectrum of Mercury atoms and derived an energy level diagram based on these measurements. We have also studied the absorption spectrum of laser dye solutions of the compounds Rhodamine B and Coumarine 307. The measurements have been taken with a Spex 270M spectrometer and have been processed and recorded in LabView. The emission spectrum of Mercury has been recorded within the range of 365 to 984 nm, where we have detected a total of 23 significant emission peaks. This is contrasted with NIST data where we have 4 out of 5 overlaps with NIST ''persistent lines'' and 7 out of 21 overlaps with NIST ''strong lines''. The absorption spectrum for Rhodamine B and Coumarine 307 show broad absorption bands which are characteristic of flourescent compounds which consist of large organic molecules.

\end{abstract}

\clearpage
\renewcommand{\contentsname}{Innehållsförteckning}
\tableofcontents

\clearpage
\pagenumbering{arabic}
\setcounter{page}{1}

\renewcommand{\thefootnote}{\arabic{footnote}}
\setcounter{footnote}{0}




%%%%%%%%%%%%%%%%%%%%%% Här börjar huvudtexten %%%%%%%%%%%%%%%%%%%%%%
%\part{Inledning}

\chapter{Inledning}
\chapter{Inledning}


%\section{Bakgrund, syfte och begränsningar}

%\paragraph{Bakgrund}
Transport inuti celler är en av grundstenarna som behövs för att cellen ska kunna verka. Exempel på en livsnödvändig intercellulär transport är hur ATP, molekylen som driver \emph{alla} biologiska processer, ska kunna ta sig från mitokondrien till alla delar av cellen. Detta är ett fall med passiv transport, där molekylerna eller partiklarna förflyttas genom att slumpvis diffundera genom cellen. Det är därför av hög vikt att kunna förstå dessa processer inuti cellerna.\todo{Man borde säga något om strängar här också.}

Partiklars rörelse i cytoplasman kan vid en första anblick tänkas kunna beskrivas med klassisk Brownsk rörelse. Studier inom området \cite{Gou_etal2014} har dock visat på avvikelser från denna teori och partiklarna verkar istället uppvisar ett sub-diffusivt beteende inuti cellen. Ytterligare skillnad i resultat kring partikelrörelser har tidigare observerats om cellen befinner sig i dvala eller i sitt normalt metabola tillstånd. En alltäckande teori för vad som kan förklara dessa observationer finns i dagsläget inte och ämnet utgör därför ett aktuellt forskningsområde.
Rörelsens stokastiska natur och cellens avancerade inre struktur ligger troligtvis bakom svårigheten man hittills stött på när man sökt en förklaringsmodell till rörelsen. Ett flertal modeller finns dock som beskriver delar av de observerade egenskaperna, bland annat ''fractional Brownian motion'' (fBm) och ''Continous Time Random Walk'' (CTRW) för partikelrrelse och ''Worm Like Chain'' (WLC) för strängrörelse som alla presenteras senare i detta arbete.

%\paragraph{Syfte} 
Syftet med denna rapport är att studera rörelser orsakade av passiv transport i celler för att utifrån denna studie förhoppningsvis kunna ge en klarare bild av cytoplasmans natur. Detta har framför allt gjorts genom att jämföra vedertagna teoretiska förklaringsmodeller med den givna datan, både för rörelsen hos partiklar genom cytoplasman och filaments (proteintrådar) rörelse i en vätska. %Filamenten befann sig alltså inte inuti en levande cell men studien av dessa borde ändå kunna ge en någorlunda bra bild av hur en sträng i cytoplasman skulle bete sig.


%\paragraph{Avgränsningar} %från planeringsrapporten
Att avbilda små partiklar och filament innebär stora svårigheter vilket har försvårat analysen av deras egenskaper. I dagsläget finns heller ingen fullständig modell som beskriver partiklars och filamentens rörelse. %Den teoretiska modellen som undersöks i det här kandidatarbetet behöver verka under vissa antaganden som begränsar dess användningsområde. 
Exempelvis kan variationer som uppstår vid betraktande av rörelser under olika tidsskalor komma att leda till svårigheter.%, bland annat i att finna en teoretisk modell som korrekt beskriver rörelsen oberoende av tidsskala. 
Således anses det mer rimligt att modellera rörelsen under antagandet att modellen i första hand beskriver rörelser för en viss tidsskala.%, förslagsvis observationer som varar i intervallet $\unit[10]{ms}$ till $\unit[10]{s}$ vilket speglar den data som analyseras i detta arbete.

Det finns idag olika teorier om vad som påverkar partiklars och filaments rörelse i cytoplasman. Vissa försöker beskriva vad som sker i celler med aktiv transport medan andra lägger mer fokus på den passiva transporten inom cellen. Då jästceller endast har passiv transport mellan celldelning.%, och datan inhämtats under mellan fas, kommer detta arbete att fokusera på just passiv transport, det vill säga diffusion av partiklar. 



\section{Datainsamling}
Datan som behandlats i detta arbete har inte samlats in under detta arbetes gång utan tillhandahölls från andra källor. Hur denna data där samlats in och vad den beskriver presenteras mer utförligt nedan.

\subsubsection{Datan för partikelrörelse i celler}
Datan som studerats för partikelrörelser i celler kommer från Max Planck Institutet i Dresden och utgörs av mätningar av positionen för fluorescerande partiklar i jästceller. Jästcellerna hade genmodifierats till att producera fluorescerande protein som lätt bildar kluster. 
Dessa kluster brukar vara av storleksordning 10--100\,nm, vilket kan jämföras med själva cellernas storlek på omkring 1\,\micro{m}.

Data för ett hundratal partiklar från olika jästceller ingick i mätserien, både för aktiva celler och celler som försatts i dvala med sänkt metabol aktivitet. Mätningen genomfördes med 100 bilder per sekund.


\subsubsection{Datan för strängrörelse i vätska}

Datan som analyserats för strängrörelse i vätska kommer från \todo{Var kommer datan från? Fråga Daniel?}... och består av filmer av aktinfilament som tillåts röra sig i en vätska. Dessa strängar hade en längd kring 10--30\,\micro{m} och befann sig i kanaler av olika bredd. Datan hade redan behandlats något så att strängens läge gav av en uppsättning vita pixlar mot en svart bakgrund.

Mätningar hade utförts på två typer av strängar: fria strängar i breda kanaler och inneslutna strängar i smala skåror. Det fanns två filmer för vardera strängtyp. Alla fyra hade filmats med 10 bilder per sekund. Rörelsen utfördes till största del i två dimensioner då skårornas djup var litet i förhållande till skårornas och filamentens bredd.





%\section{Inspiration från planeringsrapporten}

%Fördjupade studier av partikelrörelse i cellen skulle till exempel kunna leda till mer effektiva läkemedel. Vet man hur transporten inom cellen sker underlättar det arbetet med att ta fram specialdesignad medicin.


%\paragraph{Rapportens/Arbetets ändamål}
%Ur en stokastisk modell kan sedan en makroskopisk, statistisk beskrivning uppnås och det är med denna statistiska beskrivning som modellen kan jämföras med data. 

%\paragraph{Vad vi gjort}


%Men för att över huvud taget kunna analysera datan behövdes osäkerheten i mätningarna uppskattas, vilket inte är helt problemfritt då till exempel Brownsk rörelse i sig själv är en sorts brus. Brus brukar i vanliga fall hanteras genom att undersöka någon sorts medelvärde. I det här fallet kom datan från flera olika partiklar vilket gjorde att en direkt jämförelse av en undersökt parameter inte kunde göras; istället söktes först ett samband mellan storleken på partiklarna och den parameter man sökte.

%I datan fanns utöver position även en partikels intensitet i mikroskopet. Intensiteten berodde med största sannolikhet på partikelns storlek; Dock var det exakta sambandet inte helt klart vilket ger ännu en svårighet i hur datan ska analyseras. För de små partiklarna tordes intensiteten bero på volymen medan den för de större partiklarna mer borde gå mot att bero av arean. Detta då intensiteten är proportionell mot antalet ljusemitterande ämnen på partikeln som kameran ''ser''. För stora partiklar kan en del av dessa lysande ämnen döljas av andra så att kameran bara ser ljuset från den sida den är riktad mot. 
%Troligen går det dock att från intensiteterna kunna jämföra olika partiklar och på så sätt ändå kunna utnyttja den i jämförelser mellan olika partiklar. 


%Bara en liten kodsnutt som behövs när man kompilerar lokalt
%%% Local Variables: 
%%% mode: latex
%%% TeX-master: "main.tex"
%%% End: 



%\part{Huvuddel}

\chapter{Stokastiska processer och Brownsk rörelse}
%stokastiska processer och Brownsk rörelse

För att beskriva beskriva de system som undersöks i det här arbetet
behöver man ta till stokastisk analys. I vanliga fall brukar
det räcka med ordinära eller partiella differential\-ekvationer (ODE:er
eller PDE:er) för att beskriva fysikaliska stystem. Men för exempelvis
små objekt kommer termiska fluktuationer att påverka deras
betenden. Dessa termiska fluktuationer kan anses vara helt
slumpmässiga, varför de kan betraktas som \emph{stokastiska
  processer}. Påverkan på ett system från en stokastisk process leder
till att den styrande DE:n behöver modifieras med en stokastisk term,
det blir då en \emph{stokastisk differentialekvation} (SDE).

Ett exempel på när ett system består av så ''små objekt'' att termiska
fluktuationer behöver beaktas är i så kallad \emph{Brownsk
  rörelse}. Detta är ett fenomen där pollenkorn på en vattenyta såg ut
att vandra runt slumpmässigt. Fenomenet först beskrevs av Robert Brown
1828~\cite{Brown1828}, men förklarades först av Einstein
1905~\cite{Einstein1905}. Förklaringen går ut på att pollenkornen är
små nog för att när vattenmolekylerna krockar med med så överförs
tillräckligt med rörelsemängd för pollenkornen ska ses flytta på sig. 

\section{Stokastiska processer}
En \emph{stokastisk variabel} $X$ är ett objekt som kan anta värden
$x$ från en viss värdemängd $\Omega$. Vilka värden som antas styrs av
sannolikhetsfördelningen $P_X(x)$. I fallet med diskreta stokastiska
variabler är sannolikhetsfördelningen helt enkelt sannolikheten att
$X$ antar värdet $x$. Men i det här arbetet ligger fokus på
kontinuerliga stokastiska variabler. För dessa gäller att 
\begin{equation}
P(X\in[x, x+\dd{x}]) =P_X(x)\dd{x}
\end{equation}
för någon infinitesimal intervallbredd $\dd{x}$. 
I fortsättningen av detta arbete kommer ''stokastisk variabel'' att
avse en \emph{kontinuerlig} stokastisk variabel om inget annat anges.


Från detta kan en så kallad \emph{stokastisk process} definieras som en
samling av objekt som beror på en stokastisk variabel $X$ och en
deterministisk variabel, oftast betrakad som tiden\footnotemark{}
$t$. Speciellt brukar dessa objekt vara 
funktioner, $f_X(t)$. För ett givet värde $X=x$ blir alltså den
stokastiska processen en funktion $f_x(t)$, vilket medför att $F_X(t)$
definierar en samling av funktioner. 
\footnotetext{Att tiden väljs som deterministisk variabel är
  anledning till att det kallas stokastisk \emph{process}; man tänker
  sig att ett tidsförlopp som beror av den stokastiska variabeln
  utspelar sig. Mer generellt kan en godtycklig deterministisk
  variabel användas istället för tid.}  

\subsection{Statistiska verktyg för att undersöka stokastiska processer}
Stokastiska processer är som sagt slumpartade processer. Därmed kan
det vara svårt att avgöra processens natur utifrån enbart ett fåtal
observationer. För att kunna undersöka den stokastiska processen
behövs olika statistiska verktyg som exempelvis väntevärde, varians
och korrelation. 

För en stokastisk variabel $X$ och en dess sannolikhetsfördelning kan
man definiera dess \emph{väntevärde}
\begin{equation}
    \ev{X} = \int_{\Omega} x P(x) \id{x}.
\end{equation}
Något löst sett kan det betraktas som medelvärdet man förväntas få vid
upprepade mätningar av $X$. 
Väntevärdet går även att utvidga till att även omfatta funktioner av
den stokastiska variabeln. Man får då 
\begin{equation}
    \ev{f(X)} = \int_{\Omega} f(x) P(x) \id{x}.
\end{equation}
Speciellt i fallet med stokastiska processer blir väntevärdet 
\begin{equation}
    \ev{F_X(t)} = \int_{\Omega} F_x(t)P_X(x) \id{x}.
\end{equation}
Notera här att väntevärdet är beroende av $t$. 

Om väntevärdet är ett mått på vad man får som medelvärde, så behövs
även ett mått på hur spridda värden man kan tänkas få. För det används
\emph{variansen}, som går att skriva på några olika sätt
\begin{equation}
\sigma_X^2=\VAR{X} = \ev{\left(X-\ev{X} \right)^2} = \ev{X^2}-\ev{X}^2.
\end{equation}
Dock ger variansen, som man kan se, ett kvadratiskt mått på
avvikelser från medelvärdet. Därför kan det, exempelvis i sammanhang
där man vill jämföra spridningen i en mätserie, vara mer intressant
att betrakta \emph{standardavvikelsen} $\sigma_X$ som ges av roten ur
variansen.

Man kan på analogt sätt definiera en \emph{kovarians}
\begin{equation}
\COV{X}{Y}  = \ev{XY}-\ev{X}\ev{Y}.
\end{equation}
Kovariansen är ett mått på hur mycket två stokastiska variabler
samvarierar. 

I fallet med stokastiska processer kan det vara intressant att veta
hur korrelerade processerna är i till exempel tiden. För det används
korrelationsfunktionen 
\begin{equation}
C(t, t') = \frac{\COV{F_X(t)}{F_Y(t')}}{\sigma_{F_X}\sigma_{F_Y}}.
\end{equation}
Här har kovariansen delats med respektive standardavvikelse för att
korrelationsfuntionen ska ge ett värde mellan 
\todo{Detta fås av Cauchy-Schwarzs olikhet. Ska man säga det?}
$-1$ och $1$. 
Oftast brukar även translationssymmetri i $t$ gälla, varför man kan
ersätta de båda variablerna $t$ och $t'$ med deras skilnad:
\begin{equation}
C(\Delta t) = C(t, t+\Delta t).
\end{equation}


\subsubsection{Diskret data}

Som man kan se så bygger alla dessa 
\todo{Något annat än ''verktyg''?} 
verktyg på olika väntevärden. Man ser nu att för att kunna tillämpa
dessa statistiska metoder behövs ett stort statistiskt underlag av
många observationer. Detta eftersom väntevärdet är det medelvärde
som förväntas av en variabel efter tillräckligt måga observationer.





\subsection{Stokastiska differentialekvationer}
\todo[inline]{Gammalt}
En differentialekvation som innehåller termer med stokastiska processer
betecknas stokastiska differentialekvationer (SDE). Lösningen till en
SDE kommer representeras av en stokastisk process eftersom. Eftersom
många fysikaliska sy

Inom fysiken modellerar man ofta system med fluktuationer genom att betrakta
tidsutvecklingen av ett system via motsvarande differentialekvation
och man adderar sedan en stokastisk process för att representera
fluktuationen. 
Detta kallas Langevin formalism och motsvarande
stokastiska differentialekvation kallas systemets Langevin ekvation. Ett
illustrerande exempel av Langevin formalismen är fallet för brownsk
rörelse som beskrivs i avsnitt~\ref{sec:brown}.

Som tidigare nämnts så är den stokastiska
processen som beskriver systemets fluktuation oftast okänd, istället
antas att fluktuationen har vissa karakteristiska egenskaper. Exempel
på sådana egenskaper är att väntevärdet är $0$ enligt $\ev{F(t)} = 0$
eller att fluktuationen är okorrelerad i tiden
$\ev{F(t)F(t')}\propto\delta(t-t')$. 
Enligt 
\todo{Vaddå?}tidigare 
är lösningar till
stokastiska differentialekvationer stokastiska processer, och om
fluktuationen beskrivs med karakteristiska egenskaper och okänd
sannolikhetsfördelning kommer detta speglas i lösningen till
differentialekvationen. Således kommer inte sannolikhetsfördelningen
av lösning att kunna finnas, istället betraktar man motsvarande
karakteristiska egenskaper för lösningen som de hos
fluktuationen. 
\todo{De sista meningarna här  beskrivs nog tydligare
  med brownsk rörelse som exempel.} 






\section{Brownsk rörelse}\label{sec:brown}
\todo[inline]{Gammalt}
Hastigheten för en partikel som utför ren Brownsk rörelse styrs av
Langevinekvationen~\cite{Mazo_Brownian2002} 
\begin{equation} \label{eq:Brownian_SDE}
    M\dv{v}{t}=-\zeta v + F(t),
\end{equation}
där $M$ är partikelmassan, $\zeta$ en friktionskonstant och $F(t)$ en
fluktuerande kraft. Kraften utgör här det stokastiska bidraget till
differentialekvationen och är deltakorrelerat i tiden,
$\ev{F(t)F(t')}=\sigma^2\delta(t-t')$, 
med väntevärde 0, $\ev{F(t)}=0$. 
Det vill säga att den beter sig som vitt brus. 

Den fysikaliska tolkningen av denna stokastiska kraft är att partikeln
får små impulser från omgivande vätskepartiklar vilka kolliderar
slumpmässigt med den brownska partikeln.  Deltakorrelationen för
kraften uppkommer då impulserna modelleras som deltafunktioner i
tiden. Denna kraftterm kan vidare tolkas som derivatan av en
Wienerprocess i gränsen då kollisionerna infaller med hög frekvens. En
Wienerprocess är en tidskontinuerlig stokastisk process där varje
förändringssteg är oberoende av tidigare steg samtidigt som
\todo{Bättre ord än ökningar}ökningarna är normalfördelade med
väntevärde 0.
%Motivation att derivata av Wienerprocess s63

Lösningen till den stokastiska differentialekvationen \eqref{eq:Brownian_SDE} ges av \todo{Vill vi ändra till 2D?}
\begin{equation}
    v(t)=v(0)\ee^{-\nicefrac{\zeta t}{M}}+\frac{1}{M}\int^t_0 F(s)e^{-\nicefrac{\zeta (t-s)}{M}}ds.
\end{equation}
Detta får dock inte den stokastiska termen att försvinna och lösningen kan inte skrivas på en deterministisk form. För att ändå kunna göra några förutsägelser kan man titta på väntevärdet och korrelationen i tiden. Betrakta därför följande korrelation 
\begin{equation}
    \ev{v(t)v(t+\delta t)} = v(0)^2\ee^{-\nicefrac{2\zeta t}{M}}+\frac{1}{M^2}\ee^{-\nicefrac{\zeta (2t+\delta t)}{M}}\int_0^t\int_0^{t+\delta t}dt'dt''\ee^{\nicefrac{\zeta (t'+t'')}{M}}\ev{F(t')F(t'')}
\end{equation}
och utnyttja att $F(t)$ är deltakorrelerad i tid vilket ger 
\begin{equation}
    \ev{v(t)v(t+\delta t)} = v(0)^2\ee^{-\nicefrac{2\zeta t}{M}}+\frac{\sigma^2}{M^2}\ee^{-\nicefrac{\zeta (2t+\delta t)}{M}}\int_0^tdt'\ee^{\nicefrac{\zeta 2t}{M}}
\end{equation}
Korrelationen ovan kan nu enkelt beräknas och genom att låta $\delta t\to 0$ samt $t\to \infty$ fås följande samband
\begin{equation}
    \ev{v(t)^2} = \frac{\sigma^2}{2M\zeta}
\end{equation}

Med hjälp av detta samband samt ekvipartitionsteoremet $\frac{1}{2}M\ev{v^2}=\frac{1}{2}k_BT$, där $k_B$ är Boltzmanns konstant och $T$ är temperaturem, kan man nu relatera variansen $\sigma^2$ till fysikaliska storheter vilket ger 
\begin{equation}
    \sigma^2 = 2k_BT\zeta
\end{equation}


För $t \gg \nicefrac{\zeta}{M}$ blir $\nicefrac{\dd{v}}{\dd{t}}$-termen i ekvation \eqref{eq:Brownian_SDE} försumbar \todo{Visa detta?} och ekvationen kan då skrivas på formen
\begin{equation}
    \zeta \dv{x}{t}=F(t),
\end{equation}
där $x$ är partikelns position. Detta ger lösningen
\begin{equation}
    x(t)=x(0)+\frac{1}{\zeta} \int^t_0 F(s)ds.
\end{equation}
Utifrån denna lösning kan medelvärdet av den kvadrerade avvikelsen beräknas, kallat ''mean squared displacement'' (MSD), vilken blir 

\begin{equation}
    \ev{(x(t)-x(0))^2}=\frac{2k_BTt}{\zeta}
\end{equation}
där fluktuation-dissipationsteoremet gett att $\sigma^2=2k_BT\zeta$. \todo{ska vi visa flukt.diss.teoremet?} MSD:n kommer därmed att öka linjärt med tiden, något som enkelt kan jämföras med uppmätt data.




%Bara en liten kodsnutt som behövs när man kompilerar lokalt
%%% Local Variables: 
%%% mode: latex
%%% TeX-master: "main.tex"
%%% End: 

\chapter{Partikelrörelse i celler}
\section{Bakgrund}

Bakgrund teori om cellen
\begin{itemize}
    \item Cytoplasmans uppbyggnad och struktur
    \item Brownsk rörelse, CTRW, Fractional Brownian motion
    \item Metabola tillståndets påverkan på partikelrörelsen
\end{itemize}
Bakgrund för dataanalysen
\begin{itemize}
    \item Skillnad mellan energydepleted och logphase (för alla nedan)
    \item Radius of gyration = Rörlighet
    \item MSD
    \item Anisotrop miljö
    \item Korrelationsfunktioner
\end{itemize}

\section{Dataanalys och resultat}

\begin{itemize}
    \item Anisotrop miljö
    \item Skillnad mellan energydepleted och logphase (för alla nedan)
    \item Radius of gyration = Rörlighet
    \item MSD(vilken modell passar bäst)
    \item Korrelationsfunktioner
\end{itemize}

\section{Diskussion och slutsats}

\chapter{Strängars rörelse i vätskor}

Bakgrund
\begin{itemize}
    \item WLC model, ev annan modell (Måns)
    \item Protein-filament
    \item Korrelationer, tangent, tid, rum
    \item Egenmoder
\end{itemize}

Resultat som kan tas med
\begin{itemize}
    \item Uppdelningen i egenmoder, olika relaxationstid
    \item Dispersionsrelation?
    \item Skillnad mellan confined och unconfined
    
\end{itemize}


\section{Teori}

\subsection{Proteinfilament}

Aktinfilament består av glubulära aktinprotein formade som bollar vilka kopplats ihop till en lång kedja.

\subsection{Modeller för strängrörelser}

\subsubsection{Worm Like Chain-model}

\subsubsection{Månsmodell}


\subsection{Egenomder}


\subsection{Lösning arv stokastisk differentialekvation}



\section{Tillhandahållen data}

Datan som analyserats i denna andra del av arbetet kommer från \todo{Var kommer datan från?}... och består av filmer av aktinfilament som tilåts röra sig i en vätska. Dessa strängar hade en storlek kring 10--30\,\micro{m} och befann sig i kanaler av olika bredd. Datan hade redan behandlats något så att strängens läge kunde beskrivas genom lägena för ett antal punker på strängen vilka angivna som pixlar i ett koordinatsystem. Genom att bildförstoringen och kamerans pixelstorlek var känd kunde de olika strängarnas längd beräknas. Mätningarna hade utförst på två ''fria'' strängar (strängar i breda skåror) och två strängar instängda i smala skåror, alla fyra filmade med 10 bilder per sekund. Rörelsen utfördes till största del i två dimensioner då skårornas djup var litet i förhållande till skårornas och filamentens bredd.

\section{Resultat}



\section{Diskussion och slutsats}





%Bara en liten kodsnutt som behövs när man kompilerar lokalt
%%% Local Variables: 
%%% mode: latex
%%% TeX-master: "main.tex"
%%% End: 








%%%%%%%%%%%%%%%%%%%%%%%%% Källförteckning %%%%%%%%%%%%%%%%%%%%%%%%%
\newpage
%Ser till att det blir rätt namn i rubriken
\iflanguage{swedish}{\renewcommand{\bibname}{Källförteckning}}{}
\bibliographystyle{ieeetr}
\bibliography{referenser_kandidat}%kräver en fil som heter 'referenser.bib'          

%%%%%%%%%%%%%%%%%%%%%%%%%%%%% Bilagor %%%%%%%%%%%%%%%%%%%%%%%%%%%%%
\clearpage
\appendix%Resets the section counter and changes it to Alph                 
\setcounter{page}{1} %Resets the pgenumbering                               
\renewcommand*{\thepage}{A\arabic{page}}%Changes the pagenubering to 'A...'
%Ser till at det blir rätt namn
\iflanguage{swedish}{\renewcommand{\appendixname}{Bilagor}}{}
\phantomsection{}%This one is needed to make 'Appendix' show up in the TOC
\addcontentsline{toc}{part}{\appendixname} %Makes 'Appendix' show up in the TOC
%Byter tillbaks till det gamla namnet
\iflanguage{swedish}{\renewcommand{\appendixname}{Bilaga}}{}




\end{document}





%% På svenska ska citattecknet vara samma i både början och slut.
%% Använd två apostrofer (två enkelfjongar): ''.

%% Inkludera PDF-dokument
\includepdf[pages={1-}]{filnamn.pdf} %Filnamnet får INTE innehålla 'mellanslag'!

%% Figurer inkluderade som pdf-filer
\begin{figure}\centering
\centerline{ %centrerar även större bilder
\includegraphics[width=1\textwidth]{filnamn.pdf}
}
\caption{\label{fig:} }
\end{figure}

%% Figurer inkluderade med xfigs "Combined PDF/LaTeX"
\begin{figure}\centering
\resizebox{.8\textwidth}{!}{\input{filnamn.pdf_t}}
\caption{\label{fig:} }
\end{figure}

%% Figurer roterade 90 grader
\begin{sidewaysfigure}\centering
\centerline{ %centrerar även större bilder
\includegraphics[width=1\textwidth]{filnamn.pdf}
}
\caption{\label{fig:} }
\end{sidewaysfigure}


%%Om man vill lägga till något i TOC
\stepcounter{section} %Till exempel en 'section'
\addcontentsline{toc}{section}{\Alph{section}\hspace{8 pt}Labblogg} 



%Bara en liten kodsnutt som behövs när man kompilerar lokalt
%%% Local Variables: 
%%% mode: latex
%%% TeX-master: "main.tex"
%%% End: 