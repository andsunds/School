\documentclass[11pt,a4paper, notitlepage]{report}
\pdfoutput=1
%vons grund
\usepackage[T1]{fontenc}
\usepackage[utf8]{inputenc}
\usepackage[english, swedish]{babel} %OBS! Se till att vi får rätt språk.
\usepackage{amsmath}
\usepackage{lmodern}
\usepackage{units}
\usepackage{icomma}
\usepackage{color}
\usepackage{graphicx}
\graphicspath{ {bilder/} } %Gör så att man kan lägga alla bilder i en egen katalog
\usepackage{bbm}
\newcommand{\N}{\ensuremath{\mathbbm{N}}}
\newcommand{\Z}{\ensuremath{\mathbbm{Z}}}
\newcommand{\Q}{\ensuremath{\mathbbm{Q}}}
\newcommand{\R}{\ensuremath{\mathbbm{R}}}
\newcommand{\C}{\ensuremath{\mathbbm{C}}}
\newcommand{\rd}{\ensuremath{\mathrm{d}}}
\newcommand{\id}{\ensuremath{\,\rd}}
\usepackage{hyperref}

%%%%%%%%%%%%%%%%%%%%%%%Egna tillägg%%%%%%%%%%%%%%%%%%%%%%%
%%För att få referenser 'på svenska'
\usepackage[fixlanguage]{babelbib}
%\selectbiblanguage{swedish}
%\renewcommand\btxauthorcolon{:}
%%För att figurtext i underfigurer
\usepackage{subfigure} 
%%För att kunna inkludera andra PDF-dokument
\usepackage{pdfpages}
%%För att kunna ha roterade bilder
\usepackage{rotating}
%%För att kunna byta uppräkningsvisare t.ex. [label=\alph*)]
\usepackage{enumitem}
%%För att kunna lägga till 'bilagor' utan sidnumrering.
\usepackage{tocstyle}
\usetocstyle{standard}%För att få en vanlig TOC
                %no page numbers for part
\settocstylefeature[-1]{pagenumberbox}{\csname @gobble\endcsname}
\usepackage[nottoc]{tocbibind} %Puts an 'Reference' entry in the ToC.
%%För att kunna använda bra och ket
\usepackage{physics} %\bra{}\ket{} eller \expval{H}{\psi}
%%För att kunna rita snygga matriser
\usepackage{mathtools} %\begin{pmatrix*}[r] ... \end
%%För att kunna kommentera ut större stycken
%%Drar in tabell och figurtexter
\usepackage[margin=10 pt]{caption}
\usepackage{comment} %\begin{comment}
%%För att lägga in 'att göra'-noteringar i texten
\usepackage{todonotes} %\todo{...}, \todolist



%%För att inkludera MATLABkod. 
%%OBS: mcode är ett separat paket och man måste ha mcode.sty i samma
%%katalog som dokumentet.
%\usepackage[framed,numbered,autolinebreaks,useliterate]{mcode}
%\usepackage{listings} 
%\lstloadlanguages{matlab} 
%\lstset{language=matlab} 
%\lstset{literate= {å}{{\r{a}}}1 {ä}{{\"a}}1 {ö}{{\"o}}1 {Å}{{\r{A}}}1
%  {Ä}{{\"A}}1 {Ö}{{\"O}}1}%För att få svenska bokstäver från MATLAB.


%%%%%%%%%%%%%%%%%%%%%%%Formatering%%%%%%%%%%%%%%%%%%%%%%%%
%%Partiell derivata
\newcommand{\pd}{\ensuremath{\partial}}
%%Följer ISO-8601 oberoende av språk.
\usepackage{datetime2} 
%\newdateformat{specialdate}{\THEYEAR-\twodigit{\THEMONTH}-\twodigit{\THEDAY}}
%%Göra grader Celcius
\usepackage[iso, swedish]{isodate}
\newcommand{\degC}{\ifmmode \,^\circ\mathrm{C} \else $\,^\circ\mathrm{C}$ \fi}
%%Figurreferenser
\newcommand{\figref}{\figurename~\ref} 
%%Tabellreferenser
\newcommand{\tabref}{\tablename~\ref} %Stor bokstav i början

%%Det ska vara ett rakt µ i prefixet
\usepackage{upgreek}
\newcommand{\micro}{\upmu}
%%Ohm enhetskommando
\newcommand{\ohm}{\ifmmode \Upomega \else $\Upomega$ \fi}

%%'e' och 'i' ska vara upprätt
\newcommand{\e}{\mathrm{e}}
\newcommand{\ii}{\mathrm{i}}

%%För att själv bestämma marginalerna. 
\usepackage[
%            top    = 3cm,
%            bottom = 3cm,
%            left   = 3cm, right  = 3cm
]{geometry}


%%%%%%%%%%%%%%%%%%%%%%%%%%%%%%%%%%%%%%%%%%%%%%%%%%%%%%%%%%%%%%%


\newcommand{\andsunds}{andsunds@student.chalmers.se}
\newcommand{\emeeke}{emeeke@student.chalmers.se}
\newcommand{\robka}{robka@student.chalmers.se}
\newcommand{\soliver}{soliver@student.chalmers.se}

%%%%%%%%%%%%%%%%%%%%%%%%%%%%%%%%%%%%%%%%%%%%%%%%%%%%%%%%%%%%%%%





\begin{document}
\pagenumbering{roman} %%Romersk sidnumrering i början
\begin{titlepage}
\newgeometry{top=3cm, bottom=2cm}

\newcommand{\HRule}{\rule{\linewidth}{0.5mm}} % Defines a new command for the horizontal lines, change thickness here

\center % Center everything on the page
 
%------------------------------------------------------------------------------------
%	HEADING SECTIONS
%------------------------------------------------------------------------------------

\textsc{\LARGE Chalmers tekniska högskola}\\[1.5cm] % Name of your university/college
\textsc{\Large Kandidatarbetesrapport}\\[0.5cm] % Major heading such as course name
\textsc{\large Institutionen för fysik }\\[0.5cm] % Minor heading such as course title

%------------------------------------------------------------------------------------
%	TITLE SECTION
%------------------------------------------------------------------------------------

\HRule \\[0.4cm]
{ \huge \bfseries Titel}\\[0.4cm] % Title of your document
\HRule \\[1.5cm]
 
%------------------------------------------------------------------------------------
%	AUTHOR SECTION
%------------------------------------------------------------------------------------

\begin{minipage}{0.4\textwidth}
\begin{flushleft} \large
\emph{Författare:}\\
Andreas Sundström\footnotemark{} \\
Emelie Ekenstedt\footnotemark{} \\
Robin Karlsson\footnotemark{} \\
Oliver Sundell\footnotemark{}
\end{flushleft}
\end{minipage}
~
\begin{minipage}{0.4\textwidth}
\begin{flushright} \large
\emph{Handledare:} \\
Måns Henningson
\end{flushright}
\end{minipage}\\[3cm]

\addtocounter{footnote}{-4} %3=n
\stepcounter{footnote}
  \footnotetext{\href{mailto:\andsunds}{\texttt{\andsunds}}}
\stepcounter{footnote}
  \footnotetext{\href{mailto:\emeeke}{\texttt{\emeeke}}}
\stepcounter{footnote}
  \footnotetext{\href{mailto:\robka}{\texttt{\robka}}}
\stepcounter{footnote}
  \footnotetext{\href{mailto:\soliver}{\texttt{\soliver}}}



%------------------------------------------------------------------------------------
%	DATE SECTION
%------------------------------------------------------------------------------------

{\large \today}\\[1cm] % Date, change the \today to a set date if you want to be precise

%------------------------------------------------------------------------------------
%	LOGO SECTION
%------------------------------------------------------------------------------------

\includegraphics[height=5cm]{logo.pdf} % Include a department/university logo - this will require the graphicx package
 
%------------------------------------------------------------------------------------

\vfill % Fill the rest of the page with whitespace

\end{titlepage}
\restoregeometry




\setcounter{page}{2}%detta är ANDRA (2) sidan

\renewcommand{\abstractname}{Sammandrag}
\begin{abstract}


\end{abstract}

\renewcommand{\abstractname}{Abstract}
\begin{abstract}


\end{abstract}

\clearpage
\tableofcontents
\clearpage

\pagenumbering{arabic}
\setcounter{page}{1}
%\part{Inledning}
%\chapter{Test}
%\input{...}






%%%%%%%%%%%%%%%%% slutet av rapporten %%%%%%%%%%%%%%%%%
\newpage
%Ser till att det blir rätt namn i rubriken
\iflanguage{swedish}{\renewcommand{\refname}{Källförteckning}}{}
\bibliographystyle{babunsrt}
\bibliography{referenser}%kräver en fil som heter 'referenser.bib'          

\clearpage
\appendix%Resets the section counter and changes it to Alph                 
\setcounter{page}{1} %Resets the pgenumbering                               
\renewcommand*{\thepage}{A\arabic{page}}%Changes the pagenubering to 'A...'
%Ser till at det blir rätt namn
\iflanguage{swedish}{\renewcommand{\appendixname}{Bilagor}}{}
\phantomsection{}%This one is needed to make 'Appendix' show up in the TOC
\addcontentsline{toc}{part}{\appendixname} %Makes 'Appendix' show up in the TOC
%Byter tillbaks till det gamla namnet
\iflanguage{swedish}{\renewcommand{\appendixname}{Bilaga}}{}




\end{document}





%% På svenska ska citattecknet vara samma i både början och slut.
%% Använd två apostrofer (två enkelfjongar): ''.

%%För att referera till till tidigare fotnot:
%\footnotemark[\value{footnote}]

%% Inkludera PDF-dokument
%\includepdf[pages={1-}]{filnamn.pdf} %Filnamnet får INTE innehålla 'mellanslag'!

%% Figurer inkluderade som pdf-filer
%\begin{figure}\centering
%\centerline{ %centrerar även större bilder
%\includegraphics[width=1\textwidth]{filnamn.pdf}
%}
%\caption{\label{fig:} }
%\end{figure}

%% Figurer inkluderade med xfigs "Combined PDF/LaTeX"
%\begin{figure}\centering
%\resizebox{.8\textwidth}{!}{\input{filnamn.pdf_t}}
%\caption{\label{fig:} }
%\end{figure}

%% Figurer roterade 90 grader
%\begin{sidewaysfigure}\centering
%\centerline{ %centrerar även större bilder
%\includegraphics[width=1\textwidth]{filnamn.pdf}
%}
%\caption{\label{fig:} }
%\end{sidewaysfigure}


%%Om man vill lägga till något i TOC
%\stepcounter{section} %Till exempel en 'section'
%\addcontentsline{toc}{section}{\Alph{section}\hspace{8 pt}Labblogg} 

