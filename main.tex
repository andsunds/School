\documentclass[11pt,a4paper, 
english, swedish %% Make sure to put the main language last!
]{article}
\pdfoutput=1

%% Andréas's custom package 
%% (Will work for most purposes, but is mainly focused on physics.)
\usepackage{custom_as}

%% Figures can now be put in a folder: 
\graphicspath{ {figurer/} %{some_folder_name/}
}

%% If you want to change the margins for just the captions
\usepackage[margin=10 pt]{caption}

%% To add todo-notes in the pdf
\usepackage[%disable  %%this will hide all notes
]{todonotes} 

%% Change the margin in the documents
\usepackage[
%            top    = 3cm,              %% top margin
%            bottom = 3cm,              %% bottom margin
%            left   = 3cm, right  = 3cm %% left and right margins
]{geometry}


%% If you want to chage the formating of the section headers
%\renewcommand{\thesection}{...}



%%%%%%%%%%%%%%%%%%%%%%%%%%%%%%%%%%%%%%%%%%%%%%%%%%%%%%%%%%%%%%%%%%%%%%
\begin{document}%% v v v v v v v v v v v v v v v v v v v v v v v v v v
%%%%%%%%%%%%%%%%%%%%%%%%%%%%%%%%%%%%%%%%%%%%%%%%%%%%%%%%%%%%%%%%%%%%%%

%%%%%%%%%%%%%%%%%%%% vvv Internal title page vvv %%%%%%%%%%%%%%%%%%%%%

\title{Investigation of diffusion in wet paper using pulsed NMR}
\author{Andréas Sundström \and Tan Qin Yuan}
\date{\today}

\maketitle


\begin{abstract} 


\end{abstract}

%%%%%%%%%%%%%%%%%%%% ^^^ Internal title page ^^^ %%%%%%%%%%%%%%%%%%%%%
%% If you want a list of all todos
%\todolist



\section{Introduction}

Nuclear Magnetic Resonance refers to the phenomenon whereby nuclei of certain spin in the presence of an external magnetic field, undergo a transition of spin when exposed to a certain radio frequency. In the presence of a strong external magnetic field, the nuclei of the atoms of the sample undergo spin polarization. This is because any nuclei with magnetic moments that are not aligned with the direction of the magnetic field, experiences a magnetic torque which forces it to align to the direction of the magnetic field. The magnetic moment of the nuclei is perpendicular to the plane of spin of the nuclei. At the ground state (lower energy state), the magnetic moment of the nuclei is parallel to the direction of the magnetic field. However, when a certain radio frequency (Lamour frequency) is being transmitted onto the nuclei, some of the nuclei undergo Nuclear Magnetic Resonance. The nuclei on that ground state absorb photons of that frequency then transit to the excited energy state; refer to Fig 1. This happens by flipping the spin of the nuclei. The magnetic moments of the nuclei changes to anti-parallel to the direction of the magnetic field which corresponds to the excited state energy of the nuclei. If the resonance frequency (Lamour Frequency) source is turned off with the external magnetic field left on, the nuclei will then return to the ground state by reverting their spins and nuclear moments to parallel of the direction of the magnetic field. In the process, emitting photons of the Lamour Frequency. However, if both resonance frequency and external magnetic field are turned off, the nuclei will undergo random spontaneous emissions to return to its equilibrium state. At equilibrium state, the spin orientations and directions nuclear magnetic moments are random.   

\section{Method}

In our experiment 
\section{Results}

\section{Discussion}

\section{Conclusions}

\section{References}

http://hyperphysics.phy-astr.gsu.edu/hbase/Nuclear/nmr.html















%%%%%%%%%%%%%%%%%%%%%%%%%% The bibliography %%%%%%%%%%%%%%%%%%%%%%%%%%
%\newpage
%% This bibliography ueses BibTeX
\bibliographystyle{ieeetr}
\bibliography{references}%requires a file named 'references.bib'
%% Citations are as usual: \cite{example_article}

%%%%%%%%%%%%%%%%%%%%%%%%%%%%% Appendices %%%%%%%%%%%%%%%%%%%%%%%%%%%%%
\clearpage %% on a new page 
\appendix  %% This will change the page numbering to A1, A2, A3, ...;
           %% and also change the sections to A, A.1, ...; B, B.1, ...


%%%%%%%%%%%%%%%%%%%%%%%%%%%%%%%%%%%%%%%%%%%%%%%%%%%%%%%%%%%%%%%%%%%%%%
\end{document}%% ^ ^ ^ ^ ^ ^ ^ ^ ^ ^ ^ ^ ^ ^ ^ ^ ^ ^ ^ ^ ^ ^ ^ ^ ^ ^ ^
%%%%%%%%%%%%%%%%%%%%%%%%%%%%%%%%%%%%%%%%%%%%%%%%%%%%%%%%%%%%%%%%%%%%%%




%%%%  Some (useful) templates


%% På svenska ska citattecknet vara samma i både början och slut.
%% Använd två apostrofer: ''.


%% Including PDF-documents
\includepdf[pages={1-}]{filnamn.pdf} % NO blank spaces in the file name

%% Figures (pdf, png, jpg, ...)
\begin{figure}\centering
\centerline{ % centers figures larges than 1\textwidth
\includegraphics[width=.8\textwidth]{file_name.pdf}
}
\caption{}
\label{fig:}
\end{figure}

%% Figures from xfig's "Combined PDF/LaTeX"
\begin{figure}\centering
\resizebox{.8\textwidth}{!}{\input{file_name.pdf_t}}
\caption{}
\label{fig:}
\end{figure}


%% If you want to add something to the ToC
%% (Without having an actual header in the text.)
\stepcounter{section} %For example a 'section'
\addcontentsline{toc}{section}{\Alph{section}\hspace{8 pt}Labblogg} 

