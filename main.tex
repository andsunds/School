\documentclass[11pt,a4paper, twocolumn,
swedish, english %% Make sure to put the main language last!
]{article}
\pdfoutput=1

%% Andréas's custom package 
%% (Will work for most purposes, but is mainly focused on physics.)
\usepackage{custom_as}

%% Figures can now be put in a folder: 
\graphicspath{ {figures/} %{some_folder_name/}
}

%% If you want to change the margins for just the captions
\usepackage[margin=10 pt]{caption}

%% To add todo-notes in the pdf
\usepackage[%disable  %%this will hide all notes
]{todonotes} 

%% Change the margin in the documents
\usepackage[
%            top    = 3cm,              %% top margin
%            bottom = 3cm,              %% bottom margin
            left   = 2.5cm, right  = 2.5cm %% left and right margins
]{geometry}


%% If you want to chage the formating of the section headers
%\renewcommand{\thesection}{...}



%%%%%%%%%%%%%%%%%%%%%%%%%%%%%%%%%%%%%%%%%%%%%%%%%%%%%%%%%%%%%%%%%%%%%%
\begin{document}%% v v v v v v v v v v v v v v v v v v v v v v v v v v
%%%%%%%%%%%%%%%%%%%%%%%%%%%%%%%%%%%%%%%%%%%%%%%%%%%%%%%%%%%%%%%%%%%%%%

%%%%%%%%%%%%%%%%%%%% vvv Internal title page vvv %%%%%%%%%%%%%%%%%%%%%

\title{Investigation of diffusion in wet paper using pulsed NMR}
\author{Andréas Sundström \and Tan Qin Yuan}
\date{\today}

\twocolumn[
\begin{@twocolumnfalse}
\maketitle
\begin{abstract}

\end{abstract}
\end{@twocolumnfalse}
]

%%%%%%%%%%%%%%%%%%%% ^^^ Internal title page ^^^ %%%%%%%%%%%%%%%%%%%%%
%% If you want a list of all todos
%\todolist



\section{Introduction}

\todo[inline]{Something on paper and diffusion in it, and why we want
  to study this. }

\subsection{Pulsed NMR}

Nuclear Magnetic Resonance (NMR) refers to a method of probing the
spins of atomic nuclei, in an external magnetic field, by exciting
them with their resonance frequency, linked to the properties of the
nucleus and the external magnetic field. Some atomic nuclei have an
intrinsic spin, $\vb*I$, equivalent to the electron's. This energy of
the spin in an external magnetic field, $\vb*B_0$, is
\begin{equation}\label{eq:mag-energy}
E=-\gamma\vb*I\vdot\vb*B_0
\end{equation}
similar to the Zeeman effect. Here
\begin{equation}
\gamma=\frac{ge}{2m_\text{p}},
\end{equation}
where $e$ is the charge of the proton, $m_\text{p}$ is the proton
mass, and $g$ is a dimensionless spectroscopic factor dependent on the
species of nucleus. This energy coupling, among all the nuclei in a
sample, will result in an aggregate macroscopic net magnetization,
$\vb*M$, parallel to $\vb*B_0$; it is this macroscopic magnetization which
can be probed in different ways using NMR techniques. 
While in principle NMR works for any types of nuclei with spin,
hydrogen ($^1\!$H) is the most used nucleus for NMR. This is because
$^1\!$H has one of the highest values of $\gamma$ among the common
chemical elements, certainly among the nuclei common to organic
chemistry. 

The principles of \emph{pulsed} NMR is that the equilibrium
magnetization can be disturbed using a pulsed, second magnetic field;
then the subsequent decay back to the equilibrium can be
measured. Any magnetic field, $\vb*B$, will exert a twisting torque on
$\vb*M$ according to
\begin{equation}
\vb*\tau=\vb*M\cross\vb*B.
\end{equation}
Since $\vb*\tau$ is perpendicular to both $\vb*M$ and $\vb*B$, the
torque results in a precession motion around $\vb*B$ with a
precession frequency \cite[ch. 2.1]{Principles_MR1990}
\begin{equation}
\omega_\text{L} = \gamma \abs{\vb*B}= \gamma B
\end{equation}
called the Larmor frequency. 
In the context of pulsed NMR there are two magnetic fields, the static
$\vb*B_0$ and the field from the pulse $\vb*B_1$ which is
perpendicular to $\vb*B_0$. Both will exert torques on $\vb*M$,
however since $B_0\gg B_1$ the precession around $\vb*B_0$ is
dominating. And unless $\vb*B_1$ stays in phase with the precession of
$\vb*M$ the time averaged effect of $\vb*B_1$ will diminish. That is
if $\vb*B_1$ rotates\footnotemark{} with angular frequency
$\omega_0=\gamma B_0$, then $\vb*B_1$ will have a lasting impact on
the magnetization $\vb*M$. That impact will be to rotate $\vb*M$
around $\vb*B_1$, away from $\vb*B_0$. It is here where the pulses
come into play; depending on the \emph{length} of the pulses, $\vb*M$
can be rotated by different amounts. Of special interest are the pulse
lengths which rotates $\vb*M$ by $90^\circ$ or $180^\circ$ -- called
$\pi/2$ and $\pi$ pulses respectively.

\footnotetext{This would correspond to a circularly polarized
  field. In reality the fields used for $\vb*B_1$ are linearly
  polarized. However since any linearly polarized field can be
  split into two counter-rotating circularly polarized field this is
  no problem; one of the rotating component will keep in phase with the
  precession of $\vb*M$, while the other only generates very small
  perturbations. }

Now when the magnetization, $\vb*M$, has been rotated it will have a
component perpendicular to $\vb*B_0$. This perpendicular component,
$\vb*M_\perp$, rotates with angluar frequency $\omega_0$. If a pick-up
coil is placed in this plane of rotation, $\vb*M_\perp$ will induce a
voltage which can be measured, and it is this measured voltage (or
rather its envelope) which is of interest in pulsed NMR
measurements. The signal will generally decay exponentially, as
$\vb*M$ returns back to its equlibrium. The exponential decay rate,
$R$ (sometimes also given as $T=1/R$), can be said to generally
consits of three terms 
\begin{equation}
R= R_1 + R_2 + \gamma\Delta B_0.
\end{equation}
The first two terms, $R_1$ and $R_2$, are related to the sample, and
the last term is from inperfections in $B_0$.




\subsubsection{Measuring the diffusion rate}





\section{Method}

\subsection{Limitations in the equipment}
\todo[inline]{We tried dry paper and didn't see anything. Also the
  weird phenomenon with water and delays of 5\,ms.}

\section{Results}


\section{Discussion}

\section{Conclusions}

















%%%%%%%%%%%%%%%%%%%%%%%%%% The bibliography %%%%%%%%%%%%%%%%%%%%%%%%%%
%\newpage
%% This bibliography uses BibTeX
\bibliographystyle{ieeetr}
\bibliography{references}%requires a file named 'references.bib'
%% Citations are as usual: \cite{example_article}

%%%%%%%%%%%%%%%%%%%%%%%%%%%%% Appendices %%%%%%%%%%%%%%%%%%%%%%%%%%%%%
\clearpage %% on a new page 
\appendix  %% This will change the page numbering to A1, A2, A3, ...;
           %% and also change the sections to A, A.1, ...; B, B.1, ...


%%%%%%%%%%%%%%%%%%%%%%%%%%%%%%%%%%%%%%%%%%%%%%%%%%%%%%%%%%%%%%%%%%%%%%
\end{document}%% ^ ^ ^ ^ ^ ^ ^ ^ ^ ^ ^ ^ ^ ^ ^ ^ ^ ^ ^ ^ ^ ^ ^ ^ ^ ^ ^
%%%%%%%%%%%%%%%%%%%%%%%%%%%%%%%%%%%%%%%%%%%%%%%%%%%%%%%%%%%%%%%%%%%%%%



%  LocalWords:  Larmor
