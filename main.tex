\documentclass[11pt,a4paper, twocolumn,
swedish, english %% Make sure to put the main language last!
]{article}
\pdfoutput=1

%% Andréas's custom package 
%% (Will work for most purposes, but is mainly focused on physics.)
\usepackage{custom_as}

%% Figures can now be put in a folder: 
\graphicspath{ {figures/} %{some_folder_name/}
}

%% If you want to change the margins for just the captions
\usepackage[margin=10 pt]{caption}

%% To add todo-notes in the pdf
\usepackage[%disable  %%this will hide all notes
]{todonotes} 

%% Change the margin in the documents
\usepackage[
%            top    = 3cm,              %% top margin
%            bottom = 3cm,              %% bottom margin
             left   = 2.5cm, right  = 2.5cm %% left and right margins
]{geometry}


%% If you want to change the formatting of the section headers
%\renewcommand{\thesection}{...}



%%%%%%%%%%%%%%%%%%%%%%%%%%%%%%%%%%%%%%%%%%%%%%%%%%%%%%%%%%%%%%%%%%%%%%
\begin{document}%% v v v v v v v v v v v v v v v v v v v v v v v v v v
%%%%%%%%%%%%%%%%%%%%%%%%%%%%%%%%%%%%%%%%%%%%%%%%%%%%%%%%%%%%%%%%%%%%%%

%%%%%%%%%%%%%%%%%%%% vvv Internal title page vvv %%%%%%%%%%%%%%%%%%%%%

\title{Investigation of diffusion in wet paper using pulsed NMR}
\author{Andréas Sundström \and Tan Qin Yuan}
\date{\today}

\twocolumn[
\begin{@twocolumnfalse}
\maketitle
\begin{abstract}

\end{abstract}
\end{@twocolumnfalse}
]

%%%%%%%%%%%%%%%%%%%% ^^^ Internal title page ^^^ %%%%%%%%%%%%%%%%%%%%%
%% If you want a list of all todos
%\todolist



\section{Introduction}
\todo[inline]{Something about why we looked at diffusion in paper.}

Nuclear Magnetic Resonance refers to the phenomenon whereby nuclei of certain spin in the presence of an external magnetic field, undergo a transition of spin when exposed to a certain radio frequency. In the presence of a strong external magnetic field, the nuclei of the atoms of the sample undergo spin polarization. This is because any nuclei with magnetic moments that are not aligned with the direction of the magnetic field, experiences a magnetic torque which forces it to align to the direction of the magnetic field. The magnetic moment of the nuclei is perpendicular to the plane of spin of the nuclei. At the ground state (lower energy state), the magnetic moment of the nuclei is parallel to the direction of the magnetic field. However, when a certain radio frequency (Larmor frequency) is being transmitted onto the nuclei, some of the nuclei undergo Nuclear Magnetic Resonance. The nuclei on that ground state absorb photons of that frequency then transit to the excited energy state; refer to Fig 1. This happens by flipping the spin of the nuclei. The magnetic moments of the nuclei changes to anti-parallel to the direction of the magnetic field which corresponds to the excited state energy of the nuclei. If the resonance frequency (Larmor Frequency) source is turned off with the external magnetic field left on, the nuclei will then return to the ground state by reverting their spins and nuclear moments to parallel of the direction of the magnetic field. In the process, emitting photons of the Larmor Frequency. However, if both resonance frequency and external magnetic field are turned off, the nuclei will undergo random spontaneous emissions to return to its equilibrium state. At equilibrium state, the spin orientations and directions nuclear magnetic moments are random.   


\todo[inline]{Describe what we have done.}

\subsection{Pulsed NMR}

Nuclear Magnetic Resonance (NMR) refers to a method of probing the
spins of atomic nuclei, in an external magnetic field, by exciting
them with their resonance frequency, linked to the properties of the
nucleus and the external magnetic field. Some atomic nuclei have an
intrinsic spin, $\vb*I$, equivalent to the electron's. This energy of
the spin in an external magnetic field, $\vb*B_0$, is
\begin{equation}\label{eq:mag-energy}
E=-\gamma\vb*I\vdot\vb*B_0=-\gamma B_0 I_z
\end{equation}
(by convention $\vb*B_0=B_0\vu{z}$) similar to the Zeeman effect. Here
\begin{equation}
\gamma=\frac{ge}{2m_\text{p}},
\end{equation}
where $e$ is the charge of the proton, $m_\text{p}$ is the proton
mass, and $g$ is a dimensionless spectroscopic factor dependent on the
species of nucleus. This energy coupling, among all the nuclei in a
sample, will result in an aggregate macroscopic net magnetization,
$\vb*M$, parallel to $\vb*B_0$; it is this macroscopic magnetization which
can be probed in different ways using NMR techniques. 
While in principle NMR works for any types of nuclei with spin,
hydrogen ($^1\!$H) is the most used nucleus for NMR. This is because
$^1\!$H has one of the highest values of $\gamma$ among the common
chemical elements, certainly among the nuclei common to organic
chemistry. 

The principles of \emph{pulsed} NMR is that the equilibrium
magnetization can be disturbed using a pulsed, second magnetic field;
then the subsequent decay back to the equilibrium can be
measured. Any magnetic field, $\vb*B$, will exert a twisting torque on
$\vb*M$ according to
\begin{equation}
\vb*\tau=\vb*M\cross\vb*B.
\end{equation}
Since $\vb*\tau$ is perpendicular to both $\vb*M$ and $\vb*B$, the
torque results in a precession motion around $\vb*B$ with a
precession frequency \cite[ch. 2.1]{Principles_MR1990}
\begin{equation}
\omega_\text{L} = \gamma \abs{\vb*B}= \gamma B
\end{equation}
called the Larmor frequency. 
In the context of pulsed NMR there are two magnetic fields, the static
$\vb*B_0$ and the field from the pulse $\vb*B_1$ which is
perpendicular to $\vb*B_0$. Both will exert torques on $\vb*M$,
however since $B_0\gg B_1$ the precession around $\vb*B_0$ is
dominating. And unless $\vb*B_1$ stays in phase with the precession of
$\vb*M$ the time averaged effect of $\vb*B_1$ will diminish. That is
if $\vb*B_1$ rotates\footnotemark{} with angular frequency
$\omega_0=\gamma B_0$, then $\vb*B_1$ will have a lasting impact on
the magnetization $\vb*M$. That impact will be to rotate $\vb*M$
around $\vb*B_1$, away from $\vb*B_0$. It is here where the pulses
come into play; depending on the \emph{length} of the pulses, $\vb*M$
can be rotated by different amounts. Of special interest are the pulse
lengths which rotates $\vb*M$ by $90^\circ$ or $180^\circ$ -- called
$\pi/2$ and $\pi$ pulses respectively. Note that, in theory, the
$\pi/2$ pulse gives the strongest signal response while a $\pi$ pulse
should not give any signal at all.

\footnotetext{This would correspond to a circularly polarized
  field. In reality the fields used for $\vb*B_1$ are linearly
  polarized. However since any linearly polarized field can be
  split into two counter-rotating circularly polarized field this is
  no problem; one of the rotating component will keep in phase with the
  precession of $\vb*M$, while the other only generates very small
  perturbations. }

Now when the magnetization, $\vb*M$, has been rotated it will have a
component perpendicular to $\vb*B_0$. This perpendicular component,
$\vb*M_\perp$, rotates with angular frequency $\omega_0$. If a pick-up
coil is placed in this plane of rotation, $\vb*M_\perp$ will induce a
voltage which can be measured, and it is this measured voltage (or
rather its envelope) which is of interest in pulsed NMR
measurements. The signal will generally decay exponentially as
$\vb*M$ returns back to its equilibrium, the so called 
\emph{free induction decay}. The total exponential decay rate, by
convention called $R_2^*$ (sometimes also given as $T_2^*=1/R_2^*$),
can be said to generally consist of three terms 
\begin{equation}
R_2^*= R_1 + R_2 + \gamma\Delta B_0.
\end{equation}
The first two terms, $R_1$ and $R_2$, are related to the sample, and
the last term is from variations in $B_0$. These variations causes the
precession frequency to differ in different parts of the sample; the
spin precessions will therefore lose coherence and the signal will
decay. Unfortunately this last term often dominates. This means that
other methods must be used to measure $R_1$ and $R_2$ which are the
sample specific parameters. 

The first of the two intrinsic decay rates, $R_1$, the so called
\emph{longitudinal} or the \emph{spin-lattice} relaxation rate, is
related to the rate of energy loss of the spin. Since the
magnetization has been rotated the energy of the system has therefore
increased. The energy of the macroscopic magnetization is proportional
to $M_z$, similar to \eqref{eq:mag-energy}. The spin-lattice
relaxation rate is therefore related to the relaxation of $M_z$. We
will however not study or use this decay rate any further in this
report. 

%\subsubsection{The pulse echo method}
Instead we will focus on the second type of signal decay, $R_2$,
called the \emph{transverse} or the \emph{spin-spin} relaxation
rate. This decay rate is related to the coherentness of the individual
spins. As said before to measure any signal $M_\perp$ has to be
non-zero, which in turn requires the individual spins to be coherent
with each other. Consider an ideal ($\gamma\Delta B_0=0$) system
exposed by a $\pi/2$ pulse. Then directly after the pulse all the
spins have been rotated by $90^\circ$ in the same direction, which
means that $M_\perp$ is at its maximum. However If the spins, by some
intrinsic reason, tend to drift in phase when precessing, then
$M_\perp$ will decay due to loss of coherence -- similar to the signal
decay due to variations in $B_0$. It is that intrinsic individual
phase drift which gives rise to the decay of the signal corresponding
to $R_2$.  


\subsubsection{The pulse echo method and applying it to
measurments of the self diffusion rate} 

In the previous example the static magnetic field, $B_0$, was assumed
to be completely homogeneous. However since the underlying mechanism
for signal decay due to $R_2$ and $\gamma\Delta B_0$ is the same, spin
de-coherence, it can be hard to separate these two decay modes. The
pulse echo method is a method for isolating  and measuring $R_2$. 

The key difference between the two modes of decay is that spin-spin
relaxation is due to intrinsic \emph{phase drift} due to fluctuations,
while the magnetic field variation leads to different precession
\emph{frequencies}. By the ingenious method of pulse echo by Erwin
Hahn~\cite{Hahn1950} in 1950. The most common type of pulse echo
measurements are done by applying a $\pi/2$ pulse followed by a number
of $\pi$ pulses.

To understand the effect of of the susequent $\pi$ pulses assume that
$R_2=0$, i.e. that the only source of de-coherence is due to
variations in the precession frequency. The initial signal from the
$\pi/2$ pulse (at time $t=0$) will decay as usual then the $\pi$ pulse
(at $t=\tau/2$) effectively flips all spins, including the
perpendicular components, $\vb*I_\perp\to-\vb*I_\perp$, of each
spin. The spins which had precessed fsater due to having a slightly
higher precession frequence and was \emph{ahead} will now end up
\emph{behind}, but since they still have the faster frequency they
will catch up. At exactly time $t=\tau$ all the spins will meet
back up in phase, just like directly after the $\pi/2$ pulse, and an
identical signal would occur -- the ``pulse echo''. This can be
repeated with another $\pi$ pulse, usually at $t=3\tau/2$, to get
another echo. The echoes will only have the same amplitude if there
was no intrinsic rate of coherence loss. In reality the pulse echo
will decay in amplitude with rate $R_2$,
\begin{equation}
A(t)=A_0\exp[-R_2t].
\end{equation}

This is amost the whole picture. If the molucules in the sample can
diffuse to different regions with varying magnetic field,
Hahn~\cite{Hahn1950} showed that the signal will decay even faster
according to
\begin{equation}\label{eq:echo-diffusion}
A(t)=A_0\exp[-R_2t]
\exp[-K\frac{t^3}{n^2}],
\end{equation}
where $n$ is the number of $\pi$ pulses before time $t$, and $K$ is a
constant related to the diffusivity. Carr and
Purcel~\cite{Carr-Purcel1954} found the correct expresion for
\begin{equation}
K=\frac{\gamma^2D}{12}\qty(\pdv{B_0}{z})^2,
\end{equation}
where $D$ is the self diffusivity of the sample, i.e. how fast a
molecule of the sample diffuses through it. Now if the $\pi$ pulses
come at times
\begin{equation}
t_{n}=\qty(n-\tfrac{1}{2})\tau,
\end{equation}
%(the first $\pi/2$ puls being at $t=0$), 
then the echoes will occur at
time 
\begin{equation}
t_n'=n\tau.
\end{equation}
This means that \eqref{eq:echo-diffusion} can be written as
\begin{equation}
A(t')=A_0\exp[-\mathcal{R}t']
\end{equation}
since the amplitude is only relevant at the time of the echoes, and
where 
\begin{equation}
\mathcal{R}=\qty(R_2+K\tau^2).
\end{equation}
By measuring the decay rate of the multi-pulse echoes, $\mathcal{R}$,
as a function of pulse separation, $\tau$, both the spin-spin relaxation
rate, $R_2$, and the self diffusivity (indirectly), $D$, can be measured.





\section{Method}

\subsection{Limitations in the equipment}
\todo[inline]{We tried dry paper and didn't see anything. Also the
  weird phenomenon with water and delays of 5\,ms.}



\section{Results}


\section{Discussion}
\todo[inline]{Compare $T_2$ values to that of litterature.}

\section{Conclusions}




%\section{References}
%http://hyperphysics.phy-astr.gsu.edu/hbase/Nuclear/nmr.html

%%%%%%%%%%%%%%%%%%%%%%%%%% The bibliography %%%%%%%%%%%%%%%%%%%%%%%%%%
%\newpage
%% This bibliography uses BibTeX
\bibliographystyle{ieeetr}
\bibliography{references}%requires a file named 'references.bib'
%% Citations are as usual: \cite{example_article}

%%%%%%%%%%%%%%%%%%%%%%%%%%%%% Appendices %%%%%%%%%%%%%%%%%%%%%%%%%%%%%
\clearpage %% on a new page 
\appendix  %% This will change the page numbering to A1, A2, A3, ...;
           %% and also change the sections to A, A.1, ...; B, B.1, ...


%%%%%%%%%%%%%%%%%%%%%%%%%%%%%%%%%%%%%%%%%%%%%%%%%%%%%%%%%%%%%%%%%%%%%%
\end{document}%% ^ ^ ^ ^ ^ ^ ^ ^ ^ ^ ^ ^ ^ ^ ^ ^ ^ ^ ^ ^ ^ ^ ^ ^ ^ ^ ^
%%%%%%%%%%%%%%%%%%%%%%%%%%%%%%%%%%%%%%%%%%%%%%%%%%%%%%%%%%%%%%%%%%%%%%



%  LocalWords:  Larmor coherentness
