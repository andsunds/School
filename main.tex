\documentclass[11pt,a4paper, notitlepage, german, english, swedish]{report}
\pdfoutput=1

\usepackage{custom_as}

\graphicspath{ {bilder/} } %Gör så att man kan lägga alla bilder i en egen katalog

%%Drar in tabell och figurtexter
\usepackage[margin=10 pt]{caption}
%%För att lägga in 'att göra'-noteringar i texten
\usepackage[
%disable
]{todonotes} %\todo{...}

%%För att själv bestämma marginalerna. 
\usepackage[
%            top    = 3cm,
%            bottom = 3cm,
%            left   = 3cm, right  = 3cm
]{geometry}
\usepackage{amsmath}

%egna kommandon för statistik
\newcommand{\VAR}[1]{\text{var}\left[#1\right]}
\newcommand{\COV}[2]{\text{cov}\left[#1,\;#2\right]}
\newcommand{\CORR}[2]{\text{corr}\left[#1,\;#2\right]}




\begin{document}

%Alla inledande sidor finns i 'titlepages.tex'.
\renewcommand{\thefootnote}{\fnsymbol{footnote}}

%kortkommandon för mailaddresserna
\newcommand{\andsunds}{andsunds@student.chalmers.se}
\newcommand{\rigon}{rigon@student.chalmers.se}



\pagenumbering{roman} %%Romersk sidnumrering i början
\begin{titlepage}
\newgeometry{top=3cm, bottom=2cm}

\newcommand{\HRule}{\rule{\linewidth}{0.5mm}} % Defines a new command for the horizontal lines, change thickness here

\center % Center everything on the page
 
%------------------------------------------------------------------------------------
%	HEADING SECTIONS
%------------------------------------------------------------------------------------

\textsc{\huge Chalmers tekniska högskola}\\[1.5cm] % Name of university/college
\textsc{\Large Rapport, Experimentell fysik 2}\\[0.2cm] % Major heading such as course name
\textsc{\large Termodynamik -- Uppgift 3 }\\[0.5cm] % Minor heading such as course title

%------------------------------------------------------------------------------------
%	TITLE SECTION
%------------------------------------------------------------------------------------

\HRule \\[0.4cm]
{ \LARGE \bfseries 
Studier av kvicksilveratomens atomära emissionsspektra samt absorptionsspektra av laserfärgämnena Rhodamin B och Kumarin 307
}\\[0.4cm] % Title of  document
\HRule \\[1.5cm]
 
%------------------------------------------------------------------------------------
%	AUTHOR SECTION
%------------------------------------------------------------------------------------

\begin{minipage}{0.4\textwidth}
\begin{flushleft} \large
\emph{Författare:}\\
Andréas Sundström\footnotemark{} \\
Rigon Demisai\footnotemark{} 
\end{flushleft}
\end{minipage}
~
\begin{minipage}{0.4\textwidth}
\begin{flushright} \large
\emph{Labassistent:} \\
Martin Wersäll
\end{flushright}
\end{minipage}\\[3cm]

\setcounter{footnote}{0}
\stepcounter{footnote}
  \footnotetext{\href{mailto:\andsunds}{\texttt{\andsunds}}}
\stepcounter{footnote}
  \footnotetext{\href{mailto:\rigon}{\texttt{\rigon}}}



%------------------------------------------------------------------------------------
%	DATE SECTION
%------------------------------------------------------------------------------------
% Följer ISO-standarden för tidsintervall:
% https://en.wikipedia.org/wiki/ISO_8601#Time_intervals
% "Double hyphen" också ok istället för '/'. -- i LaTeX är dock lite på gränsen
{ \large
\begin{tabular}{rc}
    Laboration utförd: & 2015-12-11/15 \\[0.1cm]
    Rapport inlämnad: & \today
\end{tabular}\\[1cm]
}

%------------------------------------------------------------------------------------
%	LOGO SECTION
%------------------------------------------------------------------------------------

\includegraphics[height=5cm]{logo.pdf} % Include a department/university logo
 
%------------------------------------------------------------------------------------

\vfill % Fill the rest of the page with whitespace

\end{titlepage}
\restoregeometry


\setcounter{page}{2}%detta är ANDRA (2) sidan

\renewcommand{\abstractname}{Sammandrag}
\begin{abstract}
Vi har utfört en spektroskopisk studie av kvicksilveratomens atomära emissionsspektrum ur vilket vi kartlagt atomens energinivåer baserat på vår spektrometri. Vi har också studerat absorption i lösningar av två laserfärgämnen vid namn Rhodamin B och Kumarin 307. Mätningarna har utförts med en Spex 270M spektrometer och datainsamlingen har gjorts i LabView. Kvicksilvrets emissionsspektra är taget i intervallet 365 till 984 nm, där vi detekterat totalt 23 signifikanta emissionstoppar. Detta jämförs med NIST data där vi har 4 av 5 överlappningar med NIST ''persistent lines'' och 7 av 21 överlappningar med NIST ''strong lines''. Absorptionsspektra för Rhodamin B och Kumarin 307 visar breda absorptionsband vilket är kännetecknande för flourescerande ämnen som består av stora organiska molekyler.
\end{abstract}

\renewcommand{\abstractname}{Abstract}
\begin{abstract}

We have conducted a spectroscopic study of the emission spectrum of Mercury atoms and derived an energy level diagram based on these measurements. We have also studied the absorption spectrum of laser dye solutions of the compounds Rhodamine B and Coumarine 307. The measurements have been taken with a Spex 270M spectrometer and have been processed and recorded in LabView. The emission spectrum of Mercury has been recorded within the range of 365 to 984 nm, where we have detected a total of 23 significant emission peaks. This is contrasted with NIST data where we have 4 out of 5 overlaps with NIST ''persistent lines'' and 7 out of 21 overlaps with NIST ''strong lines''. The absorption spectrum for Rhodamine B and Coumarine 307 show broad absorption bands which are characteristic of flourescent compounds which consist of large organic molecules.

\end{abstract}

\clearpage
\renewcommand{\contentsname}{Innehållsförteckning}
\tableofcontents

\clearpage
\pagenumbering{arabic}
\setcounter{page}{1}

\renewcommand{\thefootnote}{\arabic{footnote}}
\setcounter{footnote}{0}




%%%%%%%%%%%%%%%%%%%%%% Här börjar huvudtexten %%%%%%%%%%%%%%%%%%%%%%

%\part{Inledning}

\chapter{Inledning}


%\section{Bakgrund, syfte och begränsningar}

%\paragraph{Bakgrund}
Transport inuti celler är en av grundstenarna som behövs för att cellen ska kunna verka. Exempel på en livsnödvändig intercellulär transport är hur ATP, molekylen som driver \emph{alla} biologiska processer, ska kunna ta sig från mitokondrien till alla delar av cellen. Detta är ett fall med passiv transport, där molekylerna eller partiklarna förflyttas genom att slumpvis diffundera genom cellen. Det är därför av hög vikt att kunna förstå dessa processer inuti cellerna.\todo{Man borde säga något om strängar här också.}

Partiklars rörelse i cytoplasman kan vid en första anblick tänkas kunna beskrivas med klassisk Brownsk rörelse. Studier inom området \cite{Gou_etal2014} har dock visat på avvikelser från denna teori och partiklarna verkar istället uppvisar ett sub-diffusivt beteende inuti cellen. Ytterligare skillnad i resultat kring partikelrörelser har tidigare observerats om cellen befinner sig i dvala eller i sitt normalt metabola tillstånd. En alltäckande teori för vad som kan förklara dessa observationer finns i dagsläget inte och ämnet utgör därför ett aktuellt forskningsområde.
Rörelsens stokastiska natur och cellens avancerade inre struktur ligger troligtvis bakom svårigheten man hittills stött på när man sökt en förklaringsmodell till rörelsen. Ett flertal modeller finns dock som beskriver delar av de observerade egenskaperna, bland annat ''fractional Brownian motion'' (fBm) och ''Continous Time Random Walk'' (CTRW) för partikelrrelse och ''Worm Like Chain'' (WLC) för strängrörelse som alla presenteras senare i detta arbete.

%\paragraph{Syfte} 
Syftet med denna rapport är att studera rörelser orsakade av passiv transport i celler för att utifrån denna studie förhoppningsvis kunna ge en klarare bild av cytoplasmans natur. Detta har framför allt gjorts genom att jämföra vedertagna teoretiska förklaringsmodeller med den givna datan, både för rörelsen hos partiklar genom cytoplasman och filaments (proteintrådar) rörelse i en vätska. %Filamenten befann sig alltså inte inuti en levande cell men studien av dessa borde ändå kunna ge en någorlunda bra bild av hur en sträng i cytoplasman skulle bete sig.


%\paragraph{Avgränsningar} %från planeringsrapporten
Att avbilda små partiklar och filament innebär stora svårigheter vilket har försvårat analysen av deras egenskaper. I dagsläget finns heller ingen fullständig modell som beskriver partiklars och filamentens rörelse. %Den teoretiska modellen som undersöks i det här kandidatarbetet behöver verka under vissa antaganden som begränsar dess användningsområde. 
Exempelvis kan variationer som uppstår vid betraktande av rörelser under olika tidsskalor komma att leda till svårigheter.%, bland annat i att finna en teoretisk modell som korrekt beskriver rörelsen oberoende av tidsskala. 
Således anses det mer rimligt att modellera rörelsen under antagandet att modellen i första hand beskriver rörelser för en viss tidsskala.%, förslagsvis observationer som varar i intervallet $\unit[10]{ms}$ till $\unit[10]{s}$ vilket speglar den data som analyseras i detta arbete.

Det finns idag olika teorier om vad som påverkar partiklars och filaments rörelse i cytoplasman. Vissa försöker beskriva vad som sker i celler med aktiv transport medan andra lägger mer fokus på den passiva transporten inom cellen. Då jästceller endast har passiv transport mellan celldelning.%, och datan inhämtats under mellan fas, kommer detta arbete att fokusera på just passiv transport, det vill säga diffusion av partiklar. 



\section{Datainsamling}
Datan som behandlats i detta arbete har inte samlats in under detta arbetes gång utan tillhandahölls från andra källor. Hur denna data där samlats in och vad den beskriver presenteras mer utförligt nedan.

\subsubsection{Datan för partikelrörelse i celler}
Datan som studerats för partikelrörelser i celler kommer från Max Planck Institutet i Dresden och utgörs av mätningar av positionen för fluorescerande partiklar i jästceller. Jästcellerna hade genmodifierats till att producera fluorescerande protein som lätt bildar kluster. 
Dessa kluster brukar vara av storleksordning 10--100\,nm, vilket kan jämföras med själva cellernas storlek på omkring 1\,\micro{m}.

Data för ett hundratal partiklar från olika jästceller ingick i mätserien, både för aktiva celler och celler som försatts i dvala med sänkt metabol aktivitet. Mätningen genomfördes med 100 bilder per sekund.


\subsubsection{Datan för strängrörelse i vätska}

Datan som analyserats för strängrörelse i vätska kommer från \todo{Var kommer datan från? Fråga Daniel?}... och består av filmer av aktinfilament som tillåts röra sig i en vätska. Dessa strängar hade en längd kring 10--30\,\micro{m} och befann sig i kanaler av olika bredd. Datan hade redan behandlats något så att strängens läge gav av en uppsättning vita pixlar mot en svart bakgrund.

Mätningar hade utförts på två typer av strängar: fria strängar i breda kanaler och inneslutna strängar i smala skåror. Det fanns två filmer för vardera strängtyp. Alla fyra hade filmats med 10 bilder per sekund. Rörelsen utfördes till största del i två dimensioner då skårornas djup var litet i förhållande till skårornas och filamentens bredd.





%\section{Inspiration från planeringsrapporten}

%Fördjupade studier av partikelrörelse i cellen skulle till exempel kunna leda till mer effektiva läkemedel. Vet man hur transporten inom cellen sker underlättar det arbetet med att ta fram specialdesignad medicin.


%\paragraph{Rapportens/Arbetets ändamål}
%Ur en stokastisk modell kan sedan en makroskopisk, statistisk beskrivning uppnås och det är med denna statistiska beskrivning som modellen kan jämföras med data. 

%\paragraph{Vad vi gjort}


%Men för att över huvud taget kunna analysera datan behövdes osäkerheten i mätningarna uppskattas, vilket inte är helt problemfritt då till exempel Brownsk rörelse i sig själv är en sorts brus. Brus brukar i vanliga fall hanteras genom att undersöka någon sorts medelvärde. I det här fallet kom datan från flera olika partiklar vilket gjorde att en direkt jämförelse av en undersökt parameter inte kunde göras; istället söktes först ett samband mellan storleken på partiklarna och den parameter man sökte.

%I datan fanns utöver position även en partikels intensitet i mikroskopet. Intensiteten berodde med största sannolikhet på partikelns storlek; Dock var det exakta sambandet inte helt klart vilket ger ännu en svårighet i hur datan ska analyseras. För de små partiklarna tordes intensiteten bero på volymen medan den för de större partiklarna mer borde gå mot att bero av arean. Detta då intensiteten är proportionell mot antalet ljusemitterande ämnen på partikeln som kameran ''ser''. För stora partiklar kan en del av dessa lysande ämnen döljas av andra så att kameran bara ser ljuset från den sida den är riktad mot. 
%Troligen går det dock att från intensiteterna kunna jämföra olika partiklar och på så sätt ändå kunna utnyttja den i jämförelser mellan olika partiklar. 


%Bara en liten kodsnutt som behövs när man kompilerar lokalt
%%% Local Variables: 
%%% mode: latex
%%% TeX-master: "main.tex"
%%% End: 

\input{cellbiologi.tex}



%\part{Huvuddel}

\chapter{Stokastiska processer och differentialekvationer}

För att beskriva de system som undersöks i det här arbetet
behöver man ta till stokastisk analys. I vanliga fall brukar
det räcka med ordinära eller partiella differential\-ekvationer (ODE:er
eller PDE:er) för att beskriva fysikaliska stystem. Tillräckligt små objekts beteende kommer dock att påverkas betydligt av termiska fluktuationer. Dessa termiska fluktuationer kan betraktas som helt
slumpmässiga, varför de då kan beskrivas med \emph{stokastiska processer}. Påverkan på ett system från en stokastisk process leder
till att den styrande differentialekvationen behöver modifieras med en stokastisk term,
det blir då en \emph{stokastisk differentialekvation} (SDE). Således 
ämnar följande avsnitt att introducera viktiga begrepp och metoder 
som senare används för att studera rörelsen av partiklar i celler samt 
strängar i vätskor. 

Ett exempel på när ett system består av så ''små objekt'' att termiska
fluktuationer behöver beaktas är i så kallad \emph{Brownsk rörelse}. 
Detta är ett fenomen där pollenkorn och andra små partiklar vandrar
runt tillsynes slumpmässigt på en vattenyta. Fenomenet beskrevs först av Robert Brown
1828~\cite{Brown1828}, men förklarades först senare av Einstein
1905~\cite{Einstein1905}. Förklaringen går ut på att pollenkornen är
små nog för att kollisioner med vattenmolekyler ska överföra
tillräckligt med rörelsemängd för att pollenkornens rörelseändring ska bli synbar med ett mikroskop. 


\section{Stokastiska processer}
En \emph{stokastisk variabel} $X$ är ett objekt som kan anta värden
$x$ från en viss värdemängd $\Omega$. Vilka värden som antas styrs av
sannolikhetsfördelningen $P(X=x)$. I fallet med diskreta stokastiska
variabler är sannolikhetsfördelningen helt enkelt sannolikheten att
$X$ antar värdet $x$. Men i det här arbetet ligger fokus på
kontinuerliga stokastiska variabler. För dessa gäller att sannolikheten för att X antar ett värde i intervallet $[x, x+\dd{x}]$ ges av
\begin{equation}
P(X\in[x, x+\dd{x}]) =p_X(x)\dd{x}
\end{equation}
för någon infinitesimal intervallbredd $\dd{x}$ och där $p_X$ är sannolikhetsfördelningen för $X$. 
I fortsättningen av detta arbete kommer ''stokastisk variabel'' att
avse en \emph{kontinuerlig} stokastisk variabel om inget annat anges.


Från detta kan en så kallad \emph{stokastisk process} definieras som en
samling av objekt som beror på en stokastisk variabel $X$ och en
deterministisk variabel, ofta betraktad som en tid\footnotemark{}
$t$. Speciellt brukar dessa objekt vara 
funktioner, $F_X(t)$. För ett givet värde $X=x$ blir alltså den
stokastiska processen en funktion $F_x(t)$, vilket medför att $F_X(t)$
definierar en samling av funktioner. 
\footnotetext{Att tiden väljs som deterministisk variabel är
  anledning till att det kallas stokastisk \emph{process}; man tänker
  sig att ett tidsförlopp som beror av den stokastiska variabeln
  utspelar sig. Mer generellt kan en godtycklig deterministisk
  variabel användas istället för tid.}  

\subsection{Statistiska verktyg för att undersöka stokastiska processer}
Stokastiska processer är som sagt \emph{slumpartade} processer. Därmed kan
det vara svårt att avgöra processens natur enbart utifrån ett fåtal
observationer. För att kunna undersöka en stokastisk process
behövs olika statistiska verktyg som exempelvis väntevärde, varians
och korrelation. 

För en stokastisk variabel $X$ kan man definiera ett \emph{väntevärde} med hjälp av variabelns sannolikhetsfördelning $P(x)$ enligt
\begin{equation}\label{eq:EV}
    \ev{X} = \int_{\Omega} x P(x) \id{x}.
\end{equation}
Något löst sett kan det betraktas som medelvärdet man förväntas få vid
upprepade mätningar av $X$. 
Väntevärdet går även att utvidga till att även omfatta funktioner av
den stokastiska variabeln. Man får då 
\begin{equation}\label{eq:EV_f}
    \ev{f(X)} = \int_{\Omega} f(x) P(x) \id{x}.
\end{equation}
Speciellt i fallet med stokastiska processer blir väntevärdet 
\begin{equation}\label{eq:EV_process}
    \ev{F_X(t)} = \int_{\Omega} F_x(t)P_X(x) \id{x}.
\end{equation}
Notera här att väntevärdet är beroende av $t$. 

Vidare definierar man det n:te momentet enligt 
\begin{equation}
    \ev{F(t_1)F(t_2)..F(t_n)} = \int_{\Omega} F(t_1)F(t_2)...F(t_n)P_X(x)dx.
\end{equation}
Om momentfunktionen är oberoende av en translation $t_i\to t_i+\tau$, där $i=1,2,..n$, för alla val av $n$ och $t_i$ definieras den stokastiska processen som \emph{stationär}. Speciellt har stationära processer ett  väntevärdet $\ev{F(t)}$ som är oberoende av $t$. 

Om väntevärdet är ett mått på vad man får som medelvärde, så behövs
även ett mått på hur spridda värden man kan tänkas få. För det används
\emph{variansen}, som går att formulera på några olika sätt
\begin{equation}\label{eq:VAR}
\sigma_X^2=\VAR{X} = \ev{\left(X-\ev{X} \right)^2} = \ev{X^2}-\ev{X}^2.
\end{equation}
Dock ger variansen, som man kan se, ett kvadratiskt mått på
avvikelser från medelvärdet. Därför kan det, exempelvis i sammanhang
där man vill jämföra spridningen i en mätserie, vara mer intressant
att betrakta \emph{standardavvikelsen} $\sigma_X$ som ges av roten ur
variansen.

Man kan på analogt sätt definiera en \emph{kovarians}
\begin{equation}\label{eq:COV}
\COV{X}{Y} 
= \Big\langle \big(\, X-\ev{X}\big)\;\big(\, Y-\ev{Y}\big) \Big\rangle
= \ev{XY}-\ev{X}\ev{Y}.
\end{equation}
Kovariansen är ett mått på hur mycket två stokastiska variabler
samvarierar. Speciellt syns också att $\COV{X}{X}=\VAR{X}$; alltså att
kovariansen övergår i variansen för $Y=X$. Vidare gäller att om
variablerna är \emph{statisktiskt oberoende} så är kovariansen 0.
%\todo{Ska detta betraktas som en definition av ''oberoende''?}
%\todo[color=green]{Tror inte kovariansen är noll medför stat oberoende. Bör definieras med betingad sannolikhet.}

I fallet med stokastiska processer kan det vara intressant att veta
hur korrelerade processerna är i till exempel tiden. För det används
korrelationsfunktionen 
\begin{equation}\label{eq:CORR}
c(t, t') = \frac{\COV{F_X(t)}{F_Y(t')}}{\sigma_{F_X}\sigma_{F_Y}}.
\end{equation}
Här har kovariansen delats med respektive standardavvikelse för att
korrelationsfuntionen ska ge ett värde mellan 
%\todo{Detta fås av Cauchy-Schwarzs olikhet. Ska man säga det?}\todo[color=green]{La till olikheten här}
$-1$ och $1$, där $1$ motsvarar perfekt korrelation och $-1$ perfekt antikorrelation, vilket följer av Cauchy-Schwarzs\cite{Engelberg_noise2007} olikhet 
\begin{equation}
\sigma_{F_X}^2\sigma_{F_Y}^2\geq \Big(\COV{F_X}{F_Y}\Big)^2.
\end{equation}

Om den stokastiska processen är stationär, vilket ofta gäller, innehar korrelationsfunktionen
translationssymmetri i $t$, varför man kan
ersätta de båda variablerna $t$ och $t'$ med deras differens $\Delta t$:
\begin{equation}
c(\Delta t) = C(t, t+\Delta t).
\end{equation}
Från korrelationsfunktionen för en stationär stokastisk process kan en karakteristisk tid definieras som den tid $\tau_k$ sådan att korrelationsfunktionen $c(\Delta t\geq\tau_k)$ blir försumbart liten. 

Vid studie av en stokastisk process med flera diskreta komponenter $F_i$ är det fördelaktigt att definiera en kovariansmatris som beskriver kovariansen mellan dessa komponenter. Kovariansmatrisen definieras enligt 
\begin{equation}
\label{eq:kovmatris}
    C_{ij}(t, t') = \COV{F_i(t)}{F_j(t')}.
\end{equation}
Diagonalen i $C_{ij}$ svarar så autokorrelationsfunktioner, det vill säga hur en stokastisk process korrelerar med sig själv, och avdiagonala element blir kovariansen mellan komponenenterna $F_i(t)$ och $F_j(t')$. Betraktas en stationär stokastisk process så reduceras kovariansmatrisen likt tidigare enligt $C_{ij}(t,t')\to C_{ij}(t,t+\Delta t)$ där $\Delta t = |t-t'|$. 


\subsubsection{Diskret data}
\label{seq:diskretdata}
Som man kan se så bygger alla dessa 
verktyg på olika väntevärden. Så för att kunna tillämpa
dessa statistiska metoder behövs ett stort statistiskt underlag baserat på
många observationer. Detta eftersom väntevärdet är det medelvärde
som förväntas av en variabel efter tillräckligt många observationer.

När man gör statistiska analyser använder man alltså medelvärdet i en
mätserie för att approximera väntevärden. Man får helt enkelt
\begin{equation}\label{eq:mean}
\ev{X} \approx \bar{x} = \frac{1}{N} \sum_{i=1}^N x_i,
\end{equation}
där $x_i$ är de olika observationerna av $X$ och N dess totala antal. 

När man ska beräkna variansen från ett stickprov kräver båda uttrycken
i \eqref{eq:VAR} två väntevärden. Om inte $\ev{X}$ är känt får man
alltså in två approximationer när man använder \eqref{eq:mean} för att
beräkna väntevärdena i \eqref{eq:VAR}. Detta bidrar bland annat till att en approximation av standardavvikelsen kan göras enligt
\begin{equation}
\sigma_X^2 \approx \bar{\sigma}_X^2
=  \frac{1}{N-1} \sum_{i=1}^N \left(x_i-\bar{x}\right)^2.
\end{equation}
Detta är standardfelet, $\bar{\sigma}_X$, i kvadrat beräknat med
Bessels korrektion~\cite{Rice_matstat2006}, som innebär att man använder ${N-1}$
i nämnaren istället för bara $N$. Approximationen av kovariansen
följer helt analogt från \eqref{eq:COV}.

\todo[inline]{Lägga kovariansmatrisen nedan under egen rubrik?}

Vidare kan kovariansmatrisen $C$ beräknas från observerad data av en samling stokastiska processer $A_i(t)$ där $i=1,2,..n$ och $n$ antal processer. För observerad data under en tid $T$ kan denna således beräknas enligt 
\begin{equation}
     \COV{A_i}{A_j} \approx \frac{1}{T-1}\sum_{t=1}^T \left(A_i-\bar{A_i}\right)\left(A_j-\bar{A_j}\right),
\end{equation}
från vilket det ses att $C$ är symmetrisk samt reell, givet att $A$ är en reell stokastisk process. Linjär algebra och spektralsatsen medför vidare att den allmänt icke-diagonala symmetriska samt reella kovariansmatrisen är diagonaliserbar. På matrisform fås kovariansmatrisen till att bli precis $C=AA^T$ där $A$ är en matris med de stokastiska processerna $A_i(t)$ som kolonnvektorer. Den diagonaliserade kovariansmatrisen fås då enligt 
\begin{equation}
    D = V^TCV,
\end{equation}
där $D$ är en diagonal matris med egenvärdena till $C$ och $V$ är en matris med egenvektorerna $b_\alpha$ som kolonner. De statiskt beroende variablerna $A_i(t)$ kan nu uttryckas i basen bestående av egenvektorer som
\begin{equation}
\label{eq:B}
    B_{\alpha}(t) = \sum_i V_{\alpha i}A_i(t),
\end{equation}
för $\alpha = 1,2,..n$. På matrisform kan detta beskrivas med matrisen $B_{\alpha t}$, där varje rad beskriver utvecklingen i tid av egenvektorerna till $C$. Betrakta nu kovariansen mellan komponenter till $B_{\alpha}$  enligt 
\begin{equation}
    \COV{B_{\alpha}}{B_{\beta}} = BB^T. 
\end{equation}
Från ekvation \ref{eq:B} ses att $B=V^TA$ vilket ger 
\begin{equation}
    \COV{B_{\alpha}}{B_{\beta}} = V^TAA^TV = D,
\end{equation}
där enligt tidigare $C=AA^T$, således ses att $B_{\alpha}(t)$ är statistiskt oberoende då alla avdiagonala element i kovariansmatrisen är noll och där de diagonala elementen, egenvärdena till kovariansmatrisen $C$, svarar mot variansen av $B_{\alpha}(t)$. Till skillnad från de stokastiska komponenterna $A_i$ som allmänt ej är statistiskt oberoende så är alltså $B_\alpha$ statistiskt oberoende. %Egenskaper hos kovariansmatrisen samt linjär algebra ger alltså ett recept på hur man utvinner statistiskt oberoende komponenter från en mängd statistiskt beroende. 
\todo[inline]{detta kanske inte bör ligga under diskret data. Tyckte dock det var lättast att ''visa'' att kovmatrisen var symmetrisk här. Kanske placera under strängar eller ev appendix ist?}

\subsection{Wienerprocessen}


\section{Stokastiska differentialekvationer}
En differentialekvation som innehåller termer bestående av stokastiska
processer kallas en stokastisk differentialekvation (SDE). Lösningen
till en SDE kommer representeras av en stokastisk process eftersom
minst en av de ingående termerna är stokastisk. Detta gör att man även
får en stokastisk utveckling av systemet. 

Inom fysiken modellerar man ofta system med fluktuationer genom
att betrakta tidsutvecklingen av ett system via dess styrande
differentialekvation. 
För att studera systemet under påverkan av en stokastisk fluktuation,
exempelvis brus i en elektrisk krets, lägger man till en term i
differentialekvationen som representerar fluktuationen. 
Detta kallas \emph{Langevinformalism} och motsvarande stokastiska
differentialekvation kallas systemets \emph{Langevinekvation}. 
Ett illustrerande exempel av Langevinformalismen är fallet för
Brownsk rörelse som beskrivs i avsnitt~\ref{sec:brown}.

Ett problem som här kan uppstå är att man inte känner till
fluktuationens fördelning. Oftast brukar man dock göra vissa
antaganden om fluktuationen, till exempel följande:
\begin{equation}\label{eq:white_noise}
\begin{aligned}
\ev{W_\omega(t)}&=0 \\
\ev{W_\omega(t)W_\omega(t')}&= \sigma_W^2 \, \delta(t-t'), \\
\end{aligned}
\end{equation}
\todo{Skriv om stycket!}
där $W_\omega(t)$ är den stokastiska process som beskriver
fluktuationen. Vad dessa antaganden säger är att medelvärdet av
$W_\omega(t)$ ska vara noll och att fluktuationerna ska vara oberoende i skilda tidpunkter. Vidare utgör faktiskt \eqref{eq:white_noise} definitionen 
\cite{Engelberg_noise2007} för \emph{vitt brus}. Så ur en fysikalisk synpunkt är detta ganska rimliga
antaganden för att ge en intuitiv bild av ''brus''. 

Antaganden om fluktationens 
väntevärde och korrelation men med en okänd sannolikhetsfördelning leder
till att lösningen till Langevinekvationen endast kan beskriva motsvarande
storheter. Alltså studeras under dessa antaganden oftast inte lösningen som sådan, utan 
istället motsvarande väntevärde samt 
korrelation för lösningen.  

%\subsubsection{Integrering}
%\todo[inline]{Hur och varför flytta in $\ev{\cdot}$ under integral.}





\subsection{Brownsk rörelse}\label{sec:brown}
Hastigheten för en partikel som utför ren Brownsk rörelse styrs av
Langevinekvationen~\cite{Mazo_Brownian2002} 
\begin{equation} \label{eq:Brownian_SDE}
    M\dv{v}{t}=-\zeta v + F(t),
\end{equation}
där $M$ är partikelmassan, $\zeta$ en friktionskonstant och $F(t)$ en
stokastisk, fluktuerande kraft. Kraften utgör här det stokastiska
bidraget och uppfyller egenskaperna för vitt brus enligt \eqref{eq:white_noise}.

Den fysikaliska tolkningen av denna stokastiska kraft är att partikeln
får små impulser från omgivande vätskepartiklar vilka kolliderar
slumpmässigt med partikeln.  
Denna kraftterm kan vidare tolkas som derivatan av en
Wienerprocess i gränsen då kollisionerna infaller med hög frekvens. En
Wienerprocess är en tidskontinuerlig stokastisk process där varje steg är oberoende av tidigare steg och är normalfördelat med väntevärde 0.
%Motivation att derivata av Wienerprocess s63

Lösningen till den stokastiska differentialekvationen
\eqref{eq:Brownian_SDE} ges av  
\begin{equation}
v(t)
=v(0)\ee^{-\nicefrac{\zeta t}{M}}
 +\frac{1}{M}\int^t_0 F(\tau)e^{-\nicefrac{\zeta (t-\tau)}{M}} \id\tau.
\end{equation}
Detta får dock inte den stokastiska termen att försvinna och lösningen kan inte skrivas på en deterministisk form. För att ändå kunna göra några förutsägelser kan man titta på väntevärdet och korrelationen i tiden. Betrakta därför följande korrelation, där $\delta t\geq0$,
\begin{equation}
\ev{v(t)v(t+\delta t)} 
= v(0)^2\ee^{-\nicefrac{\zeta}{M}(2t+\delta t)}
+ \frac{1}{M^2}\ee^{-\nicefrac{\zeta (2t+\delta t)}{M}}
  \int_0^t\int_0^{t+\delta t}\dd\tau\dd\tau'\, 
    \ee^{\nicefrac{\zeta (\tau+\tau')}{M}}\ev{F(\tau)F(\tau')}.
\end{equation}
Utnyttja att $F(t)$ är deltakorrelerad i tid vilket ger 
\begin{equation}
\ev{v(t)v(t+\delta t)} 
= v(0)^2\ee^{-\nicefrac{\zeta}{M}(2t+\delta t)}
 +\frac{\sigma^2}{M^2}\ee^{-\nicefrac{\zeta (2t+\delta t)}{M}}
  \int_0^t\dd\tau\ee^{\nicefrac{2\zeta \tau}{M}}.
\end{equation}
Korrelationen ovan kan nu enkelt beräknas och genom att låta $\delta t\to 0$ samt $t\to \infty$ fås följande samband
\begin{equation}
    \ev{v(t)^2} = \frac{\sigma^2}{2M\zeta}.
\end{equation}

Med hjälp av detta samband samt ekvipartitionsteoremet som gäller vid termisk jämvikt: $\frac{1}{2}M\ev{v^2}=\frac{1}{2}k_BT$, där $k_B$ är Boltzmanns konstant och $T$ är absoluta temperaturen, kan man nu relatera variansen $\sigma^2$ till fysikaliska storheter vilket ger 
\begin{equation}
    \sigma^2 = 2k_BT\zeta.
\end{equation}
Detta resultat är ett exempel på fluktuation-dissipationsteoremet som relaterar dissipationen av energi, friktionen, med fluktuationen av molekyler som träffar partikeln, \todo{Vad menar vi med detta sista? Saknas ett ord?} brownsk rörelse. 

För $t \gg \nicefrac{M}{\zeta}$ blir $\nicefrac{\dd{v}}{\dd{t}}$-termen i ekvation \eqref{eq:Brownian_SDE} \todo{Visa detta} försumbar och ekvationen kan då skrivas på formen
\begin{equation}
    \zeta \dv{x}{t}=F(t),
\end{equation}
där $x$ är partikelns position. Detta ger lösningen
\begin{equation}
    x(t)=x(0)+\frac{1}{\zeta} \int^t_0 F(\tau)\id\tau.
\end{equation}
Utifrån denna lösning kan medelvärdet av den kvadrerade avvikelsen beräknas, kallat ''mean squared displacement'' (MSD), vilken blir 
\begin{equation}\label{eq:MSD_brown}
    \ev{(x(t)-x(0))^2}=\frac{2k_BTt}{\zeta} \propto t,
\end{equation}
där enligt fluktuation-dissipationsteoremet  $\sigma^2=2k_BT\zeta$. MSD:n kommer därmed att öka linjärt med tiden, något som enkelt kan jämföras med uppmätt data.




%Bara en liten kodsnutt som behövs när man kompilerar lokalt
%%% Local Variables: 
%%% mode: latex
%%% TeX-master: "main.tex"
%%% End: 


\section{Bakgrund}

Bakgrund teori om cellen
\begin{itemize}
    \item Cytoplasmans uppbyggnad och struktur
    \item Brownsk rörelse, CTRW, Fractional Brownian motion
    \item Metabola tillståndets påverkan på partikelrörelsen
\end{itemize}
Bakgrund för dataanalysen
\begin{itemize}
    \item Skillnad mellan energydepleted och logphase (för alla nedan)
    \item Radius of gyration = Rörlighet
    \item MSD
    \item Anisotrop miljö
    \item Korrelationsfunktioner
\end{itemize}

\section{Dataanalys och resultat}

\begin{itemize}
    \item Anisotrop miljö
    \item Skillnad mellan energydepleted och logphase (för alla nedan)
    \item Radius of gyration = Rörlighet
    \item MSD(vilken modell passar bäst)
    \item Korrelationsfunktioner
\end{itemize}

\section{Diskussion och slutsats}



Bakgrund
\begin{itemize}
    \item WLC model, ev annan modell (Måns)
    \item Protein-filament
    \item Korrelationer, tangent, tid, rum
    \item Egenmoder
\end{itemize}

Resultat som kan tas med
\begin{itemize}
    \item Uppdelningen i egenmoder, olika relaxationstid
    \item Dispersionsrelation?
    \item Skillnad mellan confined och unconfined
    
\end{itemize}


\section{Teori}

\subsection{Proteinfilament}

Aktinfilament består av glubulära aktinprotein formade som bollar vilka kopplats ihop till en lång kedja.

\subsection{Modeller för strängrörelser}

\subsubsection{Worm Like Chain-model}

\subsubsection{Månsmodell}


\subsection{Egenomder}


\subsection{Lösning arv stokastisk differentialekvation}



\section{Tillhandahållen data}

Datan som analyserats i denna andra del av arbetet kommer från \todo{Var kommer datan från?}... och består av filmer av aktinfilament som tilåts röra sig i en vätska. Dessa strängar hade en storlek kring 10--30\,\micro{m} och befann sig i kanaler av olika bredd. Datan hade redan behandlats något så att strängens läge kunde beskrivas genom lägena för ett antal punker på strängen vilka angivna som pixlar i ett koordinatsystem. Genom att bildförstoringen och kamerans pixelstorlek var känd kunde de olika strängarnas längd beräknas. Mätningarna hade utförst på två ''fria'' strängar (strängar i breda skåror) och två strängar instängda i smala skåror, alla fyra filmade med 10 bilder per sekund. Rörelsen utfördes till största del i två dimensioner då skårornas djup var litet i förhållande till skårornas och filamentens bredd.

\section{Resultat}



\section{Diskussion och slutsats}





%Bara en liten kodsnutt som behövs när man kompilerar lokalt
%%% Local Variables: 
%%% mode: latex
%%% TeX-master: "main.tex"
%%% End: 


%\chapter{Resultat}

\section{Partikelrörelse i celler}

\subsection{Isometri}

%\subsection{Anisometri} Från givna data kunde en viss tendens till anisometri\todo{anisotropi?} anas då partiklarna tenderade att röra sig längs med vissa riktningar. En minstakvadratanpassning för en rät linje gjordes för att kunna transformera de givna $x$- och $y$-koordinaterna till tangent- och normalkoordinater relativt den anpassade linjen för varje partikel. För dessa nya koordinater beräknades en tidskorrelation vilket visas i \figref{fig:Korr_tn} för både aktiva celler och celler i dvala.

%Tillfällig bild, kan förbättras
%Bör definitivt ändras till eps eller pdf.
%\begin{figure}
%    \centering
%    \includegraphics[width=.8\textwidth]{Korrelation_tnkoordinat.png}
%    \caption{Korrelation i tangent- och normal-koordinaterna för aktiva celler och celler i dvala. I båda fall sjunker korrelationen initialt snabbare för cellerna i dvala. Korrelationen tycks även bestå längre i tangentialkoordinaten än normalkoordinaten vilket tyder på att det finns en föredragen väg.}
%    \label{fig:Korr_tn}
%\end{figure}

\subsection{Avvikelse från brownsk rörelse}
%Har vi exempel på avvikelse från teorin kan dessa läggas under seprata underrubriker här

\paragraph{MSD}
Utifrån teorin kring Brownsk rörelse kan vissa förutsägelser göras för partiklar som strikt följer denna typ av rörelse. Bland annat räknades det fram i avsnitt~\ref{sec:brown} att mean square displacement (MSD) för en sådan partikel skulle öka linjärt med tiden, se ekvation \todo{}\eqref{eq:MSD_brown}. Utifrån givna data för arbetet har ett potenssamband kunnat anpassas med en exponent som är skilld från och betydligt lägre än 1. Detta tyder på att partikeln inte utför en ren brownsk rörelse utan genomgår en så kallad subdiffusion, karakteriserat att partikelns MSD beror av tiden via ett potenssamband med exponent mindre än 1 men större än 0.

\begin{figure}
    \centering
    \includegraphics[width=.8\textwidth]{MSD_lilla_delta.png}
    \caption{Från filen MSD.m}
    \label{fig:MSD_ld}
\end{figure}

Vidare finns det minst två sätt att beräkna partiklarnas MSD. För stationära processer, dvs processer som inte explicit beror på när i tiden de inleds, kan man skapa ett medelvärde mellan alla möjliga mätpunkter separerade med givet tidsintervall så som beskrivet i ekvation ...
\todo{Kanske bör vi lägga till ett stycke om hur egenskaperna beräknas utifrån diskret data}
Genom att jämföra resultatet från denna typ av beräkning med att istället bara ta medelvärdet mellan alla partiklars kvadrerade radiella avvikelse från startpunkten vid given tid från start kan man avgöra om processen är stationär.
\todo{Bild för jämförelse mellan lilla och stora delta}
Båda dessa beräkningar har utförts för given data och exponentens värde i sambandet mellan MSD och tid skiljer sig/överensstämmer för lite för att någon tydlig slutsats hurvida processen är stationär eller ej ska kunna dras. \todo{Eller?}

\begin{figure}
    \centering
    \includegraphics[width=.8\textwidth]{MSD_stora_delta.png}
    \caption{Från filen storleksberoende.m}
    \label{fig:MSD_sd}
\end{figure}


%%%%%%%%%%%%%%%%%%%%%%%%%%%%%%Strängar%%%%%%%%%%%%%%%%%%%%%%%%%%%%%%


\section{Strängars rörelse i vätskor}




%Bara en liten kodsnutt som behövs när man kompilerar lokalt
%%% Local Variables: 
%%% mode: latex
%%% TeX-master: "main.tex"
%%% End: 



%\input{Obsolete/Diskussion.tex}


\chapter{Slutsats}



%Bara en liten kodsnutt som behövs när man kompilerar lokalt
%%% Local Variables: 
%%% mode: latex
%%% TeX-master: "main.tex"
%%% End: 




%%%%%%%%%%%%%%%%%%%%%%%%% Källförteckning %%%%%%%%%%%%%%%%%%%%%%%%%
\newpage
\bibliographystyle{ieeetr}
\bibliography{referenser_kandidat}

%%%%%%%%%%%%%%%%%%%%%%%%%%%%% Bilagor %%%%%%%%%%%%%%%%%%%%%%%%%%%%%
\clearpage
\appendix








\end{document}








%% På svenska ska citattecknet vara samma i både början och slut.
%% Använd två apostrofer (två enkelfjongar): ''.

%% Inkludera PDF-dokument
\includepdf[pages={1-}]{filnamn.pdf} %Filnamnet får INTE innehålla 'mellanslag'!

%% Figurer inkluderade som pdf-filer
\begin{figure}\centering
\centerline{ %centrerar även större bilder
\includegraphics[width=1\textwidth]{filnamn.pdf}
}
\caption{\label{fig:} }
\end{figure}

%% Figurer inkluderade med xfigs "Combined PDF/LaTeX"
\begin{figure}\centering
\resizebox{.8\textwidth}{!}{\input{filnamn.pdf_t}}
\caption{\label{fig:} }
\end{figure}

%% Figurer roterade 90 grader
\begin{sidewaysfigure}\centering
\centerline{ %centrerar även större bilder
\includegraphics[width=1\textwidth]{filnamn.pdf}
}
\caption{\label{fig:} }
\end{sidewaysfigure}


%%Om man vill lägga till något i TOC
\stepcounter{section} %Till exempel en 'section'
\addcontentsline{toc}{section}{\Alph{section}\hspace{8 pt}Labblogg} 



%Bara en liten kodsnutt som behövs när man kompilerar lokalt
%%% Local Variables: 
%%% mode: latex
%%% TeX-master: "main.tex"
%%% End: 