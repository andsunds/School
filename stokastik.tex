%stokastiska processer och Brownsk rörelse

För att beskriva beskriva de system som undersöks i det här arbetet
behöver man ta till stokastisk analys. I vanliga fall brukar
det räcka med ordinära eller partiella differential\-ekvationer (ODE:er
eller PDE:er) för att beskriva fysikaliska stystem. Men för exempelvis
små objekt kommer termiska fluktuationer att påverka deras
betenden. Dessa termiska fluktuationer kan anses vara helt
slumpmässiga, varför de kan betraktas som \emph{stokastiska
  processer}. Påverkan på ett system från en stokastisk process leder
till att den styrande DE:n behöver modifieras med en stokastisk term,
det blir då en \emph{stokastisk differentialekvation} (SDE).

Ett exempel på när ett system består av så ''små objekt'' att termiska
fluktuationer behöver beaktas är i så kallad \emph{Brownsk
  rörelse}. Detta är ett fenomen där pollenkorn på en vattenyta såg ut
att vandra runt slumpmässigt. Fenomenet först beskrevs av Robert Brown
1828~\cite{Brown1828}, men förklarades först av Einstein
1905~\cite{Einstein1905}. Förklaringen går ut på att pollenkornen är
små nog för att när vattenmolekylerna krockar med med så överförs
tillräckligt med rörelsemängd för pollenkornen ska ses flytta på sig. 

\section{Stokastiska processer}
En \emph{stokastisk variabel} $X$ är ett objekt som kan anta värden
$x$ från en viss värdemängd $\Omega$. Vilka värden som antas styrs av
sannolikhetsfördelningen $P_X(x)$. I fallet med diskreta stokastiska
variabler är sannolikhetsfördelningen helt enkelt sannolikheten att
$X$ antar värdet $x$. Men i det här arbetet ligger fokus på
kontinuerliga stokastiska variabler. För dessa gäller att 
\begin{equation}
P(X\in[x, x+\dd{x}]) =P_X(x)\dd{x}
\end{equation}
för någon infinitesimal intervallbredd $\dd{x}$. 
I fortsättningen av detta arbete kommer ''stokastisk variabel'' att
avse en \emph{kontinuerlig} stokastisk variabel om inget annat anges.


Från detta kan en så kallad \emph{stokastisk process} definieras som en
samling av objekt som beror på en stokastiska variabel $X$ och en
deterministisk variabel, oftast betrakad som tiden\footnotemark{}
$t$. Speciellt brukar dessa objekt vara 
funktioner, $f_X(t)$. För ett givet värde $X=x$ blir alltså den
stokastiska processen en funktion $f_x(t)$, vilket medför att $F_X(t)$
definierar en samling av funktioner. 
\footnotetext{Att tiden väljs som deterministisk variabel är
  anledning till att det kallas stokastisk \emph{process}; man tänker
  sig att ett tidsförlopp som beror av den stokastiska variabeln
  utspelar sig. Mer generellt kan en godtycklig deterministisk
  variabel användas istället för tid.}  

\subsection{Statisktiska verktyg för att undersöka stokastiska processer}
Stokastiska processer är som sagt slumpartade processer. Därmed kan
det vara svårt att avgöra processens natur utifrån enbart ett fåtal
observationer. För att kunna undersöka den stokastiska processen
behövs olika statistiska verktyg som exempelvis väntevärde, varians
och korrelation. 

För en stokastisk variabel $X$ och en dess sannolikhetsfördelning kan
man definiera dess \emph{väntevärde}
\begin{equation}
    \ev{X} = \int_{\Omega} x P(x) \id{x}.
\end{equation}
Något löst sett kan det betraktas som medelvärdet man förväntas få vid
upprepade mätningar av $X$. 
Väntevärdet går även att utvidga till att även omfatta funktioner av
den stokastiska variabeln. Man får då 
\begin{equation}
    \ev{f(X)} = \int_{\Omega} f(x) P(x) \id{x}.
\end{equation}
Speciellt i fallet med stokastiska processer blir väntevärdet 
\begin{equation}
    \ev{F_X(t)} = \int_{\Omega} F_x(t)P_X(x) \id{x}.
\end{equation}
Notera här att väntevärdet är beroende av $t$.

Om väntevärdet är ett mått på vad man får som medelvärde, så behövs
även ett mått på hur spridda värden man kan tänkas få. För det används
\emph{variansen}, som går att skriva på några olika sätt
\begin{equation}
\sigma_X^2=\VAR{X} = \ev{\left(X-\ev{X} \right)^2} = \ev{X^2}-\ev{X}^2.
\end{equation}
Dock ger variansen, som man kan se, ett kvadratiskt mått på
avvikelser från medelvärdet. Därför kan det, exempelvis i sammanhang
där man vill jämföra spridningen i en mätserie, vara mer intressant
att betrakta \emph{standardavvikelsen} $\sigma_X$ som ges av roten ur
variandsen.

Man kan på analogt sätt definiera en \emph{kovarians}
\begin{equation}
\COV{X}{Y}  = \ev{XY}-\ev{X}\ev{Y}.
\end{equation}
Kovariansen är ett mått på hur mycket två stokastiska variabler
samvarierar. 

I fallet med stokastiska processer kan det vara intressant att veta
hur korrelerade processerna är i till exempel tiden. För det används
korrelationsfunktionen 
\begin{equation}
C(t, t') = \frac{\COV{F_X(t)}{F_Y(t')}}{\sigma_{F_x}\sigma_{F_Y}}.
\end{equation}
Här har kovariansen delats med respektive standardavvikelse för att
korrelationsfuntionen ska ge ett värde mellan 
\todo{Detta fås av Cauchy-Schwarzs olikhet. Ska man säga det?}
$-1$ och $1$. 
Oftast brukar även translationssymmetri i $t$ gälla, varför man kan
ersätta de båda variablerna $t$ och $t'$ med deras skilnad:
\begin{equation}
C(\Delta t) = C(t, t+\Delta t).
\end{equation}


\subsubsection{Diskret data}

Som man kan se så bygger alla dessa 
\todo{Något annat än ''verktyg''?} 
verktyg på olika väntevärden. Man ser nu att för att kunna tillämpa
dessa statistiska metoder behövs ett stort statistiskt underlag av
många observationer. Detta eftersom väntevärdet är det medelvärde
som förväntas av en variabel efter tillräckligt måga observationer.





\subsection{Stokastiska differentialekvationer}
\todo[inline]{Gammalt}
En differentialekvation som innehåller termer med stokastiska processer
betecknas stokastiska differentialekvationer (SDE). Lösningen till en
SDE kommer representeras av en stokastisk process eftersom. Eftersom
många fysikaliska sy

Inom fysiken modellerar man ofta system med fluktuationer genom att betrakta
tidsutvecklingen av ett system via motsvarande differentialekvation
och man adderar sedan en stokastisk process för att representera
fluktuationen. 
Detta kallas Langevin formalism och motsvarande
stokastiska differentialekvation kallas systemets Langevin ekvation. Ett
illustrerande exempel av Langevin formalismen är fallet för brownsk
rörelse som beskrivs i avsnitt~\ref{sec:brown}.

Som tidigare nämnts så är den stokastiska
processen som beskriver systemets fluktuation oftast okänd, istället
antas att fluktuationen har vissa karakteristiska egenskaper. Exempel
på sådana egenskaper är att väntevärdet är $0$ enligt $\ev{F(t)} = 0$
eller att fluktuationen är okorrelerad i tiden
$\ev{F(t)F(t')}\propto\delta(t-t')$. 
Enligt 
\todo{Vaddå?}tidigare 
är lösningar till
stokastiska differentialekvationer stokastiska processer, och om
fluktuationen beskrivs med karakteristiska egenskaper och okänd
sannolikhetsfördelning kommer detta speglas i lösningen till
differentialekvationen. Således kommer inte sannolikhetsfördelningen
av lösning att kunna finnas, istället betraktar man motsvarande
karakteristiska egenskaper för lösningen som de hos
fluktuationen. 
\todo{De sista meningarna här  beskrivs nog tydligare
  med brownsk rörelse som exempel.} 






\section{Brownsk rörelse}\label{sec:brown}
\todo[inline]{Gammalt}
Hastigheten för en partikel som utför ren Brownsk rörelse styrs av
Langevinekvationen~\cite{Mazo_Brownian2002} 
\begin{equation} \label{eq:Brownian_SDE}
    M\dv{v}{t}=-\zeta v + F(t),
\end{equation}
där $M$ är partikelmassan, $\zeta$ en friktionskonstant och $F(t)$ en
fluktuerande kraft. Kraften utgör här det stokastiska bidraget till
differentialekvationen och är deltakorrelerat i tiden,
$\ev{F(t)F(t')}=\sigma^2\delta(t-t')$, 
med väntevärde 0, $\ev{F(t)}=0$. 
Det vill säga att den beter sig som vitt brus. 

Den fysikaliska tolkningen av denna stokastiska kraft är att partikeln
får små impulser från omgivande vätskepartiklar vilka kolliderar
slumpmässigt med den brownska partikeln.  Deltakorrelationen för
kraften uppkommer då impulserna modelleras som deltafunktioner i
tiden. Denna kraftterm kan vidare tolkas som derivatan av en
Wienerprocess i gränsen då kollisionerna infaller med hög frekvens. En
Wienerprocess är en tidskontinuerlig stokastisk process där varje
förändringssteg är oberoende av tidigare steg samtidigt som
\todo{Bättre ord än ökningar}ökningarna är normalfördelade med
väntevärde 0.
%Motivation att derivata av Wienerprocess s63

Lösningen till den stokastiska differentialekvationen \eqref{eq:Brownian_SDE} ges av \todo{Vill vi ändra till 2D?}
\begin{equation}
    v(t)=v(0)\ee^{-\nicefrac{\zeta t}{M}}+\frac{1}{M}\int^t_0 F(s)e^{-\nicefrac{\zeta (t-s)}{M}}ds.
\end{equation}
Detta får dock inte den stokastiska termen att försvinna och lösningen kan inte skrivas på en deterministisk form. För att ändå kunna göra några förutsägelser kan man titta på väntevärdet och korrelationen i tiden.

För $t \gg \nicefrac{\zeta}{M}$ blir $\nicefrac{\dd{v}}{\dd{t}}$-termen i ekvation \eqref{eq:Brownian_SDE} försumbar \todo{Visa detta?} och ekvationen kan då skrivas på formen
\begin{equation}
    \zeta \dv{x}{t}=F(t),
\end{equation}
där $x$ är partikelns position. Detta ger lösningen
\begin{equation}
    x(t)=x(0)+\frac{1}{\zeta} \int^t_0 F(s)ds.
\end{equation}
Utifrån denna lösning kan medelvärdet av den kvadrerade avvikelsen beräknas, kallat ''mean squared displacement'' (MSD), vilken blir 
\begin{equation}
    \ev{(x(t)-x(0))^2}=\frac{2k_BTt}{\zeta}
\end{equation}
där fluktuation-dissipationsteoremet gett att $\sigma^2=2k_BT\zeta$. \todo{ska vi visa flukt.diss.teoremet?} MSD:n kommer därmed att öka linjärt med tiden, något som enkelt kan jämföras med uppmätt data.




%Bara en liten kodsnutt som behövs när man kompilerar lokalt
%%% Local Variables: 
%%% mode: latex
%%% TeX-master: "main.tex"
%%% End: 