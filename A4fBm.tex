\chapter{Fördjupning i fBm och fGn}
\label{sec:App_fBm}

\section{Definition}

Varje fractional Brownian motion (fBm) är kopplat till en parameter H kallad Hurst parametern, $0<H<1$, som är med och avgör många av rörelsens egenskaper. Ett sätt att formellt representera fBm är att utgå från en vanlig brownsk rörelse $B(t)$ och vikta tidigare steg i rörelsen med $((t-s)^{H-\nicefrac{1}{2}})$, med $t$ och $s$ olika tidpunkter, så att fBm:n blir ett glidande medelvärde av $\dd{B}(t)$ enligt \cite{Mandelbrot_fBm1968}
\begin{equation} \label{eq:fBm_repr}
\begin{aligned}
    B_H(t,\omega)=& \frac{1}{\Gamma(H+\nicefrac{1}{2})}
    \int_{\infty}^0 \left( (t-s)^{H-\nicefrac{1}{2}}-(-s)^{H-\nicefrac{1}{2}}\right)\,\dd{B}(s,\omega) \\
    &+ \int_0^t (t-s)^{H-\nicefrac{1}{2}} \dd{B}(s,\omega)
\end{aligned}
\end{equation}
för $t>0$ utgående från startposition i 0 där $\omega$ representerar mängden av alla möjliga värden för den stokastiska funktionen och $\Gamma$ är gamma funktionen. För beskrivning för hur denna typ av integral med stokastisk integrationsvariabel kan beräknas, hänvisas till grundläggande lärobok i stokastisk analys.\todo[color=lime]{Kanske ge tips på bra bok?}
\todo[color=lime]{Onödigt formellt att presentera detta själva?}
%förklaras i avsnitt~\ref{sec:Stok_int}.
%Denna representation är dock inte unik \cite{Dieker_fBm}.

En normaliserad fBm: $B_H(t)$ karakteriseras fullständigt \cite{Dieker_fBm} av följande fem egenskaper
\begin{subequations}
\begin{align} 
    B_H(0)&=0 \\ 
    B_H(t+1)-B_H(t) &\sim N(0,1)\qcomma \forall t \label{eq:fBm_normstep}\\
    \ev{B_H(t)}&=0 \label{eq:fBm_mean} \\
    \ev{B^2_H(t)}&=t^{2H} \label{eq:fBm_var}\\
    \ev{(B_H(t_1)-B_H(t_2))^2}&=(t_1-t_2)^{2H}  \label{eq:fBm_stat_step}
\end{align}
\end{subequations}
för $t>0$ och $H$ den tidigare presenterade Hurst parametern. \todo[color=lime]{Visa varför detta måste gälla} Relationen i ekvation \eqref{eq:fBm_stat_step} medför att rörelsen har stationära ökningar. Steget mellan position $B_H(t)$ och $B_H(t+T)$
\begin{equation} \label{eq:fracGauss}
    \Delta(t) = B_H(t+T) - B_H(t)
\end{equation}
kallas \emph{fractional Gaussian noise} (fGn).

Om fBm:n inte normeras utan definieras direkt från \eqref{eq:fBm_repr} kan man visa att \cite{Dieker_fBm} variansen för \emph{icke-normerade} fBm:n $\hat{B}_H(t)$, motsvarande egenskap \eqref{eq:fBm_var}, blir 
\begin{equation} \label{eq:fBm_onorm_var}
    \ev{\hat{B}^2_H(t)} = V_H t^{2H},
\end{equation}
där $V_H = \Gamma(1-2H)\frac{\cos(\pi H)}{\pi H}$ \cite{Flandrin_fBmspektrum1989}.


\section{Korrelationer}

Kovariansen för fBm fås genom att utnyttja egenskaper för kovarians \eqref{eq:COV} enligt
\begin{equation}
    \COV{B_H(t)}{B_H(t+\Delta{t})}=\ev{B_H(t)B_H(t+\Delta{t})}-\ev{B_H(t)}\ev{B_H(t+\Delta{t})}=\ev{B_H(t)B_H(t+\Delta{t})},
\end{equation}
där $\ev{B_H(t)}=\ev{B_H(t+\Delta{t})}=0$ enligt ekvation \eqref{eq:fBm_mean}. För att beräkna det resterande väntevärdet för produkten kan man utnyttja sambandet $(B_H(t)-B_H(t+\Delta{t}))^2=B_H(t)^2 -2B_H(t)B_H(t+\Delta{t})+B_H(t+\Delta{t})^2$ för att istället beräkna
\begin{equation}
    \ev{B_H(t)B_H(t+\Delta{t})}
    =\frac{
    \ev{B_H(t)^2} + \ev{B_H(t+\Delta{t})^2} - \ev{(B_H(t)-B_H(t+\Delta{t}))^2}
    }{2}
\end{equation}
som med \eqref{eq:fBm_var} och \eqref{eq:fBm_stat_step} tillslut ger kovariansen
\begin{equation} \label{eq:fBm_cov}
    \COV{B_H(t)}{B_H(t+\Delta{t})} = \frac{1}{2}(t^{2H}+(t+\Delta{t})^{2H}-(\Delta{t})^{2H}).
\end{equation}
Kovariansen för en fBm är således tidsberoende och processen därmed icke-stationär. Detta resultat kan dock användas för att beräkna kovariansen mellan stegen, fGn, definierade i \eqref{eq:fracGauss}. På samma sätt som för kovariansen för fBm har stegen medelvärde 0 enligt  \eqref{eq:fBm_normstep} vilket ger
\begin{equation}
    \COV{\Delta(t_1)}{\Delta(t_2)} = \ev{(B_H(t_1+T)-B_H(t_1))(B_H(t_2+T)-B_H(t_2))}.
\end{equation}
där $T$ är tiden mellan två samplingspunkter. Om produkten utvecklas till fyra termer kan kovariansen i \eqref{eq:fBm_cov} användas för att få fram följande kovarians för fGn, med $t_2=t_1+\Delta{t}$
\begin{equation} \label{eq:fGn_cov}
    \COV{\Delta(t)}{\Delta(t+\Delta{t})} = \frac{1}{2}\left(-2\Delta{t}^{2H}+(\Delta{t}+T)^{2H}+(\Delta{t}-T)^{2H}\right) \equiv \gamma(\Delta{t})
\end{equation}
som endast beror av tidsseparationen mellan $t_1$ och $t_2$, vilket förväntas från en stationär fördelning.


%onormerade kovariansen mellan $\hat{B_H(t_1)}$ och $\hat{B_H(t_2)}$  


\section{PSD för icke-stationära processer}
\todo[color=lime]{Kolla så att övergången blir bra}
Som indikerat beror PSD beskriven i \eqref{eq:PSD} enbart på frekvensen och kan beräknas för stationära processer. Konceptet med PSD kan dock generaliseras till icke-stationära processer via Wigner-Ville-spektrumet (WVS) för en process $x(t)$ med kovarians funktion $\gamma_x(t_1,t_2)$ \cite{Flandrin_fBmspektrum1989}
\begin{equation} \label{eq:WVS}
V_x(t,\omega)=
\int^{\infty}_{-\infty} 
\gamma_x\!\left(t{+}\nicefrac{\tau}{2},\; t{-}\nicefrac{\tau}{2}\right) \ee^{-\ii\omega\tau} \dd\tau,
\end{equation}
WVS:en kan därmed vara tidsberoende. Det kan visas att WVS övergår i PSD för stationära processer. Detta möjliggör att PSD kan tas fram även för den icke-stationära fBm.

\section{PSD för fGn}

PSD för fGn kan räknas ut med hjälp av dess kovariansfunktion $\gamma(t_1,t_2)=\gamma(\Delta{t})$ i ekvation \eqref{eq:fGn_cov} med $t_1-t_2=\Delta{t}$ och via Wigner-Ville-spektrummet WVS i \eqref{eq:WVS}
\begin{equation}
    V_{\Delta}(t,\omega)
    =\int^{\infty}_{-\infty} 
    \gamma_{\Delta}
    \left(
    t+\frac{\tau}{2},t-\frac{\tau}{2}
    \right) 
    \ee^{-\ii\omega\tau} \dd\tau.
\end{equation}
fås med $\Delta{t}=t+\frac{\tau}{2}-(t-\frac{\tau}{2})=\tau$ vidare
\begin{equation} \label{eq:fGn_covterm}
    V_{\Delta}(t, \omega)
    =
    \int^{\infty}_{-\infty} 
    \gamma_{\Delta}(\tau) \ee^{-\ii\omega\tau} \dd\tau 
    = 
    \int^{\infty}_{-\infty} \frac{1}{2} 
    \left(
    -2\tau^{2H} + (\tau-T)^{2H} + (\tau+T)^{2H}
    \right) 
    \ee^{-\ii\omega\tau} \dd\tau.
\end{equation}
Via linjäritetsegenskaper kan nu integralens tre delar räknas var för sig där första termen ger
\begin{equation}
    \int^{\infty}_{-\infty}
    \frac{1}{2}\left(-2\tau^{2H}\right)
    \ee^{-\ii\omega\tau} \dd\tau 
    = \int^{\infty}_{-\infty} -\abs{\tau}^{2H}\ee^{-\ii\omega\tau} \dd\tau 
\end{equation}
vilket utgör fouriertransformen för $\abs{\tau}^{(2H+1)-1}$, $2H\in (0,1)$ reellt tal, vilken återfinns i \cite{BETA}
\begin{equation}
\begin{aligned}
    \int^{\infty}_{-\infty} -\abs{\tau}^{(2H+1)-1}\ee^{-\ii\omega\tau} \dd\tau 
    &= -\frac{
    2\Gamma(2H+1)
    \cos(\frac{(2H+1)\pi}{2})}{
    \abs{\omega}^{2H+1}} 
    \\ 
    &= -2\Gamma{(2H+1)} \cos\left(H\pi+\frac{\pi}{2}\right) \abs{\omega}^{-(2H+1)} 
    \\
    &= 2\Gamma{(2H+1)}\sin(H\pi) \equiv F.
\end{aligned}
\end{equation}
Andra termen kan uttryckas i termer av F med hjälp av variabelbytet $s=\tau+T$
\begin{equation}
    \int^{\infty}_{-\infty}\frac{1}{2}((\tau+T)^{2H})e^{-i\omega\tau} d\tau) = \frac{1}{2}\int^{\infty}_{-\infty} \abs{s}^{2H} \ee^{-\ii\omega(s-T)} \dd\tau = -\frac{1}{2} F \ee^{\ii\omega T} 
\end{equation}
och analogt fås för tredje termen: $-\frac{1}{2} F \ee^{-\ii\omega T}$. Läggs dessa tre termer ihop fås
\begin{equation}
\begin{aligned}
    V_{\Delta}(t,\omega)
    &= F\left(1-\frac{1}{2}(e^{i\omega T}+e^{-i\omega T})\right) 
    = F\Big( 1-\cos(\omega T) \Big) 
    = 2F\left(\sin(\frac{\omega T}{2})\right)^2 \\
    &= 
    4\Gamma{(2H+1)}
    \sin(H\pi)\left(\sin(\frac{\omega T}{2})\right)^2. \label{eq:fGn_PSD_halvklar}
\end{aligned}
\end{equation}
Genom att utnyttja följande egenskaper för gammafunktionen
\begin{align}
    \Gamma(1-z)\Gamma(z)&=\frac{\pi}{\sin(\pi z)} \\
    \Gamma(z+1)&=z\Gamma(z),
\end{align}
kan gammatermen i \eqref{eq:fGn_PSD_halvklar} skrivas om till
\begin{equation}
    \Gamma(2H+1) = 2 \Gamma(2H) = 2H\frac{\pi}{\sin(2H\pi)}\frac{1}{\Gamma(1-2H)} = \frac{H\pi}{\cos(H\pi)\sin(H\pi)\Gamma(1-2H)}.
\end{equation}
För en icke-normaliserad fBm fås en extra faktor $V_H$, se \eqref{eq:fBm_onorm_var}, i PSD:n för dess fGn: $\hat{\Delta}$ enligt
\begin{equation}
    V_{\hat{\Delta}}(t,\omega) = V_H 4\Gamma{(2H+1)}\sin(H\pi) \left( \sin(\frac{\omega T}{2}) \right) ^2 = 4 \left( \sin(\frac{\omega T}{2}) \right) ^2\abs{\omega}^{-(2H+1)},
\end{equation}
vilken saknar explicit tids beroende. För $\nicefrac{\omega T}{2} \ll 1$, det vill säga små $\omega$, kan den kvadrerade sinustermen Taylorutvecklas så att
\begin{equation}
    4 \ \frac{\left(\nicefrac{\omega T}{2} \right)^2 + \mathcal{O}(\omega^4) \right)} {\abs{\omega}^{(2H+1)}}
\end{equation}

\todo[color=lime]{Kanske även lägga till för fBm}

%För $H=\nicefrac{1}{2}$ återfås vanlig brownsk rörelse.




%Bara en liten kodsnutt som behövs när man kompilerar lokalt
%%% Local Variables: 
%%% mode: latex
%%% TeX-master: "00main.tex"
%%% End: 