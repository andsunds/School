\chapter{Fördjupning i fBm och fGn}
\label{sec:App_fBm}

\todo[inline]{Inga tomma rubriker}

\section{Definition}

Varje fractional Brownian motion (fBm) är kopplad till en parameter $H$ kallad Hurstparametern, $0<H<1$, som är med och avgör många av rörelsens egenskaper. Ett sätt att formellt representera fBm är att utgå från en vanlig brownsk rörelse $B(t)$, som matematiskt beskrivs av en Wienerprocess, och vikta tidigare steg i rörelsen med $(t-t')^{H-\nicefrac{1}{2}}$, dör $t$ och $t'$ är olika tidpunkter, så att fBm:n blir ett glidande medelvärde av $\dd{B}(t)$ enligt \cite{Mandelbrot_fBm1968}
\begin{equation} \label{eq:fBm_repr}
\begin{aligned}
    B_H(t,\Omega)=& \frac{1}{\Gamma(H+\nicefrac{1}{2})}
    \int_{\infty}^0 \left( \abs{t-t'}^{H-\nicefrac{1}{2}}-\abs{-t}^{H-\nicefrac{1}{2}}\right)\,\dd{B}(t',\Omega) \\
    &+ \int_0^t \abs{t-t'}^{H-\nicefrac{1}{2}} \dd{B}(t',\Omega)
\end{aligned}
\end{equation}
utgående från startposition i 0 där $\Omega$ representerar mängden av alla möjliga värden för den stokastiska funktionen och $\Gamma$ är gammafunktionen. För en beskrivning av hur denna typ av integral med stokastisk integrationsvariabel kan beräknas, hänvisas till någon grundläggande lärobok i stokastisk analys, exempelvis \cite{Oksendal2002}.

%Denna representation är dock inte unik \cite{Dieker_fBm}.

En normaliserad fBm, $B_H(t)$, karakteriseras fullständigt av följande fem egenskaper~\cite{Dieker_fBm}
\begin{subequations}
\begin{align} 
    B_H(0)&=0 \\ 
    B_H(t+1)-B_H(t) &\sim N(0,1)\qcomma \forall t \label{eq:fBm_normstep}\\
    \ev{B_H(t)}&=0 \label{eq:fBm_mean} \\
    \ev{B^2_H(t)}&=\abs{t}^{2H} \label{eq:fBm_var}\\
    \ev{(B_H(t_1)-B_H(t_2))^2}&=\abs{t_1-t_2}^{2H}  \label{eq:fBm_stat_step}
\end{align}
\end{subequations}
där $t>0$ och $H$ är den tidigare presenterade Hurstparametern. \todo[color=lime]{Visa varför detta måste gälla} Relationen i \eqref{eq:fBm_stat_step} medför att rörelsen har stationära steg. Stegen, som funktion av tid, mellan positionerna $B_H(t)$ och $B_H(t+T)$
\begin{equation} \label{eq:fracGauss}
    \delta(t) = B_H(t+T) - B_H(t)
\end{equation}
kallas \emph{fractional Gaussian noise} (fGn). Här är $T$ tiden mellan två samplingar i en mätserie. 

Om fBm:n inte normeras utan definieras direkt från \eqref{eq:fBm_repr} kan man visa att variansen för \emph{icke-normerade} fBm:n $\hat{B}_H(t)$, motsvarande egenskap \eqref{eq:fBm_var}, blir \cite{Dieker_fBm}
\begin{equation} \label{eq:fBm_onorm_var}
    \ev{\hat{B}^2_H(t)} = V_H \abs{t}^{2H},
\end{equation}
där $V_H = \Gamma(1-2H)\frac{\cos(\pi H)}{\pi H}$ \cite{Flandrin_fBmspektrum1989}.


\section{Korrelationer}
\todo[color=lime]{Ändra $\gamma$ och $\Delta$}
Kovariansen för fBm fås genom att utnyttja egenskaper för kovarians \eqref{eq:COV} enligt
\begin{equation}
\begin{aligned}
    \COV{B_H(t)}{B_H(t+\Delta{t})}&=\ev{B_H(t)B_H(t+\Delta{t})}-\ev{B_H(t)}\ev{B_H(t+\Delta{t})}
    \\
    &=\ev{B_H(t)B_H(t+\Delta{t})},
\end{aligned}
\end{equation}
där $\ev{B_H(t)}=\ev{B_H(t+\Delta{t})}=0$ enligt \eqref{eq:fBm_mean}. För att beräkna det resterande väntevärdet av produkten kan man utnyttja sambandet $(B_H(t)-B_H(t+\Delta{t}))^2=B_H(t)^2 -2B_H(t)B_H(t+\Delta{t})+B_H(t+\Delta{t})^2$ för att få
\begin{equation}
    \ev{B_H(t)B_H(t+\Delta{t})}
    =\frac{
    \ev{B_H(t)^2} + \ev{B_H(t+\Delta{t})^2} - \ev{(B_H(t)-B_H(t+\Delta{t}))^2}
    }{2}.
\end{equation}
Med \eqref{eq:fBm_var} och \eqref{eq:fBm_stat_step} fås tillslut kovariansen
\begin{equation} \label{eq:fBm_cov}
    \COV{B_H(t)}{B_H(t+\Delta{t})} = \frac{1}{2}\left(\abs{t}^{2H}+\abs{t+\Delta t}^{2H}-\abs{\Delta{t}}^{2H}\right).
\end{equation}
Kovariansen för en fBm är alltså tidsberoende och processen är därmed icke-stationär. 

Detta resultat kan dock användas för att beräkna kovariansen mellan stegen, fGn, definierade i \eqref{eq:fracGauss}. På samma sätt som för kovariansen av fBm har stegen medelvärde 0 enligt  \eqref{eq:fBm_normstep} vilket ger
\begin{equation}
    \COV{\delta(t_1)}{\delta(t_2)} = \Big\langle(B_H(t_1+T)-B_H(t_1))(B_H(t_2+T)-B_H(t_2))\Big\rangle.
\end{equation}
Här är $T$ tiden mellan två samplingspunkter. Om produkten utvecklas till fyra termer kan kovariansen i \eqref{eq:fBm_cov} användas för att få fram följande kovarians för fGn:
\begin{equation} \label{eq:fGn_cov}
    \COV{\delta(t)}{\delta(t+\Delta{t})} 
    = \frac{1}{2}
    \left(-2\abs{\Delta{t}}^{2H}+\abs{\Delta{t}+T}^{2H}+\abs{\Delta{t}-T}^{2H}\right) 
    \equiv \kappa(\Delta{t}),
\end{equation}
där $\Delta{t}=\abs{t_1+t_2}$.
Detta beror enbart av tidsseparationen mellan $t_1$ och $t_2$, vilket förväntas från en stationär fördelning.


\section{PSD för icke-stationära processer}
\todo[color=lime]{Ändra från WVS till S?}
Som man kan se i avsnitt~\ref{sec:PSD} beror PSD:n enbart på frekvensen och kan beräknas för stationära processer. Konceptet med PSD kan dock generaliseras till icke-stationära processer via Wigner-Ville-spektrumet (WVS) för en process $x(t)$ med kovariansfunktion $\gamma_x(t_1,t_2)$ \cite{Flandrin_fBmspektrum1989}. Wigner-Ville-spektrumet definieras som
\begin{equation} \label{eq:WVS}
V_x(t,\omega)=
\int^{\infty}_{-\infty} 
\kappa_x\left(t{+}\nicefrac{\tau}{2},\; t{-}\nicefrac{\tau}{2}\right) \ee^{-\ii \omega \tau} \dd\tau.
\end{equation}
Alltså kan WVS:en vara tidsberoende. 

Det går att visa att WVS övergår i PSD för stationära processer. Detta möjliggör att PSD kan tas fram även för den icke-stationära fBm.


\section{PSD för fGn}
\label{sec:PSD_fGn}

PSD för fGn kan räknas ut med hjälp av dess kovariansfunktion $\kappa(t_1,t_2)=\kappa(\Delta{t})$ i ekvation \eqref{eq:fGn_cov} med $\Delta{t}=t_1-t_2$ och via Wigner-Ville-spektrummet i \eqref{eq:WVS}\todo[color=cyan]{Stökig mening}
\begin{equation}
    V_{\delta}(t,\omega)
    =\int^{\infty}_{-\infty} 
    \kappa_{\delta}
    \left( t+\frac{\tau}{2},t-\frac{\tau}{2} \right) 
    \ee^{-\ii \omega\tau} \dd\tau
\end{equation}
\todo[color=cyan]{?}fås med $\Delta{t}=t+\frac{\tau}{2}-(t-\frac{\tau}{2})=\tau$ vidare
\begin{equation} \label{eq:fGn_covterm}
    V_{\delta}(t, \omega)
    =
    \int^{\infty}_{-\infty} 
    \gamma_{\delta}(\tau) \ee^{-\ii \omega\tau} \dd\tau 
    = 
    \int^{\infty}_{-\infty} \frac{1}{2} 
    \left(
    -2\abs{\tau}^{2H} + \abs{\tau-T}^{2H} + \abs{\tau+T}^{2H}
    \right) 
    \ee^{-\ii \omega\tau} \dd\tau.
\end{equation}
Här kan noteras att WVS:n nu gått över till PSD för en stationärprocess så att $V_{\delta(\omega)}=S(\omega)$ från Wiener-Khinthchine-teoremet i \eqref{eq:W-K-theorem}. Via linjäritetsegenskaper kan integralens tre delar beräknas var för sig. Den första termen utgör fouriertransformen för $\abs{\tau}^{(2H+1)-1}$, $2H\in (0,1)$, vilken återfinns i \cite{BETA}
\begin{equation}
\begin{aligned}
    \int^{\infty}_{-\infty} -\abs{\tau}^{(2H+1)-1}\ee^{-\ii \omega\tau} \dd\tau 
    &= -\frac{
    2\Gamma(2H+1)
    \cos(\frac{(2H+1)\pi}{2})}{
    \abs{\omega}^{2H+1}} 
    \\ 
    &= -2\Gamma{(2H+1)} \cos\left(H\pi+\frac{\pi}{2}\right) \abs{\omega}^{-(2H+1)} 
    \\
    &= 2\Gamma{(2H+1)}\sin(H\pi) \abs{\omega}^{-(2H+1)} \equiv F.
\end{aligned}
\end{equation}
Andra termen kan uttryckas i termer av $F$ med hjälp av variabelbytet $t'=\tau+T$
\begin{equation}
\int^{\infty}_{-\infty}\frac{1}{2} \abs{\tau+T}^{2H})\ee^{-\ii \omega\tau} \dd\tau 
= \frac{1}{2}\int^{\infty}_{-\infty} \abs{t'}^{2H} \ee^{-\ii \omega(t'-T)} \dd\tau 
= -\frac{1}{2} F \ee^{\ii \omega T}.
\end{equation}\todo[color=cyan]{Vart tar $t'$ vägen efter integralen? Det bakas in i F}
Analogt för tredje termen fås $-\frac{1}{2}F\ee^{-\ii \omega T}$. 

Läggs dessa tre termer ihop fås
\begin{equation}
\begin{aligned}
    V_{\delta}(t,\omega)
    &= F\left(1-\frac{1}{2}\left(\ee^{\ii \omega T}+\ee^{-\ii \omega T}\right)\right)
    = F\Big( 1-\cos(\omega T) \Big) 
    = 2F\left(\sin(\frac{\omega T}{2})\right)^2 \\
    &= 
    4\Gamma{(2H+1)}
    \sin(H\pi)\left(\sin(\frac{\omega T}{2})\right)^2 \abs{\omega}^{-(2H+1)}. \label{eq:fGn_PSD_halvklar}
\end{aligned}
\end{equation}
Genom att utnyttja följande egenskaper för gammafunktionen
\begin{align}
\Gamma(1-z)\Gamma(z)&=\frac{\pi}{\sin(\pi z)} \\
\Gamma(z+1)&=z\Gamma(z),
\end{align}
kan gammafaktorn i \eqref{eq:fGn_PSD_halvklar} skrivas om till
\begin{equation}
\Gamma(2H+1) 
= 2 \Gamma(2H) = 2H\frac{\pi}{\sin(2H\pi)}\frac{1}{\Gamma(1-2H)} 
= \frac{H\pi}{\cos(H\pi)\sin(H\pi)\Gamma(1-2H)}.
\end{equation}

För en icke-normerad fBm fås en extra faktor $V_H$, se \eqref{eq:fBm_onorm_var}, i PSD:n för dess fGn: $\hat{\delta}$ enligt
\begin{equation}
V_{\hat{\delta}}(t,\omega) 
= V_H 4\Gamma{(2H+1)}\sin(H\pi) \left( \sin(\frac{\omega T}{2}) \right)^2 \abs{\omega}^{-(2H+1)} 
= 4 \left( \sin(\frac{\omega T}{2}) \right) ^2\abs{\omega}^{-(2H+1)},
\end{equation}
vilken saknar explicit tidsberoende. För $\nicefrac{\omega T}{2} \ll 1$, det vill säga små $\omega$, kan den kvadrerade sinustermen Taylorutvecklas så att
\begin{equation}
    V_{\hat{\delta}}(t,\omega) \approx \frac{ \left( \nicefrac{\omega T}{2} \right)^2 + \mathcal{O}(\omega^4) }{\abs{\omega}^{(2H+1)}}.
\end{equation}
Olika beteende för olika värde på Hurstparametern $H$ framträder nu 
\begin{equation}
    \lim\limits_{\omega \to 0} V_{\hat{\Delta}}(t,\omega) = \begin{cases} 
      0 & H<\nicefrac{1}{2} \\
      \nicefrac{T^2}{4} & H=\nicefrac{1}{2} \\
      \infty & H>\nicefrac{1}{2}
   \end{cases}
\end{equation}
vilket framträder tydligt om PSD:n plottas i dB mot frekvensen.

Tillsist kan nämnas att för en process som förutsäger en kovarians $\propto t^{\alpha}$ ger \eqref{eq:fGn_covterm}-\eqref{eq:fGn_PSD_halvklar} att motsvarande PSD $\propto \omega^{-(\alpha+1)}$.

\todo[color=lime]{Kanske även lägga till för fBm}

\todo[color=lime]{Kanske något om att ej deriverbar}




%Bara en liten kodsnutt som behövs när man kompilerar lokalt
%%% Local Variables: 
%%% mode: latex
%%% TeX-master: "00main.tex"
%%% End: 