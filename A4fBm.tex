\chapter{fBm och fGn}
\todo{Förkortningar i rubriken ok?}
\label{sec:App_fBm}

\section{Definition}

En normaliserad fBm: $B_H(t)$ karakteriseras fullständigt \cite{Dieker_fBm} av följande fem egenskaper
\begin{subequations}
\begin{align} 
    B_H(0)&=0 \\ 
    B_H(t+1)-B_H(t) &\sim N(0,1), \forall t \label{eq:fBm_normstep}\\
    \ev{B_H(t)}&=0 \label{eq:fBm_mean} \\
    \ev{B^2_H(t)}&=t^{2H} \label{eq:fBm_var}\\
    \ev{(B_H(t_1)-B_H(t_2))^2}&=(t_1-t_2)^{2H}  \label{eq:fBm_stat_step}\\
\end{align}
\end{subequations}
för $t>0$ där Hurst parametern $H$ uppfyller $0< H <1$. \todo{Visa varför detta måste gälla} Relationen i ekvation \eqref{eq:fBm_stat_step} medför att rörelsen har stationära ökningar. Steget mellan position $B_H(t)$ och $B_H(t+T)$
\begin{equation} \label{eq:fracGauss}
    \Delta(t) = B_H(t+T) - B_H(t)
\end{equation}
kallas \emph{fractional Gaussian noise} (fGn).

Om fBm:n inte normeras utan definieras direkt från \eqref{eq:fBm_repr} kan man visa att \cite{Dieker_fBm} variansen för onormerade $\hat{B}_H(t)$, motsvarande egenskap \eqref{eq:fBm_var}, blir 
\begin{equation} \label{eq:fBm_onorm_var}
    \ev{\hat{B}^2_H(t)} = V_H t^{2H},
\end{equation}
där $V_H = \Gamma(1-2H)\frac{\cos(\pi H)}{\pi H}$ \cite{Flandrin_fBmspektrum1989}.


\section{Korrelationer}

Kovariansen för fBm fås genom att utnyttja egenskaper för kovariansen \eqref{eq:COV} enligt
\begin{equation}
    \COV{B_H(t)}{B_H(t+\Delta{t})}=\ev{B_H(t)B_H(t+\Delta{t})}-\ev{B_H(t)}\ev{B_H(t+\Delta{t})}=\ev{B_H(t)B_H(t+\Delta{t})},
\end{equation}
där $\ev{B_H(t)}=\ev{B_H(t+\Delta{t})}=0$ enligt ekvation \eqref{eq:fBm_mean}. För att beräkna det resterande väntevärdet för produkten kan man utnyttja sambandet $(B_H(t)-B_H(t+\Delta{t}))^2=B_H(t)^2 -2B_H(t)B_H(t+\Delta{t})+B_H(t+\Delta{t})^2$ för att istället beräkna
\begin{equation}
    \ev{B_H(t)B_H(t+\Delta{t})}=\frac{1}{2}(\ev{B_H(t)^2}+\ev{B_H(t+\Delta{t})^2}-\ev{(B_H(t)-B_H(t+\Delta{t}))^2}) 
\end{equation}
som med ekvation \eqref{eq:fBm_var} och \eqref{eq:fBm_stat_step} tillslut ger kovariansen
\begin{equation} \label{eq:fBm_cov}
    \COV{B_H(t)}{B_H(t+\Delta{t})} = \frac{1}{2}(t^{2H}+(t+\Delta{t})^{2H}-(\Delta{t})^{2H}).
\end{equation}
Kovariansen för en fBm är således tidsberoende och processen därmed icke-stationär. Detta resultat kan dock användas för att beräkna kovariansen mellan stegen, fGn, definierade i \eqref{eq:fracGauss}. På samma sätt som för kovariansen för fBm har stegen medelvärde 0 enligt ekvation \eqref{eq:fBm_normstep} vilket ger
\begin{equation}
    \COV{\Delta(t)}{\Delta(s)} = \ev{(B_H(t+T)-B_H(t))(B_H(s+T)-B_H(s))}.
\end{equation}
Om produkten utvecklas till fyra termer kan kovariansen i \eqref{eq:fBm_cov} användas för att få fram följande kovarians för fGn, med $s=t+\Delta{t}$
\begin{equation} \label{eq:fGn_cov}
    \COV{\Delta(t)}{\Delta(t+\Delta{t})} = \frac{1}{2}\left(-2\Delta{t}^{2H}+(\Delta{t}+T)^{2H}+(\Delta{t}-T)^{2H}\right)
\end{equation}
som är oberoende av t, vilket förväntas av en stationär fördelning.


%onormerade kovariansen mellan $\hat{B_H(t)}$ och $\hat{B_H(s)}$  

\section{PSD för fGn}

PSD för fGn kan räknas ut med hjälp av dess kovariansfunktion $\gamma(t_1,t_2)=\gamma(\Delta{t})$ i ekvation \eqref{eq:fGn_cov} med $t_1-t_2=\Delta{t}$ och via WVS i \eqref{eq:WVS} fås
\begin{equation}
    W_{\Delta}(t,\omega)=\int^{\infty}_{-\infty} \gamma_{\Delta}(t+\frac{\tau}{2},t-\frac{\tau}{2}) e^{-i\omega\tau} d\tau.
\end{equation}
Med $\Delta{t}=t+\frac{\tau}{2}-(t-\frac{\tau}{2})=\tau$ fås vidare
\begin{equation} \label{eq:fGn_covterm}
    W_{\Delta}(t,\omega)=\int^{\infty}_{-\infty} \gamma_{\Delta}(\tau) e^{-i\omega\tau} d\tau = \int^{\infty}_{-\infty}\frac{1}{2}(-2\tau^{2H} + (\tau-T)^{2H} + (\tau+T)^{2H}) e^{-i\omega\tau} d\tau.
\end{equation}
Via linjäritetsegenskaper kan nu integralens tre delar räknas var för sig där första termen ger
\begin{equation}
    \int^{\infty}_{-\infty}\frac{1}{2}(-2\tau^{2H})e^{-i\omega\tau} d\tau = \int^{\infty}_{-\infty} -\abs{\tau}^{2H}e^{-i\omega\tau} d\tau 
\end{equation}
vilket utgör fouriertransformen för $\abs{\tau}^{(2H+1)-1}$, $2H\in (0,1)$ reellt tal, vilken återfinns i \cite{BETA}
\begin{align}
    \int^{\infty}_{-\infty} -\abs{\tau}^{(2H+1)-1}e^{-i\omega\tau} d\tau = -\frac{2\Gamma{(2H+1)}\cos{\frac{(2H+1)\pi}{2}}}{\abs{\omega}^{2H+1}} \\ 
    = -2\Gamma{(2H+1)} \cos{(H\pi+\frac{\pi}{2})} \abs{\omega}^{-(2H+1)} = 2\Gamma{(2H+1)}\sin(H\pi) \equiv F.
\end{align}
Andra termen kan uttryckas i termer av F med hjälp av variabelbytet $s=\tau+T$
\begin{equation}
    \int^{\infty}_{-\infty}\frac{1}{2}((\tau+T)^{2H})e^{-i\omega\tau} d\tau) = \frac{1}{2}\int^{\infty}_{-\infty} \abs{s}^{2H} e^{-i\omega(s-T)} d\tau = -\frac{1}{2} F e^{i\omega T} 
\end{equation}
och analogt fås för tredje termen: $-\frac{1}{2} F e^{-i\omega T}$. Läggs dessa tre termer ihop fås
\begin{align}
    W_{\Delta}(t,\omega)&=F(1-\frac{1}{2}(e^{i\omega T}+e^{-i\omega T})) = F(1-\cos(\omega T)) = 2F(\sin(\frac{\omega T}{2}))^2 \\
    &= 4\Gamma{(2H+1)}\sin(H\pi)(\sin(\frac{\omega T}{2}))^2. \label{eq:fGn_PSD_halvklar}
\end{align} \todo{Fixa så att inte två ref}
Genom att utnyttja följande egenskaper för gammafunktionen
\begin{align}
    \Gamma(1-z)\Gamma(z)&=\frac{\pi}{\sin(\pi z)} \\
    \Gamma(z+1)&=z\Gamma(z),
\end{align}
kan gamma-termen i \eqref{eq:fGn_PSD_halvklar} skrivas om till
\begin{equation}
    \Gamma(2H+1) = 2 \Gamma(2H) = 2H\frac{\pi}{\sin(2H\pi)}\frac{1}{\Gamma(1-2H)} = \frac{H\pi}{\cos(H\pi)\sin(H\pi)\Gamma(1-2H)}.
\end{equation}
För en icke-normaliserad fBm fås en extra faktor $V_H$ från \eqref{eq:fBm_onorm_var} i PSD:n för dess fGn $\hat{\Delta}$ enligt
\begin{equation}
    W_{\hat{\Delta}}(t,\omega) = V_H 4\Gamma{(2H+1)}\sin(H\pi)(\sin(\frac{\omega T}{2}))^2 = 4 (\sin(\frac{\omega T}{2}))^2
\end{equation}

\todo{Kanske även lägga till för fBm}

%För $H=\nicefrac{1}{2}$ återfås vanlig brownsk rörelse.