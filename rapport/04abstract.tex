\cleardoublepage
\thispagestyle{plain}


\begin{abstract}
Denna rapport presenterar statistiska undersökningar av partiklars och aktinfilaments rörelser i jästceller respektive mikrokanaler i en vätska. Det är intressant att studera dynamiken för partiklar och strängar i cellers cytoplasma eftersom de är sammanlänkade med transport av proteiner och andra molekyler inuti cellen. Partikelstudierna utgår teoretiskt främst från modellerna continuous time random walk (CTRW) och fractional Brownian motion (fBm). Sedan undersöks partikelrörelserna genom att mäta diffusionshastigheten genom cellen, men även eventuella anisotropier och stegens frekvensspektrum studeras. 
Undersökningen av aktinfilaments dynamik utgår från worm-like chain-modellen (WLC), där fria och instängda filament studeras separat. Huvudsakligen undersöks filamentens styvhet samt sambandet mellan rumsliga svängningar och relaxationstid.
Partikelstudien drar slutsatsen att ingen av modellerna för partikelrörelse ger en fullständig beskrivning av de observerade egenskaperna. Vidare verkar partiklarna uppleva en isotrop miljö inuti jästcellen, vilket tyder på att det inte finns några asymmetriska strukturer på den undersökta längdskalan. 
För strängarna visas att den observerade styvheten är större för instängda filament, vilket var väntat. Vidare erhölls ett avtagande samband mellan vågtalet och relaxationstiden, som också förutsades av WLC-modellen; dock gick det inte att kvantitativt bekräfta det teoretiska sambandet. Filamentens svängningar visas även kunna representeras väl av okorrelerade egenmoder, men fortsatta teoretiska studier krävs för att kunna dra slutsatser från dessa resultat. 

%Dessa egenmoder tas fram genom diagonalisering av en kovariansmatris med beteende likt harmoniska svängningar.

\end{abstract}


\begin{otherlanguage}{english}
\begin{abstract}

This report presents statistical studies on particle and actin filament motion in yeast cells and microchannels respectively. Particle and filament dynamics is closely related to transport of proteins and other molecules through the cell cytoplasm, which encourage studies on the subject.
Continuous time random walk (CTRW) and fractional Brownian motion (fBm) provides the theoretical foundation for the analysis of the particle motion. This is primarily done by measuring the rapidity of the particle diffusion through the cell and the step's frequency spectrum as well as investigating potential anistropic behaviour within the cell.
The study on actin filament dynamics is mainly based on the worm-like chain model (WLC), where free and confined filaments have been examined separately. The study focuses on certain features of the filaments such as their rigidity and the relation between spatial oscillations and relaxation time. 
The study on particle dynamics draws the conclusion that none of the particle motion models completely describes the observed characteristics. What is also found is that the particles appears to experience an isotropic environment, indicating an absence of asymmetrical structures on the studied length scale.
The observed rigidity of the strings is shown to be larger for the confined strings, which was expected. Furthermore, a decreasing relation between wavenumber and relaxation time was found, which was predicted by the WLC-model; the theoretical prediction could not be quantitatively confirmed. The oscillations of the filaments is shown to be well represented by uncorrelated eigenmodes, but further theoretical studies are needed to interpret these results. 


\end{abstract}
\end{otherlanguage}

\clearpage
\thispagestyle{plain}

%Bara en liten kodsnutt som behövs när man kompilerar lokalt
%%% Local Variables: 
%%% mode: latex
%%% TeX-master: "00main.tex"
%%% End: 