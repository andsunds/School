\renewcommand{\thefootnote}{\fnsymbol{footnote}}

%kortkommandon för mailaddresserna
\newcommand{\andsunds}{andsunds@student.chalmers.se}
\newcommand{\rigon}{rigon@student.chalmers.se}



\pagenumbering{roman} %%Romersk sidnumrering i början
\begin{titlepage}
\newgeometry{top=3cm, bottom=2cm}

\newcommand{\HRule}{\rule{\linewidth}{0.5mm}} % Defines a new command for the horizontal lines, change thickness here

\center % Center everything on the page
 
%------------------------------------------------------------------------------------
%	HEADING SECTIONS
%------------------------------------------------------------------------------------

\textsc{\huge Chalmers tekniska högskola}\\[1.5cm] % Name of university/college
\textsc{\Large Rapport, Experimentell fysik 2}\\[0.2cm] % Major heading such as course name
\textsc{\large Termodynamik -- Uppgift 3 }\\[0.5cm] % Minor heading such as course title

%------------------------------------------------------------------------------------
%	TITLE SECTION
%------------------------------------------------------------------------------------

\HRule \\[0.4cm]
{ \LARGE \bfseries 
Studier av kvicksilveratomens emissionsspektra samt absorptionsspektra av laserfärgämnena Rhodamin B och Kumarin 307
}\\[0.4cm] % Title of  document
\HRule \\[1.5cm]
 
%------------------------------------------------------------------------------------
%	AUTHOR SECTION
%------------------------------------------------------------------------------------

\begin{minipage}{0.4\textwidth}
\begin{flushleft} \large
\emph{Författare:}\\
Andréas Sundström\footnotemark{} \\
Rigon Demisai\footnotemark{} 
\end{flushleft}
\end{minipage}
~
\begin{minipage}{0.4\textwidth}
\begin{flushright} \large
\emph{Labassistent:} \\
Martin Wersäll
\end{flushright}
\end{minipage}\\[3cm]

\setcounter{footnote}{0}
\stepcounter{footnote}
  \footnotetext{\href{mailto:\andsunds}{\texttt{\andsunds}}}
\stepcounter{footnote}
  \footnotetext{\href{mailto:\rigon}{\texttt{\rigon}}}



%------------------------------------------------------------------------------------
%	DATE SECTION
%------------------------------------------------------------------------------------
% Följer ISO-standarden för tidsintervall:
% https://en.wikipedia.org/wiki/ISO_8601#Time_intervals
% "Double hyphen" också ok istället för '/'. -- i LaTeX är dock lite på gränsen
{ \large
\begin{tabular}{rc}
    Laboration utförd: & 2015-12-11/15 \\[0.1cm]
    Rapport inlämnad: & \today
\end{tabular}\\[1cm]
}

%------------------------------------------------------------------------------------
%	LOGO SECTION
%------------------------------------------------------------------------------------

\includegraphics[height=5cm]{logo.pdf} % Include a department/university logo
 
%------------------------------------------------------------------------------------

\vfill % Fill the rest of the page with whitespace

\end{titlepage}
\restoregeometry


\setcounter{page}{2}%detta är ANDRA (2) sidan

\renewcommand{\abstractname}{Sammandrag}
\begin{abstract}
Den här rapporten beskriver en spektroskopisk studie av
kvicksilveratomens atomära emissionsspektrum samt en studie av
laserfärgämnena rhodamin~B och kumarin~307 och deras
absorptionsspektra.
Ur emisionspektrumet från Hg har vissa av atomens energinivåer kartlagts,
baserat på kvanmekaniska regler och jämförelse med tidigare data på
Hg.  Kvicksilvrets emissionsspektra
är taget i intervallet 350--1100\,nm, där detekterades totalt 23
emissionstoppar, varefter 13 övergångar kunde identifieras.
Absorptionsspektrumen för rhodamin~B och kumarin~307 visar breda
absorptionsband vilket är kännetecknande för flourescerande ämnen som
består av stora organiska molekyler. 
Mätningarna har utförts med en Spex~270M spektrometer och
datainsamlingen har gjorts i LabView.
\end{abstract}

\renewcommand{\abstractname}{Abstract}
\begin{abstract}
This report describes a spectroscopic study of the atomic emision
spectrum of mercury, and also a study of the laser dyes Rhodamine~D
and Coumarin~307 and their absorption spectrum.
Som of the energylevels of Hg have been idetified from the emission
spectrum, based on quantum mecanichal rules and comparison with erlier
data on Hg. The emission spectrum of mercury was taken in the interval
of 350--1100\,nm, there a total of 23 emission lines were detected,
from which 13 different atomic transitions could be identified. 
The absorption spectrum of Rhodamin~B and Coumarin~307  show wide
absorption bands which are characteristic of big organic molecules.
THe measurements were made with a Spex~270M spectrometer and the data
collection was done through LabVIEW.
\end{abstract}

\clearpage
\renewcommand{\contentsname}{Innehållsförteckning}

\tableofcontents


\clearpage
\pagenumbering{arabic}
\setcounter{page}{1}

\renewcommand{\thefootnote}{\arabic{footnote}}
\setcounter{footnote}{0}
