\chapter{Slutsats}

De observerade partiklarna i jästcellerna undergår subdiffusion, det vill säga diffunderar långsammare än en vanlig brownsk rörelse. Därmed ger varken denna modell eller Ornstein-Uhlenbeck-modellen en överensstämmande beskrivning av diffusionstakten. Ornstein-Uhlenbeck-modellen kan dock användas för att beskriva skillnaden i asfärisitet mellan de två cellfaserna \emph{aktiv} och \emph{i dvala}.

Även om fBm och CTRW båda förutsäger subdiffusion kan dessa modeller inte beskriva alla de observerade egenskaperna hos partiklarna. Utifrån anpassningar till partiklarnas MSD, PSD och asfärisitet kan rörelsens Hurstparameter, kopplad till fBm, tas fram. Dessa värden på $H$ överlappar inte vilket tyder på brister i fBm som modell för partikelrörelse i celler. CTRW utgör ingen stationär process något som den observerade processen verkar vara...

%Bara en liten kodsnutt som behövs när man kompilerar lokalt
%%% Local Variables: 
%%% mode: latex
%%% TeX-master: "00main.tex"
%%% End: 