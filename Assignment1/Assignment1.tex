\documentclass[11pt,letter, swedish, english
]{article}
\pdfoutput=1

\usepackage{../custom_as}
\usepackage[makeroom
]{cancel}
\graphicspath{{figures/}}

\swapcommands{\Delta}{\varDelta}
\swapcommands{\Omega}{\varOmega}

%%Drar in tabell och figurtexter
\usepackage[margin=10 pt]{caption}
%%För att lägga in 'att göra'-noteringar i texten
\usepackage{todonotes} %\todo{...}

%%För att själv bestämma marginalerna. 
\usepackage[
%            top    = 2.5cm,
%            bottom = 3cm,
%            left   = 3cm, right  = 3cm
]{geometry}

%%För att ändra hur rubrikerna ska formateras


\renewcommand{\thefootnote}{\fnsymbol{footnote}}

\renewcommand{\thesubsection}{\arabic{section} (\alph{subsection})}
\renewcommand{\thesubsubsection}{\arabic{section} (\alph{subsection},\,\roman{subsubsection})}


\begin{document}




%%%%%%%%%%%%%%%%% vvv Inbyggd titelsida vvv %%%%%%%%%%%%%%%%%

\title{Numerical solutions to PDE's -- AMATH\,741 \\
Assignment 1}
\author{Andréas Sundström}
\date{\today}

\maketitle

%%%%%%%%%%%%%%%%% ^^^ Inbyggd titelsida ^^^ %%%%%%%%%%%%%%%%%

\section{Advection-diffusion equation}

Here we will consider the equation
\begin{equation}\label{eq:1_start}
u_t+au_x=\sigma u_{xx}\qcomma
a>0\qcomma \sigma>0
\end{equation}
on some domain.
The solutions are assumed to be of the form
\begin{equation}\label{eq:1_sol1}
u(x, t)=\sum_k \hat{u}(k, t)\ee^{\ii kx}.
\end{equation}
(The exact specification of $k$ in the sum is not given here. That is
because the values of $k$ depends on the IC or BC.)

\subsection{Finding a solution}
To find the form of each term in \eqref{eq:1_sol1}, we can omit the
sum in \eqref{eq:1_sol1}, since the governing PDE is linear. We get
\begin{equation}\label{eq:1_disp}
\hat{u}_t + \ii ka\, \hat{u}
=-\sigma k^2\hat{u},
\end{equation}
where we've also canceled the factor $ee^{\ii kx}$ appearing in all of
the terms in the dispersion relation \eqref{eq:1_disp}. We see now
that \eqref{eq:1_disp} is a DE in $t$ with the solution
\begin{equation}
\hat{u}(k, t)= c(k)\, \ee^{-(\ii ka+\sigma k^2)t},
\end{equation}
where $c(k)$ is some function of $k$ (that depends on the I.C.).
The solution to \eqref{eq:1_start} thus becomes
\begin{equation}\label{eq:1_sol2}
u(x, t) = \sum_k c(k)\, \ee^{\ii k(x-at)}\ee^{-\sigma k^2t}.
\end{equation}

The initial conditions come into play on determining $c(k)$. Let's say
that
\begin{equation}
u(x, 0)=\varphi(x)
\end{equation}
on the domain of the problem. Then, setting $t=0$ in
\eqref{eq:1_sol2}, we get
\begin{equation}
\sum_k c(k)\, \ee^{\ii kx}=\varphi(x)
\end{equation}
This is a Fourier series expansion of $\varphi$; for this to work we
need either of these two cases to be satisfied
\begin{enumerate}[label=\Roman*.]
\item the domain of the problem is bounded with length $P$,
\item or $\varphi$ is periodic with period $P$.
\end{enumerate}
The allowed frequencies will be $k=n/P$, $n\in\Z$.
And the coefficients are the regular Fourier coefficients
\begin{equation}
c(k)=\frac{1}{P}\int_{x_0}^{x_0+P} \varphi(x)\ee^{-\ii kx}\id{x}.
\end{equation}

An example of a possible IC of the first kind would be 
\begin{equation}
\varphi_\mathrm{I}(x)=1-|x|
\end{equation}
on the finite domain $x\in[-1, 1]$. An expample of the second kind of
IC could be
\begin{equation}
\varphi_\mathrm{II}(x)=\cos(x)+\cos(2x)
\end{equation}
on $x\in\R$.


\subsection{Periodic boundary conditions}
Periodic bondary conditions are equivalent to changing from case I to
case II. 

What will happen is that each frequency will be attenuated according
to $\ee^{-\sigma k^2t}$, and then Fourier analysis tells us that the
(space domain) solution will be shifted\footnotemark{} by $akt$. 

\footnotetext{See part (c) of this problem for more details.}

\todo[inline]{Plot!}

\subsection{Relating this solution to the solutions of the original PDE's}






\section{Higher order approximations of derivatives}




\section{Round-off errors}



\section{Numerical solution of the heat equation}







\end{document}




%  LocalWords:  MFT MF Advection PDE's AMATH
