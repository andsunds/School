\documentclass[11pt,letter, swedish, english
]{article}
\pdfoutput=1

\usepackage{../custom_as}
\usepackage[makeroom
]{cancel}
\graphicspath{{figures/}}

\swapcommands{\Delta}{\varDelta}
\swapcommands{\Omega}{\varOmega}

%%Drar in tabell och figurtexter
\usepackage[margin=10 pt]{caption}
%%För att lägga in 'att göra'-noteringar i texten
\usepackage{todonotes} %\todo{...}

%%För att själv bestämma marginalerna. 
\usepackage[
%            top    = 2.5cm,
%            bottom = 3cm,
%            left   = 3cm, right  = 3cm
]{geometry}

%%För att ändra hur rubrikerna ska formateras


\renewcommand{\thefootnote}{\fnsymbol{footnote}}

\newcommand{\Tc}{\ensuremath{T_{\text{c}}}}
\newcommand{\eF}{\ensuremath{\epsilon_{\text{F}}}}
\newcommand{\wD}{\ensuremath{\omega_{\text{D}}}}

%\usepackage{tikz}

\begin{document}

%\tikzstyle{every picture}+=[remember picture]
%\tikzstyle{na} = [shape=rectangle,inner sep=0pt,text depth=0pt]



%%%%%%%%%%%%%%%%% vvv Inbyggd titelsida vvv %%%%%%%%%%%%%%%%%

\title{Statistical Physics 2 -- PHYS\,705 \\
Assignment 1}
\author{Andréas Sundström}
\date{\today}

\maketitle

%%%%%%%%%%%%%%%%% ^^^ Inbyggd titelsida ^^^ %%%%%%%%%%%%%%%%%

\section{Mean field magnetic susceptibility}
\renewcommand{\thesubsection}{\arabic{section} (\roman{subsection})}

In this problem, we will study the magnetic susceptibility, $\chi$, in
view of the MFT approach to the Ising model. We will look at the
asymptotic behaviour of $\chi$ for $T\to0^+$ and $T\to\Tc$.

We begin from the MF equation
\begin{equation}\label{eq:1_MFE}
M=\tanh(\frac{MJ + B}{T}),
\end{equation}
and the definition of susceptibility
\begin{equation}\label{eq:1_chi}
\chi:=\eval{\dv{M}{B}}_{B=0}.
\end{equation}

By applying \eqref{eq:1_chi} to \eqref{eq:1_MFE}, we get
\begin{equation}
\chi=\frac{1}{\cosh^2(MJ/T)}\, 
\qty(\frac{J}{T}\chi+\frac{1}{T}),
\end{equation}
or in other words
\begin{equation}\label{eq:1_chi_exact}
\chi=\frac{1}{T\cosh^2(MJ/T)-\Tc}.
\end{equation}
In the last step we also used the fact that $\Tc=J$ using the MFT approach.

\subsection{Zero temperature limit}
As $T\to0^+$, the leading order behaviour of the hyperbolic cosine is
\begin{equation}
\cosh^2\qty(\frac{MJ}{T})\sim \frac{1}{4}\exp(2\frac{\abs{M}J}{T}).
\end{equation}
This exponential growth will dominate over the factor $T$ in
front and the constant $-\Tc$ in the denominator. And \eqref{eq:1_MFE}
also shows us that $M\to\pm1$ as $T\to0^+$. 

The asymptotic behaviour of \eqref{eq:1_chi_exact} will be 
\begin{equation}
\chi\sim \frac{4}{T}\exp(-2\frac{J}{T})
\qcomma \text{as}\ T\to0.
\end{equation}
And to be clear, this asymptotic behaviour results in $\chi(T)\to0$ as
$T\to0$. 


\subsection{Critical temperature limit}






\section{Phase transitions in Landau theory}
\renewcommand{\thesubsection}{\arabic{section} (\alph{subsection})}
\renewcommand{\thesubsubsection}{\arabic{section} (\alph{subsection},\,\roman{subsubsection})}

Here, we will use the Landau free energy
\begin{equation}
F=-bM+tM^2+uM^4+M^6,
\end{equation}
where $b=B/T$, $t=(T-\Tc)/\Tc$ and $u$ is some coefiicient with no
restrictions. 

\subsection{Generic forms of $F$}
There will be, depending on how precise you want to be, around four
different generic forms of $F$, and they occur at
\begin{enumerate}[label=\roman*.]
\item $t,\,u\ge0$ %$\quad\Longrightarrow\quad$ just one minimum at $M=0$
\item $t<0$, $u$ any value; and $t=0$, $u<0$
%$\quad\Longrightarrow\quad$ local maximum at $M=0$ and two minimums with $M^2{>}\,0$.
\item $0<t<u^2/4$, $u<0$
\item $t>u^2/4$, $u<0$
\end{enumerate}



\section{1D Ising model}



\section{Ising model generalization}





\end{document}




%  LocalWords:  MFT MF
