\documentclass[11pt,letter, swedish, english
]{article}
\pdfoutput=1

\usepackage{../custom_as}
\usepackage[makeroom
]{cancel}
\graphicspath{{figures/}}

\swapcommands{\Delta}{\varDelta}
\swapcommands{\Omega}{\varOmega}

%%Drar in tabell och figurtexter
\usepackage[margin=10 pt]{caption}
%%För att lägga in 'att göra'-noteringar i texten
\usepackage{todonotes} %\todo{...}

%%För att själv bestämma marginalerna. 
\usepackage[
%            top    = 2.5cm,
%            bottom = 3cm,
%            left   = 3cm, right  = 3cm
]{geometry}

%%För att ändra hur rubrikerna ska formateras


\renewcommand{\thefootnote}{\fnsymbol{footnote}}

\newcommand{\Tc}{\ensuremath{T_{\text{c}}}}
\newcommand{\eF}{\ensuremath{\epsilon_{\text{F}}}}
\newcommand{\wD}{\ensuremath{\omega_{\text{D}}}}

%\usepackage{tikz}

\begin{document}

%\tikzstyle{every picture}+=[remember picture]
%\tikzstyle{na} = [shape=rectangle,inner sep=0pt,text depth=0pt]



%%%%%%%%%%%%%%%%% vvv Inbyggd titelsida vvv %%%%%%%%%%%%%%%%%

\title{Statistical Physics 2 -- PHYS\,705 \\
Assignment 1}
\author{Andréas Sundström}
\date{\today}

\maketitle

%%%%%%%%%%%%%%%%% ^^^ Inbyggd titelsida ^^^ %%%%%%%%%%%%%%%%%

\section{Mean field magnetic susceptibility}
\renewcommand{\thesubsection}{\arabic{section} (\roman{subsection})}

In this problem, we will study the magnetic susceptibility, $\chi$, in
view of the MFT approach to the Ising model. We will look at the
asymptotic behaviour of $\chi$ for $T\to0^+$ and $T\to\Tc$.

We begin from the MF equation
\begin{equation}\label{eq:1_MFE}
M=\tanh(\frac{MJ + B}{T}),
\end{equation}
and the definition of susceptibility
\begin{equation}\label{eq:1_chi}
\chi:=\eval{\dv{M}{B}}_{B=0}.
\end{equation}

By applying \eqref{eq:1_chi} to \eqref{eq:1_MFE}, we get
\begin{equation}
\chi=\frac{1}{\cosh^2(MJ/T)}\, 
\qty(\frac{J}{T}\chi+\frac{1}{T}),
\end{equation}
or in other words
\begin{equation}\label{eq:1_chi_exact}
\chi=\frac{1}{T\cosh^2(M\Tc/T)-\Tc}.
\end{equation}
In the last step we also used the fact that $\Tc=J$ using the MFT approach.

\subsection{Zero temperature limit}
As $T\to0^+$, the leading order behaviour of the hyperbolic cosine is
\begin{equation}
\cosh^2\qty(\frac{M\Tc}{T})\sim \frac{1}{4}\exp(2\frac{\abs{M}\Tc}{T}).
\end{equation}
This exponential growth will dominate over the factor $T$ in
front and the constant $-\Tc$ in the denominator. And \eqref{eq:1_MFE}
also shows us that $M\to\pm1$ as $T\to0^+$. 

The asymptotic behaviour of \eqref{eq:1_chi_exact} will be 
\begin{equation}
\chi\sim \frac{4}{T}\exp(-2\frac{\Tc}{T})
\qcomma \text{as}\ T\to0.
\end{equation}
And to be clear, this asymptotic behaviour results in $\chi(T)\to0$ as
$T\to0$. 


\subsection{Critical temperature limit}
We begin with aproaching the limit from below, $T<\Tc$. We do still
evaluate the derivative for $\chi$ at $B=0$, so the MFT value for $M$
can be used:
\begin{equation}
M=\sqrt{3}\frac{T}{\Tc}\sqrt{-t}.
\end{equation}
Plugging this into the exact expression \eqref{eq:1_chi_exact} and
Taylor expanding the hyperbolic cosine gives
\begin{equation}
\begin{aligned}
\chi(T\to\Tc^-)\sim&\frac{1}{T\qty[1+3(T/\Tc)^2(\Tc-T)/\Tc]-\Tc}\\
=&\frac{1}{T-3(T/\Tc)^3(T-\Tc) -\Tc}\\
=&\frac{1}{T-\Tc}\times\frac{1}{1-3(T/\Tc)^3}\\
\sim&\frac{1}{2}\times\frac{1}{\Tc-T}=\frac{1}{2\Tc}|t|^{-1}.
\end{aligned}
\end{equation}
If we approach the limit from the other end, we know that $M=0$. So
\eqref{eq:1_chi_exact} is just
\begin{equation}
\chi(T\to\Tc^+)=\frac{1}{T-\Tc}=\frac{1}{\Tc} |t|^{-1}.
\end{equation}

To conclude this part of the problem we note that in both cases, we
can write
\begin{equation}
\chi=C_\pm |t|^{-\gamma},
\end{equation}
where $\gamma=1$ and $C_+/C_-=1/2$.



\section{Phase transitions in Landau theory}
\renewcommand{\thesubsection}{\arabic{section} (\alph{subsection})}
\renewcommand{\thesubsubsection}{\arabic{section} (\alph{subsection},\,\roman{subsubsection})}

Here, we will use the Landau free energy
\begin{equation}\label{eq:2_F(M)}
F=-bM+tM^2+uM^4+M^6,
\end{equation}
where $b=B/T$, $t=(T-\Tc)/\Tc$ and $u$ is some coefiicient with no
restrictions. 

\subsection{Generic forms of $F$}
There will be, depending on how precise you want to be, around four
different generic forms of $F$, and they occur at
\begin{enumerate}[label=\Roman*.]
\item $t\ge0$, $u\ge0$ \\
Here, it's obvious that all we can get is a single minimum at $M=0$.
\item $t<0$, $u$ any value; or $t=0$, $u<0$\\
For $t<0$, the initial behaviour around $M=0$ is to drop down, but at
large $|M|$ we still get $F\to+\infty$. So we must have two minima at
$M\neq0$, and a local maximum at $M=0$. The same is true for $t=0$ and
$u<0$. 
\item $0<t<u^2/4$, $u<0$\footnotemark{}\addtocounter{footnote}{-1}\\
Here, we initially get a growing $F$, near $M=0$, but the effects of a
negative enough $u$ will then come into play and force down $F(m)$ to
two minima at $M\neq0$ which are stronger than the one at $M=0$.
\item $t>u^2/4$, $u<0$\footnotemark{}\\
In this case, there might not even be two minima at $M\neq0$, but if
they excist they are weaker than the one at $M=0$.
\end{enumerate}
These four cases encompass all of the $u{-}T$ plane. The generic forms
of $F(M)$ can be seen in \figref{fig:2a}.

\footnotetext{The difference between these two cases will be
  further elaborated on in part (b) of this problem.}

\begin{figure}
\centering
\input{figures/generic_forms.tex}
\caption{The four (or five) different generic forms of $F(M)$, as
  given by \eqref{eq:2_F(M)}. }
\label{fig:2a}
\end{figure}

\subsection{First order transition, when $u<0$}
As illustrated in the difference between case III and IV, we see that
when $u<0$ it is indeed possible to get a scenario when the global
minimum of $F(M)$ jumps from $M=0$ to $\pm M_0\neq0$. The transition
occurs when the equation 
\begin{equation}
tM^2+uM^4+M^6=M^2(t+uM^2+M^4)=0
\end{equation}
goes from only one real solution to three/five. This is because we
know that the minimum at $M=0$ gives $F(M=0)=0$. So if we were to get
more solutions to this equation, that would mean that $F$ can assume
negative values. Ergo, there are some other minima which are stronger
than the one at $M=0$. 

We are therefore looking for when
\begin{equation}
t+uM^2+M^4=0
\end{equation}
starts having real solutions. These solutions are
\begin{equation}
M^2=\frac{-u\pm\sqrt{u^2-4t}}{2},
\end{equation}
and we see that this occurs only if $t\le u^2/4$. We are now in case
III, where the \emph{global} minima is away from $M=0$, but $M=0$ is
still however a \emph{local} minimum. This means that there is a
chance that the system will remain in $M=0$ all the way until $t=0$,
when we enter case II. In case II, $M=0$ is now a local maximum so the
system will quickly find its way to one of the two minima with
non-zero $M$. 

% Note also that the temperature at which this sudden jump in
% magetization happens is at $t=u^2/4>0$. That is the phase
% transition occurs at a temerature above $\Tc$.



\section{1D Ising model}



\section{Ising model generalization}





\end{document}




%  LocalWords:  MFT MF
