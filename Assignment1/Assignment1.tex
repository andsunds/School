\documentclass[11pt,letter, swedish, english
]{article}
\pdfoutput=1

\usepackage{../custom_as}



%%Drar in tabell och figurtexter
\usepackage[margin=10 pt]{caption}
%%För att lägga in 'att göra'-noteringar i texten
\usepackage{todonotes} %\todo{...}

%%För att själv bestämma marginalerna. 
\usepackage[
%            top    = 3cm,
%            bottom = 3cm,
%            left   = 3cm, right  = 3cm
]{geometry}

%%För att ändra hur rubrikerna ska formateras
\renewcommand{\thesubsection}{\arabic{section} (\alph{subsection})}

\renewcommand{\thesubsubsection}{\arabic{section} (\alph{subsection}\roman{subsubsection})}

% \newcommand{\cbox}[2][cyan]
% {\mathchoice
% 	{\setlength{\fboxsep}{0pt}\colorbox{#1}{$\displaystyle#2$}}
% 	{\setlength{\fboxsep}{0pt}\colorbox{#1}{$\textstyle#2$}}
% 	{\setlength{\fboxsep}{0pt}\colorbox{#1}{$\scriptstyle#2$}}
% 	{\setlength{\fboxsep}{0pt}\colorbox{#1}{$\scriptscriptstyle#2$}}
% }
% \newcommand{\grande}{\cbox{\phantom{\frac{1}{xx}}}}


\begin{document}

%%%%%%%%%%%%%%%%% vvv Inbyggd titelsida vvv %%%%%%%%%%%%%%%%%
% \begin{titlepage}
\title{Statistical Physics -- PHYS\,704 \\
Assignment 1}
\author{Andréas Sundström}
\date{\today}

\maketitle

%%%%%%%%%%%%%%%%% ^^^ Inbyggd titelsida ^^^ %%%%%%%%%%%%%%%%%

%Om man vill ha en lista med vilka todo:s som finns.
%\todolist

\section{Same temperature in equlibrium}


\section{A PDE for the entropy}
We are given to show that the entropy of a system can be described as
\begin{equation}
S \stackrel{?}{=} N\qty(\pdv{S}{N})_{V,E} 
+ V\qty(\pdv{S}{V})_{N,E}
+ E\qty(\pdv{S}{E})_{V,N} =: \sigma,
\end{equation}
where $\sigma$ has been defined as the middle expression. Note here
that since $S=S(N, V, E)$ so must also $\sigma=\sigma(N, V, E)$. And
since $S$, $N$, $V$ and $E$ are all \emph{extensive} properties, the
partial derivatives of $S$ must all be \emph{intensive}; this
makes $\sigma$ \emph{extensive} as well. 

Armed with this knowlege we can now start to investigate
$\sigma$. We begin by differentiating $\sigma$ to see what appears:
\begin{equation}\label{eq:d_sigma}
\begin{aligned}
\qty(\pdv{\sigma}{N})_{V,E} 
=& \qty(\pdv{S}{N})_{V,E} + N\qty(\pdv[2]{S}{N})_{V,E}\\ 
&+ V \qty(\pdv{N}\qty[\qty(\pdv{S}{V})_{N,E}])_{V,E}
+ E \qty(\pdv{N}\qty[\qty(\pdv{S}{E})_{N,V}])_{V,E}.
\end{aligned}
\end{equation}
Like for $S$ the partial derivatives of $\sigma$ is also
intensive. This means that if we were to scale the system by an
arbitrary amount $\alpha$, \eqref{eq:d_sigma} will become\footnotemark{}
\begin{equation}
\begin{aligned}
\qty(\pdv{\sigma}{N})_{V,E} 
=& \qty(\pdv{S}{N})_{V,E} + \alpha N\qty(\pdv[2]{S}{N})_{V,E}\\ 
&+ \alpha V \qty(\pdv{N}\qty[\qty(\pdv{S}{V})_{N,E}])_{V,E}
+ \alpha E \qty(\pdv{N}\qty[\qty(\pdv{S}{E})_{N,V}])_{V,E}.
\end{aligned}
\end{equation}
Since this is true for all $\alpha>0$, we must conclude that
\begin{equation}
\qty(\pdv{\sigma}{N})_{V,E} 
= \qty(\pdv{S}{N})_{V,E}.
\end{equation}
\footnotetext{More or less the definition of an extensive property is
  that if the whole system is scaled by $\alpha$, the extensive
  properties will be scaled by the same amount, e.g. 
  $E\mapsto E^*=\alpha E$. Where as in intensive property will be
  unaffected, e.g. $T\mapsto T^*=T$. }
The same argument can be applied to the two other derivative to get
\begin{equation}
\qty(\pdv{\sigma}{V})_{N,E} = \qty(\pdv{S}{V})_{N,E}
\qcomma
\qty(\pdv{\sigma}{E})_{V,N} = \qty(\pdv{S}{E})_{V,N}.
\end{equation}
In other words all of the partial derivatives of $S$ and $\sigma$ are
equal. By the uniqueness of derivatives, we now have
\begin{equation}
S=\sigma+S_0,
\end{equation}
for some possible constant $S_0$. 

The value of $S_0$ can be found by studying the limit 
$(N, V, E)\to0$. We obviously have
$\sigma(N, V, E)\to 0$ as $(N, V, E)\to0$,
and the entropy $S$ must also tend to~0 as $(N, V, E)\to0$. Therefore
it's clear that $S_0=0$, meaning that
\begin{equation}
S=\sigma
= N\qty(\pdv{S}{N})_{V,E} 
+ V\qty(\pdv{S}{V})_{N,E}
+ E\qty(\pdv{S}{E})_{V,N}.
\end{equation}
\qed



\section{Partition function of a relativistic gas}

\subsection*{Some relations derived from the partitionfunction}



\section{A quantum rotor}

\subsection{The eigenstatesand eigenvalues to the Hamiltonian}


\subsection{The density matrix}








\end{document}





%% På svenska ska citattecknet vara samma i både början och slut.
%% Använd två apostrofer (två enkelfjongar): ''.


%% Inkludera PDF-dokument
\includepdf[pages={1-}]{filnamn.pdf} %Filnamnet får INTE innehålla 'mellanslag'!

%% Figurer inkluderade som pdf-filer
\begin{figure}\centering
\centerline{ %centrerar även större bilder
\includegraphics[width=1\textwidth]{filnamn.pdf}
}
\caption{}
\label{fig:}
\end{figure}

%% Figurer inkluderade med xfigs "Combined PDF/LaTeX"
\begin{figure}\centering
\resizebox{.8\textwidth}{!}{\input{filnamn.pdf_t}}
\caption{}
\label{fig:}
\end{figure}

%% Figurer roterade 90 grader
\begin{sidewaysfigure}\centering
\centerline{ %centrerar även större bilder
\includegraphics[width=1\textwidth]{filnamn.pdf}
}
\caption{}
\label{fig:}
\end{sidewaysfigure}


%%Om man vill lägga till något i TOC
\stepcounter{section} %Till exempel en 'section'
\addcontentsline{toc}{section}{\Alph{section}\hspace{8 pt}Labblogg} 

