\documentclass[11pt,letter, swedish, english
]{article}
\pdfoutput=1

\usepackage{../custom_as}
\usepackage[makeroom
]{cancel}
\graphicspath{{figures/}}

\swapcommands{\Phi}{\varPhi}
\swapcommands{\Omega}{\varOmega}

\newcommand{\enaught}{\ensuremath\varepsilon_0}

%%Drar in tabell och figurtexter
\usepackage[margin=10 pt]{caption}
%%För att lägga in 'att göra'-noteringar i texten
\usepackage{todonotes} %\todo{...}

%%För att själv bestämma marginalerna. 
\usepackage[
%            top    = 2.5cm,
%            bottom = 3cm,
%            left   = 3cm, right  = 3cm
]{geometry}

%%För att ändra hur rubrikerna ska formateras


%\renewcommand{\thefootnote}{\fnsymbol{footnote}}

%\newcommand{\Tc}{\ensuremath{T_{\text{c}}}}
\newcommand{\sign}{\ensuremath{\,\text{sign}}}

%\usepackage{tikz}


\renewcommand{\thesubsection}{\arabic{section} (\alph{subsection})}
\renewcommand{\thesubsubsection}{\arabic{section} (\alph{subsection},\,\roman{subsubsection})}


\begin{document}

%\tikzstyle{every picture}+=[remember picture]
%\tikzstyle{na} = [shape=rectangle,inner sep=0pt,text depth=0pt]



%%%%%%%%%%%%%%%%% vvv Inbyggd titelsida vvv %%%%%%%%%%%%%%%%%

\title{E\&M -- PHYS\,706 \\
Assignment 1}
\author{Andréas Sundström}
\date{\today}

\maketitle

%%%%%%%%%%%%%%%%% ^^^ Inbyggd titelsida ^^^ %%%%%%%%%%%%%%%%%

\section{Electric potential and filed along the $z$ axis}
\begin{figure}\centering
\input{figures/1_geometry.pdf_t}
\caption{The geometry of this problem. Two point charges, each of
  charge $+Q$, are located at $z=\pm R$, and a circular ring centered
  atround the origin with a total charge of $-2Q$.}
\label{fig:1_geometry}
\end{figure}

In this problem, we are tasked to find the electric field and
potential along the $z$~axis, from a geometry given in
\figref{fig:1_geometry}. 

\subsection{The electro static potential}
By the superposition principles we can easili solve this problem in
parts by combining the potentials from each source
$\Phi=\Phi_{+R}+\Phi_{-R}+\Phi_{\text{ring}}$. 

Let's begin with the teo easiest terms. The two terms from the point
charges, which we all know is just
\begin{equation}\label{eq:1_Phi_dot}
\Phi_{\pm{R}}(\vb*r=z\vu{z})=\frac{Q}{4\pi\enaught}
\frac{1}{\abs{\vb*r\mp R\vu{z}}}
=\frac{Q}{4\pi\enaught} \frac{1}{\abs{z\mp R}}.
\end{equation}

Next up is the ring charge. In this case we could easily argue that
the symmetry of the circular charge will make the potential due to
that ring 
\begin{equation}\label{eq:1_Phi_ring}
\Phi_\text{ring}(\vb*r=z\vu{z})=\frac{-2Q}{4\pi\enaught}
\frac{1}{\sqrt{z^2+R^2}},
\end{equation}
where $\sqrt{z^2+R^2}$ is the distance from any point of the ring to
the $z$~axis at $z$. But let's do it more rigorously this time. In
cylindrical coordinates the charge distribution of that ring is
\begin{equation}
\rho(r, \phi, z)=\lambda \delta(z)\delta(r-R),
\end{equation}
where $\lambda=-2Q/(2\pi R)$ is the uniform linear charge density
along the ring. 
Therefore the potential, at any point in space, due to that ring is
\begin{equation}
\Phi_\text{ring}(y, x, z)=\frac{\lambda}{4\pi\enaught}
\int_0^{2\pi}\rd\phi'\int_{-\infty}^\infty\rd{z'}\int_0^\infty\rd{r'}\,r'
\frac{\delta(z)\delta(r-R)}{\abs{\vb*r-\vb*r'}}
\end{equation}
which upon evaluating the $\delta$ functions yields
\begin{equation}
\Phi_\text{ring}(y, x, z)
= \frac{\lambda}{4\pi\enaught} \int_0^{2\pi}\rd\phi'
\frac{R}{\sqrt{(x-R\cos\phi')^2+(y-R\sin\phi'^2)+z^2}}.
\end{equation}
From here, we can clearly see that by setting $x=y=0$ the expression
in the denominator simplifies to $\sqrt{z^2+R^2}$, which is
independent of $\phi'$; we thus end up with exactly
\eqref{eq:1_Phi_ring} as the potential due to the ring on the
$z$~axis.  

The end result from all three sources therefore becomes
\begin{align}\label{eq:1_Phi_a}
\Phi(\vb*r=z\vu{z})=&\frac{Q}{4\pi\enaught}
\qty[\frac{1}{\abs{z-R}}+\frac{1}{\abs{z+R}}
-\frac{2}{\sqrt{z^2+R^2}}]
\\\label{eq:1_Phi_b}
=&\frac{Q}{4\pi\enaught z}
\qty[\frac{1}{\abs{1-R/z}}+\frac{1}{\abs{1+R/z}}
-\frac{2}{\sqrt{1+(R/z)^2}}].
\end{align}

\subsection{The electric field along the $z$ axis}
To calculate the $E$~field along the $z$~axis we use
$\vb*E=-\grad\Phi$, meaning that $E_z=-\pdv*{\Phi}{z}$. This
derivative is easiest done using \eqref{eq:1_Phi_a}. We begin by
rewriting the absolute values\footnotemark, then we differentiate:
\begin{equation}
\begin{aligned}
E_z(\vb*r=z\vu{z})=&
-\frac{Q}{4\pi\enaught} \pdv{z}\qty[
\frac{\sign(z-R)}{(z-R)} +\frac{\sign(z+R)}{(z+R)}
-\frac{2}{\sqrt{z^2+R^2}}]\\
=&-\frac{Q}{4\pi\enaught} \qty[
-\frac{\sign(z-R)}{(z-R)^2} -\frac{\sign(z+R)}{(z+R)^2}
+\frac{2z}{(z^2+R^2)^{3/2}}].
\end{aligned}
\end{equation}
If we want to, we can rewrite this expression as
\begin{equation}
E_z(\vb*r=z\vu{z})
=\frac{Q}{4\pi\enaught z^2} \qty[
\frac{\sign(z-R)}{(1-R/z)^2} +\frac{\sign(z+R)}{(1+R/z)^2}
-\frac{2\sign(z)}{(1+(R/z)^2)^{3/2}}].
\end{equation}
\footnotetext{We are not concerned about the derivatives of the
  ``sign'' functions at their switching point since the rest of the
  expression is not analytic there either. }


\subsection{Asymptotic behaviour of the potential along the $z$ axis}
To find the asymptotic behaviour as $z\to+\infty$, we will need to use
the following Taylor expansion
\begin{equation}
(1+x)^\alpha=1+\alpha x+\frac{\alpha(\alpha-1)}{2}x^2+\order{x^3}
\end{equation}
for sufficiently small $x$. 

In our case, with $\zeta=R/z$, that would correspond to
\begin{equation}
(1\pm\zeta)^{-1}=1\mp\zeta+\zeta^2+\order{\zeta^3}
\end{equation}
and
\begin{equation}
\qty(1+\zeta^2)^{-1/2}=1-\frac{1}{2}\zeta^2+\order{\zeta^4}
\end{equation}
Subsituting these expansions into \eqref{eq:1_Phi_b} gives us the
asymptotic behaviour
\begin{equation}
\Phi(\vb*r=z\vu{z})\sim
\frac{Q}{4\pi\enaught z}
\qty[\qty(1-\zeta+\zeta^2)+\qty(1+\zeta+\zeta^2)
-2\qty(1-\frac{1}{2}\zeta^2)]
=\frac{3Q}{4\pi\enaught}\frac{R^2}{z^3},
\end{equation}
as $\zeta\to0$ or $z\to\infty$.




\section{Spherically symmetric charge distributions}
\renewcommand{\thesubsection}{\arabic{section} (\roman{subsection})}


\section{Mean charge distribution of a hydrogen atom}
\renewcommand{\thesubsection}{\arabic{section} (\alph{subsection})}


\section{Capacitances}



\section{Method of mirror charges}






\end{document}




%  LocalWords:  MFT MF Ising
