\documentclass[11pt,letter, swedish, english
]{article}
\pdfoutput=1

\usepackage{../custom_as}



%%Drar in tabell och figurtexter
\usepackage[margin=10 pt]{caption}
%%För att lägga in 'att göra'-noteringar i texten
\usepackage{todonotes} %\todo{...}

%%För att själv bestämma marginalerna. 
\usepackage[
%            top    = 3cm,
%            bottom = 3cm,
%            left   = 3cm, right  = 3cm
]{geometry}

%%För att ändra hur rubrikerna ska formateras
\renewcommand{\thesubsection}{\arabic{section} (\alph{subsection})}

\renewcommand{\thesubsubsection}{\arabic{section} (\alph{subsection},\,\roman{subsubsection})}

% \newcommand{\cbox}[2][cyan]
% {\mathchoice
% 	{\setlength{\fboxsep}{0pt}\colorbox{#1}{$\displaystyle#2$}}
% 	{\setlength{\fboxsep}{0pt}\colorbox{#1}{$\textstyle#2$}}
% 	{\setlength{\fboxsep}{0pt}\colorbox{#1}{$\scriptstyle#2$}}
% 	{\setlength{\fboxsep}{0pt}\colorbox{#1}{$\scriptscriptstyle#2$}}
% }
% \newcommand{\grande}{\cbox{\phantom{\frac{1}{xx}}}}

\renewcommand{\thefootnote}{\fnsymbol{footnote}}

\begin{document}

%%%%%%%%%%%%%%%%% vvv Inbyggd titelsida vvv %%%%%%%%%%%%%%%%%
% \begin{titlepage}
\title{Statistical Physics -- PHYS\,704 \\
Assignment 1}
\author{Andréas Sundström}
\date{\today}

\maketitle

%%%%%%%%%%%%%%%%% ^^^ Inbyggd titelsida ^^^ %%%%%%%%%%%%%%%%%

%Om man vill ha en lista med vilka todo:s som finns.
%\todolist

\section{Same temperature in equlibrium}
We have a closed system with two subsystems, $A$ and $B$, at
equlibrium with each other. Now we want to prove that the temperatures
of the two subsystems are equal, with the fact that entropy and energy
are extensive properties. 

Since the two subsystems are at equlibrium we know that the entropy of
the total system is maximized, therefore $S_A+S_B=S$ is constant. And
by energy conservation $E_A+E_B=E$ is also constant. 

We are now ready to write down the temperature of the subsystems:
\begin{equation}
\frac{1}{T_A}=\qty(\pdv{S_A}{E_A})_{N,V} 
=\bigg(\pdv{S_A}{E_B}\overbrace{\pdv{E_B}{E_A}}^{=-1}\bigg)_{N,V} 
=-\bigg(\pdv{S_B}{E_B}\overbrace{\pdv{S_A}{S_B}}^{=-1}\bigg)_{N,V} 
=+\qty(\pdv{S_B}{E_B})_{N,V} = \frac{1}{T_B}.
\end{equation}
\qed

\section{A PDE for the entropy}
We are given to show that the entropy of a system can be described as
\begin{equation}
S \stackrel{?}{=} N\qty(\pdv{S}{N})_{V,E} 
+ V\qty(\pdv{S}{V})_{N,E}
+ E\qty(\pdv{S}{E})_{V,N},
\end{equation}
using the fact that $S$, $N$, $V$ and $E$ are all additive or
extensive properties of a system. This means that if we scale the
system by any amount $\alpha>0$, these properties will scale with the
same amount:
\begin{align}
&N\longmapsto\alpha N\qcomma V\longmapsto \alpha V 
\qcomma E\longmapsto \alpha E\qcomma S\longmapsto \alpha S\\
\label{eq:scale_S}
\Longrightarrow\quad& S(\alpha N, \alpha V, \alpha E) = \alpha S(N, V, E).
\end{align}



Since this is true for any $\alpha$ we can choose $\alpha=1+\epsilon$,
where $|\epsilon|\ll1$\footnotemark{}. The LHS of \eqref{eq:scale_S}
can now be expanded in it's Taylor series
\begin{equation}
\begin{aligned}
S\Big((1+\epsilon) N, (1+\epsilon) V, (1+\epsilon) E\Big) =& 
S(N, V, E)\\
&+ \epsilon N\qty(\pdv{S}{N})_{\!V,E}\!
+ \epsilon V\qty(\pdv{S}{V})_{\!N,E}\!
+ \epsilon E\qty(\pdv{S}{E})_{\!V,N}\!
+ \order{\epsilon^2},
\end{aligned}
\end{equation}
meanwhile the RHS is just $(1+\epsilon)S(N, V, E)$.
\footnotetext{More precicely, choose $\epsilon$ sufficiently small so
  that we're within the radius of conversion of the Taylor series of
  $S$. }
All of this is true for all $\epsilon$, especially in the limit
$\epsilon\to0$, where the $\order{\epsilon^2}$ terms can be
disregarded. We thus get the equation
\begin{equation}
\begin{aligned}
S(N, V, E) + \epsilon S(N, V, E) =&
S(N, V, E)\\
&+\epsilon
\left(N\qty(\pdv{S}{N})_{\!V,E}\!
+ V\qty(\pdv{S}{V})_{\!N,E}\!
+ E\qty(\pdv{S}{E})_{\!V,N}\right).
\end{aligned}
\end{equation}
Here we can cancel out the $S$ terms and then the $\epsilon$'s, which
ends up giving us
\begin{equation}
S(N, V, E) =
N\qty(\pdv{S}{N})_{\!V,E}\!
+ V\qty(\pdv{S}{V})_{\!N,E}\!
+ E\qty(\pdv{S}{E})_{\!V,N}.
\end{equation}
\qed


\section{Partition function of a relativistic gas}
Given a relativistic gas, with energy-mumentum relation
$\epsilon=c\abs{p}$, where $c$ is the speed of light in vacuum, we
want to find the partition function of this gas. 

The partition function is given by
\begin{equation}
\begin{aligned}
Z&=\sum_i\ee^{\nicefrac{-\epsilon_i}{T}}
=\sum_i\exp\bigg(\!-\frac{c}{T}
\hspace{-5pt}\sum_{\stackrel{\text{\tiny all}}{\text{\tiny particles}}}\hspace{-5pt}
|\vb{p}_i|\bigg)\\
&\to \int\frac{\dd[3N]{q}\dd[3N]{p}}{N!h^{3N}}
\exp(-\frac{c}{T}\sum_{i=1}^N \abs{\vb{p}_i}).%\\
%&=\frac{1}{N!h^{3N}} \int\dd[3N]{q} 
%\int\dd[3N]{p} \prod_{i=1}^N \exp(-\frac{c}{T}\abs{\vb{p}_i}).
\end{aligned}
\end{equation}
Here we easily see that the integrand is independent of $q$, making
the $q$ integral evaluate to $V^N$. Then we can also note that the
integrand is just a product where each factor, belonging to a separate
partice, is independent of each others. This allows us to split up the
$p$ integral into $N$ identical integrals.
\begin{equation}
\begin{aligned}
Z&=\frac{V^N}{N!h^{3N}}
\qty(\int\!\dd[3]{p} \ee^{\nicefrac{c|\vb{p}|}{T}})^N\\
&=\frac{V^N}{N!h^{3N}}
\qty(\int_0^{2\pi}\!\dd{\phi}\int_0^\pi\!\dd{\theta}\sin\theta
\int_0^\infty p^2\dd{p}\ee^{\nicefrac{cp}{T}})^{\!\!N}.
\end{aligned}
\end{equation}
The $\phi$ and $\theta$ integral yiels the usual $4\pi$ and the $p$
integral can easily be solved by substituting $\xi=\nicefrac{cp}{T}$,
giving
\begin{equation}
Z=\frac{V^N}{N!h^{3N}}
\Bigg(4\pi \qty(\frac{T}{c})^3
\underbrace{\int_0^\infty \xi^2\ee^{-\xi}\id\xi}_{=2!}\Bigg)^{\!N}
=\frac{1}{N!} \qty[8\pi V \qty(\frac{T}{hc})^3]^N.
\end{equation}
\qed

\subsection*{Some relations derived from the partitionfunction}
We're also told to show that
\begin{equation}
PV=NT=\frac{1}{3}E \quad\text{in general;}
\quad\text{and } PV^{\frac{4}{3}}=\text{const.}
\quad\text{for an adiabatic process.}
\end{equation}

We begin by calculating the energy in the sytem via the formula
\begin{equation}\label{eq:3_E}
\begin{aligned}
E&=-\qty(\pdv{\ln(Z)}{\beta})_{N,V}=T^2\qty(\pdv{\ln(Z)}{T})_{N,V}\\
&=T^2\qty(\pdv{T})_{N,V}\qty[-\ln(N!)+N\ln(8\pi V (hc)^{-3})+3N\ln(T)].
&=3NT
\end{aligned}
\end{equation}
The preassure can be given by differentiating the Helmholtz free energy
$F=-T\ln(Z)$ with respect to $V$.
thus giving us
\begin{equation}\label{eq:3_PV}
\begin{aligned}
PV&=-V\qty(\pdv{F}{V})_{T}\\
&=V\qty(\pdv{V})_{T}
\qty[T\qty(-\ln(N)+N\ln(8\pi\qty(\frac{T}{hc})^3) + N\ln(V))]\\
&=NT.
\end{aligned}
\end{equation}
This concludes the proof of the first equations.

For the adiabatic process we know that entropy is constant. Once again,
we make use of th Helmholtz free energy, but this time we need to use
the relationship $F=E-TS$ as well. The constant entropy is then
\begin{equation}
\begin{aligned}
S&=\frac{E-F}{T}=\begin{Bmatrix*}[l]
E=3NT\\F=-T\ln(Z)
\end{Bmatrix*}\\
&=\underbrace{3N-\ln(N!) +N\ln(8\pi(hc)^{-3})+N}_\text{All constants}
\ln(VT^3).
\end{aligned}
\end{equation}
In other words, $VT^3=\text{const.}$, and with \eqref{eq:3_PV} we get
$V^4P^3=\text{const.}$, which is equivalent to what we set out to
prove. 
\qed


\section{A quantum rotor}
For a quantum rotor we have the Hamiltonian
\begin{equation}
H = -\frac{\hbar^2}{2I} \dv[2]{\theta},
\end{equation}
where $\theta\in[0,2\pi)$ is an angular variable. The fact that
$\theta$ is an angular variable gives us a periodig boundry condition
on all wavefunctions:
\begin{equation}
\psi(\theta+m2\pi)=\psi(\theta)\qcomma \forall m\in\Z.
\end{equation}


\subsection{The eigenstates and eigenvalues to the Hamiltonian}
To find the eigenstate, we just use the equation
$H\phi_n(\theta)=E_n\phi_n(\theta)$. With the peiodic boundry
condition, it's easy to se that
\begin{equation}
\phi_n(\theta)= A\ee^{\ii n\theta}+B\ee^{-\ii n\theta}
\end{equation}
with the eigenvalue
\begin{equation}
E_n=\frac{\hbar^2n^2}{2I}
\end{equation}
is a solution to the eigenvalue problem. By applying orthonormality
for all eigenstates, we see that one possible choise
of basis/set of eigenfunctions is
\begin{equation}
\phi_n(\theta)=\frac{1}{\sqrt{2\pi}}\ee^{\ii n\theta}\qcomma n\in\Z.
\end{equation}


\subsection{The density matrix}
To find the density matrix, $\rho$, we use the following expression
\begin{equation}
\rho = \sum_k p_k \op{\phi_k},
\end{equation}
where $\ket{\phi_k}$ is some basis which diagonalizes $\rho$, and
$p_k$ is the probability to find the state in the state of
$\ket{\phi_k}$. Thankfully the system considered here is in
equlibrium, which gives us $\comm{H}{\rho}=0$. This is why
$\ket{\phi_n}$, which was eigenstates to $H$, is used here as well. 

But we're asked to find the density matrix in the
$\theta$ representation, which is
\begin{equation}
\rho(\theta, \theta') = \mel{\theta'}{\rho}{\theta}
= \sum_k p_k \ip{\theta'}{\phi_k}\ip{\phi_k}{\theta}
= \sum_k p_k \phi_k^*(\theta')\phi_k(\theta)
= \sum_k p_k \frac{\ee^{-\ii k (\theta'-\theta)}}{2\pi}.
\end{equation}
Now, we just have to find
\begin{equation}
p_k = \frac{\ee^{\nicefrac{-E_k}{T}}}{Z} 
=\frac{\ee^{-\frac{\hbar^2}{2IT}k^2}}{Z} 
=\frac{\ee^{-\epsilon k^2}}{Z}, 
\end{equation}
where $\epsilon=\nicefrac{\hbar^2}{(2IT)}$, and
\begin{equation}
Z=\sum_{k=-\infty}^\infty\ee^{-\epsilon k^2}.
\end{equation}
This last sum is clearly convergent, but lacks a closed
expression. All this leavs us with
\begin{equation}
\rho(\theta, \theta')
= \frac{1}{2\pi Z} 
\sum_{k=-\infty}^\infty \exp(-\ii k(\theta'-\theta) -\epsilon k^2)
= \frac{1}{2\pi Z} 
\sum_{k=-\infty}^\infty \exp(-\ii k\Delta\theta -\epsilon k^2),
\end{equation}
for an arbitrary temperature, where $\Delta\theta=\theta'-\theta$. It
is however posible to get the behaiviour in the limits of high and low
temperature. 

\subsubsection{High tempereature limit}
In the high tempererature limit $\epsilon\propto\nicefrac{1}{T}$
becomes very small, and the difference between each term in the sum
becomes small as well, justifying a transition to an integral. Note
here that we'll get an integral expression in both $Z$ and $\rho$.

Let's begin with the partition function
\begin{equation}
Z\to \int_0^{2\pi}\frac{\dd{q}}{h^1}\int_{-\infty}^\infty\ee^{-\epsilon k^2} \dd{k} 
= \frac{2\pi}{h}\sqrt{\frac{\pi}{\epsilon}}.
\end{equation}
For the $q$ integral it's over all of the $\theta$ space, that is from
$0$ to $2\pi$. 
The $k$ integral is just a standard integral with a well known
value. 

And now, on to the whole expression for the density matrix
\begin{equation}\label{eq:4_hi_temp}
\rho(\Delta\theta) \to \frac{1}{2\pi Z}
\int_0^{2\pi}\frac{\dd{q}}{h^1}
\int_{-\infty}^\infty\ee^{\ii k\Delta\theta}\ee^{-\epsilon k^2} \dd{k} 
= \frac{1}{2\pi}\frac{h}{2\pi}\sqrt{\frac{\epsilon}{\pi}}\cdot
\frac{2\pi}{h}\sqrt{\frac{\pi}{\epsilon}}\ee^{-\nicefrac{(\Delta\theta)^2}{4\epsilon}}
= \frac{1}{2\pi}\ee^{-\nicefrac{(\Delta\theta)^2}{4\epsilon}}.
\end{equation}
Here the fact that the integral is the Fourier transform of
$\ee^{-\epsilon k^2}$ was used to evaluate the it. This is also
a well known transfrom which can be looked up in some work of
reference, e.g. Råde\,\&\,Westergren, \textit{Mathematics Handbook for
Science and Engineering}, ed. 5:12, section 13.2.

Pleas do also note, that the $q$ integral is not an integral over
any of our two angle variables $\theta$ or $\theta'$, eaven though
it's over the domain of $\theta$. Therefore $q$ was used as the
integration variable instead. 



\subsubsection{Low temperature limit}
In the low tempereature limit $\epsilon\propto\nicefrac{1}{T}$ becomes
very large, and causes more or less all terms to vanish in the sums. To
keep things interesting though, let's keep the terms up to
$k=\pm1$. This gives us
\begin{equation}
Z\approx 1+2\ee^{-\epsilon}
\end{equation}
and
\begin{equation}\label{eq:4_low_temp}
\begin{aligned}
\rho(\Delta\theta)&\approx \frac{1}{2\pi}
\frac{1+\ee^{-\epsilon}\Big(\ee^{-\ii\Delta\theta}+\ee^{+\ii\Delta\theta}\Big)}{1+2\ee^{-\epsilon}}
\approx\frac{1}{2\pi}
\qty(1-2\ee^{-\epsilon})\qty(1+2\ee^{-\epsilon}\cos(\Delta\theta))\\
&\approx \frac{1}{2\pi}\qty(1 + 2\ee^{-\epsilon}\big(\cos(\Delta\theta)-1\big)).
\end{aligned}
\end{equation}
Here all $\order{\ee^{-2\epsilon}}$ have been discarded. If we were to
discard eaven the $\order{\ee^{-\epsilon}}$ terms, we would get a
constant behaviour with respect to $\Delta\theta$.


\subsection*{Comment regarding the normalization of $\rho(\Delta\theta)$}
We should always stop and ask oursevles whether or not the final
answer is reasonable or feasible. One such question is wheter the total
probability of the system is still~1, in the given solutions for the
low and high temperature limits. At a first glance it might look like
it's not; however with some closer investigation we can see that the
total probability is indeed~1 for both the limits.

To show this, we begin by remembering that the total probability of the
system beeing in some state, is given by the \emph{trace} of the
density matrix. Which in our case would be to integrate over all
$\theta$'s with $\theta'=\theta$, or $\Delta\theta=0$. From this it's
clear that the $\rho$'s from both \eqref{eq:4_hi_temp} and
\eqref{eq:4_low_temp} gives us a trace of~$1$. So there is no
violation of total nomalization of $\rho$. 


\end{document}





%% På svenska ska citattecknet vara samma i både början och slut.
%% Använd två apostrofer (två enkelfjongar): ''.


%% Inkludera PDF-dokument
\includepdf[pages={1-}]{filnamn.pdf} %Filnamnet får INTE innehålla 'mellanslag'!

%% Figurer inkluderade som pdf-filer
\begin{figure}\centering
\centerline{ %centrerar även större bilder
\includegraphics[width=1\textwidth]{filnamn.pdf}
}
\caption{}
\label{fig:}
\end{figure}

%% Figurer inkluderade med xfigs "Combined PDF/LaTeX"
\begin{figure}\centering
\resizebox{.8\textwidth}{!}{\input{filnamn.pdf_t}}
\caption{}
\label{fig:}
\end{figure}

%% Figurer roterade 90 grader
\begin{sidewaysfigure}\centering
\centerline{ %centrerar även större bilder
\includegraphics[width=1\textwidth]{filnamn.pdf}
}
\caption{}
\label{fig:}
\end{sidewaysfigure}


%%Om man vill lägga till något i TOC
\stepcounter{section} %Till exempel en 'section'
\addcontentsline{toc}{section}{\Alph{section}\hspace{8 pt}Labblogg} 

