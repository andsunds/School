\chapter{Inledning}


%\section{Bakgrund, syfte och begränsningar}

%\paragraph{Bakgrund}
Transport inuti celler är en av grundstenarna för cellers överlevnad. Exempel på en livsnödvändig intracellulär transport är hur ATP, molekylen som driver \emph{alla} biologiska processer, kan ta sig från mitokondrien till alla delar av cellen. Detta är ett fall av passiv transport, där molekylerna eller partiklarna förflyttas genom att slumpvis diffundera genom cellen. Det är därför viktigt att kunna beskriva transport inuti cellerna för att förstå cellens inre processer.
I cellens inre finns även trådliknande strukturer, uppbyggda av proteinfilament, som ger både stadga och möjliggör en aktiv transport inom cellen. \todo[color=lime]{Varför studera filamentrörelse?}Dessa filaments dynamik kan därför också vara av intresse att studera.


Som en första approximation skulle partiklars rörelse i cytoplasman kunna beskrivas med klassisk brownsk rörelse. Studier av diffusion i celler [\cite{Hofling&Franosch2013}, \cite{Dix_Crowdingeffects2008}, \cite{Gou_etal2014}, \cite{Parry_etal2014}] har dock visat på avvikelser från denna teori, där partiklarna diffunderar långsammare än förväntat inuti cellen. Ytterligare skillnad i partikelrörelser har tidigare observerats om cellen befinner sig i dvala eller i sitt normala metabola tillstånd \cite{Parry_etal2014}. En alltäckande teori för vad som kan förklara dessa observationer finns i dagsläget inte och ämnet utgör därför ett aktuellt forskningsområde. Rörelsens stokastiska natur och cellens avancerade inre struktur är troligtvis varför det har varit svårt att finna en modell för rörelsen. Ett flertal modeller finns dock som beskriver delar av de observerade egenskaperna, bland annat ''fractional Brownian motion'' (fBm) \cite{Mandelbrot_fBm1968} och ''Continous Time Random Walk'' (sv. tidskontinuerlig slumpvandring) (CTRW) \cite{Hofling&Franosch2013} för partikelrörelse och ''Worm Like Chain'' (WLC) \cite{Milstein2013} för strängrörelse som alla presenteras senare i detta arbete.
%Rörelsens stokastiska natur och cellens avancerade inre struktur ligger troligtvis bakom svårigheten man hittills stött på när man sökt en förklaringsmodell till rörelsen.


%\paragraph{Syfte} 
Syftet med denna rapport är att studera rörelser orsakade av passiv transport i celler för att utifrån denna studie förhoppningsvis kunna ge en klarare bild av cytoplasmans natur. Detta görs genom att jämföra vedertagna teoretiska diffusionsmodeller med den tillgängliga datan, både för rörelsen hos partiklar genom cytoplasman och filaments (proteintrådar) rörelse i en vätska. %Filamenten befann sig alltså inte inuti en levande cell men studien av dessa borde ändå kunna ge en någorlunda bra bild av hur en sträng i cytoplasman skulle bete sig.


%\paragraph{Avgränsningar}
Att avbilda små partiklar och filament innebär stora svårigheter vilket har försvårat analysen av deras egenskaper. I dagsläget finns heller ingen fullständig modell som beskriver partiklars och filaments rörelse \todo[color=lime]{Någon källa för strängarna}[\cite{Hofling&Franosch2013}, []]. %Den teoretiska modellen som undersöks i det här kandidatarbetet behöver verka under vissa antaganden som begränsar dess användningsområde. 
Exempelvis kan variationer som uppstår vid betraktande av rörelser under olika tidsskalor komma att leda till svårigheter. %, bland annat i att finna en teoretisk modell som korrekt beskriver rörelsen oberoende av tidsskala. 
Således är det mer rimligt att modellera rörelsen under antagandet att modellen i första hand beskriver rörelser under en viss tidsskala.%, förslagsvis observationer som varar i intervallet $\unit[10]{ms}$ till $\unit[10]{s}$ vilket speglar den data som analyseras i detta arbete.

Det finns idag olika teorier om vad som påverkar partiklars och filaments rörelse i cytoplasman. Vissa \cite{Gou_etal2014} försöker beskriva vad som sker i celler med aktiv transport medan andra \cite{Parry_etal2014} lägger mer fokus på den passiva transporten inom cellen, mycket beroende på vilken celltyp som studerats. Då jästceller endast har aktiv transport under celldelning, undersöks framför allt den passiva transporten i den här studien. Detta eftersom den tillgängliga datan var insamlad i log-fas under korta tidsperioder så att endast ett fåtal celler delade sig under datainsamlingen. 








%\section{Inspiration från planeringsrapporten}

%Fördjupade studier av partikelrörelse i cellen skulle till exempel kunna leda till mer effektiva läkemedel. Vet man hur transporten inom cellen sker underlättar det arbetet med att ta fram specialdesignad medicin.


%\paragraph{Rapportens/Arbetets ändamål}
%Ur en stokastisk modell kan sedan en makroskopisk, statistisk beskrivning uppnås och det är med denna statistiska beskrivning som modellen kan jämföras med data. 

%\paragraph{Vad vi gjort}


%Men för att över huvud taget kunna analysera datan behövdes osäkerheten i mätningarna uppskattas, vilket inte är helt problemfritt då till exempel brownsk rörelse i sig själv är en sorts brus. Brus brukar i vanliga fall hanteras genom att undersöka någon sorts medelvärde. I det här fallet kom datan från flera olika partiklar vilket gjorde att en direkt jämförelse av en undersökt parameter inte kunde göras; istället söktes först ett samband mellan storleken på partiklarna och den parameter man sökte.

%I datan fanns utöver position även en partikels intensitet i mikroskopet. Intensiteten berodde med största sannolikhet på partikelns storlek; Dock var det exakta sambandet inte helt klart vilket ger ännu en svårighet i hur datan ska analyseras. För de små partiklarna tordes intensiteten bero på volymen medan den för de större partiklarna mer borde gå mot att bero av arean. Detta då intensiteten är proportionell mot antalet ljusemitterande ämnen på partikeln som kameran ''ser''. För stora partiklar kan en del av dessa lysande ämnen döljas av andra så att kameran bara ser ljuset från den sida den är riktad mot. 
%Troligen går det dock att från intensiteterna kunna jämföra olika partiklar och på så sätt ändå kunna utnyttja den i jämförelser mellan olika partiklar. 


%Bara en liten kodsnutt som behövs när man kompilerar lokalt
%%% Local Variables: 
%%% mode: latex
%%% TeX-master: "00main.tex"
%%% End: 