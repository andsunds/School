\input{template_files/packages}
\usepackage{units}
%\usepackage[T1]{fontenc}
%\usepackage[utf8]{inputenc}
\usepackage{physics}

\newcommand{\ee}{\mathrm{e}}
\newcommand{\ii}{\mathrm{i}}

\title{H1b: MD simulation -- dynamic properties}
\author{Andr\'eas Sundstr\"om and Linnea Hesslow}
\date{\today}

\begin{document}

\input{template_files/titlepage}

\section*{Introduction}

Already in antiquity people studied the effect of particles impinging on other
particles. Since then the art has developed \ldots\
(\emph{If you like to do so, you may take the opportunity to put the  methods
in a wider perspective here.}) Here is a random reference.\cite{lamport94}

\section*{Task 1}
We determined the theoretical lattice parameter ....

Figure~\ref{fig1} shows the potential energy as a function of the lattice parameter. We used a quadratic fit to find the minimum energy, and obtained $V_{\rm eq} \approx \unit[65.38]{\AA^3}$. This corresponds to the equilibrium lattice parameter $a_{\rm eq} \approx \unit[4.029]{\AA}$ at \unit[0]{K}, which we took as the initial lattice parameter for the following tasks.  

\begin{figure}[!ht]
\begin{center}
  \includegraphics[width=0.7\textwidth]{../figures/potential_energy} 
  \caption{The potential energy per unit cell for aluminum as a function of the lattice parameter cubed.}
  \label{fig1}
\end{center}
\end{figure}

We find that figure~\ref{fig1} looks similar to the figure~1 in the homework problem file. 

\section*{Task 5}
Equation (82) in MD lecture notes:


\begin{align}
\Delta_{\rm MSD}(t) &= \lim_{T \rightarrow \infty} \frac{1}{T} \int_0^{T} d t' \frac{1}{N_{\rm atoms}} \sum_{i=0}^{N_{\rm atoms}-1} \left[{\bf r}_i(t+t') - {\bf r}_i(t') \right]^2 \\ &\Rightarrow \nonumber
\\
\Delta_{\rm MSD}(t_k) &\approx
\frac{1}{N_T -k}\frac{1}{N_{\rm atoms}} \sum_{j=0}^{N_T-k-1} \sum_{i=0}^{N_{\rm atoms}-1} \left[{\bf r}_i(t_{k+j}) - {\bf r}_i(t_j) \right]^2 
\end{align}

To determine M, we used mean of ... for t  > ...


\section*{Task 7}

% \subsection*{Sort of proof}
% Discrete auto-correlation function
% \begin{equation}
% \varPhi_j = \frac{1}{N}\sum_{i=0}^{N-1} v_{i+j}v_{i}
% \end{equation}

% Discrete Fourier transform,
% \begin{equation}
% \hat{f}_k = \sum_{i=0}^{N-1} f_i \exp(\ii 2\pi \frac{ik}{N}),
% \end{equation}
% We can define a discrete power spectrum as
% \begin{equation}
% \begin{aligned}
% \hat{P}_k =& \frac{1}{N} \abs{\hat{v}_k}^2
% =\frac{1}{N} \hat{v}_k\bar{\hat{v}}_k\\
% =&\frac{1}{N} \sum_{i=0}^{N-1} v_i\exp(\ii 2\pi \frac{ik}{N})
% \sum_{i'=0}^{N-1} v_{i'}\exp(-\ii 2\pi \frac{i'k}{N})
% \end{aligned}
% \end{equation}
% We can now do a relabeling of the index $i=i'+j$, and use periodic
% indices $v_{i+N}=v_{i}$
% \begin{equation}
% \begin{aligned}
% \hat{P}_k =& \frac{1}{N} \abs{\hat{v}_k}^2
% =\frac{1}{N} \hat{v}_k\bar{\hat{v}}_k\\
% =&\frac{1}{N} \sum_{i'=0}^{N-1}\sum_{j=-i'}^{N-1-i'} v_{j+i'}v_{i'}
% \exp(\ii 2\pi \frac{jk}{N}).
% \end{aligned}
% \end{equation}
% However, since the sum over $j$ reaches all indices in the periodic
% index domain, we can change the summation to going from $j=0$ to
% $N-1$, and then also swap the order of summation, which gives
% \begin{equation}
% \hat{P}_k =\sum_{j=0}^{N-1} \exp(\ii 2\pi \frac{jk}{N}).
% \frac{1}{N}\sum_{i'=0}^{N-1} v_{j+i'}v_{i'}.
% \end{equation}
% It is now clear that this power spectrum is the discrete Fourier
% transform of the discrete auto-correlation function,
% \begin{equation}\label{eq7:Wiener-Khinthchine}
% \hat{P}_k =\sum_{j=0}^{N-1} \varPhi_j \exp(\ii 2\pi \frac{jk}{N}).
% \end{equation}
% In reality, $v_i$ is not periodic, however we might still use the
% periodic indices for notational convenience.

\subsection*{What we did}

We calculated the discrete auto-correlation function similarly to the
MSD, 
\begin{equation}
\varPhi_j = \frac{1}{N-j}\sum_{i=0}^{N-j-1} \ev{v_{i+j}v_{i}},
\end{equation}
where $j=0,1,\ldots,N-1$ and the average is taken over all atoms.
We then preceded to numerically approximate the integral
\begin{equation}
\hat{\varPhi}(f) = 2\int_0^{\infty}\dd{t}\,
\varPhi(t) \cos(2\pi ft)
\approx 2\int_0^{T_{\rm s}}\dd{t}\,\varPhi(t) \cos(2\pi ft)
\end{equation}
using a trapeziodal method in \textsc{Matlab}, with a frequency range
$f=0$ to $f=1/(2\Delta{t})=f_{\rm Nyqvist}$, and frequency steps
$\Delta{f}=1/T_{\rm s}$, where $T_{\rm s}$ is a time at about half the
simulation end time. This is to avoid including noisy data in
$\varPhi(t)$ at later times, where the statistics are poor.

We then calculated the powerspectrum according to
\begin{equation}
\begin{aligned}
\hat{P}(\omega) =& \lim_{T\to\infty}\frac{1}{T}
\ev{\abs{\int_0^{T}\dd{t}\,v(t)\ee^{\ii\omega t}}^2}\\
\approx& \frac{1}{T}
\ev{\abs{\int_0^{T}\dd{t}\,v(t)\ee^{\ii\omega t}}^2}\\
\Longrightarrow\quad
\hat{P}_k=& \frac{1}{T}
\ev{\abs{\frac{T}{N} \sum_{i=0}^{N-1} v_i\exp(\ii2\pi\frac{ik}{N})}^2}
=\frac{T}{N} \ev{\abs{\hat{\vb*v}_k}^2}
\end{aligned}
\end{equation}
where the averages is taken over all atoms, and
\begin{equation}
\hat{\vb*v}_k = \sqrt{N}\sum_{i=0}^{N-1} \vb*v_i \exp(\ii2\pi\frac{ik}{N})
\end{equation}
is the discrete Fourier transform of $v_i$.

When we compare $\hat{\varPhi}_k$ and $\hat{P}_k$ in Figure~\ref{...},
we find that they are very similar, as, indeed, they should be
acording to the Wiener-Khinthchine theorem.







\section*{Problem 1}
As a starting point we first look at scattering from a hard-sphere
potential. We also consider the Lennard--Jones potential, which is depicted
in Figure~\ref{fig11}. (\emph{Always refer to Figures in the text.})

\begin{figure}[!ht]
\begin{center}
  \includegraphics[width=0.7\textwidth]{template_files/LJ} 
  \caption{The Lennard--Jones potential.
  Make sure you label and have units on all axes! Also make sure that labels etc.\
  are legible and that, if you print in black and white, that you use different line
  styles when required to differentiate between curves. In \textsc{matlab}
  you can export any figure to an .eps file from File $\rightarrow$
  Export\ldots\ in the Figure window.}
  \label{fig1}
\end{center}
\end{figure}
  
\section*{Problem 2}

In the following we give an example of how to produce a table.
Use the code for Table~\ref{tab1} as a template.

\begin{table}[!ht]
  \begin{center}
    \caption{A dummy table}
    \begin{tabular}{l|c|c}\hline\hline
      \textbf{Col.~1} & \textbf{Col.~2} & \textbf{Col.~3} \\ \hline
      the & quick & brown \\ 
      fox & jumps & over \\ 
      the & lazy  & dog \\ 
      \hline\hline
    \end{tabular}
    \label{tab1}
  \end{center}
\end{table}

\section*{Problem 3}

If you find some part of the code particularly interesting you may 
include it in the text, otherwise it should be included in the appedix.
If you do want to include code the following commands will print
the text directly, with no \LaTeX~commands executed:

\begin{lstlisting}[language=matlab]
% Hello world ten times in MATLAB
for i = 1 : 10
  fprintf('Hello world %d!\n',i);
end
\end{lstlisting}

\begin{lstlisting}[language=python]
# Hello world ten times in Python
for i in range(10):
  print 'Hello world %d!' % i
\end{lstlisting}

\section*{Problem 4}
At some point it may be appropriate to include equations. It is done in the
following way:

\begin{equation}
  V(r) = 4\epsilon \left[ \left( \frac{\sigma}{r} \right)^{12} - 
    \left(\frac{\sigma}{r} \right)^{6} \right]
\end{equation}

Do number and reference all your equations.

\section*{Concluding discussion}

Use your favourite flavor of \LaTeX{} to compile the file:
\begin{verbatim}
xelatex template.tex
pdflatex template.tex
latex template.tex
\end{verbatim}
should all work.
If you use \verb+pdflatex+ or \verb+xelatex+, included figures need to be in
\verb+pdf+, \verb+jpg+, or \verb+png+ format. If you want to include eps
figures, you can easily convert them to \verb+pdf+ using the command
\begin{verbatim}
ps2pdf -dEPSCrop figure.eps figure.pdf
\end{verbatim}

\begin{thebibliography}{69}
\bibitem{lamport94} Leslie Lamport, \emph{\LaTeX: A Document Preparation
System}. Addison Wesley, Massachusetts, 2nd Edition, 1994.
\end{thebibliography}

\newpage

\appendix

\section{Source Code}

Include all source code here in the appendix. Keep the code formatting clean,
use indentation, and comment your code to make it easy to understand. Also,
break lines that are too long. (Keep them under 80 characters!)

%\subsection{Calculating pi using matlab: \texttt{pi.m}}
%\lstinputlisting[language=matlab,numbers=left]{template_files/pi.m}

%\subsection{Calculating pi using python: \texttt{pi.py}}
%\lstinputlisting[language=python,numbers=left]{template_files/pi.py}


% \subsection{Main program task 1: \texttt{main\_T1.c}}
% \lstinputlisting[language=c,numbers=left]{../code/main_T1.c}

% \subsection{Main program  Task 2: \texttt{main\_T2.c}}
% \lstinputlisting[language=c,numbers=left]{../code/main_T2.c}

% \subsection{Temperature and pressure equilibration for tasks 3-7 : \texttt{main\_T3.c}}
% \lstinputlisting[language=c,numbers=left]{../code/main_T3.c}

% \subsection{Production runs for tasks 3-7 : \texttt{main\_Prod.c}}
% \lstinputlisting[language=c,numbers=left]{../code/main_Prod.c}

% \subsection{Production runs for tasks 3-7 : \texttt{main\_Prod.c}}
% \lstinputlisting[language=c,numbers=left]{../code/main_Prod.c}


% \subsection{Misc functions : \texttt{funcs.c}}
% \lstinputlisting[language=c,numbers=left]{../code/funcs.c}

% \section{Auxiliary }
% \subsection{Makefile}
% \lstinputlisting[language=bash,numbers=left]{../code/Makefile}

% \section{Matlab scripts}
% \subsection{Analysis scripts for tasks 3-7: \texttt{Al\_energies.m}}
% \lstinputlisting[language=matlab,numbers=left]{../m_scripts/H1_analysis.m}

% \subsection{Improve figure appearance: \texttt{ImproveFigureCompPhys.m}}
% \lstinputlisting[language=matlab,numbers=left]{../m_scripts/ImproveFigureCompPhys.m}

% \subsection{Change size of figures: \texttt{setFigureSize.m}}
% \lstinputlisting[language=matlab,numbers=left]{../m_scripts/setFigureSize.m}


\end{document}

%%% Local Variables:
%%% mode: latex
%%% TeX-master: t
%%% End:

%  LocalWords:  MSD
