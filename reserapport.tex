\documentclass[11pt,a4paper, english, swedish]{article}
\pdfoutput=1

\usepackage{custom_as}


%%Drar in tabell och figurtexter
\usepackage[margin=10 pt]{caption}
%%För att lägga in 'att göra'-noteringar i texten
\usepackage{todonotes} %\todo{...}

%%För att själv bestämma marginalerna. 
\usepackage[
%            top    = 3cm,
%            bottom = 3cm,
%            left   = 3cm, right  = 3cm
]{geometry}

%%För att ändra hur rubrikerna ska formateras
%\renewcommand{\thesection}{...}



\begin{document}

\title{Reserapport -- University of Waterloo (UW) 
\\ \Large Waterloo, Ontario, Kanada \large(utanför Toronto)}
\author{Andréas Sundström -- F}
\date{2016--2017}

\maketitle

\addtocounter{section}{-1}
\section{I Sverige}
Det är lite pilligt att fixa allt, men förhoppningsvis kan det här
avsnittet ge lite information om vad som behöver/kan ordnas här hemma.

\subsection{Språktest}
Om man läser äldre reserapporter från UW så verkar vissa ha klarat sig
utan ett språkprov. Så var det inte den här gången. I efterhand borde
jag ha varit mer aktiv med att undersöka detta, men som det föll sig
nu var det så att när vi hade fått kontrakt med vår koordinator på
Chalmers fick vi reda på att vi skulle kontakta UW\footnotemark{}.
Då fick vi veta om att det visst skulle krävas ett språkprov. 
\footnotetext{I skrivande stund var det Ibi Brown som man får tag i
  via
  \href{mailto:studyabroad@uwaterloo.ca}{\nolinkurl{studyabroad@uwaterloo.ca}}. }

Man kan här notera att detta fick vi reda på den 5:e februari och att
ansökan skulle in den 1:a mars. Nu gick visserligen
\href{http://www.folkuniversitetet.se/Las-mer-om-sprak/Sprakexamina/IELTS/IELTS-Goteborg/}{IELTS}
i Göteborg den 20:e februari, vilket ledde till en hektisk
tvåveckorsperiod med att försöka gräva upp sina gamla
engelka. \emph{Så om man är intresserad av att söka sig till ett
  utbyte bör man kolla upp med just det universitetet vad som kommer
  att krävas, så att man hinner förbereda sig lite inför eventuella
  prov.}

Själva provet i sig kan i stora drag sammanfattas med ''tänk
nationella i Engelska~B, men med den strängaste läraren du någonsin
träffat''. Allt är väldigt stikt. De kolla ditt pass hela tiden
under provet, man får inte använda sin egna penna, man får inte ens ha
med sig sitt armbandsur. Men bli heller \emph{inte för oroliga}, för
de flesta svenskar är det här provet inga problem. 

En sak till också om provet: det kostar 2\,250~SEK (för
närvarande). Inte det roligaste att skrapa ihop för att skriva ett
prov. Man får var beredd på att saker kommer att kosta; det är sådant
som resebidraget ska täcka, men de pengarna kommer inte förrän i
November.\todo{Uppdatera när pengarna faktiskt kom.}

\subsection{Ansökan till UW}
Själva ansökan skulle in redan den 1:e mars, men den var mest bara att
fylla i personuppgifter och vilket porgram man sökte till. Efter den
ansökan så kommer man få ett konto\footnotemark{} där man ska ladda
upp saker som kursval, studeresultat och sitt resultat i
engelskprovet. 

\footnotetext{Motsvarande typ Studentportalen, fast ganska mycket
  enklare/simplare.} 

Om man söker som non-degree student så behövdes inga referenser, trots
att det såg ut som det på ansöknigssidan. Man ska, om man gjort
allting rätt, inte helle behöva beta ansökningsavgiften på 100~CAD.

\subsubsection{Besked}
Både jag och Marcus sökte mestadels fysikkurser, och det verkar ha
gjort handläggningstiden lite längre än normalt. Oavsett vad så fick
vi inte våra antagningsbesked förrän \emph{9:e maj}. 

Det tog alltså drgyt två månader att få beskedet. Men det var inga
problem. Vi hann fixa allt som följer efter beskedet under tiden som
fanns kvar. 

\subsection{Söka bostad}
Jag vet inte hur högt tryck det är på bostäder där, men när jag och
Marcus skulle söka hängde vi på låset för när bostadskön
öppnade. Systemet blev överbelastat precis när bostadskön öppnade och
det dröjde nära 45~minuter innan jag kunde skicka in min ansökan.

Med det sagt, verkade det inte ha varit några problem att få sitt
boende. Vi fick besked två dagar senare att vi blivit erbjudna en
bostad. Då måste man också betala en 
\todo{Får man tillbaks pengarna?}handpenning på 500~CAD. Så var beredd
med pengar på kortet, för om man inte betalar inom en vecka tappar man
sitt erbjudande.
Allt detta skedde 10--12:e maj.

\subsection{Visum -- ''study permit''}
För att få studera i Kanada mer än 6~månader måste man ha ett ''study
permit''. Detta går att söka helt elektroniskt. Först skapar man ett
konto hos
\href{http://www.cic.gc.ca/english/information/applications/student.asp}{CIC
(kanadenskska migrationsmyndigheten)}. Sen laddar man upp de dokument
som krävs. Det ska inte vara några större bekymmer att fixa det.
De enda dokumenten som var lite oklara vad man behövde var ''Exchange
Letter'' och dokumenten som styrker din finansiella ställning.

För de finansiella dokumenten visade det sig räck med kontoutdrag som
visar hur mycket pengar man har tillgängligt. Det går också att söka
ett intyd från CSN om att man är berättigad studiemedel för den
utbildninga man söker. \todo{Räckte detta?}

Man kommer att få ett ''Letter of Acceptance'' från UW, detta ska med
i studietillståndsansökan. Men utöver det ska det även med ett
''Exchange Letter''. På CIC:s hemsida är det inte helt klart varifrån
man ska få tag i det. Men för oss verkade det funka att be Chalmrs
utbyteskoordinator\footnotemark{} att skriva ett intyg om att vi var
antagna till ett utbytesprogram. (Kolla vad som ska stå med när ni gör
er ansökan.) 
\footnotetext{I skrivande stund var det Ann-Marie Danielsson Alatalo, \href{mailto:Ann-Marie.Danielsson-Alatalo@chalmers.se}{Ann-Marie.Danielsson-Alatalo@chalmers.se}.}

\subsubsection{Övrig visering}
Tänk på att också skaffa en
\href{http://www.esta.us/sweden.html}{ESTA} för att få komma in i
USA. Det kostar bara 14~USD (i nuläget), och gäller i två år.


\subsection{Bank/finans}
Det är inga problem att överleva i Kanada med enbart en svanskbank och
bankkort. Men många banker tar ut ett valutaväxlingspåslag på 1--2\,\%
för köp i utländska valutor. 
%Detta kommer att ackumuleras till en hel del pengar under ett års utgifter. 


% \emph{Kolla med din bank vad den tar för avgifter och påslag.} Det kan
% vara värt att titta efter alternativ. 
% De alternativen jag har tittat på är: 
% \begin{itemize}
% \item Forex betal- och kreditkort. Det här kortet har inget
%   valutaväxlingspåslag, men för att kunna få det måste man ha en
%   taxerad inkomst (det räcker inte med studiemedel). Så om man har
%   jobbat och tjänat pengar här hemma kan man aänsöka om det här
%   kortet, annars är det ingen idé.
% \item Forex bankkort. Man öppnar ett bankkonto hos Forex, och så får
%   man ett bankkort kopplat till kontot. Dock är valutaväxlingspåslaget
%   1\,\% (det lägsta jag har hittat bland för mig tillgängliga alternativ). 
% \end{itemize}

\subsubsection{Betalning till Universitetet}
Tänk på att det forfarande är en del avgifter till universitetet som
måste betalas innan terminen. T.ex. ska hyran för \emph{hela terminen}
betalas till universitetet (om man bor på ett boende ordnat av dem). 


\subsection{Flyg}
Vi flög Iceland Air från Landvetter till Toronto. Den huvudsakliga
anledningen till detta valet var att de tillåter en att ta med två
incheckade bagage à \unit[23]{kg}, vilket kan behövas (själv landade
mina väskor på 22\,kg och 23\,kg).

\subsubsection{Transit från flygplatsen till Waterloo}
Från Toronto Pearson Airport går det s.k. ''shuttle buses'' till
Waterloo. Den vi åkte med hette Airways Transit och gick från
flygplatsen till universitetet. Med det skagt ska man dock komma ihåg
att universitetsområdet är väldigt stort och det kan vara långt från
avsläppningsplatsen till boendet. 

\section{Inledande intryck}



\end{document}







