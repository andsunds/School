\documentclass[11pt,a4paper, english, swedish]{article}
\pdfoutput=1

\usepackage{custom_as}


%%Drar in tabell och figurtexter
\usepackage[margin=10 pt]{caption}
%%För att lägga in 'att göra'-noteringar i texten
\usepackage{todonotes} %\todo{...}

%%För att själv bestämma marginalerna. 
\usepackage[
%            top    = 3cm,
%            bottom = 3cm,
%            left   = 3cm, right  = 3cm
]{geometry}

%%För att ändra hur rubrikerna ska formateras
%\renewcommand{\thesection}{...}



\begin{document}

\title{Reserapport -- University of Waterloo (UW) 
\\ \Large Waterloo, Ontario, Kanada \large(utanför Toronto)}
\author{Andréas Sundström -- F}
\date{2016--2017}

\maketitle

\addtocounter{section}{-1}
\section{Förberedelser Sverige}
Det är lite pilligt att fixa allt, men förhoppningsvis kan det här
avsnittet ge lite information om vad som behöver/kan ordnas här hemma.

\subsection{Språktest}
Om man läser äldre reserapporter från UW så verkar vissa ha klarat sig
utan ett språkprov. Så var det inte den här gången. I efterhand borde
jag ha varit mer aktiv med att undersöka detta, men som det föll sig
nu var det så att när vi hade fått kontrakt med vår koordinator på
Chalmers fick vi reda på att vi skulle kontakta UW\footnotemark{}.
Då fick vi veta om att det visst skulle krävas ett språkprov. 
\footnotetext{I skrivande stund var det Ibi Brown som man får tag i
  via
  \href{mailto:studyabroad@uwaterloo.ca}{\nolinkurl{studyabroad@uwaterloo.ca}}. }

Man kan här notera att detta fick vi reda på den 5:e februari och att
ansökan skulle in den 1:a mars. Nu gick visserligen
\href{http://www.folkuniversitetet.se/Las-mer-om-sprak/Sprakexamina/IELTS/IELTS-Goteborg/}{IELTS}
i Göteborg den 20:e februari, vilket ledde till en hektisk
tvåveckorsperiod med att försöka gräva upp sina gamla
engelka. \emph{Så om man är intresserad av att söka sig till ett
  utbyte bör man kolla upp med just det universitetet vad som kommer
  att krävas, så att man hinner förbereda sig lite inför eventuella
  prov.}

Själva provet i sig kan i stora drag sammanfattas med ''tänk
nationella i Engelska~B, men med den strängaste läraren du någonsin
träffat''. Allt är väldigt stikt. De kolla ditt pass hela tiden
under provet, man får inte använda sin egna penna, man får inte ens ha
med sig sitt armbandsur. Men bli heller \emph{inte för oroliga}, för
de flesta svenskar är det här provet inga problem. 

En sak till också om provet: det kostar 2\,250~SEK (för
närvarande). Inte det roligaste att skrapa ihop för att skriva ett
prov. Man får var beredd på att saker kommer att kosta; det är sådant
som resebidraget ska täcka, men de pengarna kom inte förrän i
November.

% \subsection{Ansökan till UW}
% Själva ansökan skulle in redan den 1:e mars, men den var mest bara att
% fylla i personuppgifter och vilket porgram man sökte till. Efter den
% ansökan så kommer man få ett konto\footnotemark{} där man ska ladda
% upp saker som kursval, studeresultat och sitt resultat i
% engelskprovet. 

% \footnotetext{Motsvarande typ Studentportalen, fast ganska mycket
%   enklare/simplare.} 

% Om man söker som non-degree student så behövdes inga referenser, trots
% att det såg ut som det på ansöknigssidan. Man ska, om man gjort
% allting rätt, inte helle behöva beta ansökningsavgiften på 100~CAD.

% \subsubsection{Besked}
% Både jag och Marcus sökte mestadels fysikkurser, och det verkar ha
% gjort handläggningstiden lite längre än normalt. Oavsett vad så fick
% vi inte våra antagningsbesked förrän \emph{9:e maj}. 

% Det tog alltså drgyt två månader att få beskedet. Men det var inga
% problem. Vi hann fixa allt som följer efter beskedet under tiden som
% fanns kvar. 


\subsection{Visum -- ''Study permit''}
För att få studera i Kanada mer än 6~månader måste man ha ett ''study
permit''. Detta går att söka helt elektroniskt, men det kostar 150\,CAD.

Först skapar man ett konto hos
\href{http://www.cic.gc.ca/english/information/applications/student.asp}{CIC
(kanadenskska migrationsmyndigheten)}. Sen laddar man upp de dokument
som krävs. Det ska inte vara några större bekymmer att fixa det.
De enda dokumenten som var lite oklara vad man behövde var ''Exchange
Letter'' och dokumenten som styrker din finansiella ställning.

För de finansiella dokumenten visade det sig räcka med kontoutdrag som
visar hur mycket pengar man har tillgängligt. Det går också att söka
ett intyd från CSN om att man är berättigad studiemedel för den
utbildninga man söker. 

Man kommer att få ett ''Letter of Acceptance'' från UW, detta ska med
i studietillståndsansökan. Men utöver det ska det även med ett
''Exchange Letter''. På CIC:s hemsida är det inte helt klart varifrån
man ska få tag i det. Men för oss verkade det funka att be Chalmrs
utbyteskoordinator\footnotemark{} att skriva ett intyg om att vi var
antagna till ett utbytesprogram. (Kolla vad som ska stå med när ni gör
er ansökan.) 
\footnotetext{I skrivande stund var det Ann-Marie Danielsson Alatalo,
  \href{mailto:Ann-Marie.Danielsson-Alatalo@chalmers.se}{Ann-Marie.Danielsson-Alatalo@chalmers.se}.} 

\subsubsection{Övrig visering}
Tänk på att också skaffa en
\href{http://www.esta.us/sweden.html}{ESTA} för att få komma in i
USA. Det kostar bara 14~USD (i nuläget), och gäller i två år.


\subsection{Bank/finans}
Det är inga problem att överleva i Kanada med enbart en svensk bank och
bankkort. Men många banker tar ut ett valutaväxlingspåslag på 1--2\,\%
för köp i utländska valutor. 

Jag skulle dock rekommendrea att skaffa ett kreditkort (svenskt). För
vissa ställen, som biluthyrningar, kräver att man anväder just ett
\emph{kredit}kort. 


\subsubsection{Betalning till Universitetet}
Tänk på att det forfarande är en del avgifter till universitetet som
måste betalas innan terminerna börjar. T.ex. ska hyran för hela terminen
betalas till universitetet (om man bor på ett boende ordnat av
dem). Bor man inte på ett ''on campus''-boende behöver man fortfarande
betala för en försäkring och studentkårsavgift. 

Betlningar till universitetet sker via Western Union. Och man betalar
från en svensk bank. Det kan gå att göra det via internetbank. Men på
grund av begränsningar hos min bank var jag tvungen att be mina
föräldrar att göra själva betalningen. 


\subsection{Flyg}
Vi flög med Iceland Air från Landvetter till Toronto. Den huvudsakliga
anledningen till detta valet var att de tillåter en att ta med två
incheckade bagage à \unit[23]{kg}, vilket kan behövas (själv landade
mina väskor på 22\,kg och 23\,kg).

\subsubsection{Transit från flygplatsen till Waterloo}
Från Toronto Pearson Airport går det s.k. ''shuttle buses'' till
Waterloo. Den vi åkte med hette Airways Transit och gick från
flygplatsen till universitetet. Med det skagt ska man dock komma ihåg
att universitetsområdet är väldigt stort och det kan vara långt från
avsläppningsplatsen till boendet. 







\section{Bostad}
Jag bodde ihop med Marcus, också från Chalmers, i vad som hette
Columbi Lake village (CLV). Det är lite dyrare än andra ställen, men
det är stort -- två våningar och källare, eget kök och tvättrum.  

\subsection{Söka bostad}
Jag vet inte hur högt tryck det är på bostäder där, men när jag och
Marcus skulle söka hängde vi på låset för när bostadskön
öppnade. Systemet blev överbelastat precis när bostadskön öppnade och
det dröjde nära 45~minuter innan jag kunde skicka in min ansökan.

Med det sagt, verkade det inte ha varit några problem att få sitt
boende. Vi fick besked två dagar senare att vi blivit erbjudna en
bostad. Då måste man också betala en 
handpenning på 500~CAD. Så var beredd
med pengar på kortet, för om man inte betalar inom en vecka tappar man
sitt erbjudande.
Allt detta skedde 10--12:e maj.




\section{Vid sidan om studierna}
Kitchener-Waterloo (KW) är två samanvuxna städer med totalt ca
300\,000~invånare -- alltså lite mindre än Göteborg. Men staden i sig
är rätt så trist faktiskt. Kring universitetet finns det en del
aktiviteter, men mitt bästa tips är att resa runt utanför KW. Toronto
är ju t.ex. inte allt för långt bort.

En annan sak man inte ska missa är att besöka ett gäng
nationalparker. Hyr en bil och åk till någon av många parker i
närheten. Ibland anordnas det även lite längre resor med
övernattning. 

Tyvärr är kollektivtrafiken inte direkt värdsklass, men tycker man om
att åka tåg finns det speciella tågpass man kan köpa. Dessa gör att
tågresor inte blir helt orimligt dyra, vilke de kan vara i annat
fall. Det finns ett pass för resor längs med den så kallade
''Québec--Windsor''-korridoren; med det passet kan man ta sig till
många av de större säderna i östra Kanada. 



\section{Diverse anmärknigar}

En tråkig aspekt av Nordamerika är att många städer i allmänhet är
byggda kring att man her en bil. 







\section{Kurser}
På UW:s hemsida säger de att 0,5~credit på graduate level motsvarar
9--10~ECTS. Så tre kurser per termin, eller lite mer, är vad man ska
ta. Det ska dock noteras att mastersstudier där skiljer sig från i
Sverige. På UW ska en mastersstudent ta \emph{totalt} fyra (4) kurser
för sin examen; istället läggs mycket mer tid på forskning och
undervisnig. Så de flesta mastersstudenter på UW tar sällan mer än två
kurser parallellt. Som utbytesstudent behöver man dock fortfarande
klämma in minst tre kurser per termin för att komma upp i sina
30~ECTS. 

De fysikkurserna jag tog samläses med University of Guelph, vilket
betydde att kurserna gick på videolänk. Iblan var föreläsaren hos oss
och ibland var han i Guelph. Överlag gick det ganska bra med
videolänken, men vissa tillfällen tappade vi kontakten och var tvungna
att avbryta föreläsningara. 

\subsection{Fall term (höstterminen)}
Jag tog tre kurser den här terminen. Och ärligt talat så var de tre
kurserna lite mycket att orka med i. Detta pågrund av att alla tre
kurserna var rätt så tunga i hemläxor -- jag lämnade i medel in
10~(\LaTeX{}ade) sidor per hemläxa och med fem läxor i varje kurs blri det en del
jobb. Jag skulle tro att mindre teroretiska kurser inte har det lika
tufft med läxor.

\paragraph{PHYS\,701 -- Quantum Mechanics 1}
En fortsättningskusr i kvantmekanik -- innehåller bl.a. lite
vägintegraler, spridningsteori och relativistisk kvantmekanik. Man
behöver defiitivt ha läst en introkurs för att ta den här kursen. Som
jag har upplevt det så är de kvantkurser man tar på F lite mindre än
vad andra fysikstudenter brukar ha som bakgrund\footnotemark{}. Med
det sagt så var det inte så svårt att hänga med, även om de började
strax ovanför vad man får i grundutbildningen på F. 
\footnotetext{På UW tar man tre kvantkurser som under
  graduate. Bl.a. har man redan läst relativistisk kvantmekanik när
  man kommer från en grundutbildning i fysik på UW.}

Kursen hölls av Jim Martin, en lite udda men trevlig professor. En bra
resurs vid läxläsnig är att maila och fråga om man kan komma förbi och
få lite hjälp med läxan. 


\paragraph{PHYS\,704 -- Statistical Physics 1}
Som namnet anger så är det en kurs i statistisk mekanik. Den börjar
men lite repetition av vad som ingick i termon på F, men fortsätter
djupare med ''secod quantaization'' och superfluiditet och
supraledning. Att ha sett lite stat. mek. tidiagre är ju hjälpsamt.

Även här var professorn, Anton Burkov, mycket trevlig och hjälpsam om
man behöver hjälp med hemläxorna. 


\paragraph{AMATH732 -- Asymptotic Analysis and Perturbation Theory}
En rolig kurs om man är intresserad av uppskattningar och
approximationsmetoder. Kursen börjar med hur man hittar approximativa
lösningar till polynom, för att sen fortsätta vidare till
differentialekvationer och asymptotiska lösnigar av dito. 
Inga särskilda förkunskaper behövs, annat än grundläggande kunskap i
ODE:er. 

Det speciella med den här kursen var att det bara var totalt fyra (4)
studenter som gick den mitt år. Här var professorn lite svårare att
bara gå och fråga, men med den lilla klassen blev det naturligt en
lite studiecirkel som man kunde få hjälp från. 


\subsection{Winter term (vårterminen)}


\paragraph{PHYS705 -- Statistical Physics 2}


\paragraph{PHYS706 --  Electromagnetic Theory}

\paragraph{AMATH741 -- Numerical Solutions of PDE's}




\end{document}







