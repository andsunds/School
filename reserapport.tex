\documentclass[11pt,a4paper, english, swedish]{article}
\pdfoutput=1

\usepackage{custom_as}


%%Drar in tabell och figurtexter
\usepackage[margin=10 pt]{caption}
%%För att lägga in 'att göra'-noteringar i texten
\usepackage{todonotes} %\todo{...}

%%För att själv bestämma marginalerna. 
\usepackage[
%            top    = 3cm,
%            bottom = 3cm,
%            left   = 3cm, right  = 3cm
]{geometry}

%%För att ändra hur rubrikerna ska formateras
%\renewcommand{\thesection}{...}



\begin{document}

\title{Reserapport -- University of Waterloo (UW)}
\author{Andréas Sundström -- F}
\date{2016--2017}

\maketitle

\addtocounter{section}{-1}
\section{Inför utbytet}
Ett ganska standard 

\subsection{Språktest}
Om man läser äldre reserapporter från UW så verkar vissa ha klarat sig
utan ett språkprov. Så var det inte den här gången. I efterhand borde
jag ha varit mer aktiv med att undersöka detta, men som det föll sig
nu var det så att när vi hade fått kontrakt med vår koordinator på
Chalmers fick vi reda på att vi skulle kontakta UW\footnotemark{}.
Då fick vi veta om att det visst skulle krävas ett språkprov. 
\footnotetext{I skrivande stund var det Ibi Brown som man får tag i
  via
  \href{mailto:studyabroad@uwaterloo.ca}{\nolinkurl{studyabroad@uwaterloo.ca}}. }

Man kan här notera att detta fick vi reda på den 5:e februari och att
ansökan skulle in den 1:a mars. Nu gick visserligen
\href{http://www.folkuniversitetet.se/Las-mer-om-sprak/Sprakexamina/IELTS/IELTS-Goteborg/}{IELTS}
i Göteborg den 20:e februari, vilket ledde till en hektisk
tvåveckorsperiod med att försöka gräva upp sina gamla
engelka. \emph{Så om man är intresserad av att söka sig till ett
  utbyte bör man kolla upp med just det universitetet vad som kommer
  att krävas, så att man hinner förbereda sig lite inför eventuella
  prov.}

Själva provet i sig kan i stora drag sammanfattas med ''tänk
nationella i Engelska~B, men med den strängaste läraren du någonsin
träffat''. Allt är väldigt stikt. De kolla ditt pass hela tiden
under provet, man får inte använda sin egna penna, man får inte ens ha
med sig sitt armbandsur. 

En sak till också om provet: det kostar 2\,250~SEK. Inte det roligaste
att skrapa ihop för att skriva ett prov. Man får var beredd på att saker
kommer att kosta; det är sådant som resebidraget ska täcka, men de
pengarna kommer inte förrän i November.\todo{Uppdatera när pengarna
  faktiskt kom.}

\subsection{Ansökan}
Själva ansökan skulle in redan den 1:e mars, men den var mest bara att
fylla i personuppgifter och vilket porgram man sökte till. Efter den
ansökan så kommer man få ett konto\footnotemark{} där man ska ladda
upp saker som kursval, studeresultat och sitt resultat i
engelskprovet. 

\footnotetext{Motsvarande typ Studentportalen, fast ganska mycket
  enklare/simplare.} 

Om man söker som non-degree student så behövdes inga referenser, trots
att det såg ut som det på ansöknigssidan. Man ska, om man gjort
allting rätt, inte helle behöva beta ansökningsavgiften på 100~CAD.





\end{document}





%% På svenska ska citattecknet vara samma i både början och slut.
%% Använd två apostrofer (två enkelfjongar): ''.


%% Inkludera PDF-dokument
\includepdf[pages={1-}]{filnamn.pdf} %Filnamnet får INTE innehålla 'mellanslag'!

%% Figurer inkluderade som pdf-filer
\begin{figure}\centering
\centerline{ %centrerar även större bilder
\includegraphics[width=1\textwidth]{filnamn.pdf}
}
\caption{\label{fig:} }
\end{figure}

%% Figurer inkluderade med xfigs "Combined PDF/LaTeX"
\begin{figure}\centering
\resizebox{.8\textwidth}{!}{\input{filnamn.pdf_t}}
\caption{\label{fig:} }
\end{figure}

%% Figurer roterade 90 grader
\begin{sidewaysfigure}\centering
\centerline{ %centrerar även större bilder
\includegraphics[width=1\textwidth]{filnamn.pdf}
}
\caption{\label{fig:} }
\end{sidewaysfigure}


%%Om man vill lägga till något i TOC
\stepcounter{section} %Till exempel en 'section'
\addcontentsline{toc}{section}{\Alph{section}\hspace{8 pt}Labblogg} 

