\chapter{Partikelrörelse i celler}

Partikelrörelse i cellers cytoplasma är något som länge fascinerat forskare inom området. En enkel modell som modellerar partikelrörelserna som klassisk brownsk rörelse kan inte förklara de observerade egenskaperna där en långsammare diffusion är ett av de mest tydliga exemplen. Andra modeller har tagits fram i avsikt att försöka förklara rörelsen men i dagsläget finns inte någon universell modell som kan beskriva rörelsen fullständigt. 

Detta kapitel presenterar inledningsvis tre möjliga förklaringsmodeller till partikelrörelsen:  Ornstein-Uhlenbeck-processen, CTRW och fBm som alla tre har kopplingar till klassisk brownsk rörelse. 

I denna studie undersöks om given data över partikelrörelse i jästceller har egenskaper, bland annat MSD och asphericity, som kan beskrivas av dessa modeller. Både teoretiska beräkningar och simuleringar används för att göra jämförelsen. Den stora skillnaden mellan CTRW och fBm är att CTRW är en icke-stationär process medan fBm är stationär, något som används för att avgöra deras relevans som förklaringsmodeller. \todo{O-U är stationär.}

Några av de viktigaste resultaten som presenteras i detta kapitel är att rörelsen verkar vara stationär och att ... \todo{Här kan man fylla i det man tycker är relevant}



\section{Modeller för partikelrörelse i vätskor}

Till en första approximation skulle rörelsen för en partikel i en jästcells cytoplasma kunna beskrivas med klassisk brownsk rörelse. Partiklarna krockar där med mindre partiklar från omgivningen och rörelsen kan då beskrivas med en stokastisk gaussisk propagator~\cite{Einstein1905}
\begin{equation}
P(x,t)=\frac{1}{\sqrt{4\pi Dt}}e^{-\nicefrac{x^2}{4Dt}}
\end{equation} %Obs n=1 då endast en partikel följs
som utgörs av sannolikhetstätheten för partikelns förflyttning $x$ vid en given tid $t$, där konstanten $D$ är dess diffusionskonstant. Om partikeln betraktas som en sfär med radie $r$ nedsänkt i en vätska med dynamisk viskositet $\mu$ och absolut temperatur $T$ kan $D$ approximeras till~\cite{Einstein1905}
\begin{equation}
D=\frac{k_\text{B} T}{6\pi \mu r}.
\end{equation}
%Man betraktar då cytoplasman som en homogen vätska. %Denna teori bygger dock på att man har termisk jämvikt och att partiklar rör sig i en enbart viskös vätska, två kriterier som inte uppfylls i cytoplasman bland annat på grund av mitokondriernas energiutvinning.

Experiment\cite{Midtveldt_etal2016} har dock visat att diffusionen av partiklar i celler är fundamentalt långsammare än förväntat från klassisk brownsk rörelse. Det finns ett par verktyg man kan använda för att undersöka detta. Ett av dem är MSD. Som visades i avsnitt~\ref{sec:brown} är MSD:n för brownsk rörelse proportionell mot förlupen tid. Speciellt kan \eqref{eq:MSD_brown} skrivas om till
\begin{equation}
\ev{x(t)^2} \propto t,
\end{equation}
där $x(0)=0$. Vad \cite{Midtveldt_etal2016} då har funnit i tidigare studier är att exponenten på $t$ är lägre än~$1$, vilket tyder på att andra förklaringsmodeller kan behövas. %Två sådana kandidater är ''Continuous Time Random Walk'' (CTRW) och ''Fractional Brownian motion''.



%\subsubsection{Brownsk rörelse} \label{sec:Brownsk}
%\todo[inline]{Skriv ihop kort sammanfattning av det som står i stokastikavsnittet}
%\todo[]{Förklara varför anomal transport ens är intressant att studera.}


\subsection{Ornstein-Uhlenbeck-process}
En första utvidgning av modellen för brownsk rörelse fås genom att lägga till en återförande term som drar partikeln mot någon medelpunkt. Detta är en så kallad Ornstein-Uhlenbeck-process
\begin{equation}\label{eq:SDE_o-u}
\pd_t x = -\gamma ( x-\bar{x} ) + \pd_t W,
\end{equation}
där $\gamma$ är en tidsskala som styr hur hårt bunden partikeln är till medelpositionen $\bar{x}$, och $\pd_t W$ är den stokastiska drivningen av partikeln som uppfyller $\ev{\pd_t W}=0$ och $\ev{\pd_t W(t)\pd_t W(t')} = \sigma^2\delta(t-t')$. Detta är en inte helt orimlig modell eftersom en partikel i en cell rimligtvis inte kan vandra hur långt som helst. Typiskt kommer en partikel att tränga ihop cytoplasmans beståndsdelar i den riktning som den rör sig åt, vilket leder till en återförande kraft.

\todo[inline]{Kolla om dessa ekvationer stämmer.}
Med \eqref{eq:SDE_o-u} och metoderna från avsnitt~\ref{sec:brown} kan modellens kovarians och MSD härledas. Kovariansen i gränsen mot ett stationärtillstånd ($\gamma t\gg 1$) blir då
\begin{equation}\label{eq:COV_o-u}
\ev{x(t)x(t+\Delta{t})} \approx \frac{\sigma^2}{2\gamma} \ee^{-\gamma\Delta{t}},
\end{equation}
där $\bar{x}$ har satts till 0. Vidare kan också MSD beräknas till:\todo{ev. faktor 2 här enligt måns?}
\begin{equation}\label{eq:MSD_o-u}
\ev{x(t)^2} 
\approx \frac{\sigma^2}{2\gamma} \left( 1-\ee^{-2\gamma t} \right),
%\approx \frac{\sigma^2}{2\gamma}t\qcomma \text{för } \gamma t\ll1
\end{equation}
med $\bar{x}=0$. Notera att både \eqref{eq:COV_o-u} och \eqref{eq:MSD_o-u} ger att variansen, då $\gamma t\gg 1$, blir konstant lika med~$\nicefrac{\sigma^2}{2\gamma}$ eftersom $\ev{x(t)}=\bar{x}=0$.

%Notera likheten här till den styrande SDE:n för brownsk rörelse.

%\todo[inline]{Kommer mera så småningom...}
  
\subsection{Continuous Time Random Walk (CTRW)}
En möjlig förklaringsmodell för anomal transport i celler utgörs av CTRW (Continuous-Time Random Walk)~\cite{Hofling&Franosch2013}. Här beskrivs rörelsemönstret av att partiklarna under majoriteten av tiden sitter bundna till olika nätliknande strukturer för att sedan plötsligt ta sig vidare till en ny position efter en viss tid. Partiklarnas steglängder samt väntetider mellan steg är stokastiska variabler och för härledning av partikelns egenskaper se bilaga~\ref{app:CTRW}. Givet att väntevärdet av steglängden är noll så blir MSD för CTRW vid stora $t$
\begin{equation}\label{eq:CTRW_MSD}
    \ev{x(t)^2} \propto \frac{\sigma^2t^{\alpha}}{\Gamma (\alpha+1)}\quad 0<\alpha<1,
\end{equation}
där $\sigma$ är variansen för stegen, $\alpha$ är en konstant kopplad till sannolikhetsfördelningen för tiden mellan stegen och $\Gamma$ är gammafunktionen. Eftersom $0<\alpha<1$ utför CTRW subdiffusion vilket skiljer sig från den klassiska brownska rörelsen. Denna anomala transport uppstår då medelväntetiden mellan två hopp blir oändlig så att centrala gränsvärdessatsen därmed ej uppfylls. Summan av de stokastiska variablerna går således ej mot att bli normalfördelad, något som är grundläggande i teorin kring brownsk rörelse.
\todo{Lägg till mer om var konstanterna kommer ifrån och ev fler förutsägningar}


\begin{comment}
Positionsändringen och väntetiden beskrivs av två oberoende stokastiska variabler. 
MSD för denna typ av rörelse blir\cite{Barkai_MSDCTRW2007}
\begin{equation}
    \ev{x^2(t)} \approx \frac{\ev{\Delta x^2}}{A \Gamma(1+\alpha)}t^{\alpha} + \frac{2\ev{\Delta x}^2}{\Gamma(1+2\alpha)A^2} t^{2\alpha}, 
\end{equation}
där $A$ är en konstant som dyker upp i fördelningsfunktionen för väntetiderna, $\Gamma$ är gammafunktionen, $\alpha$ en konstant som uppfyller $0<\alpha<1$ och $\Delta x$ steget mellan två positioner. Om $\ev{\Delta x}=0 $ försvinner andra termen och rörelsens MSD blir proportionell mot $t^\alpha$. Eftersom $\alpha < 1$ blir MSD:n inte linjär i tiden, utan ändringstakten kommer att avta med tiden och rörelsen skiljer sig från den klassiska brownska rörelsen.
\end{comment}

Teorin har visat sig kunna beskriva vissa aspekter av partikelrörelse i nät av F-aktinfilament\cite{Barkai_CTRW}. Om partikelns radie var av samma storleksordning som genomsnittliga maskstorleken bands partiklarna tillfälligt i nätet, bortsett från termiska fluktuationer, för att sedan ta sig igenom en maska och fastna i ett nytt hålrum i nätet. MSD:n för dessa partiklar uppfyllde ett $t^{\alpha}$-beroende som i \eqref{eq:CTRW_MSD}. Partiklar med radier som var betydligt mindre än maskstorleken hade ett rörelsemönster mer likt brownsk rörelse medan de större partiklarna fastnade i nätet utan att kunna ta sig loss.
%Om en liknande nätstruktur kan uppstå i jästceller kan man alltså förvänta sig olika resultat vad gäller rörelsen för partiklar av olika storlek.

%Lästips:
%https://faculty.biu.ac.il/~barkaie/PCCPreview.pdf
%https://faculty.biu.ac.il/~barkaie/Pathways.pdf


\subsection{Fractional Brownian Motion (fBm)}

En annan modell som kan undersökas är fractional Brownian motion~\cite{Mandelbrot_fBm1968} som bygger på superpositioner av brownska processer med brus som uppvisar en beständig korrelation i tiden. 

Ett sätt att representera fBm är att utgå från en vanlig brownsk rörelse $B(t)$ och vikta tidigare steg i rörelsen med $((t-s)^{H-\nicefrac{1}{2}})$ \todo{vad är t,s och H? utveckla} så att fBm:n blir ett \todo{glidande medelvärde?}rörligt medelvärde av $\dd{B}(t)$ enligt \cite{Mandelbrot_fBm1968}
\begin{equation} \label{eq:fBm_repr}
\begin{aligned}
    B_H(t,\omega)=& \frac{1}{\Gamma(H+\nicefrac{1}{2})} 
    \int_{\infty}^0 \left( (t-s)^{H-\nicefrac{1}{2}}-(-s)^{H-\nicefrac{1}{2}}\right)\,\dd{B}(s,\omega) \\
    &+ \int_0^t (t-s)^{H-\nicefrac{1}{2}} \dd{B}(s,\omega)
\end{aligned}
\end{equation}
för $t>0$ utgående från startposition i 0 där $\omega$ representerar mängden av alla möjliga värden för den stokastiska funktionen och $\Gamma$ är gamma-funktionen. Hur denna typ av integral med stokastisk integrationsvariabel kan utföras förklaras i avsnitt~\ref{sec:Stok_int}.
%Denna representation är dock inte unik \cite{Dieker_fBm}.


Mätningar på diffusionsartade stokastiska processer har visat på starkt beroende även mellan steg som är separerade långt i tid~\cite{Mandelbrot_fBm1968} något som inte kan förklaras med den brownska rörelsens exponentiellt avtagande korrelation \todo{Eller är det de okorrelerade stegen man avser här?}i \eqref{eq:Brown_korr}. Detta motiverar införandet av denna modell som kandidat för att beskriva   diffusion i celler. Modellen förutsäger också en minskad diffusionstakt jämfört med vanlig brownsk rörelse, något som observerats i verkliga celler~\cite{Hofling&Franosch2013}.

En normerad fBm $B_H(t)$ kan karakteriseras~\cite{Dieker_fBm} helt från att $B_H(t)$ har stationära och gaussiskt fördelade steg för $t>0$ och $B_H(0)=0$ samt att
\begin{equation}
\begin{aligned}
    \ev{B_H(t)}&=0 \\
    \ev{B^2_H(t)}&=t^{2H}
\end{aligned}
\end{equation}
för $t\geq 0$, där Hurstparametern $H$ uppfyller $0< H <1$. För $H=\nicefrac{1}{2}$ återfås vanlig brownsk rörelse. Stationära ökningar medför att 
\begin{equation}
    \ev{(B_H(t_1)-B_H(t_2))^2} = \abs{t_1-t_2}^{2H},
\end{equation}
eller ekvivalent
\begin{equation}
    \ev{B_H(t_1)^2}+\ev{B_H(t_2)^2}-2\ev{B_H(t_1)B_H(t_2)} = \abs{t_1-t_2}^{2H}.
\end{equation}
Kovariansen för fBm är således
\begin{equation}
\ev{B_H(t_1)B_H(t_2)}
= \frac{1}{2} \left(t_1^{2H}+t_2^{2H}-\abs{t_2-t_1}^{2H}\right),
\end{equation} \todo{Fixa så att appendix och denna stämmer överens i notation}
vilket ger att för $H<\nicefrac{1}{2}$ fås positiv korrelation för två positioner i rörelsen och för  $H>\nicefrac{1}{2}$ fås negativ korrelation. Den negativa korrelationen ger brusigt utseende åt plotten över rörelsen då rörelsen ofta byter riktning medan den positiva korrelationen ger kurvan ett mer slätt utseende som tenderar att dra iväg i en bestämd riktning. Man kan även visa att fBm är den enda gaussiska process som uppträder självliknande \todo{Källa}[], det vill säga att $B_H(at)$ och $a^H B_H(t)$ har samma ändligtdimensionella fördelning.

Stegen mellan varje positionsändring i en normerad fBm:  $\Delta{B_H}_k=B_H(k+1)-B_H(k)$ är standardnormalfördelad för varje $k$ men tillskillnad från ren brownsk rörelse är stegen i allmänhet inte oberoende. Ovan utpekades dessa ökningars stationära egenskap som en av de karakteristiska dragen för fBm. Att ökningarna är stationära innebär att de saknar explicit tidsberoende, de beror bara på tidsintervallets storlek.

Även om fBm har stationära ökningar är processen i sig inte stationär \cite{Flandrin_fBmspektrum1989}, vilket ger en tidsberoende spektraltäthet. Istället kan det vara av intresse att betrakta de stationära stegen för en fBm
\begin{equation}
    \Delta{B_H}(t)=B_H(t+T)-B_H(t)
\end{equation} 
kallat fractional Gaussian noise (fGn). PSD för dessa räknas fram i bilaga~\ref{sec:App_fBm} och blir
\begin{equation}
    W_{\Delta{B_H}}(t,\omega)=(\sin{\frac{\omega T}{2}})^2 \abs{\omega}^{-(2H+1)}
\end{equation}
dvs oberoende av tid.
    



\section{Intensitets- och storleksberoende samt brusunersökning}
%Om värdena skiljer sig åt väsentligt för små och stora partiklar -> CTRW mer trolig. 

Den studerade datan består av ett hundratal olika partiklar i respektive fas. Alla dessa partiklarna har olika storlek, vilket speglas i att de får olika hög intensitet när de filmas i mikroskop. I nuläget finns ingen bra kännedom om hur en partikels intensitet förhåller sig till dess radie -- \cite{Parry_etal2014} som hade samma sorts partiklar som här gjorde ett försök att jämföra med partiklar med känd storlek. Den här studien kommer inte att försöka reda ut något samband mellan en partikels intensitet och storlek, men för att kunna ta medelvärden som inte påverkas av partikelstorlek behövs ändå vissa samband mellan exempelvis intensitet och hur mycket partikeln rör sig. 

För det första behövs en definition på ''hur mycket partikeln rör sig''. En sådan definition bör spegla den egenskap som man vill undersöka via medelvärdering. I de flesta fall, exempelvis MSD, visar sig positionens varians vara ett lämpligt mått att använda för normering. Dock används inte variansen som direkt normering eftersom det egentligen är \emph{intensiteten} som ska korrigeras för.

\subsubsection{Undersökning av mätbrus}
En viktig aspekt i alla fysikaliska mätningar är hur mycket brus datan innehåller. Så även i den här studien. Dock kan det verka motsägelsefullt att försökta uppskatta hur mycket brus man har i data som består av slumpvandrande partiklar -- också det en sorts ''brus''. Det går dock att göra några försök. 

Mätbruset kan uppskattas genom att ansätta en uppmätt position på formen
\begin{equation}
\hat{x}_i = x_i(I) + \sigma_\text{brus}\eta_i,
\end{equation}
där $x(I)$ och $\sigma_\text{brus}\eta_i$ är ett oberoende normalfördelat mätbrus med standardavvikelse $\sigma_\text{brus}$. Vidare antas även att partiklarnas verkliga rörelse minskar med ökad storlek så att mätbruset blir dominant hos partiklar med stora intensiteter:
\begin{equation}\label{eq:approx_noise}
\hat{x}_i \approx \sigma_\text{brus}\eta_i \qcomma \text{för stora } I.
\end{equation}
Med detta går det nu att uppskatta mätbrusets styrka, $\sigma_\text{brus}$, genom att undersöka exempelvis steglängdens asymptotiska beteende. 

Om man undersöker medelsteglängden $\Delta{\bar{r}}$ under antagandet att \eqref{eq:approx_noise} gäller finner man att
\begin{equation}\label{eq:delta_r_noise}
\begin{aligned}
\Delta{\bar{r}} =& 
\sqrt{ \ev{(\hat{x}_{i+1}-\hat{x}_{i})^2 + (\hat{y}_{i+1}-\hat{y}_{i})^2 }}
\\
\approx& 
\sigma_\text{brus} \ev{ \left(\eta^{(x)}_{i+1} - \eta^{(x)}_{i}\right)^2 + \left(\eta^{(y)}_{i+1}-\eta^{(y)}_{i}\right)^2 }^{\nicefrac{1}{2}} \\
=&2\sigma_\text{brus}.
\end{aligned}
\end{equation}
Här har faktumet att en summa av oberoende stokastiska variabler får en varians som är summan av variansen använts för att evaluera den andra raden i \eqref{eq:delta_r_noise}.
På samma sätt kan även partikelns standardavvikelse beräknas i gränsen med stora $I$
\begin{equation}\label{eq:sigma_r_noise}
\begin{aligned}
\sigma_r =& 
\sqrt{\VAR{\hat{x}} + \VAR{\hat{y}_{i}} }\\
\approx& \sigma_\text{brus} \sqrt{\VAR{\eta^{(x)}} + \VAR{\eta^{(y)}} }\\
=&\sqrt{2}\,\sigma_\text{brus}.
\end{aligned}
\end{equation}



\subsection{Resultat -- }
De trendlinjer som har anpassats till $\sigma_r$ är gjorda som potenssamband mellan $\sigma_r$ och~$I$; på så sätt erhålls linjer i dubbellogplott. Från \figref{fig:storleksberoende} ses att $\sigma_r^{\text{(dvala)}} \propto I^{-0,41}$ och $\sigma_r^{(\text{log-fas})} \propto I^{-0,17}$. Det verkar alltså finnas en skillnad mellan celler i dvala och i log-fas med avseende på hur intensitetsberoendet ser ut.

%Brus
Eftersom \eqref{eq:delta_r_noise} har en något större faktor framför $\sigma_\text{brus}$ används $\Delta{\bar{r}}$ för att bestämma $\sigma_\text{brus}$ -- i \figref{fig:storleksberoende} syns också att det finns ett tydligare trendbrott i medelsteglängderna än i standardavvikelserna. Det bestämda värdet på brusnivån, $\sigma_\text{brus}$, från $\Delta{\bar{r}}$ kan sedan användas i \eqref{eq:sigma_r_noise} för att se om $\sigma_r$ ser ut att gå mot det beräknade värdet $\sqrt{2}\sigma_\text{brus}$. Detta borde i så fall synas i form av en avvikelse från trendlinjen där den korsar brusnivån.
I fallet med celler i dvala, \figref{fig:storleksberoende}~(a) och (c), ser det ut att kunna stämma, men data saknas tyvärr i just det intressanta området. Partiklarna från celler i log-fasen, \figref{fig:storleksberoende}~(b) och (d), har ett mycket svagare intensitetsberoende och där går inte att säga något om huruvida den beräknade brusnivån stämmer. 

Egentligen är den enda indikationen på att det skulle finnas en brusnivå utplaningen\footnotemark{} i \figref{fig:storleksberoende}~(c). Och om det rör sig om mätbrus borde $\sigma_\text{brus}$ vara samma för partiklar i båda cellerfaserna.\todo{nämner vi att data från de olika faserna är tagna på samma sätt? därmed bör bruset vara samma} Sammantaget ger denna brusanalys alltså att mätbrusets storlek blir $\sigma_\text{brus}=\unit[0,5]{nm} \pm 10\,\%$. Här har osäkerheten beräknas genom att ta standardavvikelsen i $\Delta\bar{r}$ för de partiklar vars intensitet översteg $12$\,intensitetsenheter i \figref{fig:storleksberoende}.

\footnotetext{Läsaren påminns om att $I$-axeln är logaritmerad, vilket leder till att området från 5--20 ser betydligt mindre ut i jämförelse med hela intervallet än vad det är. Detta gör tyvärr att utplaningen inte ser ut att vara så betydelsefull. }


\begin{figure}\centering
   \subfigure[][]{\input{bilder/partiklar/std_ed.tex}}
   \subfigure[][]{\input{bilder/partiklar/std_lp.tex}}
   \subfigure[][]{\input{bilder/partiklar/medelsteg_ed.tex}}
   \subfigure[][]{\input{bilder/partiklar/medelsteg_lp.tex}}
\caption{Varje enskild partikels standardavvikelse $\sigma_r$ och medelsteglängd $\Delta{\bar{r}}$ plottade mot deras intensitet $I$ i mikroskopet. 
Med standardavvikelse menas hur stor standardavvikelse partikeln hade i sin position enligt $\sigma_r=(\sigma_x+\sigma_y)^{1/2}$, där $\sigma_x$ och $\sigma_y$ är standardavvikelserna i $x$- respektive $y$-koordinaten.
%En partikels intensitet svarar mot dess storlek, så det vore rimligt att deras steglängder och standardavvikelse asymptotiskt går mot $0$.
Från steglängdens asymptotiska betende för stora intensiteter uppskattades mätbrusets storlek; detta verkar dock bara går för partiklar i celler i dvala. 
Anledningen till att brusnivån skiljer sig åt mellan medelsteg och standardavvikelse är att det kommer in olika faktorer för ett mätbrus i de olika måtten -- se \eqref{eq:delta_r_noise} och \eqref{eq:sigma_r_noise}. }
\label{fig:storleksberoende}
\end{figure}



\section{Undersökning av olika typer av MSD}

I det här avsnittet jämförs data med de tidigare nämnda modellernas förutsägelser. Bland annat räknades det fram i avsnitt~\ref{sec:brown} att MSD:n för en sådan partikel skulle öka linjärt med tiden, se \eqref{eq:MSD_brown}.

Vidare finns det minst två sätt att beräkna partiklarnas MSD. För stationära processer, det vill säga processer som inte explicit beror på när i tiden de inleds, kan man skapa ett medelvärde mellan alla möjliga mätpunkter separerade med givet tidsintervall $\Delta{t}$ enligt
\begin{equation} \label{eq:MSD_S}
    S(\Delta t)= \frac{1}{N}\sum^N_{i=1}\frac{1}{m}\sum^m_{j=1}(x_i(t_j+\Delta t)-x_i(t_j))^2
\end{equation} 
där $N$ är antalet partiklar och $m$ antalet möjliga tidsintervall av längd $\Delta t$. 
Man kan också betrakta
\begin{equation} \label{eq:MSD_s}%Byta plats?
    s(\Delta t)= \frac{1}{N}\sum^N_{i=1}(x_i(\Delta t)-x_i(0))^2
\end{equation} 
som inte kräver att processen är stationär.
Genom att jämföra resultatet mellan dessa två typer av MSD kan man avgöra om processen är stationär. Eftersom en stationär process är oberoende av tidstranslation och därmed borde ge samma resultat för $S$ och $s$. 

De givna ekvationerna är för ändliga och diskreta datamängder, men $S$ och $s$ skulle också kunna beskrivas teoretiskt med en väntevärdesbeskrivning.

En annan skillnad som kan uppstå när man beräknar MSD med \eqref{eq:MSD_S} respektive \eqref{eq:MSD_s} är att i en praktisk tillämpning, med begränsad datamängd, så kan $S(\Delta{t})$ bli betydligt jämnare än $s(\Delta{t})$. Detta på grund av att \eqref{eq:MSD_S} innehåller betydligt fler termer i varje medelvärde. 

%\subsubsection{Databehandling} \todo[inline]{Utveckling pågår}

\subsection{Resultat -- MSD skiljer sig mellan de olika faserna}

För att undersöka om partikelrörelsen utgör en stationär process beräknas den stationära och icke-stationära MSD:n för partiklarna enligt ekvation \eqref{eq:MSD_S} och \eqref{eq:MSD_s}. Ett potenssamband har anpassats till resultaten i de båda beräkningarna och visas i \figref{fig:MSD}. 
Ingen väsentlig skillnad i exponenten kan uttydas från anpassningen mellan de båda MSD typerna, vilket tyder på att processen är stationär. Modellen torde därför inom detta område beskrivas bättre av den stationära fBm snarare än den icke-stationära CTRW.

Jämförs exponenterna i potensanpassningen mellan cellerna i dvala och log-fas, (a) respektive (b) i \figref{fig:MSD}, finns en viss skillnad. Exponenten för energydepleted cellerna är \todo{Samma antal värdesiffror i figurerna?} 0,65 mot logphase-cellernas 0,80. Båda processeran skiljer sig därmed från klassisk brownk rörelse där den förväntade exponenten är 1 enligt \eqref{eq:MSD_brown}. Båda processerna är därmed exempel på subdiffusion, karakteriserat av att partikelns MSD beror av tiden via ett potenssamband med exponent mindre än 1.



MSD:n från \eqref{eq:MSD_s} är som förväntat mer ojämn än den stationära då färre medelvärden tas. Samtidigt är linjen tillräckligt tydlig för att anpassningen ska kunna göras med liten felgräns.

\begin{figure}\centerline{
\subfigure[][]{
\input{bilder/partiklar/MSD_ed.tex}
}
\subfigure[][]{
\input{bilder/partiklar/MSD_lp.tex}
}}
\caption{Dubbellogplott av de olika sorternas MSD, beräknade enligt \eqref{eq:MSD_S} respektive \eqref{eq:MSD_s}, för de olika cellfaserna, dvala (a) och aktiv (b). På grund av normeringen som användes kommer $S(\Delta{t})$ och  $s(\Delta{t})$ att ha samma värde i $\Delta{t}=\unit[10]{s}$. Detta är dock av liten betydelse eftersom det intressanta att undersöka är exponenten på $\Delta{t}$ (lutningen i dessa grafer). I de olika cellfaserna syns att lutningarna i diagrammen är samma inom varje fas, men skiljger sig åt mellan (a) och (b). Detta tyder på ett stationärt beteende, men att det ändå finns en tydlig skilnad mellan aktiva celler och de i dvala. Notera också att $S$ är jämnare än $s$, vilket bara beror på att $S$ innehåller fler termer som medelvärderats. }
\label{fig:MSD}
\end{figure}



\section{Spektraltäthet}
%Ett annat verktyg för att undersöka partilarnas rörelser är att titta på vad för spektraltäthet partiklarnas position har. Spektraltätheten erhålls genom att ta Fouriertransformen av partikelns position som funktion av tid. På så sätt erhålls ett mått på hur mycket av varje frekvens som finns. 

\todo{för position eller steg?}


\todo[inline]{Mer kommer sen.}



\subsection{Resultat -- }






\section{Anisotropi i partikelrörelsen}
Om man misstänker att cellens inre struktur påverkar partikelrörelserna skulle eventuell anisotropi kunna ge ett bidrag till dessa. Alltså ifall det finns föredragna riktningar för partikeln att röra sig i. Ett sätt att undersöka detta är genom att betrakta koordinaternas \todo{Byta till ''matris''!}kovarianstensor som är en tensor med de olika koordinaternas statistiska andramoment:
\begin{equation}
C_{ij} = \frac{1}{N} \sum_{n=1}^{N} 
\left(r_i^{(n)}r_j^{(n)} -\bar{r}_i\bar{r}_j \right),
\end{equation}
där $r_i^{(n)}$ är den $i$:te koordinaten ($x$ eller $y$) i den $n$:te datapunkten och $\bar{r}_i$ är den $i$:te koordinatens medelvärde. På matrisform kan $C$ skrivas som
\begin{equation}\label{eq:C_matris}
C=
\begin{bmatrix} 
\ev{x^2}-\ev{x}^2 & \ev{xy}-\ev{x}\ev{y}\\
\ev{yx}-\ev{x}\ev{y} & \ev{y^2}-\ev{y}^2
\end{bmatrix}.
\end{equation}

Notera hur snarlik $C$ är med tröghetstensorn för en kropps olika tröghetsmoment som uppkommer i mekaniken. Och precis som i mekaniken kan man hitta två principalaxlar genom att ta fram dess egenvektorer och diagonalisera tensorn. I mekaniken svarar principalaxlarna mot de riktningar av rörelsemängdsmoment som är oberoende av rotation längs andra axlar. I de här statistiska sammanhangen finns det en liknande tolkning nämligen att principalaxlarna svarar mot de riktningar i cellen där rörelserna är oberoende av varandra. Egenvärdena i dessa två riktningarna svarar mot variansen i den oberoende koordinaten längs med den riktningen.

För att ur kovarianstensorn få ut ett mått på hur isotropt partikeln rör sig kan man undersöka asymmetrimåttet~\cite{Rudnick_Asphericity1986}
\begin{equation}\label{eq:asph.}
    A_d=\frac{\sum_{j=1}^d\sum_{i<j} 
\ev{(\lambda_i-\lambda_j)^2} }{
(d-1) \ev{(\sum_{j=1}^d \lambda_j)^2}}
\end{equation}
i $d$ dimensioner, kallat \emph{asfärisitet}. Egenvärdena själva svarar som sagt mot variansen i de olika riktningarna, vilket motsvarar hur mycket partikeln har rört sig i respektive rikting. Hade positionerna varit helt isotropt fördelade så hade alltså $A=0$, om partikeln bara hade rört sig längs en linje blir istället $A=1$.
I två dimensioner förenklas \eqref{eq:asph.} till 
\begin{equation}\label{eq:asph._2D}
    A=\frac{\ev{(\lambda_1 - \lambda_2)^2}}{\ev{(\lambda_1 + \lambda_2)^2}}.
\end{equation}

\subsubsection{Tillämpning av asfärisiteten}
Man kan enkelt visa att egenvärdena till kovarianstensorn i \eqref{eq:C_matris} uppfyller~\cite{Hong_asymmetri1998}
\begin{equation}
\begin{aligned}
    \lambda_1+\lambda_2 &= \ev{x^2}+\ev{y^2} \\
    \abs{\lambda_1-\lambda_2} &= 
        \sqrt{\left(\ev{x^2}-\ev{y^2}\right)^2 + 4\ev{xy}^2}.
\end{aligned}
\end{equation} \todo{Fixa så att medelvärde dras bort}
För en given fördelningsfunktion med ändliga moment blir det därmed nu möjligt att beräkna denna kvot teoretiskt.

För en obegränsad Wienerprocess i $d$ dimensioner fås 
\begin{equation} \label{eq:Asphericity_Brownian}
    A_d^\text{(Wierner)}=\frac{2(d+2)}{5d+4}.
\end{equation}
För två dimensioner fås $A_2^\text{(Wierner)}=\nicefrac{4}{7}$. Detta innebär att även långvariga slumpvandringar i snitt kommer att uppvisa viss anisotropi.

För fBm fås en formel i två dimensioner som gäller i gränsen där antalet mätpunkter blir mycket stort ~\cite{Hong_asymmetri1998}
\begin{equation} \label{eq:A_fBm}
A=2-
\frac{\frac{1}{2(H+1)^2}}{\frac{1}{2(H+1)^2}+\frac{2H+1}{4(4H+1)}-\frac{1}{4H+3}-\frac{\Gamma^2(2H+2)}{\Gamma(4H+4)}},
\end{equation}
där $H$ är Hurst parametern som karakteriserar rörelsen och $\Gamma$ är gammafunktionen. För $H=\nicefrac{1}{2}$ fås en vanlig brownsk rörelse med motsvarande  asymmetrimått $A=\nicefrac{4}{7}$ vilket stämmer överens med ekvation \eqref{eq:Asphericity_Brownian} för $d=2$. Samma värde fås även approximativt för CTRW-modellen \cite{Ernst_ACTRW2012} då denna rörelse vid varje tidsögonblick ser ut som brownsk rörelse. \todo{Vad menas här? Kolla upp.}

%Även http://iopscience.iop.org/article/10.1088/0305-4470/19/4/004/pdf



 
\subsection{Simulering och numerisk beräkning av asymmetrimått} 
Istället för att bara direkt beräkna $A$ enligt \eqref{eq:asph._2D} för den undersökta datan och några olika modeller kan man också studera fördelningen av måttet 
\begin{equation}
\mathcal{A} =
\frac{(\lambda_1 - \lambda_2)^2}{(\lambda_1 + \lambda_2)^2}.
\end{equation}
Här ska det dock påpekas att det inte går att få $A$ från  $\ev{\mathcal{A}}$ eftersom \eqref{eq:asph._2D} består av ett väntevärde i både täljare och nämnare. Eftersom $\mathcal{A}$ alltså skiljer sig så från $A$ går det heller inte lika enkelt att göra teoretiska beräkningar, vilket gör att fördelningen av $\mathcal{A}$ får undersökas med Monte Carlo-simuleringar.


%Eftersom en mätning bara kan bestå av ett ändligt antal datapunkter kan även en äkta, isotrop brownsk rörelse ge upphov till $\mathcal{A}>1$. Vidare är \emph{kvoten} mellan egenvärden inte linjär, vilket gör en teoretisk analys av fördelningarna mycket svår. För att ändå kunna säga något om den riktiga datan kan man ganska enkelt ta fram en fördelning för $\mathcal{A}$ för några olika modeller genom simulering. 

De modeller som testats är en vanlig Wienerprocess, en Wienerprocess med ett ''mätbrus'' pålagt och en Ornstein-Uhlenbeck-process. Från dessa processer samt från datan över partikelrörelsen erhölls olika fördelningar av $\mathcal{A}$. 


I Wienerprocessen simulerades partikelns position i varje ny tidpunkt genom att gå ett normalfördelat steg från positionen i den förra tidpunkten. Detta kan skrivas som
\begin{equation}\label{eq:sim_wiener}
x_{i+1} = x_i + \delta 
\qcomma  \delta \sim N(0, \sigma)
\end{equation}
och där $x_1=0$. Sedan gör man samma sak för $y$. Eftersom $\mathcal{A}$ ger ett mått på hur mycket partikeln \emph{föredrar en viss riktning}, finner man att värdet på $\sigma$ inte påverkar fördelningen -- så länge $\sigma$ är samma för både $x$ och $y$. Detta blir uppenbart när man tänker på att \eqref{eq:sim_wiener} kan skrivas som 
\begin{equation}
x_{i+1} =\sigma\,\hat{x}_{i+i} = \sigma\,\left( \hat{x}_i + \hat{\delta} \right) 
\qcomma  \hat{\delta} \sim N(0, 1).
\end{equation}
Alltså att $\sigma$ bara är en multiplikativ konstant framför positionen.

För att istället simulera en Wienerprocess med mätbrus användes \eqref{eq:sim_wiener} för att först simulera själva Wienerprocessen för att sedan \emph{efteråt} addera en slumpad brusterm $\eta \sim N(0, \sigma')$ till varje $x_i$. Den väsentliga skillnaden mot en ren Wienerprocess är att mätbruset $\eta$ inte påverkar nästkommande position. Till skillnad från den rena Wienerprocessen så kommer värdena på $\sigma$ och $\sigma'$ att påverka $\mathcal{A}$, men enligt samma argument som ovan kommer bara kvoten $\nicefrac{\sigma'}{\sigma}$ vara det som påverkar. Detta gör att simuleringarna med mätbrus går att genomföra med olika värden på parametern $\nicefrac{\sigma'}{\sigma}$ för att få olika fördelningar av $\mathcal{A}$. 

Ornstein-Uhlenbeck-processen i \eqref{eq:SDE_o-u} simulerades genom att använda derivatan för tidsutveckling och genom att sätta $\bar{x}=0$. Man får då
\begin{equation}
x_{i+1} = x_i + \Delta{t}\,\pd_{t}x  = (1-k) x_i +  \delta 
\qcomma  \delta \sim N(0, \sigma),
\end{equation}
där $x_1=0$. Notera här att den styrande parametern i simuleringarna är $k=\gamma\Delta{t}$, samt att $\sigma$ på samma sätt som för den rena Wienerprocessen kan väljas godtyckligt utan att påverka $\mathcal{A}$. 

Eftersom resultatet för asymmetrimåttet i ekvation \eqref{eq:A_fBm} gäller för riktigt långa mätningar medan den data som analyseras i detta arbete är ganska begränsad kan man försöka simulera fram sannolikheter istället. Genom att simulera många mätserier för samma antal steg och antal partiklar som i datan kan man bedöma sannolikheten att det framräknade asymmetrimåttet till exempel skulle ha kommit från en ren klassisk brownsk rörelse. 


\subsection{Resultat -- partikelrörelsen är minst lika isotrop som Wiernerprocessen}
Som kan ses i \figref{fig:asymmetri} skiljer sig partiklar från celler i dvala och log-fas. Båda fallen har fördelningar som svarar mot en rörelse som är minst lika isotrop som Wiernerprocessen -- under de undersökta tidsskalorna. Vidare ses att partiklarna från celler i dvala får en fördelning som desutom skiljer sig från Wienerprocessens. Detta tyder på att cytoplasman har olika betende i de olika studerade cellfaserna. 

Om istället det riktiga asfärisitetsmåttet, $A$, betrakts erhålls värdena \todo{Fylla i.}


En anledning att studera isotropin är att bedöma hurvida det finns en föredragen rikting i cellerna. Hade det varit fallet så skulle det kunna antyda om strukturer i cellen som påverkade partikelrörelsen; då hade det även varit befogat att studera rörelserna längs denna föredragna rikting separat. Eftersom detta nu inte är aktuellt bör partiklarna betraktas röra sig isotropt under de undersökta tidsskalorna. Härav har ingen hänsyn till föredragna riktningar tagits i de andra studierna som gjorts på partikelrörelse. 

%Genom att spela upp datan över partikelpositionerna som en film blir det tydligt att vissa partiklar stöter på större strukturer i cytoplasman som de inte kan rubba ur vägen utan måste ta sig runt. För dessa partiklar ter sig cytoplasman alltså inte helt isotrop på större tidsskalor. \todo{Vad kan man säga mer här?}

%Mer isotropt än vid brownsk rörelse. Påverkas kanske av cellens inre utformning och hålls på plats


\begin{comment}
\begin{table}[]
\centering
\begin{tabular}{|c|c|c|c|}
\hline
     &  Log-fas & Dvala & fBm\\ \hline
    $A$  & 0,41 & 0,49 & \\ \hline
\end{tabular}
\caption{Caption}
\label{tab:asp_values}
\end{table}
\end{comment}

\begin{figure}\centering
\input{bilder/partiklar/isotropi_asymmetri.tex}
\caption{Fördelning av asymmetrimåttet $\mathcal{A}$, visat som sannolikheten att $\mathcal{A}$ är \emph{större} än ett visst värde $a$.
Graferna visar att de partiklar från celler i log-fas, i det här avseendet, som en Wiernerprocess, men att partiklarna från celler i dvala beter sig mer isotropt än en vanlig Wienerprocess. Läsaren påminns om att en helt cirkulärt symmetrisk fördelning svarar mot $\mathcal{A}=0$, medan $\mathcal{A}=1$ svarar mot att positionerna ligger på en linje.}
\label{fig:asymmetri}
\end{figure}



\section{Diskussion}
\todo[inline]{Inga tomma rubriker}

\subsection{MSD}
%Klippt från resultatdelen
Då MSD:n ger en hint om hur snabbt partiklarna sprids verkar partiklarna i cellerna i dvala diffundera långsammare än partiklarna i de aktiva cellerna. De undersökta cellernas metabola tillstånd verkar alltså påverka diffusionen i cytoplasman vilket bekräftar resultat från tidigare studier~\cite{Parry_etal2014}. Dessa studier har, liksom här, utförts på celler utan motorprotein. Detta utesluter förklaringsmodellen där de utpekas som största bidragsfaktor som möjlig förklaring det det observerade fenomenet. 

\subsection{Ornstein-Uhlenbeck-modellen}
\todo[inline]{Varsågod att ändra så mycket ni vill.}

\subsection{Brusnivån är väldigt låg}


%Om sambandet mellan intensitet och partikelstorleken påverkar resultatet

%Byter man till normal- och tangential-koordinater bygger man in ett bias som påverkar resultaten till att det verkar finnas egenskaper som egentligen inte finns. 
%Av slumpskäl kommer vissa vägar att vara mer raka än andra medan andra får mer symmetriska banor.


%Bara en liten kodsnutt som behövs när man kompilerar lokalt
%%% Local Variables: 
%%% mode: latex
%%% TeX-master: "00main.tex"
%%% End: 