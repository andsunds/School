\documentclass[11pt,a4paper, 
swedish, english %% Make sure to put the main language last!
]{article}
\pdfoutput=1

%% Andréas's custom package 
%% (Will work for most purposes, but is mainly focused on physics.)
\usepackage{../custom_as}

%% Figures can now be put in a folder: 
\graphicspath{ {figures/} %{some_folder_name/}
}

%% If you want to change the margins for just the captions
\usepackage[margin=10 pt]{caption}

%% To add todo-notes in the pdf
\usepackage[%disable  %%this will hide all notes
]{todonotes} 

%% Change the margin in the documents
\usepackage[
%            top    = 3cm,              %% top margin
%            bottom = 3cm,              %% bottom margin
%            left   = 2.5cm, right  = 2.5cm %% left and right margins
]{geometry}


%% If you want to change the formatting of the section headers
%\renewcommand{\thesection}{...}

\newcommand{\EM}{\text{EM}}
\newcommand{\E}{\text{E}}
%\newcommand{\GR}{\text{GR}}
%\newcommand{\SR}{\text{SR}}


%%%%%%%%%%%%%%%%%%%%%%%%%%%%%%%%%%%%%%%%%%%%%%%%%%%%%%%%%%%%%%%%%%%%%%
\begin{document}%% v v v v v v v v v v v v v v v v v v v v v v v v v v
%%%%%%%%%%%%%%%%%%%%%%%%%%%%%%%%%%%%%%%%%%%%%%%%%%%%%%%%%%%%%%%%%%%%%%


%% If you want to use an external file for the title page
%\renewcommand{\thefootnote}{\fnsymbol{footnote}}

%kortkommandon för mailaddresserna
\newcommand{\andsunds}{andsunds@student.chalmers.se}
\newcommand{\rigon}{rigon@student.chalmers.se}



\pagenumbering{roman} %%Romersk sidnumrering i början
\begin{titlepage}
\newgeometry{top=3cm, bottom=2cm}

\newcommand{\HRule}{\rule{\linewidth}{0.5mm}} % Defines a new command for the horizontal lines, change thickness here

\center % Center everything on the page
 
%------------------------------------------------------------------------------------
%	HEADING SECTIONS
%------------------------------------------------------------------------------------

\textsc{\huge Chalmers tekniska högskola}\\[1.5cm] % Name of university/college
\textsc{\Large Rapport, Experimentell fysik 2}\\[0.2cm] % Major heading such as course name
\textsc{\large Termodynamik -- Uppgift 3 }\\[0.5cm] % Minor heading such as course title

%------------------------------------------------------------------------------------
%	TITLE SECTION
%------------------------------------------------------------------------------------

\HRule \\[0.4cm]
{ \LARGE \bfseries 
Studier av kvicksilveratomens atomära emissionsspektra samt absorptionsspektra av laserfärgämnena Rhodamin B och Kumarin 307
}\\[0.4cm] % Title of  document
\HRule \\[1.5cm]
 
%------------------------------------------------------------------------------------
%	AUTHOR SECTION
%------------------------------------------------------------------------------------

\begin{minipage}{0.4\textwidth}
\begin{flushleft} \large
\emph{Författare:}\\
Andréas Sundström\footnotemark{} \\
Rigon Demisai\footnotemark{} 
\end{flushleft}
\end{minipage}
~
\begin{minipage}{0.4\textwidth}
\begin{flushright} \large
\emph{Labassistent:} \\
Martin Wersäll
\end{flushright}
\end{minipage}\\[3cm]

\setcounter{footnote}{0}
\stepcounter{footnote}
  \footnotetext{\href{mailto:\andsunds}{\texttt{\andsunds}}}
\stepcounter{footnote}
  \footnotetext{\href{mailto:\rigon}{\texttt{\rigon}}}



%------------------------------------------------------------------------------------
%	DATE SECTION
%------------------------------------------------------------------------------------
% Följer ISO-standarden för tidsintervall:
% https://en.wikipedia.org/wiki/ISO_8601#Time_intervals
% "Double hyphen" också ok istället för '/'. -- i LaTeX är dock lite på gränsen
{ \large
\begin{tabular}{rc}
    Laboration utförd: & 2015-12-11/15 \\[0.1cm]
    Rapport inlämnad: & \today
\end{tabular}\\[1cm]
}

%------------------------------------------------------------------------------------
%	LOGO SECTION
%------------------------------------------------------------------------------------

\includegraphics[height=5cm]{logo.pdf} % Include a department/university logo
 
%------------------------------------------------------------------------------------

\vfill % Fill the rest of the page with whitespace

\end{titlepage}
\restoregeometry


\setcounter{page}{2}%detta är ANDRA (2) sidan

\renewcommand{\abstractname}{Sammandrag}
\begin{abstract}
Vi har utfört en spektroskopisk studie av kvicksilveratomens atomära emissionsspektrum ur vilket vi kartlagt atomens energinivåer baserat på vår spektrometri. Vi har också studerat absorption i lösningar av två laserfärgämnen vid namn Rhodamin B och Kumarin 307. Mätningarna har utförts med en Spex 270M spektrometer och datainsamlingen har gjorts i LabView. Kvicksilvrets emissionsspektra är taget i intervallet 365 till 984 nm, där vi detekterat totalt 23 signifikanta emissionstoppar. Detta jämförs med NIST data där vi har 4 av 5 överlappningar med NIST ''persistent lines'' och 7 av 21 överlappningar med NIST ''strong lines''. Absorptionsspektra för Rhodamin B och Kumarin 307 visar breda absorptionsband vilket är kännetecknande för flourescerande ämnen som består av stora organiska molekyler.
\end{abstract}

\renewcommand{\abstractname}{Abstract}
\begin{abstract}

We have conducted a spectroscopic study of the emission spectrum of Mercury atoms and derived an energy level diagram based on these measurements. We have also studied the absorption spectrum of laser dye solutions of the compounds Rhodamine B and Coumarine 307. The measurements have been taken with a Spex 270M spectrometer and have been processed and recorded in LabView. The emission spectrum of Mercury has been recorded within the range of 365 to 984 nm, where we have detected a total of 23 significant emission peaks. This is contrasted with NIST data where we have 4 out of 5 overlaps with NIST ''persistent lines'' and 7 out of 21 overlaps with NIST ''strong lines''. The absorption spectrum for Rhodamine B and Coumarine 307 show broad absorption bands which are characteristic of flourescent compounds which consist of large organic molecules.

\end{abstract}

\clearpage
\renewcommand{\contentsname}{Innehållsförteckning}
\tableofcontents

\clearpage
\pagenumbering{arabic}
\setcounter{page}{1}

\renewcommand{\thefootnote}{\arabic{footnote}}
\setcounter{footnote}{0}



%%%%%%%%%%%%%%%%%%%% vvv Internal title page vvv %%%%%%%%%%%%%%%%%%%%%
\title{Gravitation and Cosmology -- FFM 071
\\ {\Large Hand-in set 2} }
\author{Andréas Sundström}
\date\today%{2018-02-13}

\maketitle

%%%%%%%%%%%%%%%%%%%% ^^^ Internal title page ^^^ %%%%%%%%%%%%%%%%%%%%%
%% If you want a list of all todos
%\todolist

\renewcommand{\thesubsection}{\arabic{section} (\roman{subsection})}
\setcounter{section}{8}

\section{3D anti-de Sitter}

The $\text{AdS}_3$ metric
\begin{equation}\label{eq1:metric1}
\rd{s}^2=-\rd{u}^2-\rd{v}^2+\rd{x}^2+\rd{y}^2
\end{equation}
under the constraint
\begin{equation}\label{eq1:constraint}
-u^2-v^2+x^2+y^2=-R^2,
\end{equation}
can be rewritten using the change of variables
\begin{equation}\label{eq1:var-change}
\begin{aligned}
&u=\sqrt{R^2+r^2}\cos(t/R)
\qc\qquad &x=r\cos\phi, \\
&v=\sqrt{R^2+r^2}\sin(t/R)
\qc\qquad &y=r\sin\phi.
\end{aligned}
\end{equation}
Firstly it is easy to confirm that \eqref{eq1:var-change} satify
\eqref{eq1:constraint} by noting that
\begin{equation}
u^2+v^2=R^2+r^2\qand
x^2+y^2=r^2.
\end{equation}
Then to calculate the metric, we need
\begin{equation}
\rd{u}=\pdv{u}{r}\rd{r}+\pdv{u}{t}\rd{t}+\pdv{u}{\phi}\rd\phi
=\frac{r\cos(t/R)\rd{r}}{\sqrt{R^2+r^2}}
-\frac{\sqrt{R^2+r^2}}{R}\sin(t/R)\rd{t}
\end{equation}
and similarly for
\begin{equation}
\rd{v}=\frac{r\sin(t/R)\rd{r}}{\sqrt{R^2+r^2}}
+\frac{\sqrt{R^2+r^2}}{R}\cos(t/R)\rd{t},
\end{equation}
which leads to
\begin{equation}
\rd{u}^2+\rd{v}^2=
\frac{r^2\rd{r}^2}{R^2+r^2}+\frac{R^2+r^2}{R^2}\rd{t}^2
\end{equation}
Then since we already recognize the polar coordiantes for $x$ and $y$,
we can write down
\begin{equation}
\rd{x}^2+\rd{y}^2=\rd{r}^2+r^2\rd\phi^2.
\end{equation}
We can now write down the metric \eqref{eq1:metric1} in terms of the
new coodinates
\begin{equation}
\begin{aligned}
\rd{s}^2=&-\frac{R^2+r^2}{R^2}\rd{t}^2
+\qty[1-\frac{r^2}{R^2+r^2}]\rd{r}^2+r^2\rd\phi^2\\
=&-\frac{R^2+r^2}{R^2}\rd{t}^2
+\frac{R^2}{R^2+r^2}\rd{r}^2+r^2\rd\phi^2
\end{aligned}
\end{equation}

The proper distance from $r=0$ to $r=\infty$ can easily be calcualted
as
\begin{equation}
S=\int_{r=0}^{r=\infty}\rd{s}=
\begin{Bmatrix}
\rd{t}=0\\
\rd{\phi}=0
\end{Bmatrix}
=\int_0^\infty\rd{r}
\overbrace{\frac{R}{\sqrt{R^2+r^2}}}^{\sim R/r\text{ for }r\gg R}
 =\infty.
\end{equation}
Meanwhile the time it takes for a light signal to traverses the same
path is
\begin{equation}
T=\int_{r=0}^{r=\infty}\rd{t}
=\int_0^\infty\rd{r}\frac{R^2}{R^2+r}
=\frac{\pi R}{2},
\end{equation}
where $\rd{t}$ have been calculated using
\begin{equation}
0=\rd{\tau^2}=\frac{R^2+r^2}{R^2}\rd{t}^2
-\frac{R^2}{R^2+r^2}\rd{r}^2
\end{equation}
for light.



\section{Maximally symmetric metric?}
We have a metric given by
\begin{equation}
\rd{s}^2=-\rd{t}^2+\rd{x}^2+\rd{y}^2+\rd{z}^2
-4\cosh(\frac{x}{2})
\qty[\cosh(\frac{x}{2})(\rd{t}+\rd{x})-\sinh(\frac{x}{2})\rd{y}]\rd{x},
\end{equation}
which in matrix form gives
\begin{equation}
g_{\mu\nu}=
\begin{bmatrix}
-1 & -2\cosh^2(\frac{x}{2}) &0 &0\\
-2\cosh^2(\frac{x}{2}) &1-4\cosh^2(\frac{x}{2}) &+\sinh(x)&0\\
0&+\sinh(x)&1&0\\
0&0&0&1
\end{bmatrix}
\end{equation}












%%%%%%%%%%%%%%%%%%%%%%%%%%%%%%%%%%%%%%%%%%%%%%%%%%%%%%%%%%%%%%%%%%%%%%
\end{document}%% ^ ^ ^ ^ ^ ^ ^ ^ ^ ^ ^ ^ ^ ^ ^ ^ ^ ^ ^ ^ ^ ^ ^ ^ ^ ^ ^
%%%%%%%%%%%%%%%%%%%%%%%%%%%%%%%%%%%%%%%%%%%%%%%%%%%%%%%%%%%%%%%%%%%%%%



%%% Local Variables:
%%% mode: latex
%%% TeX-master: t
%%% End:

%  LocalWords:  contravariant Stereographic covariant variational
%  LocalWords:  azimuthal affine parametrization Ricci
