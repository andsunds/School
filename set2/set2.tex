\documentclass[11pt,a4paper, 
swedish, english %% Make sure to put the main language last!
]{article}
\pdfoutput=1

%% Andréas's custom package 
%% (Will work for most purposes, but is mainly focused on physics.)
\usepackage{../custom_as}

%% Figures can now be put in a folder: 
\graphicspath{ {figures/} %{some_folder_name/}
}

%% If you want to change the margins for just the captions
\usepackage[margin=10 pt]{caption}

%% To add todo-notes in the pdf
\usepackage[%disable  %%this will hide all notes
]{todonotes} 

%% Change the margin in the documents
\usepackage[
%            top    = 3cm,              %% top margin
%            bottom = 3cm,              %% bottom margin
%            left   = 2.5cm, right  = 2.5cm %% left and right margins
]{geometry}


%% If you want to change the formatting of the section headers
%\renewcommand{\thesection}{...}

\newcommand{\EM}{\text{EM}}
\newcommand{\E}{\text{E}}
%\newcommand{\GR}{\text{GR}}
%\newcommand{\SR}{\text{SR}}


%%%%%%%%%%%%%%%%%%%%%%%%%%%%%%%%%%%%%%%%%%%%%%%%%%%%%%%%%%%%%%%%%%%%%%
\begin{document}%% v v v v v v v v v v v v v v v v v v v v v v v v v v
%%%%%%%%%%%%%%%%%%%%%%%%%%%%%%%%%%%%%%%%%%%%%%%%%%%%%%%%%%%%%%%%%%%%%%


%% If you want to use an external file for the title page
%\renewcommand{\thefootnote}{\fnsymbol{footnote}}

%kortkommandon för mailaddresserna
\newcommand{\andsunds}{andsunds@student.chalmers.se}
\newcommand{\rigon}{rigon@student.chalmers.se}



\pagenumbering{roman} %%Romersk sidnumrering i början
\begin{titlepage}
\newgeometry{top=3cm, bottom=2cm}

\newcommand{\HRule}{\rule{\linewidth}{0.5mm}} % Defines a new command for the horizontal lines, change thickness here

\center % Center everything on the page
 
%------------------------------------------------------------------------------------
%	HEADING SECTIONS
%------------------------------------------------------------------------------------

\textsc{\huge Chalmers tekniska högskola}\\[1.5cm] % Name of university/college
\textsc{\Large Rapport, Experimentell fysik 2}\\[0.2cm] % Major heading such as course name
\textsc{\large Termodynamik -- Uppgift 3 }\\[0.5cm] % Minor heading such as course title

%------------------------------------------------------------------------------------
%	TITLE SECTION
%------------------------------------------------------------------------------------

\HRule \\[0.4cm]
{ \LARGE \bfseries 
Studier av kvicksilveratomens atomära emissionsspektra samt absorptionsspektra av laserfärgämnena Rhodamin B och Kumarin 307
}\\[0.4cm] % Title of  document
\HRule \\[1.5cm]
 
%------------------------------------------------------------------------------------
%	AUTHOR SECTION
%------------------------------------------------------------------------------------

\begin{minipage}{0.4\textwidth}
\begin{flushleft} \large
\emph{Författare:}\\
Andréas Sundström\footnotemark{} \\
Rigon Demisai\footnotemark{} 
\end{flushleft}
\end{minipage}
~
\begin{minipage}{0.4\textwidth}
\begin{flushright} \large
\emph{Labassistent:} \\
Martin Wersäll
\end{flushright}
\end{minipage}\\[3cm]

\setcounter{footnote}{0}
\stepcounter{footnote}
  \footnotetext{\href{mailto:\andsunds}{\texttt{\andsunds}}}
\stepcounter{footnote}
  \footnotetext{\href{mailto:\rigon}{\texttt{\rigon}}}



%------------------------------------------------------------------------------------
%	DATE SECTION
%------------------------------------------------------------------------------------
% Följer ISO-standarden för tidsintervall:
% https://en.wikipedia.org/wiki/ISO_8601#Time_intervals
% "Double hyphen" också ok istället för '/'. -- i LaTeX är dock lite på gränsen
{ \large
\begin{tabular}{rc}
    Laboration utförd: & 2015-12-11/15 \\[0.1cm]
    Rapport inlämnad: & \today
\end{tabular}\\[1cm]
}

%------------------------------------------------------------------------------------
%	LOGO SECTION
%------------------------------------------------------------------------------------

\includegraphics[height=5cm]{logo.pdf} % Include a department/university logo
 
%------------------------------------------------------------------------------------

\vfill % Fill the rest of the page with whitespace

\end{titlepage}
\restoregeometry


\setcounter{page}{2}%detta är ANDRA (2) sidan

\renewcommand{\abstractname}{Sammandrag}
\begin{abstract}
Vi har utfört en spektroskopisk studie av kvicksilveratomens atomära emissionsspektrum ur vilket vi kartlagt atomens energinivåer baserat på vår spektrometri. Vi har också studerat absorption i lösningar av två laserfärgämnen vid namn Rhodamin B och Kumarin 307. Mätningarna har utförts med en Spex 270M spektrometer och datainsamlingen har gjorts i LabView. Kvicksilvrets emissionsspektra är taget i intervallet 365 till 984 nm, där vi detekterat totalt 23 signifikanta emissionstoppar. Detta jämförs med NIST data där vi har 4 av 5 överlappningar med NIST ''persistent lines'' och 7 av 21 överlappningar med NIST ''strong lines''. Absorptionsspektra för Rhodamin B och Kumarin 307 visar breda absorptionsband vilket är kännetecknande för flourescerande ämnen som består av stora organiska molekyler.
\end{abstract}

\renewcommand{\abstractname}{Abstract}
\begin{abstract}

We have conducted a spectroscopic study of the emission spectrum of Mercury atoms and derived an energy level diagram based on these measurements. We have also studied the absorption spectrum of laser dye solutions of the compounds Rhodamine B and Coumarine 307. The measurements have been taken with a Spex 270M spectrometer and have been processed and recorded in LabView. The emission spectrum of Mercury has been recorded within the range of 365 to 984 nm, where we have detected a total of 23 significant emission peaks. This is contrasted with NIST data where we have 4 out of 5 overlaps with NIST ''persistent lines'' and 7 out of 21 overlaps with NIST ''strong lines''. The absorption spectrum for Rhodamine B and Coumarine 307 show broad absorption bands which are characteristic of flourescent compounds which consist of large organic molecules.

\end{abstract}

\clearpage
\renewcommand{\contentsname}{Innehållsförteckning}
\tableofcontents

\clearpage
\pagenumbering{arabic}
\setcounter{page}{1}

\renewcommand{\thefootnote}{\arabic{footnote}}
\setcounter{footnote}{0}



%%%%%%%%%%%%%%%%%%%% vvv Internal title page vvv %%%%%%%%%%%%%%%%%%%%%
\title{Gravitation and Cosmology -- FFM 071
\\ {\Large Hand-in set 2} }
\author{Andréas Sundström}
\date\today%{2018-02-13}

\maketitle

%%%%%%%%%%%%%%%%%%%% ^^^ Internal title page ^^^ %%%%%%%%%%%%%%%%%%%%%
%% If you want a list of all todos
%\todolist

\renewcommand{\thesubsection}{\arabic{section} (\roman{subsection})}
\setcounter{section}{8}

\section{3D anti-de Sitter}

The $\text{AdS}_3$ metric
\begin{equation}\label{eq9:metric1}
\rd{s}^2=-\rd{u}^2-\rd{v}^2+\rd{x}^2+\rd{y}^2
\end{equation}
under the constraint
\begin{equation}\label{eq9:constraint}
-u^2-v^2+x^2+y^2=-R^2,
\end{equation}
can be rewritten using the change of variables
\begin{equation}\label{eq9:var-change}
\begin{aligned}
&u=\sqrt{R^2+r^2}\cos(t/R)
\qc\qquad &x=r\cos\phi, \\
&v=\sqrt{R^2+r^2}\sin(t/R)
\qc\qquad &y=r\sin\phi.
\end{aligned}
\end{equation}
Firstly it is easy to confirm that \eqref{eq9:var-change} satisfy
\eqref{eq9:constraint} by noting that
\begin{equation}
u^2+v^2=R^2+r^2\qand
x^2+y^2=r^2.
\end{equation}
Then to calculate the metric, we need
\begin{equation}
\rd{u}=\pdv{u}{r}\rd{r}+\pdv{u}{t}\rd{t}+\pdv{u}{\phi}\rd\phi
=\frac{r\cos(t/R)\rd{r}}{\sqrt{R^2+r^2}}
-\frac{\sqrt{R^2+r^2}}{R}\sin(t/R)\rd{t}
\end{equation}
and similarly for
\begin{equation}
\rd{v}=\frac{r\sin(t/R)\rd{r}}{\sqrt{R^2+r^2}}
+\frac{\sqrt{R^2+r^2}}{R}\cos(t/R)\rd{t},
\end{equation}
which leads to
\begin{equation}
\rd{u}^2+\rd{v}^2=
\frac{r^2\rd{r}^2}{R^2+r^2}+\frac{R^2+r^2}{R^2}\rd{t}^2
\end{equation}
Then since we already recognize the polar coordinates for $x$ and $y$,
we can write down
\begin{equation}
\rd{x}^2+\rd{y}^2=\rd{r}^2+r^2\rd\phi^2.
\end{equation}
We can now write down the metric \eqref{eq9:metric1} in terms of the
new coordinates
\begin{equation}
\begin{aligned}
\rd{s}^2=&-\frac{R^2+r^2}{R^2}\rd{t}^2
+\qty[1-\frac{r^2}{R^2+r^2}]\rd{r}^2+r^2\rd\phi^2\\
=&-\frac{R^2+r^2}{R^2}\rd{t}^2
+\frac{R^2}{R^2+r^2}\rd{r}^2+r^2\rd\phi^2
\end{aligned}
\end{equation}

The proper distance from $r=0$ to $r=\infty$ can easily be calculated
as
\begin{equation}
S=\int_{r=0}^{r=\infty}\rd{s}=
\begin{Bmatrix}
\rd{t}=0\\
\rd{\phi}=0
\end{Bmatrix}
=\int_0^\infty\rd{r}
\overbrace{\frac{R}{\sqrt{R^2+r^2}}}^{\sim R/r\text{ for }r\gg R}
 =\infty.
\end{equation}
Meanwhile the time it takes for a light signal to traverses the same
path is
\begin{equation}
T=\int_{r=0}^{r=\infty}\rd{t}
=\int_0^\infty\rd{r}\frac{R^2}{R^2+r}
=\frac{\pi R}{2},
\end{equation}
where $\rd{t}$ have been calculated using
\begin{equation}
0=\rd{\tau^2}=\frac{R^2+r^2}{R^2}\rd{t}^2
-\frac{R^2}{R^2+r^2}\rd{r}^2
\end{equation}
for light.



\section{Maximally symmetric metric?}
We have a metric given by
\begin{equation}\label{eq10:ds}
\rd{s}^2=-\rd{t}^2+\rd{x}^2+\rd{y}^2+\rd{z}^2
-4\cosh(\frac{x}{2})
\qty[\cosh(\frac{x}{2})(\rd{t}+\rd{x})-\sinh(\frac{x}{2})\rd{y}]\rd{x},
\end{equation}
which in matrix form gives
\begin{equation}
g_{\mu\nu}=
\begin{bmatrix}
-1 & -(1+\cosh x) &0 &0\\
-(1+\cosh x) &-(1+2\cosh x) &+\sinh x&0\\
0&+\sinh x &1&0\\
0&0&0&1
\end{bmatrix}.
\end{equation}
Here we have used the hyperbolic identities
\begin{equation}
2\cosh^2\qty(\frac{x}{2})\equiv 1+\cosh x
\qand
2\cosh(\frac{x}{2})\sinh(\frac{x}{2})
\equiv \sinh x.
\end{equation}

To see if this is maximally symmetric, we calculate the Riemann
tensor. To do that we need the affine connection
\begin{equation}
\Gamma^\rho_{\mu\nu}=\frac{1}{2}g^{\rho\sigma}
\qty(\pd_{\mu}g_{\nu\sigma}+\pd_{\nu}g_{\mu\sigma}-\pd_{\sigma}g_{\mu\nu}).
\end{equation}
However we inside the parenthesis only the elements with $\mu\nu=xx$
survive. To prove this we take $\mu\nu=x\alpha$, where $\alpha\neq x$,
this gives
\begin{equation}
\pd_{x}g_{\alpha\sigma}+\cancelto{0}{\pd_{\alpha}}g_{x\sigma}-\pd_{\sigma}g_{x\alpha}
=\pd_{x}g_{\alpha\sigma}-\pd_{\sigma}g_{x\alpha}.
\end{equation}
The only way for either of the two remaining terms to survive is if
$\sigma=x$, but in that case the two terms are identical and cancel
each other. Then for $\mu\nu=\alpha\beta$, where $\alpha,\beta\neq x$,
we get
\vspace{-1em}
\begin{equation}
\cancelto{0}{\pd_{\alpha}}g_{\beta\sigma}
+\cancelto{0}{\pd_{\beta}}g_{\alpha\sigma}
-\cancel{\pd_{\sigma}g_{\alpha\beta}}=0.
\end{equation}
The last term cancel because the only non-constant elements in
$g_{\mu\nu}$ are on either $\mu=x$ or $\nu=x$, which is not the case
for $\mu\nu=\alpha\beta$. Hence the only non-vanishing elements of the
affine connection are $\Gamma^\rho_{xx}$. This then means that the
Riemann tensor,
\begin{equation}
{R^\rho}_{\sigma\mu\nu}=
\pd_{\mu}\Gamma^\rho_{\sigma\nu}-\pd_{\nu}\Gamma^\rho_{\sigma\mu}
+\Gamma^\rho_{\mu\tau}\Gamma^\tau_{\sigma\nu}
-\Gamma^\rho_{\nu\tau}\Gamma^\tau_{\sigma\mu},
\end{equation}
actually vanishes. To see this more clearly, we note that the
anti-symmetry in $\mu\nu$ means that $\mu\neq\nu$, but due to the
restriction to $\Gamma^\rho_{xx}$ we must have one of $\mu$ or $\nu$
to be $x$. Without loss of generality, we choose $\mu=x$ and
$\nu=\alpha\neq x$, and we get
\begin{equation}
{R^\rho}_{\sigma x\alpha}=
\pd_{x}\cancelto{0}{\Gamma^\rho_{\sigma\alpha}}
-\cancelto{0}{\pd_{\alpha}}\Gamma^\rho_{\sigma x}
+\Gamma^\rho_{x\tau}\cancelto{0}{\Gamma^\tau_{\sigma\alpha}}
-\cancelto{0}{\Gamma^\rho_{\alpha\tau}}\Gamma^\tau_{\sigma x}=0.
\end{equation}
This is in other words a flat space-time, and is therefore maximally
symmetric.
% Note to self: See Weinberg p. 380:
% "Thus any flat metric admits $N(N+1)/2$ independent Killing vectors,
% and is therefore maximally symmetric."


We will therefore now try to get \eqref{eq10:ds} into the familiar
form of the Minkowski metric,
\begin{equation}\label{eq10:Mink1}
\rd{s'}^2=-\rd{t'}^2+\rd{x'}^2+\rd{y'}^2+\rd{z'}^2.
\end{equation}
To do this we introduce the new variables $t'=t'(t,x)$, $x'=x'(x)$,
$y'=y'(x,y)$, and $z'=z'(z)=z$. For $z'$ the choice is obvious, and
for the other three variables it is relatively easy to convince
yourself that that they should only depend on $x$ and their non-primed
counterpart. WE can now write out the metric \eqref{eq10:Mink1}
(ignoring $z'$ and $z$ since we already know that transformation) 
\begin{equation}%\label{eq10:Mink2}
\begin{aligned}
\rd{s'}^2=&
-\qty(\pdv{t'}{t})^2\rd{t}^2-2\pdv{t'}{t}\pdv{t'}{x}\id{t}\id{x}
-\qty(\pdv{t'}{x})^2\rd{x}^2\\
&+\qty(\pdv{x'}{x})^2\rd{x}^2\\
&+\qty(\pdv{y'}{x})^2\rd{x}^2+2\pdv{y'}{x}\pdv{y'}{y}\id{x}\id{y}
+\qty(\pdv{y'}{y})^2\rd{y}^2.
\end{aligned}
\end{equation}
For this to equal the original metric \eqref{eq10:ds}, we must have
\begin{equation}\label{eq10:cond}
\begin{cases}
\qty(\pdv{t'}{t})^2=1\\
\qty(\pdv{y'}{y})^2=1\\
\qty(\pdv{x'}{x})^2-\qty(\pdv{t'}{x})^2+\qty(\pdv{y'}{x})^2
=-(1+2\cosh x)\\
\pdv{t'}{t}\pdv{t'}{x}=1+\cosh x\\
\pdv{y'}{x}\pdv{y'}{y}=\sinh x.
\end{cases}
\end{equation}
If we choose the positive solutions to the first two equations, we get
\begin{equation}
\begin{cases}
\qty(\pdv{t'}{t})=+1\\
\qty(\pdv{y'}{y})=+1\\
\end{cases}
\quad\Longrightarrow\quad
\begin{cases}
t'(t,x)=t+f(x)\\
y'(x,y)=y+g(x),
\end{cases}
\end{equation}
which means that the last two equations in \eqref{eq10:cond} become
\begin{equation}
\begin{cases}
\dv{f}{x}=1+\cosh x\\
\dv{g}{x}=\sinh x
\end{cases}
\quad\Longrightarrow\quad
\begin{cases}
t'(t,x)=t+x+\sinh x\\
y'(x,y)=y+\cosh x.
\end{cases}
\end{equation}
It is now straight forward to show that the middle equation in
\eqref{eq10:cond} becomes
\begin{equation}
\qty(\pdv{x'}{x})^2
\cancel{-(1+2\cosh x)}
\overbrace{-\cosh^2 x+\sinh^2x}^{=-1}
=\cancel{-(1+2\cosh x)},
\end{equation}
which means that we also here can choose $\pdv*{x'}{x}=+1$. Every
thing is self-consistent, meaning that the full transformation of
variables 
\begin{equation}
\begin{cases}
t'=t'(t,x)=t+x+\sinh x\\
x'=x'(x)=x\\
y'=y'(x,y)=y+\cosh x\\
z'=z'(z)=z
\end{cases}
\end{equation}
results in $\rd{s'}^2$ from \eqref{eq10:Mink1} and $\rd{s}^2$ from
\eqref{eq10:ds} being equal.



\section{The non-flat torus}
On the non-flat torus embedded in three dimensions 
\begin{equation}
\begin{cases}
x=r_2A(\phi_2)\cos\phi_1\\
y=r_2A(\phi_2)\sin\phi_1\\
z=r_2\sin\phi_2
\end{cases}
\qq{with}
A(\phi_2)=r_1+r_2\cos\phi_2,
\end{equation}
and the metric, given by 
\begin{equation}
\rd{s}^2=\rd{x}^2+\rd{y}^2+\rd{z}^2
=A^2(\phi_2)\rd\phi_1^2+{r_2}^2\rd\phi_2^2,
\end{equation}
is
\begin{equation}
g_{\mu\nu}=
\begin{bmatrix}
A^2(\phi_2)&0\\
0&{r_2}^2
\end{bmatrix}.
\end{equation}
From this it is a simple task to compute the Affine connection
\begin{equation}
\Gamma^\rho_{\mu\nu}=\frac{1}{2}g^{\rho\sigma}
\qty(\pd_{\mu}g_{\nu\sigma}+\pd_{\nu}g_{\mu\sigma}-\pd_{\sigma}g_{\mu\nu}),
\end{equation}
whose only non-zero elements are 
\begin{equation}\label{eq11:affine}
\begin{aligned}
\Gamma^1_{12}=\Gamma^1_{21}
=&\frac{1}{2}g^{11}\pd_2g_{11}
=-\frac{r_2\sin\phi_2}{A(\phi_2)}\\
\Gamma^2_{11}=&-\frac{1}{2}g^{22}\pd_2g_{11}
=\frac{A(\phi_2)\sin\phi_2}{r_2}.
\end{aligned}
\end{equation}
Here we have used $\pd_2A^2(\phi_2)=-2A(\phi_2){r_2}\sin\phi_2$. Then
the Riemann tensor,
\begin{equation}
{R^\rho}_{\sigma\mu\nu}=
\pd_{\mu}\Gamma^\rho_{\sigma\nu}-\pd_{\nu}\Gamma^\rho_{\sigma\mu}
+\Gamma^\rho_{\mu\tau}\Gamma^\tau_{\sigma\nu}
-\Gamma^\rho_{\nu\tau}\Gamma^\tau_{\sigma\mu},
\end{equation}
is also seen to only have these few non-zero elements
\begin{equation}
\begin{aligned}
{R^1}_{212}=-{R^1}_{221}=&-\pd\Gamma^1_{21}-\Gamma^1_{21}\Gamma^1_{21}
=\frac{r_2}{A(\phi_2)}\cos\phi_2,\\
{R^2}_{112}=-{R^1}_{121}=&-\pd\Gamma^2_{11}+\Gamma^2_{11}\Gamma^1_{21}
=-\frac{A(\phi_2)}{r_2}\cos\phi_2.
\end{aligned}
\end{equation}
This gives the Ricci tensor
\begin{equation}
R_{\mu\nu}={R^\rho}_{\mu\rho\nu}={R^1}_{\mu1\nu}+{R^2}_{\mu2\nu}
=\begin{bmatrix}
\frac{A(\phi_2)}{r_2}\cos\phi_2&0\\
0&\frac{r_2}{A(\phi_2)}\cos\phi_2
\end{bmatrix},
\end{equation}
and the Ricci scalar
\begin{equation}
R=g^{\mu\nu}R_{\mu\nu}=g^{11}R_{11}+g^{22}R_{22}
=\frac{2\cos\phi_2}{r_2A(\phi_2)}
=\frac{2\cos\phi_2}{r_2(r_1+r_2\cos\phi_2)}.
\end{equation}
We now also note that $R_{\mu\nu}=g_{\mu\nu}R/2$ as is the case for
maximally symmetric spaces.

\todo[inline]{Still need to calculate the Killing vectors. Can't seem
  to separate the varaibles properly.}



\section{Wormholes}
Here we have the metric
\begin{equation}\label{eq12:ds}
\rd{s}^2=-\rd{t}^2+\rd{r}^2
+\qty(r^2+a^2)\qty(\rd\theta^2+\sin^2\theta\id\phi^2),
\end{equation}
giving 
\begin{equation}
g_{\mu\nu}=\text{Diag}
\begin{bmatrix}
-1,& 1,& (r^2+a^2),& (r^2+a^2)\sin^2
\end{bmatrix}.
\end{equation}
We now want to find the geodesic equations
\begin{equation}
\dv[2]{x^\mu}{\tau}+\Gamma^\mu_{\rho\nu}\dv{x^\rho}{\tau}\dv{x^\nu}{\tau}=0. 
\end{equation}
To do that we need the affine connection which, since $g_{\mu\nu}$ is
diagonal and only depends on $r$ and $\theta$, is relatively straight
forward to compute. Its only non-zero elements are
\begin{equation}\label{eq12:Gamma}
\begin{aligned}
&\Gamma^r_{\theta\theta}=-r\qc
\Gamma^r_{\phi\phi}=-r\sin^2\theta\\
&\Gamma^{\theta}_{r\theta}=\Gamma^{\theta}_{\theta r}
=\frac{r}{r^2+a^2}\qc
\Gamma^{\theta}_{\phi\phi}=-\sin\theta\,\cos\theta\\
&\Gamma^{\phi}_{r\phi}=\Gamma^{\phi}_{\phi r}=\frac{r}{r^2+a^2}\qc
\Gamma^\phi_{\theta\phi}=\Gamma^\phi_{\phi\theta}=\cot\theta,
\end{aligned}
\end{equation}
which gives the geodesic equations
\begin{equation}
\begin{cases}
\dv[2]{t}{\tau}=0\\
\dv[2]{r}{\tau}-r\qty(\dv{\theta}{\tau})^2
-r\sin^2\theta\,\qty(\dv{\phi}{\tau})^2=0\\
\dv[2]{\theta}{\tau}+\frac{2r}{r^2+a^2}\dv{r}{\tau}\dv{\theta}{\tau}
-\sin\theta\cos\theta\qty(\dv{\phi}{\tau})^2=0\\
\dv[2]{\phi}{\tau}+\frac{2r}{r^2+a^2}\dv{r}{\tau}\dv{\phi}{\tau}
+\cot\theta\dv{\phi}{\tau}\dv{\theta}{\tau}=0.
\end{cases}
\end{equation}
The last two equations can be rewritten, by multiplying with
$(r^2+a^2)$ (and also with $\sin^2\theta$ for the last equation), to
\begin{equation}
\begin{cases}
\dv{\tau}\qty[\qty(r^2+a^2)\dv{\theta}{\tau}]
=\qty(r^2+a^2)\sin\theta\cos\theta\qty(\dv{\phi}{\tau})^2\\[0.5ex]
\dv{\tau}\qty[\qty(r^2+a^2)\sin^2\theta\dv{\phi}{\tau}]=0.
\end{cases}
\end{equation}

\todo[inline]{Find wormhole solution!}

Using \eqref{eq12:Gamma} we can easily calculate the Riemann tensor
whose non-zero elements are
\begin{equation}
\begin{aligned}
&{R^r}_{\theta r\theta}=-{R^r}_{\theta\theta r}
=-\frac{a^2}{r^2+a^2} \qc
{R^r}_{\phi r\phi}=-{R^r}_{\phi\phi r}
=-\frac{a^2\sin^2\theta}{R^2+a^2}\\
&{R^\theta}_{r\theta r}=-{R^\theta}_{rr\theta}
=-\frac{a^2}{(r^2+a^2)^2}\qc
{R^\theta}_{\phi\theta\phi}=-{R^\theta}_{\phi\phi\theta}
=\frac{a^2\sin^2\theta}{r^2+a^2},\\
&{R^\phi}_{r\phi r}=-{R^\phi}_{rr\phi}
=-\frac{a^2}{(r^2+a^2)^2}\qc
{R^\phi}_{\theta\phi\theta}=-{R^\phi}_{\theta\theta\phi}
=\frac{a^2}{r^2+a^2}.
\end{aligned}
\end{equation}
From this we can compute the Ricci tensor and scalar. It is clear from
the Riemann tensor above that $R_{\mu\nu}={R^\rho}_{\mu\rho\nu}$ must
be diagonal and is given by
\begin{equation}
R_{\mu\nu}=\text{Diag}
\begin{bmatrix}
0,& -\frac{2a^2}{(r^2+a^2)^2},& 0, &0
\end{bmatrix},
\end{equation}
and then the curvature tensor trivially becomes
\begin{equation}
R=g^{\mu\nu}R_{\mu\nu}=-\frac{2a^2}{(r^2+a^2)^2}.
\end{equation}
We also see that $a\to0$ recovers a flat space, as we would expect
from the metric.

We now can write down the Einstein tensor as
\begin{equation}
G_{\mu\nu}=R_{\mu\nu}-\frac{R}{2}g_{\mu\nu}
=\frac{a^2}{(r^2+a^2)^2}\,\text{Diag}\!
\begin{bmatrix}
-1,& -1,&
(r^2+a^2),&
(r^2+a^2)\sin^2\theta
\end{bmatrix}.
\end{equation}
There are however problems with this Einstein tensor. According to
Einstein's equations 
\begin{equation}
G_{\mu\nu}=8\pi G T_{\mu\nu},
\end{equation}
and therefore since $G_{00}<0$ also $T_{00}<0$, but $T_{00}$
represents the energy density which can not be negative. 

\todo[inline]{Find all isometries!}





%%%%%%%%%%%%%%%%%%%%%%%%%%%%%%%%%%%%%%%%%%%%%%%%%%%%%%%%%%%%%%%%%%%%%%
\end{document}%% ^ ^ ^ ^ ^ ^ ^ ^ ^ ^ ^ ^ ^ ^ ^ ^ ^ ^ ^ ^ ^ ^ ^ ^ ^ ^ ^
%%%%%%%%%%%%%%%%%%%%%%%%%%%%%%%%%%%%%%%%%%%%%%%%%%%%%%%%%%%%%%%%%%%%%%



%%% Local Variables:
%%% mode: latex
%%% TeX-master: t
%%% End:

%  LocalWords:  contravariant Stereographic covariant variational
%  LocalWords:  azimuthal affine parametrization Ricci Minkowski
