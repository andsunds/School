\documentclass[11pt,a4paper, 
swedish, english %% Make sure to put the main language last!
]{article}
\pdfoutput=1

%% Andréas's custom package 
%% (Will work for most purposes, but is mainly focused on physics.)
\usepackage{../custom_as}

%% Figures can now be put in a folder: 
\graphicspath{ {figures/} %{some_folder_name/}
}

%% If you want to change the margins for just the captions
\usepackage[margin=10 pt]{caption}

%% To add todo-notes in the pdf
\usepackage[%disable  %%this will hide all notes
]{todonotes} 

%% Change the margin in the documents
\usepackage[
%            top    = 3cm,              %% top margin
%            bottom = 3cm,              %% bottom margin
%            left   = 2.5cm, right  = 2.5cm %% left and right margins
]{geometry}


%% If you want to change the formatting of the section headers
%\renewcommand{\thesection}{...}

\newcommand{\EM}{\text{EM}}
\newcommand{\E}{\text{E}}
%\newcommand{\GR}{\text{GR}}
%\newcommand{\SR}{\text{SR}}


%%%%%%%%%%%%%%%%%%%%%%%%%%%%%%%%%%%%%%%%%%%%%%%%%%%%%%%%%%%%%%%%%%%%%%
\begin{document}%% v v v v v v v v v v v v v v v v v v v v v v v v v v
%%%%%%%%%%%%%%%%%%%%%%%%%%%%%%%%%%%%%%%%%%%%%%%%%%%%%%%%%%%%%%%%%%%%%%


%% If you want to use an external file for the title page
%\renewcommand{\thefootnote}{\fnsymbol{footnote}}

%kortkommandon för mailaddresserna
\newcommand{\andsunds}{andsunds@student.chalmers.se}
\newcommand{\rigon}{rigon@student.chalmers.se}



\pagenumbering{roman} %%Romersk sidnumrering i början
\begin{titlepage}
\newgeometry{top=3cm, bottom=2cm}

\newcommand{\HRule}{\rule{\linewidth}{0.5mm}} % Defines a new command for the horizontal lines, change thickness here

\center % Center everything on the page
 
%------------------------------------------------------------------------------------
%	HEADING SECTIONS
%------------------------------------------------------------------------------------

\textsc{\huge Chalmers tekniska högskola}\\[1.5cm] % Name of university/college
\textsc{\Large Rapport, Experimentell fysik 2}\\[0.2cm] % Major heading such as course name
\textsc{\large Termodynamik -- Uppgift 3 }\\[0.5cm] % Minor heading such as course title

%------------------------------------------------------------------------------------
%	TITLE SECTION
%------------------------------------------------------------------------------------

\HRule \\[0.4cm]
{ \LARGE \bfseries 
Studier av kvicksilveratomens atomära emissionsspektra samt absorptionsspektra av laserfärgämnena Rhodamin B och Kumarin 307
}\\[0.4cm] % Title of  document
\HRule \\[1.5cm]
 
%------------------------------------------------------------------------------------
%	AUTHOR SECTION
%------------------------------------------------------------------------------------

\begin{minipage}{0.4\textwidth}
\begin{flushleft} \large
\emph{Författare:}\\
Andréas Sundström\footnotemark{} \\
Rigon Demisai\footnotemark{} 
\end{flushleft}
\end{minipage}
~
\begin{minipage}{0.4\textwidth}
\begin{flushright} \large
\emph{Labassistent:} \\
Martin Wersäll
\end{flushright}
\end{minipage}\\[3cm]

\setcounter{footnote}{0}
\stepcounter{footnote}
  \footnotetext{\href{mailto:\andsunds}{\texttt{\andsunds}}}
\stepcounter{footnote}
  \footnotetext{\href{mailto:\rigon}{\texttt{\rigon}}}



%------------------------------------------------------------------------------------
%	DATE SECTION
%------------------------------------------------------------------------------------
% Följer ISO-standarden för tidsintervall:
% https://en.wikipedia.org/wiki/ISO_8601#Time_intervals
% "Double hyphen" också ok istället för '/'. -- i LaTeX är dock lite på gränsen
{ \large
\begin{tabular}{rc}
    Laboration utförd: & 2015-12-11/15 \\[0.1cm]
    Rapport inlämnad: & \today
\end{tabular}\\[1cm]
}

%------------------------------------------------------------------------------------
%	LOGO SECTION
%------------------------------------------------------------------------------------

\includegraphics[height=5cm]{logo.pdf} % Include a department/university logo
 
%------------------------------------------------------------------------------------

\vfill % Fill the rest of the page with whitespace

\end{titlepage}
\restoregeometry


\setcounter{page}{2}%detta är ANDRA (2) sidan

\renewcommand{\abstractname}{Sammandrag}
\begin{abstract}
Vi har utfört en spektroskopisk studie av kvicksilveratomens atomära emissionsspektrum ur vilket vi kartlagt atomens energinivåer baserat på vår spektrometri. Vi har också studerat absorption i lösningar av två laserfärgämnen vid namn Rhodamin B och Kumarin 307. Mätningarna har utförts med en Spex 270M spektrometer och datainsamlingen har gjorts i LabView. Kvicksilvrets emissionsspektra är taget i intervallet 365 till 984 nm, där vi detekterat totalt 23 signifikanta emissionstoppar. Detta jämförs med NIST data där vi har 4 av 5 överlappningar med NIST ''persistent lines'' och 7 av 21 överlappningar med NIST ''strong lines''. Absorptionsspektra för Rhodamin B och Kumarin 307 visar breda absorptionsband vilket är kännetecknande för flourescerande ämnen som består av stora organiska molekyler.
\end{abstract}

\renewcommand{\abstractname}{Abstract}
\begin{abstract}

We have conducted a spectroscopic study of the emission spectrum of Mercury atoms and derived an energy level diagram based on these measurements. We have also studied the absorption spectrum of laser dye solutions of the compounds Rhodamine B and Coumarine 307. The measurements have been taken with a Spex 270M spectrometer and have been processed and recorded in LabView. The emission spectrum of Mercury has been recorded within the range of 365 to 984 nm, where we have detected a total of 23 significant emission peaks. This is contrasted with NIST data where we have 4 out of 5 overlaps with NIST ''persistent lines'' and 7 out of 21 overlaps with NIST ''strong lines''. The absorption spectrum for Rhodamine B and Coumarine 307 show broad absorption bands which are characteristic of flourescent compounds which consist of large organic molecules.

\end{abstract}

\clearpage
\renewcommand{\contentsname}{Innehållsförteckning}
\tableofcontents

\clearpage
\pagenumbering{arabic}
\setcounter{page}{1}

\renewcommand{\thefootnote}{\arabic{footnote}}
\setcounter{footnote}{0}



%%%%%%%%%%%%%%%%%%%% vvv Internal title page vvv %%%%%%%%%%%%%%%%%%%%%
\title{Gravitation and Cosmology -- FFM 071
\\ {\Large Hand-in set 1} }
\author{Andréas Sundström}
\date{\today}

\maketitle

%%%%%%%%%%%%%%%%%%%% ^^^ Internal title page ^^^ %%%%%%%%%%%%%%%%%%%%%
%% If you want a list of all todos
%\todolist


\renewcommand{\thesubsection}{\arabic{section} (\alph{subsection})}
\section{Some Lorentz transformations}
\subsection{Proper time}
To show that the proper time is invariant, we just use the definition
of $\rd\tau$ and the transformed coordinates 
$\rd{x'}^\alpha={\Lambda^\alpha}_\gamma \rd{x}^{\gamma}$,
\begin{equation}
\rd{\tau'}^2=-\eta_{\alpha\beta}\rd{x'}^{\alpha}\rd{x'}^{\beta}
=-\eta_{\alpha\beta}({\Lambda^\alpha}_\gamma \rd{x}^{\gamma})
({\Lambda^\beta}_\delta\rd{x}^{\delta})
=-\eta_{\alpha\beta}{\Lambda^\alpha}_\gamma{\Lambda^\beta}_\delta
\rd{x}^{\gamma}\rd{x}^{\delta}.
\end{equation}
next we use the condition that the Lorentz transform must satisfy
\begin{equation}
\eta_{\alpha\beta}{\Lambda^\alpha}_\gamma{\Lambda^\beta}_\delta
=\eta_{\gamma\delta},
\end{equation}
which therefore yields
\begin{equation}
\rd{\tau'}^2=\eta_{\gamma\delta}\rd{x}^{\gamma}\rd{x}^{\delta}
=\rd\tau^2.
\end{equation}
Since we have now concluded that $\rd\tau^2$ is a scalar (invariant)
under Lorentz transforms, so must $\rd\tau=\sqrt{\rd\tau^2}$.

\subsection{Relativistic particle velocity}
For a Lorentz boost with $\vb*v=(v,0,0)$, the transformations are as
follows
\begin{equation}
\begin{cases}
\rd{t}'=\gamma(\rd{t}-v\rd{x})\\
\rd{x}'=\gamma(\rd{x}-v\rd{t})\\
\rd{y}'=\rd{y}\\
\rd{z}'=\rd{z}
\end{cases}
\qand
\begin{cases}
\rd{t}=\gamma(\rd{t}'+v\rd{x}')\\
\rd{x}=\gamma(\rd{x}'+v\rd{t}')\\
\rd{y}=\rd{y}'\\
\rd{z}=\rd{z}'.
\end{cases}
\end{equation}
The velocity of a particle 
\begin{equation}
\vb*u=\qty(\dv{x}{t},\dv{y}{t},\dv{z}{t})
=(u_x,u_y,u_z)=(u\cos\theta,u\sin\theta,0)
\end{equation}
in the first frame, will transform as
\begin{equation}\label{eq1:upx1}
u'_x=\dv{x'}{t'}=\dv{x'}{t}\dv{t}{t'}
=\qty(\pdv{x'}{t}+\pdv{x'}{x}\pdv{x}{t})
\qty(\pdv{t}{t'}+\pdv{t}{x'}\pdv{x'}{t'})
=\gamma^2(-v+u_x)(1+vu'_x)
\end{equation}
and
\begin{equation}\label{eq1:upy1}
u'_y=\dv{y'}{t'}=\dv{y'}{t}\dv{t}{t'}
=\qty(\pdv{y'}{y}\pdv{y}{t})
\qty(\pdv{t}{t'}+\pdv{t}{x'}\pdv{x'}{t'})
=\gamma u_y(1+vu'_x).
\end{equation}
Solving for $u'_x$ in \eqref{eq1:upx1} yields
\begin{equation}
u'_x=\frac{u_x-v}{\gamma^{-2}-v(u_x-v)}
=\frac{u_x-v}{1-v^2-v(u_x-v)}
=\frac{u_x-v}{1-u_x v},
\end{equation}
similarly with \eqref{eq1:upy1}
\begin{equation}
u'_y=\gamma u_y\qty(1+v\frac{u_x-v}{1-u_x v})
=\frac{u_y}{\gamma(1-u_x v)}.
\end{equation}
The transform of $\theta$, the angle $\vb*u$ makes with the $x$ axis,
is easily given by 
\begin{equation}
\tan\theta'=\frac{u_y'}{u_x'}
=\qty(\frac{u\sin\theta}{\gamma(1-u\cos\theta v)})
\qty(\frac{1-u\cos\theta v}{u\cos\theta-v})
=\frac{\sin\theta}{\gamma(\cos\theta-\nicefrac{v}{u})}.
\end{equation}



\section{The stress tensor}
To show the conservation laws for the stress tensor\footnotemark{}
\footnotetext{I call the stress tensor of a particle
  ${T_n}^{\alpha\beta}$, with subscript $n$, and use $x_n(\tau)$ and
  $p_n$ for the trajectory and momentum of the particle, to more
  clearly distinguish what ``belongs'' to the particle and what are
  free variables.} 
\begin{equation}
{T_n}^{\alpha\beta}(x)=\int\rd\tau\,{p_n}^\alpha
\dv{{x_n}^\beta}{\tau}\delta^4\big(x-x_n(\tau)\big).
\end{equation}
The divergence of the stress tensor therefore becomes
\begin{equation}\label{eq2:pdvT1}
\pd_\beta{T_n}^{\alpha\beta}(x)=\int\rd\tau\,{p_n}^\alpha
\dv{{x_n}^\beta}{\tau}\,\pdv{x^\beta}\delta^4\big(x-x_n(\tau)\big).
\end{equation}
We note that the derivatives only act on the delta function since
$x_n$ and $p_n$ are the trajectory and momentum of the particle, while
the derivatives act on the free variable $x$. Next we note that for
any function $f$ 
\begin{equation}
\pdv{x}\Big[f(x-y)\Big]=-\pdv{y}\Big[f(x-y)\Big],
\end{equation}
which means that \eqref{eq2:pdvT1} can be written as
\vspace{-1em}
\begin{equation}
\begin{aligned}
\pd_\beta{T_n}^{\alpha\beta}(x)=&-\int\rd\tau\,{p_n}^\alpha
\overbrace{\dv{{x_n}^\beta}{\tau}\,\pdv{{x_n}^\beta}}^{=\dv*\tau}
\delta^4\big(x-x_n(\tau)\big)\\
=&-\int\rd\tau\,{p_n}^\alpha\dv{\tau}\delta^4\big(x-x_n(\tau)\big).
\end{aligned}
\end{equation}
Integration by parts now yields
\begin{equation}
\begin{aligned}
\pd_\beta{T_n}^{\alpha\beta}(x)=&
-\int\rd\tau\,{p_n}^\alpha\dv{\tau}\delta^4\big(x-x_n(\tau)\big)\\
=&-\cancel{\qty[{p_n}^\alpha\delta^4\big(x-x_n(\tau)\big)]}
+\int\rd\tau\,\dv{{p_n}^\alpha}{\tau}\delta^4\big(x-x_n(\tau)\big).
\end{aligned}
\end{equation}
The first term from the integration by parts cancel since the
integration is over ``all $\tau$'' and therefore $x_n\neq x$ at the
end points, which means that the delta function is zero.

Now if the particle interacts with an electromagnetic field, it will
experience a Lorentz force
\begin{equation}
\dv{{p_n}^\alpha}{\tau}=e_n\eta_{\beta\gamma}F^{\alpha\beta}
\dv{{x_n}^\gamma}{\tau}={F^{\alpha}}_\gamma \,e_n{u_n}^\gamma.
%={F^{\alpha}}_\gamma J^\gamma.
\end{equation}
This means that
\begin{equation}\label{eq2:pdvT}
\pd_\beta{T_n}^{\alpha\beta}(x)=\int\rd\tau\,{F^{\alpha}}_\gamma e_n{u_n}^\gamma
\delta^4\big(x-x_n(\tau)\big)
={F^{\alpha}}_\gamma e_n{u_n}^\gamma
\delta^3\big(\vb*x-\vb*x_n(\tau)\big)
={F^{\alpha}}_\gamma J^\gamma
\end{equation}
which in general is non-vanishing. Note that since we are assuming
that this particle is the only particle the current is indeed
$J^\alpha=e_n{u_n}^\alpha\delta^3\big(\vb*x-\vb*x_n(\tau)\big)$, and
if we were to have more particles all the above calculations would be
exactly the same (but with summation signs in front of them). 

We now take a look at the electromagnetic stress tensor
\begin{equation}
{T_{\EM}}^{\alpha\beta}={F^{\alpha}}_\gamma F^{\beta\gamma}
-\frac{1}{4}\eta^{\alpha\beta}F_{\gamma\delta}F^{\gamma\delta},
\end{equation}
and then Maxwell's equations will come in handy later
\begin{subequations}
\begin{equation}\label{eq:Maxwell}
\pd_\alpha F^{\alpha\beta}=-J^\beta,
\end{equation}
\begin{equation}\label{eq:Bianchi}
%\pd_{[\alpha}F_{\beta\gamma]}=
\pd_\alpha F_{\beta\gamma} +\pd_\beta F_{\gamma\alpha} +\pd_\gamma F_{\beta\alpha} 
=0.
\end{equation}
\end{subequations}
Its divergence therefore is
\begin{equation}
\begin{aligned}
\pd_{\beta}{T_{\EM}}^{\alpha\beta}=&
{F^{\alpha}}_\gamma \qty(\pd_\beta F^{\beta\gamma})
+\qty(\pd_\beta{F^{\alpha}}_\gamma) F^{\beta\gamma}
-\frac{1}{4}\eta^{\alpha\beta}\pd_\beta\qty[F_{\gamma\delta}F^{\gamma\delta}]\\
=&-{F^{\alpha}}_\gamma J^\gamma
+F^{\beta\gamma} \,\pd_\beta{F^{\alpha}}_\gamma
-\frac{1}{2}F_{\gamma\delta}\,\pd^\alpha F^{\gamma\delta}.
\end{aligned}
\end{equation}
The first term is given directly from \eqref{eq:Maxwell}.
We now take a look at the last two terms. Firstly we can raise and
lower the betas and gammas which are contracted in the second term, 
$F^{\beta\gamma} \,\pd_\beta{F^{\alpha}}_\gamma
=F_{\beta\gamma} \,\pd^\beta F^{\alpha\gamma}$, then in the last term
we are free to rename the two contracted indices $\gamma\to\beta$ and
$\delta\to\gamma$. We now have
\begin{equation}
F_{\beta\gamma} \,\pd^\beta F^{\alpha\gamma}
-\frac{1}{2}F_{\beta\gamma}\,\pd^\alpha F^{\beta\gamma}=
\frac{1}{2}\qty( F_{\beta\gamma}\pd^\beta F^{\alpha\gamma}
-F_{\beta\gamma}\pd^\beta F^{\gamma\alpha})
-\frac{1}{2}F_{\beta\gamma}\pd^\alpha F^{\beta\gamma}.
\end{equation}
Here we used the anti-symmetry of the EM tensor, 
$F^{\alpha\gamma}=-F^{\gamma\alpha}$. Now we do some more index
gymnastics by interchanging $\beta\leftrightarrow\gamma$ in the first
term, $F_{\gamma\beta}\pd^\gamma F^{\alpha\beta}
=-F_{\beta\gamma}\pd^\gamma F^{\alpha\beta}$. 
In the end what we have is
\begin{equation}
F_{\beta\gamma} \,\pd^\beta F^{\alpha\gamma}
-\frac{1}{2}F_{\beta\gamma}\,\pd^\alpha F^{\beta\gamma}=
-\frac{1}{2}F_{\beta\gamma}\qty( 
\pd^\gamma F^{\alpha\beta}
+\pd^\beta F^{\gamma\alpha}
+\pd^\alpha F^{\beta\gamma}).
\end{equation}
The expression inside the parentheses is just the contravariant version
of the RHS of \eqref{eq:Bianchi} and therefore vanishes.
So what we are left with for the divergence is
\begin{equation}
\pd_{\beta}{T_{\EM}}^{\alpha\beta}=-{F^{\alpha}}_\gamma J^\gamma,
\end{equation}
which is exactly $-\pd_\beta{T_n}^{\alpha\beta}$ from \eqref{eq2:pdvT}.

Therefore since $\pd_\beta$ is a linear operator we know that
\begin{equation}
\pd_{\beta}\qty[{T_{\EM}}^{\alpha\beta}+{T_n}^{\alpha\beta}]=0.
\end{equation}
This is physically reasonable since the total stress tensor, which
represents the total energy and momentum-current density, should
reasonably be conserved in a system. The two terms
${T_{\EM}}^{\alpha\beta}$ and ${T_n}^{\alpha\beta}$ are not conserved
separately since the particle and EM field exchanges energy and
momentum with each other, which also leads to the above conclusion
that there is no \emph{net} change in the total energy or momentum of
the system.





\renewcommand{\thesubsection}{\arabic{section} (\roman{subsection})}
\section{Time dilation}
\swapcommands{\phi}{\varphi}
\subsection{Two separate effects}
For a satellite in circular orbit around the Earth, 
%as in \figref{fig:orbit}, 
the time dilation can be seen as two separate
effects. Firstly the slowing down of time from the orbital speed due
to special relativity, and secondly a speed-up from weaker gravity due
to general relativity. It is not really correct to take these two as
separate effects, but in a case like the earth where both the orbital
speed and gravitation is extremely small (in natural units) this
approximation holds.

We can begin with the effects of gravity. The strength of gravity in
the weak field limit is given by the Newtonian potential
\begin{equation}\label{eq3:phi}
\phi(r)=-\frac{\mu}{r},
\end{equation}
where $\mu=GM$, $G$ being the constant of gravity and $M$ the mass
of the Earth. For a distant observer, unaffected by Earth's gravity,
a stationary clock (both the Earth and the clock being stationary in
the observer's frame of reference) only the 00 component of the metric
affects the time dilation, since the gravitational field is static in
this frame of reference, this component is in the weak field limit
$g_{00}=-(1+2\phi)$. The time dilation therefore becomes 
\begin{equation}
\rd\tau=
\overbrace{(-g_{\mu\nu}\rd{x}^\mu\rd{x}^\nu)^{1/2}}
^{=(-g_{00})^{1/2}\rd{t},\text{ static clock and field}}
=\Big(1+2\phi(r)\Big)^{1/2}\rd{t}
\approx\qty(1-\frac{\mu}{r})\rd{t},
\end{equation}
where $\rd{x}^\mu$ and $\rd{t}$ are in the coordinate system of the distant
observer, and $\rd\tau$ is the proper time recorded by that stationary
clock. Note that since $\rd\tau<\rd{t}$, the distant observer
will see the clock tick \emph{slower}\footnotemark{} by the amount
given above.  
\footnotetext{Say the clock is ticking with a time $\rd\tau$ between
  ticks. The distant observer sees the clock above Earth tick with a
  time $\rd{t}$ between ticks in his or her frame of reference.
  Therefore since $\rd{t}>\rd\tau$ the Earth clock seems slower than
  an identical clock at the distant observer.}

To get the SR time dilation, we need the orbital speed which is
handily calculated using Newtonian mechanics to be
\begin{equation}
v^2=\frac{GM}{R}=\frac{\mu}{R},
\end{equation}
where $R$ is the orbital radius. Only accounting for the SR effects of
the orbital speed that time dilation becomes
\begin{equation}
\rd\tau=\gamma^{-1}\,\rd{t}=\qty(1-v^2)^{1/2}\rd{t}
=\qty(1-\frac{\mu}{R})^{1/2}\rd{t}
\approx\qty(1-\frac{\mu}{2R})\rd{t}.
\end{equation}
Once again $\rd\tau$ is the proper time of a clock moving at a speed
$v$, without gravity, while $\rd{t}$ is its time recorded by the
distant observer. 

In this weak field and slow moving limit the real proper time of the
satellite should therefore be the product of the two effects
\begin{equation}
\rd\tau_\text{sat}=\qty(1-\frac{\mu}{R})\qty(1-\frac{\mu}{2R})\rd{t}
\approx\qty(1-\frac{3\mu}{2R})\rd{t}.
\end{equation}
This we want to compare to a stationary clock on the surface of the
Earth, at a distance $R_\E$ from the center of the Earth, which only
has the GR effects 
\begin{equation}
\rd\tau_\E=\eval{\rd\tau}_{R_\E}=\qty(1-\frac{\mu}{R_\E})\rd{t}.
\end{equation}
Setting these two equal, we get
\begin{equation}\label{eq3:R}
R=\frac{3}{2}R_\E.
\end{equation}


\subsection{But there is really only one time dilation}
Since the previous section felt a bit hand-wavy, we can also redo the
same calculations a bit more rigorously. We actually had the answer in
front of our noses from the very beginning. We start off with the
square of the proper time compared to the time of the distant observer 
\begin{equation}\label{eq3:dtaudt}
\qty(\frac{\rd\tau}{\rd{t}})^2=
-g_{\mu\nu}\frac{\rd{x}^\mu}{\rd{t}}\frac{\rd{x}^\nu}{\rd{t}}.
\end{equation}
Next, in the weak field approximation we can write the metric as
\begin{equation}
g_{\mu\nu}=\eta_{\mu\nu}+h_{\mu\nu}\qc h_{\mu\nu}\ll1.
\end{equation}
Since we know that the satellite is slow,
\begin{equation}
1=\dv{x^0}{t}\gg\dv{x^i}{t}= v^i
\end{equation}
we know that the only correction to the metric which survives to first
order is $h_{00}=-2\phi$, the rest of them get multiplied by a factor
$v^i\sim v\ll1$.

We can now evaluate \eqref{eq3:dtaudt} to be
\begin{equation}
\qty(\frac{\rd\tau}{\rd{t}})^2\approx
-g_{00}+\eta_{ij}\frac{\rd{x}^i}{\rd{t}}\frac{\rd{x}^j}{\rd{t}}
=1+2\phi-v^2.
\end{equation}
For the orbiting satellite this becomes
\begin{equation}
\qty(\frac{\rd\tau_\text{sat}}{\rd{t}})^2
=1-2\frac{\mu}{R}-\frac{\mu}{R}
=1-3\frac{\mu}{R},
\end{equation}
and for the stationary clock on the surface of the Earth we get
\begin{equation}
\qty(\frac{\rd\tau_\text{E}}{\rd{t}})^2=1-2\frac{\mu}{R}.
\end{equation}
For these two proper times to be the same, we recover \eqref{eq3:R}
from the previous calculation.


\section{Stereographic projection}
\swapcommands{\phi}{\varphi}

\begin{figure}\centering
\resizebox{.6\textwidth}{!}{
\input{figures/stereographic.pdf_t}}
\caption{}
\label{fig:stereographic}
\end{figure}

\figref{fig:stereographic} shows the geometry of a stereographic
projection, from the south pole of the unit sphere onto the equatorial
plane, for two points on the sphere with the same polar angle
$\phi$. From the \emph{theorem of inscribed angles} it is clear that
the vertical angle from the south pole is exactly half the azimuthal
angle $\theta$. This means that the distance from $(x,y)$ to the
origin, $r$, must be given by $r=\tan(\theta/2)$. Therefore,
expressing $x$ and $y$ in polar coordinates we get
\begin{equation}
\begin{cases}
x=\tan\tfrac{\theta}{2}\,\cos\phi\\
y=\tan\tfrac{\theta}{2}\,\sin\phi.
\end{cases}
\end{equation}

Here we want to find the metric 
\begin{equation}
\rd{s}^2=\rd\theta^2+\sin^2\theta\,\rd\phi^2
\end{equation}
in terms of the $(x,y)$ coordinates. As a good first start, we can
begin to compute
\begin{equation}
\rd{x}=\pdv{x}{\theta}\rd\theta+\pdv{x}{\phi}\rd\phi
=\frac{\cos\phi\;\rd\theta}{2\cos^2(\theta/2)}
-\tan\frac{\theta}{2}\,\sin\phi\,\rd\phi
\end{equation}
and
\begin{equation}
\rd{y}=\pdv{y}{\theta}\rd\theta+\pdv{y}{\phi}\rd\phi
=\frac{\sin\phi\;\rd\theta}{2\cos^2(\theta/2)}
+\tan\frac{\theta}{2}\,\cos\phi\,\rd\phi.
\end{equation}
From here we easily calculate
\begin{equation}
\rd{x}^2+\rd{y}^2=
\frac{\rd\theta^2}{4\cos^4(\theta/2)} 
+\tan^2\frac{\theta}{2}\;\rd\phi^2
\end{equation}
by using $\cos^2+\sin^2\equiv1$ and seeing that the cross
terms cancel. To simplify this, we use the trigonometric identities
\begin{equation}
\cos^2(\theta/2)\equiv\frac{1+\cos\theta}{2}
\end{equation}
and
\begin{equation}
\tan\frac{\theta}{2}\equiv \frac{\sin\theta}{1+\cos\theta},
\end{equation}
which gives
\begin{equation}
\rd{x}^2+\rd{y}^2
=\frac{\rd\theta^2+\sin^2\theta\,\rd\phi^2}{(1+\cos\theta)^2} 
=\frac{\rd{s}^2}{(1+\cos\theta)^2}.
\end{equation}
We are almost there. We just need to use one more trigonometric
identity,
\begin{equation}
r^2=\tan^2\frac{\theta}{2}\equiv
\frac{1-\cos\theta}{1+\cos\theta}
=\frac{1+1-(1+\cos\theta)}{1+\cos\theta}
=\frac{2}{1+\cos\theta}-1.
\end{equation}
Solving for $(1+\cos\theta)$ yields
\begin{equation}
1+\cos\theta=\frac{2}{1+r^2},
\end{equation}
and we see that we can write down the metric for this coordinate
system as
\begin{equation}\label{eq4:ds2-sphere}
\rd{s}^2=4\,\frac{\,\rd{x}^2+\rd{y}^2\,}{(1+r^2)^2},
\end{equation}
with $r^2=x^2+y^2$.

To show that this is an Einstein metric, I refer to
Section~\ref{sec:geometries} where we have already calculate the Ricci
tensor for a flat, a spherical and a hyperbolic geometry. Using
\eqref{eq8:g} and (\ref{eq8:Rmunu}a-c), we see that in
that coordinate system $R_{\mu\nu}=kr_0^{-2}g_{\mu\nu}$, which in our
geometry with $k=1$ (sphere) and $r_0=1$ (of radius 1) becomes
\begin{equation}\label{eq4:R-g}
R_{\mu\nu}=g_{\mu\nu}.
\end{equation}
This was in that coordinate system, but since \eqref{eq4:R-g} is a
\emph{tensor equation} it holds in \emph{any} coordinate system,
including this. Similarly when we take the metric
\begin{equation}\label{eq4:ds2-hyper}
\rd{s}^2=4\,\frac{\,\rd{x}^2+\rd{y}^2\,}{(1-r^2)^2}
\end{equation}
which belongs to a hyperbolic surface, we once again use \eqref{eq8:g}
and (\ref{eq8:Rmunu}a-c), but with $k=-1$ and $r_0=1$. This leads to
\begin{equation}
R_{\mu\nu}=-g_{\mu\nu},
\end{equation}
which of course also holds in this coordinate system. We can therefore
conclude that both \eqref{eq4:ds2-sphere} and \eqref{eq4:ds2-hyper}
are Einstein metrics.


\section{Variational method}

\subsection{The basic action}
We have the action
\begin{equation}\label{eq5:S}
S[x]=\int_A^B\rd\tau=\int_A^B\rd\sigma\,\dv{\tau}{\sigma}
=\int_A^B\rd\sigma\,
\sqrt{-g_{\mu\nu}\dv{x^\mu}{\sigma}\dv{x^\nu}{\sigma}}.
\end{equation}
If we vary this action we get
\vspace{-1em}
\begin{equation}\label{eq5:dS1}
\begin{aligned}
\delta{S}[x]=&\int_A^B\overbrace{\rd\sigma}^{\rd\tau(\dv*{\sigma}{\tau})}
\overbrace{\frac{1}{2\sqrt{-g_{\mu\nu}\dv{x^\mu}{\sigma}\dv{x^\nu}{\sigma}}}}
^{=(2\dv*{\tau}{\sigma})^{-1}}
\delta{x}^\rho\pd_\rho\qty[-g_{\mu\nu}\dv{x^\mu}{\sigma}\dv{x^\nu}{\sigma}
]\\
=&-\frac{1}{2}\int_A^B\rd\tau\,
\qty(\dv{\sigma}{\tau})^2 \delta{x}^\rho
\qty[(\pd_{\rho}g_{\mu\nu})\dv{x^\mu}{\sigma}\dv{x^\nu}{\sigma}
+2g_{\mu\nu}\qty(\pd_{\rho}\dv{x^\mu}{\sigma})\dv{x^\nu}{\sigma}
].
\end{aligned}
\end{equation}
From here we note that
\begin{equation}
\dv{x^\mu}{\sigma}+\dv{\delta{x}^\mu}{\sigma}
=\dv{\sigma}\qty[x^\mu+\delta{x}^\mu]
=\dv{x^\mu}{\sigma}+\delta{x}^\rho\pd_\rho\dv{x^\mu}{\sigma},
\end{equation}
which means that \eqref{eq5:dS1} can be written as
\begin{equation}\label{eq5:dS2}
\begin{aligned}
\delta{S}[x]=&-\frac{1}{2}\int_A^B\rd\tau\,\qty(\dv{\sigma}{\tau})^2 
\qty[ \delta{x}^\rho(\pd_{\rho}g_{\mu\nu})
\dv{x^\mu}{\sigma}\dv{x^\nu}{\sigma}
+2g_{\mu\nu}\dv{\delta{x}^\mu}{\sigma}\dv{x^\nu}{\sigma}
]\\
=&-\frac{1}{2}\int_A^B\rd\tau\,
\qty[ \delta{x}^\rho(\pd_{\rho}g_{\mu\nu})
\dv{x^\mu}{\tau}\dv{x^\nu}{\tau}
+2g_{\mu\nu}\dv{\delta{x}^\mu}{\tau}\dv{x^\nu}{\tau}
].
\end{aligned}
\end{equation}
We now integrate the last term by parts, yielding
\begin{equation}
\begin{aligned}
\int_A^B\rd\tau\,g_{\mu\nu}\dv{\delta{x}^\mu}{\tau}\dv{x^\nu}{\tau}
=&\cancel{\qty[g_{\mu\nu}\delta{x}^\mu\dv{x^\nu}{\tau}]_A^B}
-\int_A^B\rd\tau\,\delta{x}^\mu
\qty(\dv{g_{\mu\nu}}{\tau}\dv{x^\nu}{\tau}
+g_{\mu\nu}\dv[2]{x^\nu}{\tau})\\
=&-\int_A^B\rd\tau\,\delta{x}^\mu
\qty((\pd_\sigma g_{\mu\nu})\dv{x^\sigma}{\tau}\dv{x^\nu}{\tau}
+g_{\mu\nu}\dv[2]{x^\nu}{\tau}).
\end{aligned}
\end{equation}
The term $[\ldots]_A^B$ vanish, since we require that
$\delta{x}^\mu=0$ at the end points -- otherwise the path would no
longer be from $A$ to $B$.

We can rename the index $\mu\to\rho$ and then plug this back into
\eqref{eq5:dS2}, which gives
\begin{equation}%\label{eq5:dS2}
\begin{aligned}
\delta{S}[x]=&\frac{1}{2}\int_A^B\rd\tau\,\delta{x}^\rho
\qty[ -(\pd_{\rho}g_{\mu\nu})\dv{x^\mu}{\tau}\dv{x^\nu}{\tau}
+2(\pd_\sigma g_{\rho\nu})\dv{x^\sigma}{\tau}\dv{x^\nu}{\tau}
+2g_{\rho\nu}\dv[2]{x^\nu}{\tau} ]\\
=&\frac{1}{2}\int_A^B\rd\tau\,\delta{x}^\rho 
\qty[ -(\pd_{\rho}g_{\mu\nu})\dv{x^\mu}{\tau}\dv{x^\nu}{\tau}
+2(\pd_\sigma g_{\rho\nu})\dv{x^\sigma}{\tau}\dv{x^\nu}{\tau}
+2g_{\rho\nu}\dv[2]{x^\nu}{\tau} ].
\end{aligned}
\end{equation}
We want to find the path that minimizes the action, meaning that
$\delta{S}[x]=0$, or in other words
\begin{equation}\label{eq5:big-geodesic}
2(\pd_\sigma g_{\rho\nu})\dv{x^\sigma}{\tau}\dv{x^\nu}{\tau}
-(\pd_{\rho}g_{\mu\nu})\dv{x^\mu}{\tau}\dv{x^\nu}{\tau}
+2g_{\rho\nu}\dv[2]{x^\nu}{\tau} =0.
\end{equation}
Now we take a look at the first term
\begin{equation}
\begin{aligned}
2(\pd_\sigma g_{\rho\nu})\dv{x^\sigma}{\tau}\dv{x^\nu}{\tau}=&
\underbrace{(\pd_{\sigma}g_{\rho\nu})
  \dv{x^\sigma}{\tau}\dv{x^\nu}{\tau}}
_{\sigma\to\mu,\;\nu\to\sigma}
+\underbrace{(\pd_{\sigma}g_{\rho\nu})
  \dv{x^\sigma}{\tau}\dv{x^\nu}{\tau}}
_{\nu\to\mu}\\
=&(\pd_\mu g_{\rho\sigma})\dv{x^\mu}{\tau}\dv{x^\sigma}{\tau}
+(\pd_\sigma g_{\rho\mu})\dv{x^\sigma}{\tau}\dv{x^\mu}{\tau}\\
=&\Big(\pd_\mu g_{\rho\sigma}+\pd_\sigma g_{\rho\mu}\Big)
\dv{x^\mu}{\tau}\dv{x^\sigma}{\tau}.
\end{aligned}
\end{equation}
Then setting $\nu\to\sigma$ in the middle term of
\eqref{eq5:big-geodesic}, yields
\begin{equation}
\Big(\pd_\mu g_{\rho\sigma}+\pd_\sigma g_{\rho\mu}
-\pd_{\rho}g_{\mu\sigma}\Big)
\dv{x^\mu}{\tau}\dv{x^\sigma}{\tau}
+2g_{\rho\nu}\dv[2]{x^\nu}{\tau} =0
\end{equation}
Then by multiplying everything by $\frac{1}{2}g^{\tau\rho}$ we get
\begin{equation}
0=\dv[2]{x^\tau}{\tau} +
\frac{1}{2}g^{\tau\rho}
\Big(\pd_\mu g_{\rho\sigma}+\pd_\sigma g_{\rho\mu}
-\pd_{\rho}g_{\mu\sigma}\Big)
\dv{x^\mu}{\tau}\dv{x^\sigma}{\tau}
=\dv[2]{x^\tau}{\tau} +
\Gamma^\tau_{\mu\sigma}
\dv{x^\mu}{\tau}\dv{x^\sigma}{\tau},
\end{equation}
which is the geodesic equation just as expected.


\subsection{Another action}
In this part of the problem we have a slightly different Lagrangian
for the action,
\begin{equation}\label{eq5:Lagrangian2}
S'[x]=\int_A^B\rd\tau\,\mathcal{L}(x,\dot{x})
=\int_A^B\rd\tau\,\Big(-g_{\mu\nu}(x)\dot{x^\mu}\dot{x^\nu}\Big),
\end{equation}
where $\dot{x}=\dv*{x}{\tau}$. With the Euler-Lagrange equations we
get
\vspace{-1ex}
\begin{equation}
\begin{aligned}
0=&
\dv{\tau}\qty[\pdv{\mathcal{L}}{\dot{x}^\rho}]
-\dv{\mathcal{L}}{{x}^\rho}
=\dv{\tau}\bigg[-2g_{\mu\nu}
\overbrace{\pdv{\dot{x}^\mu}{\dot{x}^\rho}}^{=\delta^\mu_\rho}
\dot{x^\nu}\bigg]
-\Big(-(\pd_{\rho}g_{\mu\nu})\dot{x}^{\mu}\dot{x}^\nu\Big)\\
=&-2g_{\nu\rho}\ddot{x}^\nu
-2\dv{g_{\nu\rho}}{\tau}\dot{x}^\nu
+(\pd_{\rho}g_{\mu\nu})\dot{x}^{\mu}\dot{x}^\nu\\
=&-2g_{\nu\rho}\dv[2]{x^\nu}{\tau}
-2(\pd_{\sigma}g_{\nu\rho})\dv{x^\sigma}{\tau}\dv{x^\nu}{\tau}
+(\pd_{\rho}g_{\mu\nu})\dv{x^{\mu}}{\tau}\dv{x^\nu}{\tau}.
\end{aligned}
\end{equation}
This last step is exactly minus the LHS of \eqref{eq5:big-geodesic},
which we just showed resulted in the geodesic equation. We can
therefore now say that \eqref{eq5:Lagrangian2} also gives the same
geodesic equation. However, one rather serious drawback of
\eqref{eq5:Lagrangian2} is that it is \emph{not} parametrization
invariant
\begin{equation}
S'[x]=\int_A^B\rd\sigma\,\dv{\tau}{\sigma}
\qty(-g_{\mu\nu}\dv{x^\mu}{\tau}\dv{x^\nu}{\tau})
=\int_A^B\rd\sigma\,\dv{\sigma}{\tau}
\qty(-g_{\mu\nu}\dv{x^\mu}{\sigma}\dv{x^\nu}{\sigma}).
\end{equation}
There pops up an unwanted factor $\dv*{\sigma}{\tau}$ with this
Lagrangian, which is avoided by the square root in \eqref{eq5:S}.

\section{Covariant Laplacian}
\swapcommands{\phi}{\varphi}
The covariant Laplacian acting on a scalar field is given by
\begin{equation}
\Box\phi=\nabla_\mu\nabla^\mu\phi=\nabla_\mu(g^{\mu\nu}\nabla_\nu)\phi
=(g^{\mu\nu}\nabla_\mu)\nabla_\nu\phi
=\nabla^\nu\nabla_\nu\phi.
\end{equation}
We are free to move around the metric tensor as we wish because of the
\emph{metric postulate}.

Let's use the middle expression to write out the Laplacian
explicitly. First of all, we note that $\nabla_\nu\phi=\pd_\nu\phi$
since there is no indices for the affine connection to act on. Next
we get
\begin{equation}
\Box\phi=\pd_\mu(g^{\mu\nu}\pd_\nu)\phi
+\Gamma^\mu_{\mu\rho}(g^{\rho\nu}\pd_{\nu}\phi).
\end{equation}
From here we use the determinant formula
\begin{equation}\label{eq:det-formula}
\Gamma^\mu_{\mu\rho}=\frac{1}{2}
\Big(\cancel{g^{\mu\sigma}\pd_\mu g_{\rho\sigma}}
-\cancel{g^{\mu\sigma}\pd_\sigma g_{\mu\rho}} \Big)
+\frac{1}{2}g^{\mu\sigma}\pd\rho g_{\mu\sigma}
=\frac{1}{2}g^{-1}\pd_\rho g,
\end{equation}
where $g=\det(g_{\mu\nu})$ is the determinant of the metric.

In the even more concrete example of polar coordinates on a flat 2D
surface, we have the metric
\begin{equation}
g_{\mu\nu}=
\begin{bmatrix}
1&0\\
0&r^2
\end{bmatrix}
\qand
g^{\mu\nu}=
\begin{bmatrix}
1&0\\
0&r^{-2}
\end{bmatrix},
\end{equation}
resulting in $g=r^2$ and
\begin{equation}
\Gamma^\mu_{\mu\rho}=\frac{1}{2}g^{-1}
\begin{bmatrix}
\pd_r g\\\pd_\varphi g
\end{bmatrix}
=
\begin{bmatrix}
r^{-1}\\0
\end{bmatrix}.
\end{equation}
We can now write down the covariant Laplacian on a flat surface with
polar coordinates as
\begin{equation}
\Box\phi=\pd_r(g^{rr}\pd_r\phi)
+\pd_\varphi(g^{\varphi\varphi}\pd_\varphi\phi)
+\Gamma^{\mu}_{\mu r}(g^{rr}\pd_r\phi)
=\Big[\pd_{rr}+r^{-1}\pd_r+r^{-2}\pd_{\varphi\varphi}
\Big]\phi,
\end{equation}
which is precisely what we expected from previous knowledge on
multivariate calculus. 


\section{A geodesic}
To find the geodesics of the two dimensional metric
\begin{equation}
\rd\tau^2=t^{-2}\,\rd{t}^2 - t^{-2}\,\rd{x}^2,
\end{equation}
we begin by noting that it only depends on $t$. This means that if we
use the Lagrangian from \eqref{eq5:Lagrangian2}
\begin{equation}
\mathcal{L}=-g_{\mu\nu}\dv{x^\mu}{\tau}\dv{x^\nu}{\tau}
=t^{-2}\,\dot{t}^2 - t^{-2}\,\dot{x}^2,
\end{equation}
then we can use the Euler-Lagrange equation
\begin{equation}
\dv{\tau}\qty[\pdv{\mathcal{L}}{\dot{x}}]
-\cancelto{0}{\dv{\mathcal{L}}{x}}=0.
\end{equation}
This means that
\begin{equation}
\dv{\tau}\qty[-2t^{-2}\dot{x}]=0
\end{equation}
or in other words
\begin{equation}
\dot{x}=t^2C,
\end{equation}
where $C$ is a constant. We can also write
\begin{equation}
\dot{t}=\dv{t}{\tau}=\dv{t}{x}\dv{x}{\tau}
=\dv{t}{x}\,t^2C.
\end{equation}
Now, for a time-like ($\rd{t}>\rd{x}$) geodesic we get
\begin{equation}
+1=t^{-2}\,\qty(\dv{t}{\tau})^2
- t^{-2}\,\qty(\dv{x}{\tau})^2
=t^{-2}\,\qty(t^2C\dv{t}{x})^2
- t^{-2}\qty(t^2C)^2
\end{equation}
which leads to
\begin{equation}
\qty(\dv{t}{x})^2=\frac{1+C^2t^2}{C^2t^2}
\quad\Longrightarrow\quad
\dv{x}{t}=\frac{Ct}{\sqrt{1+C^2t^2}}<1\qfor t>0.
\end{equation}
This has the solution
\begin{equation}
x(t)=\sqrt{1+C^2t^2}+x_0,
\end{equation}
for some constant of integration $x_0$. If we consider a light cone
emanating from the origin $(t,x)=(0,0)$, and since we want the
geodesic to lie inside the light cone, $x_0=-1$ and we see that indeed
$x(t)<t$ for $t>0$.




\section{Some curved geometries}
\label{sec:geometries}
\swapcommands{\phi}{\varphi}
Given the metrics
\begin{equation}\label{eq8:ds2}
\rd{s}^2=\frac{\rd{r}^2}{1-k\frac{r^2}{r_0^2}}+r^2\rd\phi^2
\qc k=-1,0,+1,
\end{equation}
we want to compute the curvature scalar
\begin{equation}
R=g^{\mu\nu}R_{\mu\nu},
\end{equation}
where
\begin{equation}
R_{\mu\nu}=R^{\rho}_{\mu\rho\nu}
\end{equation}
and
\begin{equation}
R^{\rho}_{\sigma\mu\nu}
=\pd_\mu\Gamma^{\rho}_{\nu\sigma}-\pd_\nu\Gamma^{\rho}_{\mu\sigma}
+\Gamma^\rho_{\mu\tau}\Gamma^\tau_{\nu\sigma}
-\Gamma^\rho_{\nu\tau}\Gamma^\tau_{\mu\sigma}.
\end{equation}
The metric tensor is diagonal and has the form
\begin{equation}\label{eq8:g}
g_{\mu\nu}=
\begin{bmatrix}
\qty(1-kr^2/r_0^2)^{-1}&0\\
0&r^{2}
\end{bmatrix}
\qand
g^{\mu\nu}=
\begin{bmatrix}
\qty(1-kr^2/r_0^2)&0\\
0&r^{-2}
\end{bmatrix}.
\end{equation}
Then we need the affine connection
\begin{equation}
\Gamma^\rho_{\mu\nu}=\frac{1}{2}g^{\rho\sigma}
\Big(
\pd_{\mu}g_{\nu\sigma} +\pd_{\nu}g_{\mu\sigma} -\pd_{\sigma}g_{\mu\nu}
\Big).
\end{equation}
Since $g_{\mu\nu}$ is diagonal we will only have
$g^{\rho\sigma}=g^{\rho\rho}$ in the first factor, which means that
\begin{equation}
\Gamma^r_{\mu\nu}=\frac{1}{2}g^{rr}
\Big(
\pd_{\mu}g_{\nu r} +\pd_{\nu}g_{\mu r} -\pd_{r}g_{\mu\nu}
\Big)
\end{equation}
and
\begin{equation}
\Gamma^\phi_{\mu\nu}=\frac{1}{2}g^{\phi\phi}
\Big(
\pd_{\mu}g_{\nu \phi} +\pd_{\nu}g_{\mu \phi}
-\cancelto{0}{\pd_{\phi}g_{\mu\nu}}\hspace{1em}
\Big)
\end{equation}
We also point out that $g_{\mu\nu}$ only depends on $r$, meaning that
only $\pd_r$ survives. With these two comments in mind we get
\begin{subequations}
\begin{equation}
\Gamma^{r}_{r\phi}=\Gamma^{r}_{\phi r}=\frac{1}{2}g^{rr}
\Big(
\pd_{r}\cancelto{0}{g_{\phi r}}\hspace{1em}
+\cancelto{0}{\pd_{\phi}g_{rr}}\hspace{1em}
-\pd_{r}\cancelto{0}{g_{r\phi}}\hspace{1em}
\Big)
=0,
\end{equation}
\begin{equation}
\Gamma^{r}_{rr}=\frac{1}{2}g^{rr}\pd_r g_{rr}
=k\frac{r}{r_0^2}\qty(1-k\frac{r^2}{r_0^2})^{-1},
\end{equation}
\begin{equation}
\Gamma^{r}_{\phi\phi}=-\frac{1}{2}g^{rr}\pd_r g_{\phi\phi}
=-r\qty(1-k\frac{r^2}{r_0^2}),
\end{equation}
\begin{equation}
\Gamma^\phi_{rr}=\Gamma^{\phi}_{\phi\phi}=0,
\end{equation}
\begin{equation}
\Gamma^\phi_{r\phi}=\Gamma^{\phi}_{\phi r}
=\frac{1}{2}g^{\phi\phi}\pd_rg_{\phi\phi}
=\frac{1}{r}.
\end{equation}
\end{subequations}


For the Riemann tensor, we begin by noting that it is anti-symmetric
in its last two indices,
$R^{\rho}_{\sigma\mu\nu}=-R^{\rho}_{\sigma\nu\mu}$, which means that
we directly get $R^{\rho}_{\sigma rr}=R^{\rho}_{\sigma\phi\phi}=0$ and
we only have to look at
\begin{equation}
R^{\rho}_{\sigma r\phi}=-R^{\rho}_{\sigma \phi r}
=\pd_{r}\Gamma^{\rho}_{\phi\sigma}
-\cancelto{0}{\pd_\phi\Gamma^{\rho}_{r\sigma}}\hspace{1em}
+\Gamma^\rho_{r\tau}\Gamma^\tau_{\phi\sigma}
-\Gamma^\rho_{\phi\tau}\Gamma^\tau_{r\sigma}.
\end{equation}
For $\rho=r$ we get
\begin{equation}
R^{r}_{\sigma r\phi}=-R^{r}_{\sigma \phi r}
=\pd_{r}\Gamma^{r}_{\phi\sigma}
+\Gamma^r_{rr}\Gamma^r_{\phi\sigma}
-\Gamma^r_{\phi\phi}\Gamma^\phi_{r\sigma}
\end{equation}
and for $\rho=\phi$ we get
\begin{equation}
R^{\phi}_{\sigma r\phi}=-R^{\phi}_{\sigma \phi r}
=\pd_{r}\Gamma^{\phi}_{\phi\sigma}
+\Gamma^\phi_{r\phi}\Gamma^\phi_{\phi\sigma}
-\Gamma^\phi_{\phi r}\Gamma^r_{r\sigma}.
%=\pd_{r}\Gamma^{\phi}_{\phi\sigma}
%+\frac{1}{r}\Big(
%\Gamma^\phi_{\phi\sigma}-\Gamma^r_{r\sigma}
%\Big).
\end{equation}
We can now write down the remaining elements
\begin{subequations}
\begin{equation}
R^{r}_{r r\phi}=-R^{r}_{r \phi r}
=\pd_{r}\cancelto{0}{\Gamma^{r}_{\phi r}}
+\Gamma^r_{rr}\cancelto{0}{\Gamma^r_{\phi r}}
-\Gamma^r_{\phi\phi}\cancelto{0}{\Gamma^\phi_{rr}}=0,
\end{equation}
\begin{equation}
R^{r}_{\phi r\phi}=-R^{r}_{\phi \phi r}
=\pd_{r}\Gamma^{r}_{\phi\phi}
+\Gamma^r_{rr}\Gamma^r_{\phi\phi}
-\Gamma^r_{\phi\phi}\Gamma^\phi_{r\phi}
=\ldots=\frac{kr^2}{r_0^2},
\end{equation}
\begin{equation}
R^{\phi}_{\phi r\phi}=-R^{\phi}_{\phi \phi r}
=\pd_{r}\cancelto{0}{\Gamma^{\phi}_{\phi\phi}}
+\Gamma^\phi_{r\phi}\cancelto{0}{\Gamma^{\phi}_{\phi\phi}}
-\Gamma^\phi_{\phi r}\cancelto{0}{\Gamma^r_{r\phi}}
=0,
\end{equation}
\begin{equation}
R^{\phi}_{r r\phi}=-R^{\phi}_{r \phi r}
=\pd_{r}\Gamma^{\phi}_{\phi r}
+\Gamma^\phi_{r\phi}\Gamma^\phi_{\phi r}
-\Gamma^\phi_{\phi r}\Gamma^r_{rr}
=\ldots=-\frac{k}{r_0^2}\qty(1-k\frac{r^2}{r_0^2})^{-1}.
\end{equation}
\end{subequations}
With these few non-zero elements the Ricci tensor is easily calculated
\begin{equation}
R_{\mu\nu}=R^\rho_{\mu\rho\nu}=R^r_{\mu r\nu}+R^\phi_{\mu\phi\nu}.
\end{equation}
We therefore get
\begin{subequations}\label{eq8:Rmunu}
\begin{equation}
R_{rr}=\cancelto{0}{R^r_{r r r}}+R^\phi_{r \phi r}
=\frac{k}{r_0^2}\qty(1-k\frac{r^2}{r_0^2})^{-1},
\end{equation}
\begin{equation}
R_{\phi\phi}=R^r_{\phi r\phi}+\cancelto{0}{R^{\phi}_{\phi\phi\phi}}\hspace{1em}
=\frac{kr^2}{r_0^2},
\end{equation}
\begin{equation}
R_{r\phi}=R_{\phi r} =0.
\end{equation}
\end{subequations}
And lastly the curvature scalar becomes
\begin{equation}\label{eq8:R}
R=g^{\mu\nu}R_{\mu\nu}=g^{rr}R_{rr}+g^{\phi\phi}R_{\phi\phi}
=\frac{2k}{r_0^2}.
\end{equation}

The value of the curvature scalar $R$ seems very reasonable since
$k=0$ is obviously a flat surface. To see what the other two
geometries are, $k=\pm1$, we first set $k=1$ and change the
coordinates to 
\begin{equation}\label{eq8:theta}
\begin{cases}
r= r_0\sin\theta,\\
\phi=\phi.
\end{cases}
\end{equation}
This means that
\begin{equation}
\theta=\arcsin(\frac{r}{r_0})
\quad\Longrightarrow\quad
\rd\theta^2=\qty(\frac{\,\rd{r}}{r_0\sqrt{1-r^2/r_0^2}})^2
=\frac{\rd{r}^2}{r_0^2-r^2},
\end{equation}
and the metric \eqref{eq8:ds2} becomes 
\begin{equation}\label{eq8:ds2+1}
\eval{\rd{s}^2}_{k=1}=\frac{r_0^2\,\rd{r}^2}{r_0^2-r^2}+r^2\rd\phi^2
=r_0^2\,\rd\theta^2+r_0^2\sin^2\theta\,\rd\phi^2.
\end{equation}
For $k=-1$, we see that changing $\sin\to\sinh$ in \eqref{eq8:theta}
yields
\begin{equation}
\theta=\sinh^{-1}\qty(\frac{r}{r_0})
\quad\Longrightarrow\quad
\rd\theta^2=\qty(\frac{\,\rd{r}}{r_0\sqrt{1+r^2/r_0^2}})^2
=\frac{\rd{r}^2}{r_0^2+r^2},
\end{equation}
the the metric then becomes
\begin{equation}\label{eq8:ds2-1}
\eval{\rd{s}^2}_{k=-1}=\frac{r_0^2\,\rd{r}^2}{r_0^2+r^2}+r^2\rd\phi^2
=r_0^2\,\rd\theta^2+r_0^2\sinh^2\theta\,\rd\phi^2.
\end{equation}
These two metrics, \eqref{eq8:ds2+1} and \eqref{eq8:ds2-1}, belongs to
a sphere of radius $r_0$ and a corresponding hyperbolic surface. So it
is clear that the curvature scalar \eqref{eq8:R} behave just as
expected. 







%%%%%%%%%%%%%%%%%%%%%%%%%%%%%%%%%%%%%%%%%%%%%%%%%%%%%%%%%%%%%%%%%%%%%%
\end{document}%% ^ ^ ^ ^ ^ ^ ^ ^ ^ ^ ^ ^ ^ ^ ^ ^ ^ ^ ^ ^ ^ ^ ^ ^ ^ ^ ^
%%%%%%%%%%%%%%%%%%%%%%%%%%%%%%%%%%%%%%%%%%%%%%%%%%%%%%%%%%%%%%%%%%%%%%



%%% Local Variables:
%%% mode: latex
%%% TeX-master: t
%%% End:

%  LocalWords:  contravariant Stereographic covariant variational
%  LocalWords:  azimuthal affine parametrization Ricci
