\documentclass[11pt,a4paper, 
swedish, english %% Make sure to put the main language last!
]{article}
\pdfoutput=1

%% Andréas's custom package 
%% (Will work for most purposes, but is mainly focused on physics.)
\usepackage{../custom_as}

%% Figures can now be put in a folder: 
\graphicspath{ {figures/} %{some_folder_name/}
}

%% If you want to change the margins for just the captions
\usepackage[margin=10 pt]{caption}

%% To add todo-notes in the pdf
\usepackage[%disable  %%this will hide all notes
]{todonotes} 

%% Change the margin in the documents
\usepackage[
%            top    = 3cm,              %% top margin
%            bottom = 3cm,              %% bottom margin
%            left   = 2.5cm, right  = 2.5cm %% left and right margins
]{geometry}


%% If you want to change the formatting of the section headers
%\renewcommand{\thesection}{...}

\newcommand{\EM}{\text{EM}}
\newcommand{\E}{\text{E}}

%%%%%%%%%%%%%%%%%%%%%%%%%%%%%%%%%%%%%%%%%%%%%%%%%%%%%%%%%%%%%%%%%%%%%%
\begin{document}%% v v v v v v v v v v v v v v v v v v v v v v v v v v
%%%%%%%%%%%%%%%%%%%%%%%%%%%%%%%%%%%%%%%%%%%%%%%%%%%%%%%%%%%%%%%%%%%%%%


%% If you want to use an external file for the title page
%\renewcommand{\thefootnote}{\fnsymbol{footnote}}

%kortkommandon för mailaddresserna
\newcommand{\andsunds}{andsunds@student.chalmers.se}
\newcommand{\rigon}{rigon@student.chalmers.se}



\pagenumbering{roman} %%Romersk sidnumrering i början
\begin{titlepage}
\newgeometry{top=3cm, bottom=2cm}

\newcommand{\HRule}{\rule{\linewidth}{0.5mm}} % Defines a new command for the horizontal lines, change thickness here

\center % Center everything on the page
 
%------------------------------------------------------------------------------------
%	HEADING SECTIONS
%------------------------------------------------------------------------------------

\textsc{\huge Chalmers tekniska högskola}\\[1.5cm] % Name of university/college
\textsc{\Large Rapport, Experimentell fysik 2}\\[0.2cm] % Major heading such as course name
\textsc{\large Termodynamik -- Uppgift 3 }\\[0.5cm] % Minor heading such as course title

%------------------------------------------------------------------------------------
%	TITLE SECTION
%------------------------------------------------------------------------------------

\HRule \\[0.4cm]
{ \LARGE \bfseries 
Studier av kvicksilveratomens atomära emissionsspektra samt absorptionsspektra av laserfärgämnena Rhodamin B och Kumarin 307
}\\[0.4cm] % Title of  document
\HRule \\[1.5cm]
 
%------------------------------------------------------------------------------------
%	AUTHOR SECTION
%------------------------------------------------------------------------------------

\begin{minipage}{0.4\textwidth}
\begin{flushleft} \large
\emph{Författare:}\\
Andréas Sundström\footnotemark{} \\
Rigon Demisai\footnotemark{} 
\end{flushleft}
\end{minipage}
~
\begin{minipage}{0.4\textwidth}
\begin{flushright} \large
\emph{Labassistent:} \\
Martin Wersäll
\end{flushright}
\end{minipage}\\[3cm]

\setcounter{footnote}{0}
\stepcounter{footnote}
  \footnotetext{\href{mailto:\andsunds}{\texttt{\andsunds}}}
\stepcounter{footnote}
  \footnotetext{\href{mailto:\rigon}{\texttt{\rigon}}}



%------------------------------------------------------------------------------------
%	DATE SECTION
%------------------------------------------------------------------------------------
% Följer ISO-standarden för tidsintervall:
% https://en.wikipedia.org/wiki/ISO_8601#Time_intervals
% "Double hyphen" också ok istället för '/'. -- i LaTeX är dock lite på gränsen
{ \large
\begin{tabular}{rc}
    Laboration utförd: & 2015-12-11/15 \\[0.1cm]
    Rapport inlämnad: & \today
\end{tabular}\\[1cm]
}

%------------------------------------------------------------------------------------
%	LOGO SECTION
%------------------------------------------------------------------------------------

\includegraphics[height=5cm]{logo.pdf} % Include a department/university logo
 
%------------------------------------------------------------------------------------

\vfill % Fill the rest of the page with whitespace

\end{titlepage}
\restoregeometry


\setcounter{page}{2}%detta är ANDRA (2) sidan

\renewcommand{\abstractname}{Sammandrag}
\begin{abstract}
Vi har utfört en spektroskopisk studie av kvicksilveratomens atomära emissionsspektrum ur vilket vi kartlagt atomens energinivåer baserat på vår spektrometri. Vi har också studerat absorption i lösningar av två laserfärgämnen vid namn Rhodamin B och Kumarin 307. Mätningarna har utförts med en Spex 270M spektrometer och datainsamlingen har gjorts i LabView. Kvicksilvrets emissionsspektra är taget i intervallet 365 till 984 nm, där vi detekterat totalt 23 signifikanta emissionstoppar. Detta jämförs med NIST data där vi har 4 av 5 överlappningar med NIST ''persistent lines'' och 7 av 21 överlappningar med NIST ''strong lines''. Absorptionsspektra för Rhodamin B och Kumarin 307 visar breda absorptionsband vilket är kännetecknande för flourescerande ämnen som består av stora organiska molekyler.
\end{abstract}

\renewcommand{\abstractname}{Abstract}
\begin{abstract}

We have conducted a spectroscopic study of the emission spectrum of Mercury atoms and derived an energy level diagram based on these measurements. We have also studied the absorption spectrum of laser dye solutions of the compounds Rhodamine B and Coumarine 307. The measurements have been taken with a Spex 270M spectrometer and have been processed and recorded in LabView. The emission spectrum of Mercury has been recorded within the range of 365 to 984 nm, where we have detected a total of 23 significant emission peaks. This is contrasted with NIST data where we have 4 out of 5 overlaps with NIST ''persistent lines'' and 7 out of 21 overlaps with NIST ''strong lines''. The absorption spectrum for Rhodamine B and Coumarine 307 show broad absorption bands which are characteristic of flourescent compounds which consist of large organic molecules.

\end{abstract}

\clearpage
\renewcommand{\contentsname}{Innehållsförteckning}
\tableofcontents

\clearpage
\pagenumbering{arabic}
\setcounter{page}{1}

\renewcommand{\thefootnote}{\arabic{footnote}}
\setcounter{footnote}{0}



%%%%%%%%%%%%%%%%%%%% vvv Internal title page vvv %%%%%%%%%%%%%%%%%%%%%
\title{Gravitation and Cosmology -- FFM 071
\\ {\Large Hand-in set 1} }
\author{Andréas Sundström}
\date{\today}

\maketitle

%%%%%%%%%%%%%%%%%%%% ^^^ Internal title page ^^^ %%%%%%%%%%%%%%%%%%%%%
%% If you want a list of all todos
%\todolist


\renewcommand{\thesubsection}{\arabic{section} (\alph{subsection})}
\section{Some Lorentz transformations}
\subsection{Proper time}
To show that the proper time is invariant, we just use the definition
of $\rd\tau$ and the transformed coordinates 
$\rd{x'}^\alpha={\Lambda^\alpha}_\gamma \rd{x}^{\gamma}$,
\begin{equation}
\rd{\tau'}^2=-\eta_{\alpha\beta}\rd{x'}^{\alpha}\rd{x'}^{\beta}
=-\eta_{\alpha\beta}({\Lambda^\alpha}_\gamma \rd{x}^{\gamma})
({\Lambda^\beta}_\delta\rd{x}^{\delta})
=-\eta_{\alpha\beta}{\Lambda^\alpha}_\gamma{\Lambda^\beta}_\delta
\rd{x}^{\gamma}\rd{x}^{\delta}.
\end{equation}
next we use the condition that the Lorentz transform must satisfy
\begin{equation}
\eta_{\alpha\beta}{\Lambda^\alpha}_\gamma{\Lambda^\beta}_\delta
=\eta_{\gamma\delta},
\end{equation}
which therefore yields
\begin{equation}
\rd{\tau'}^2=\eta_{\gamma\delta}\rd{x}^{\gamma}\rd{x}^{\delta}
=\rd\tau^2.
\end{equation}
Since we have now concluded that $\rd\tau^2$ is a scalar (invariant)
under Lorentz transforms, so must $\rd\tau=\sqrt{\rd\tau^2}$.

\subsection{Relativistic particle velocity}
For a Lorentz boost with $\vb*v=(v,0,0)$, the transformations are as
follows
\begin{equation}
\begin{cases}
\rd{t}'=\gamma(\rd{t}-v\rd{x})\\
\rd{x}'=\gamma(\rd{x}-v\rd{t})\\
\rd{y}'=\rd{y}\\
\rd{z}'=\rd{z}
\end{cases}
\qand
\begin{cases}
\rd{t}=\gamma(\rd{t}'+v\rd{x}')\\
\rd{x}=\gamma(\rd{x}'+v\rd{t}')\\
\rd{y}=\rd{y}'\\
\rd{z}=\rd{z}'.
\end{cases}
\end{equation}
The velocity of a particle 
\begin{equation}
\vb*u=\qty(\dv{x}{t},\dv{y}{t},\dv{z}{t})
=(u_x,u_y,u_z)=(u\cos\theta,u\sin\theta,0)
\end{equation}
in the first frame, will transform as
\begin{equation}\label{eq1:upx1}
u'_x=\dv{x'}{t'}=\dv{x'}{t}\dv{t}{t'}
=\qty(\pdv{x'}{t}+\pdv{x'}{x}\pdv{x}{t})
\qty(\pdv{t}{t'}+\pdv{t}{x'}\pdv{x'}{t'})
=\gamma^2(-v+u_x)(1+vu'_x)
\end{equation}
and
\begin{equation}\label{eq1:upy1}
u'_y=\dv{y'}{t'}=\dv{y'}{t}\dv{t}{t'}
=\qty(\pdv{y'}{y}\pdv{y}{t})
\qty(\pdv{t}{t'}+\pdv{t}{x'}\pdv{x'}{t'})
=\gamma u_y(1+vu'_x).
\end{equation}
Solving for $u'_x$ in \eqref{eq1:upx1} yields
\begin{equation}
u'_x=\frac{u_x-v}{\gamma^{-2}-v(u_x-v)}
=\frac{u_x-v}{1-v^2-v(u_x-v)}
=\frac{u_x-v}{1-u_x v},
\end{equation}
similarly with \eqref{eq1:upy1}
\begin{equation}
u'_y=\gamma u_y\qty(1+v\frac{u_x-v}{1-u_x v})
=\frac{u_y}{\gamma(1-u_x v)}.
\end{equation}
The transform of $\theta$, the angle $\vb*u$ makes with the $x$ axis,
is easily given by 
\begin{equation}
\tan\theta'=\frac{u_y'}{u_x'}
=\qty(\frac{u\sin\theta}{\gamma(1-u\cos\theta v)})
\qty(\frac{1-u\cos\theta v}{u\cos\theta-v})
=\frac{\sin\theta}{\gamma(\cos\theta-\nicefrac{v}{u})}.
\end{equation}



\section{The stress tensor}
To show the conservation laws for the stress tensor\footnotemark{}
\footnotetext{I call the stress tensor of a particle
  ${T_n}^{\alpha\beta}$, with subscript $n$, and use $x_n(\tau)$ and
  $p_n$ for the trajectory and momentum of the particle, to more
  clearly distinguish what ``belongs'' to the particle and what are
  free variables.} 
\begin{equation}
{T_n}^{\alpha\beta}(x)=\int\rd\tau\,{p_n}^\alpha
\dv{{x_n}^\beta}{\tau}\delta^4\big(x-x_n(\tau)\big).
\end{equation}
The divergence of the stress tensor therefore becomes
\begin{equation}\label{eq2:pdvT1}
\pd_\beta{T_n}^{\alpha\beta}(x)=\int\rd\tau\,{p_n}^\alpha
\dv{{x_n}^\beta}{\tau}\,\pdv{x^\beta}\delta^4\big(x-x_n(\tau)\big).
\end{equation}
We note that the derivatives only act on the delta function since
$x_n$ and $p_n$ are the trajectory and momentum of the particle, while
the derivatives act on the free variable $x$. Next we note that for
any function $f$ 
\begin{equation}
\pdv{x}\Big[f(x-y)\Big]=-\pdv{y}\Big[f(x-y)\Big],
\end{equation}
which means that \eqref{eq2:pdvT1} can be written as
\vspace{-1em}
\begin{equation}
\begin{aligned}
\pd_\beta{T_n}^{\alpha\beta}(x)=&-\int\rd\tau\,{p_n}^\alpha
\overbrace{\dv{{x_n}^\beta}{\tau}\,\pdv{{x_n}^\beta}}^{=\dv*\tau}
\delta^4\big(x-x_n(\tau)\big)\\
=&-\int\rd\tau\,{p_n}^\alpha\dv{\tau}\delta^4\big(x-x_n(\tau)\big).
\end{aligned}
\end{equation}
Integration by parts now yields
\begin{equation}
\begin{aligned}
\pd_\beta{T_n}^{\alpha\beta}(x)=&
-\int\rd\tau\,{p_n}^\alpha\dv{\tau}\delta^4\big(x-x_n(\tau)\big)\\
=&-\cancel{\qty[{p_n}^\alpha\delta^4\big(x-x_n(\tau)\big)]}
+\int\rd\tau\,\dv{{p_n}^\alpha}{\tau}\delta^4\big(x-x_n(\tau)\big).
\end{aligned}
\end{equation}
The first term from the integration by parts cancel since the
integration is over ``all $\tau$'' and therefore $x_n\neq x$ at the
end points, which means that the delta function is zero.

Now if the particle interacts with an electromagnetic field, it will
experience a Lorentz force
\begin{equation}
\dv{{p_n}^\alpha}{\tau}=e_n\eta_{\beta\gamma}F^{\alpha\beta}
\dv{{x_n}^\gamma}{\tau}={F^{\alpha}}_\gamma \,e_n{u_n}^\gamma.
%={F^{\alpha}}_\gamma J^\gamma.
\end{equation}
This means that
\begin{equation}\label{eq2:pdvT}
\pd_\beta{T_n}^{\alpha\beta}(x)=\int\rd\tau\,{F^{\alpha}}_\gamma e_n{u_n}^\gamma
\delta^4\big(x-x_n(\tau)\big)
={F^{\alpha}}_\gamma e_n{u_n}^\gamma
\delta^3\big(\vb*x-\vb*x_n(\tau)\big)
={F^{\alpha}}_\gamma J^\gamma
\end{equation}
which in general is non-vanishing. Note that since we are assuming
that this particle is the only particle the current is indeed
$J^\alpha=e_n{u_n}^\alpha\delta^3\big(\vb*x-\vb*x_n(\tau)\big)$, and
if we were to have more particles all the above calculations would be
exactly the same (but with summation signs in front of them). 

We now take a look at the electromagnetic stress tensor
\begin{equation}
{T_{\EM}}^{\alpha\beta}={F^{\alpha}}_\gamma F^{\beta\gamma}
-\frac{1}{4}\eta^{\alpha\beta}F_{\gamma\delta}F^{\gamma\delta},
\end{equation}
and then Maxwell's equations will come in handy later
\begin{subequations}
\begin{equation}\label{eq:Maxwell}
\pd_\alpha F^{\alpha\beta}=-J^\beta,
\end{equation}
\begin{equation}\label{eq:Bianchi}
%\pd_{[\alpha}F_{\beta\gamma]}=
\pd_\alpha F_{\beta\gamma} +\pd_\beta F_{\gamma\alpha} +\pd_\gamma F_{\beta\alpha} 
=0.
\end{equation}
\end{subequations}
Its divergence therefore is
\begin{equation}
\begin{aligned}
\pd_{\beta}{T_{\EM}}^{\alpha\beta}=&
{F^{\alpha}}_\gamma \qty(\pd_\beta F^{\beta\gamma})
+\qty(\pd_\beta{F^{\alpha}}_\gamma) F^{\beta\gamma}
-\frac{1}{4}\eta^{\alpha\beta}\pd_\beta\qty[F_{\gamma\delta}F^{\gamma\delta}]\\
=&-{F^{\alpha}}_\gamma J^\gamma
+F^{\beta\gamma} \,\pd_\beta{F^{\alpha}}_\gamma
-\frac{1}{2}F_{\gamma\delta}\,\pd^\alpha F^{\gamma\delta}.
\end{aligned}
\end{equation}
The first term is given directly from \eqref{eq:Maxwell}.
We now take a look at the last two terms. Firstly we can raise and
lower the betas and gammas which are contracted in the second term, 
$F^{\beta\gamma} \,\pd_\beta{F^{\alpha}}_\gamma
=F_{\beta\gamma} \,\pd^\beta F^{\alpha\gamma}$, then in the last term
we are free to rename the two contracted indices $\gamma\to\beta$ and
$\delta\to\gamma$. We now have
\begin{equation}
F_{\beta\gamma} \,\pd^\beta F^{\alpha\gamma}
-\frac{1}{2}F_{\beta\gamma}\,\pd^\alpha F^{\beta\gamma}=
\frac{1}{2}\qty( F_{\beta\gamma}\pd^\beta F^{\alpha\gamma}
-F_{\beta\gamma}\pd^\beta F^{\gamma\alpha})
-\frac{1}{2}F_{\beta\gamma}\pd^\alpha F^{\beta\gamma}.
\end{equation}
Here we used the anti-symmetry of the EM tensor, 
$F^{\alpha\gamma}=-F^{\gamma\alpha}$. Now we do some more index
gymnastics by interchanging $\beta\leftrightarrow\gamma$ in the first
term, $F_{\gamma\beta}\pd^\gamma F^{\alpha\beta}
=-F_{\beta\gamma}\pd^\gamma F^{\alpha\beta}$. 
In the end what we have is
\begin{equation}
F_{\beta\gamma} \,\pd^\beta F^{\alpha\gamma}
-\frac{1}{2}F_{\beta\gamma}\,\pd^\alpha F^{\beta\gamma}=
-\frac{1}{2}F_{\beta\gamma}\qty( 
\pd^\gamma F^{\alpha\beta}
+\pd^\beta F^{\gamma\alpha}
+\pd^\alpha F^{\beta\gamma}).
\end{equation}
The expression inside the parentheses is just the contravariant version
of the RHS of \eqref{eq:Bianchi} and therefore vanishes.
So what we are left with for the divergence is
\begin{equation}
\pd_{\beta}{T_{\EM}}^{\alpha\beta}=-{F^{\alpha}}_\gamma J^\gamma,
\end{equation}
which is exactly $-\pd_\beta{T_n}^{\alpha\beta}$ from \eqref{eq2:pdvT}.

Therefore since $\pd_\beta$ is a linear operator we know that
\begin{equation}
\pd_{\beta}\qty[{T_{\EM}}^{\alpha\beta}+{T_n}^{\alpha\beta}]=0.
\end{equation}
This is physically reasonable since the total stress tensor, which
represents the total energy and momentum-current density, should
reasonably be conserved in a system. The two terms
${T_{\EM}}^{\alpha\beta}$ and ${T_n}^{\alpha\beta}$ are not conserved
separately since the particle and EM field exchanges energy and
momentum with each other, which also leads to the above conclusion
that there is no \emph{net} change in the total energy or momentum of
the system.





\renewcommand{\thesubsection}{\arabic{section} (\roman{subsection})}
\section{Time dialation}

\begin{figure}\centering
%\resizebox{.8\textwidth}{!}{
\input{figures/Laika.pdf_t}%}
\caption{}
\label{fig:}
\end{figure}



%%%%%%%%%%%%%%%%%%%%%%%%%%%%%%%%%%%%%%%%%%%%%%%%%%%%%%%%%%%%%%%%%%%%%%
\end{document}%% ^ ^ ^ ^ ^ ^ ^ ^ ^ ^ ^ ^ ^ ^ ^ ^ ^ ^ ^ ^ ^ ^ ^ ^ ^ ^ ^
%%%%%%%%%%%%%%%%%%%%%%%%%%%%%%%%%%%%%%%%%%%%%%%%%%%%%%%%%%%%%%%%%%%%%%




%%%%  Some (useful) templates


%% På svenska ska citattecknet vara samma i både början och slut.
%% Använd två apostrofer: ''.


%% Including PDF-documents
\includepdf[pages={1-}]{filnamn.pdf} % NO blank spaces in the file name

%% Figures (pdf, png, jpg, ...)
\begin{figure}\centering
\centerline{ % centers figures larges than 1\textwidth
\includegraphics[width=.8\textwidth]{file_name.pdf}
}
\caption{}
\label{fig:}
\end{figure}

%% Figures from xfig's "Combined PDF/LaTeX"
\begin{figure}\centering
\resizebox{.8\textwidth}{!}{\input{file_name.pdf_t}}
\caption{}
\label{fig:}
\end{figure}


%% If you want to add something to the ToC
%% (Without having an actual header in the text.)
\stepcounter{section} %For example a 'section'
\addcontentsline{toc}{section}{\Alph{section}\hspace{8 pt}Labblogg} 


%  LocalWords:  contravariant
