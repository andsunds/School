\chapter{CTRW och subdiffusion}
\todo{Sidnumrering i appendix stämmer inte?}

Continous time random walk är modell där rörelsen beskrivs av 
\begin{equation}
    X(t) = \sum_{i=1}^{n(t)} x_i
\end{equation}
där $x_i$ identiskt oberoende fördelade steg med sannolikhetsfördelning $p(x)$. Till skillnad mot en enklare slumpvandring så sker inte stegen med jämna tidsintervall, utan tiden mellan stegen är slumpmässiga med en sannolikhetsfördelning $\psi (t)$. Vidare betecknas sannolikheten att $n$ stycken steg har skett efter en tid $t$ som $P'(n,t)$ och $p_n(x)$ som sannolikheten att befinna sig vid $x$ efter $n$ stycken steg. Givet dessa definitioner är sannolikheten att befinna sig vid $x$ efter en tid $t$ precis 
\begin{equation}
\label{eq:P(x,t)}
    P(x,t) =\sum_{n=0}^{\infty} P'(n,t)p_n(x).
\end{equation}

För att beräkna godtyckliga moment av $X(t)$, av speciellt intresse är väntevärdet och variansen, så betraktar vi fouriertransformen av $P(x,t)$
\begin{equation}\label{eq:four}
\tilde{P}(k,t) = \int_{-\infty}^{\infty} P(x,t)\ee^{\ii kx}dx.
\end{equation}
Från fouriertransformen kan godtyckligt moment beräknas enligt 
\begin{equation}\label{eq:moment}
\ev{X(t)^n} = (-i)^n\frac{\pd^n \tilde{P}(k,t)}{\pd k^n}\big |_{k=0}
\end{equation}. 

Det visar sig att 
\begin{equation}
p_n(x) = p(x)*p(x)...*p(x),
\end{equation}
där $*$ betecknar faltning och $p(x)$ är faltad $n$ gånger med sig själv. \todo{motivera} Således är 
\begin{equation}\label{eq:Pfour}
P(k,t) =\sum_{n=0}^{\infty} P'(n,t)p(k)^n,
\end{equation}
där $p(k)$ är fouriertransformen av $p(x)$ och vi har utnyttjat faltning av två funktioner i $x$-rummet motsvarar multiplikation i $k$-rummet. Det underlättar nu genom att beskriva $P'(n,t)$ i termer av sannolikhetsfördelningen $\psi (t)$ som enligt motsvarar sannolikheten att ett steg sker efter en tid $t$. Detta görs enklast genom att notera att sannolikheten att inget steg sker under en tid $t$ är 
\begin{equation}\label{eq:nojump}
\Psi (t) = 1-\int_0^t \psi(t')dt' = \int_t^{\infty} \psi(t)dt. 
\end{equation}
Givet detta är sannolikheten att ett steg sker under en tid $t$
\begin{equation}
P'(1,t) = \int_0^t\psi(t')\Psi(t-t')dt',
\end{equation}
där vi alltså summerar sannolikheten att ett steg sker vid $t'$ och sedan inget mer under en tid $t-t'$ över alla $t'$. Således är $P'(1,t) = \psi(t)*\Psi(t)$ och från detta inses
\begin{equation}
P(n,t) = \psi(t)*\psi(t)*...*\psi(t)*\Psi(t).
\end{equation}
Detta uttryck kan förenklas genom ensidig laplacetransformera i $t$ enligt 
\begin{equation}
P'(n,s) = \int_0^\infty P'(n,t)\ee^{-st},
\end{equation}
vilket ger 
\begin{equation}
P(n,s) = \psi(s)^n\Psi(s),
\end{equation}
där $\psi(s)$ är laplacetransformen av $\psi(t)$ och likvärdigt för $\Psi(s)$ samt. Från definitionen \eqref{eq:nojump} fås 
\begin{equation}
\Psi(s) = \frac{1-\psi(s)}{s}.
\end{equation}
Vi är nu redo att presentera \eqref{eq:four} där vi utnyttjar laplacetransformen av $P(n,t)$ vilket ger 
\begin{equation}
P(k,s) = \sum_{n=0}^{\infty} \frac{1-\psi(s)}{s}[\psi(s)p(k)]^n,
\end{equation}
och slutligen med hjälp av en geometrisk summa får vi 
\begin{equation}\label{eq:master}
P(k,s) = \frac{1-\psi(s)}{s}\frac{1}{1-\psi(s)p(k)}.
\end{equation}

Genom att applicera \eqref{eq:moment} på detta samband kan väntevärdet samt variansen för $X(t)$ beräknas vilket visas nedan. Väntevärdet blir 
\begin{equation}
\ev{X(s)} = -i\frac{\pd \tilde{P}(k,s)}{\pd k}\big |_{k=0},
\end{equation}
som med \eqref{eq:master} ger 
\begin{equation}
\ev{X(s)} = \frac{1-\psi(s)}{s}\frac{\psi(s)(-i\frac{\pd p(k)}{\pd k})}{(1-\psi(s)p(k))^2}\bigg |_{k=0}.
\end{equation}
och ser vid en första anblick inte ut som framsteg. Men uttrycket förenklas genom att inse att 
\begin{equation}
(-i\frac{\pd p(k)}{\pd k})\bigg |_{k=0} = \ev{x}, \quad p(k)\bigg |_{k=0} = 1. 
\end{equation}
Väntevärdet blir således 
\begin{equation}\label{eq:medel}
\ev{X(s)} = \frac{\psi(s)\ev{x}}{s(1-\psi(s))},
\end{equation}
med antagandet $\ev{x}\equiv l< \infty$. Analogt kan variansen beräknas enligt 
\begin{equation}
\ev{X(s)^2} = -\frac{\pd^2 \tilde{P}(k,s)}{\pd k^2}\bigg |_{k=0},
\end{equation}
vilket efter en liten exercis ger 
\begin{equation}\label{eq:var}
\ev{X(s)^2} = \frac{\ev{x^2}\psi(s)}{s(1-\psi(s))}+\frac{2\ev{x}\psi(s)}{s(1-\psi(s))^2},
\end{equation}
där $\ev{x^2}\equiv \sigma_x^2< \infty$. Vi har således uttryck väntevärdet samt variansen för $X(s)$ i termer av transformen av sannolikhetsfördelningen $\psi(t)$. 

\todo[inline]{exempel på normal samt anomal diffusion kommer nedan.}
\section{Normal diffusion}
Vid normal diffusion är väntetiden $\tau\equiv\ev{t}$ finit och väntevärde samt varians kan beräknas i gränsen då $t\to\infty$ vilket är ekvivalent med $s\to0$. I denna gräns gäller sannolikhetsfördelning 
\begin{equation}
\psi(s) \sim 1-\tau+s+\frac{\ev{t^2}s^2}{2} 
\end{equation}

