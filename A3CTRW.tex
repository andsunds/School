\newcommand{\Pk}{\widehat{P}}
\newcommand{\Pks}{\widetilde{\widehat{P}}}
\newcommand{\pk}{\hat{p}}
\newcommand{\PPs}{\widetilde{P'}}
\newcommand{\psis}{\widetilde{\psi}}
\newcommand{\PSIs}{\widetilde{\Psi}}


\chapter{Fördjupning i CTRW och subdiffusion}


\label{app:CTRW} 
Följande bilaga följer till stor del från \cite{Chu_2003} och \cite{Vahey_2006}. Continuous time random walk är modell där rörelsen för till exempel en partikel beskrivs av 
\begin{equation}
    X(t) = \sum_{i=1}^{n(t)} x_i,
\end{equation}
där $x_i$ är identiskt oberoende fördelade steg med sannolikhetsfördelning $p(x)$. Till skillnad mot en enklare slumpvandring så sker inte stegen med jämna tidsintervall, utan tiden mellan stegen är slumpmässiga och följer en sannolikhetsfördelning $\psi (t)$, se \figref{fig:CTRW}. Hädanefter antas att $\psi(t)$ är identiska samt oberoende för alla steg, det finns teorier som studerar så kallad ''aging CTRW''\cite{Barkai_ACTRW2002} där detta inte gäller. Vidare betecknas sannolikheten att $n$ stycken steg har skett efter en tid $t$ som $P'(n,t)$ och $p_n(x)$ som sannolikheten att befinna sig vid $x$ efter $n$ stycken steg. Givet dessa definitioner är sannolikheten att befinna sig vid $x$ efter en tid $t$ precis 
\begin{equation}
\label{eq:P(x,t)}
    P(x,t) =\sum_{n=0}^{\infty} P'(n,t)p_n(x).
\end{equation}

För att beräkna godtyckliga moment av $X(t)$, av speciellt intresse är väntevärdet och andramomentet, så betraktas fouriertransformen av $P(x,t)$
\begin{equation}\label{eq:four}
\Pk (k,t) = \int_{-\infty}^{\infty} P(x,t)\ee^{\ii kx}\id{x}.
\end{equation}
Från fouriertransformen kan godtyckligt moment beräknas enligt 
\begin{equation}\label{eq:moment}
\ev{X(t)^n} = (-i)^n \eval{\frac{\pd^n \Pk (k,t)}{\pd k^n}}_{k=0}
\end{equation}. 

Sannolikheten för att nå $x$ efter $n$ antal steg visar sig beskrivas av en faltning $p(x)^{*n}$, där notationen betyder 
\begin{equation}
p_n(x) = p(x)*p(x)*\ldots*p(x),
\end{equation}
%\begin{comment}
\begin{figure}
    \centering
    \input{bilder/bilaga/ctrw.pdf_t}
    \caption{En realisation av CTRW där partikeln hoppar från $x_0$ till $x_1$ efter en tid $t_1$ och till $x_2$ efter en tid $t_2$. Till skillnad mot enkel slumpvandring då $t_1\equiv t_2$ så är i allmänhet $t_1\neq t_2$ för CTRW.}
    \label{fig:CTRW}
\end{figure}
%\end{comment}
$*$ betecknar faltning och $p(x)$ är faltad $n$ gånger med sig själv. Vilket inses lättast genom att betrakta två steg
\begin{equation}\label{eq:nfaltning}
    p_2(x) = \int_{-\infty}^{\infty} p(x-x')p(x')\id{x'},
\end{equation}
där man summerar över alla $x'$ sannolikheten att första steget sker till en position $x'$ och andra steget till $x$ från $x'$. Från detta inses det att \eqref{eq:nfaltning} kan generaliseras för $n$ antal steg enligt
\begin{equation}\label{eq:Pfour}
\Pk (k,t) =\sum_{n=0}^{\infty} P'(n,t)\pk (k)^n,
\end{equation}
där $\pk (k)$ är fouriertransformen av $p(x)$ och egenskapen att faltning i $x$-rummet motsvarar multiplikation i $k$-rummet använts. Det underlättar nu att beskriva $P'(n,t)$ i termer av sannolikhetsfördelningen $\psi (t)$ som enligt tidigare motsvarar sannolikheten att ett steg sker efter en tid $t$. Detta görs enklast genom att notera att sannolikheten att inget steg sker under en tid $t$ är 
\begin{equation}\label{eq:nojump}
\Psi (t) \equiv 1-\int_0^t \psi(t')\id{t'} = \int_t^{\infty} \psi(t')\id{t'}. 
\end{equation}
Givet detta är sannolikheten att ett steg sker under en tid $t$
\begin{equation}\label{eq:psifalt}
P'(1,t) = \int_0^t\psi(t')\Psi(t-t')\id{t'},
\end{equation}
där sannolikheten att ett steg sker vid $t'$ och sedan inget mer under en tid $t-t'$ summeras över alla $t'$. Sannolikhetsfördelningen $\psi (t)$ väljs till att vara identiskt noll för $t<0$. Således är \eqref{eq:psifalt} ekvivalent med $P'(1,t) = \psi(t)*\Psi(t)$ och generaliserar till
\begin{equation}
P'(n,t) = \psi(t)^{*n}*\Psi(t).
\end{equation}
Detta uttryck kan förenklas genom ensidig laplacetransformera i $t$ enligt 
\begin{equation}
\PPs (n,s) = \int_0^\infty P'(n,t)\ee^{-st}\id{t},
\end{equation}
vilket ger 
\begin{equation}
\PPs (n,s) = \psis(s)^n\PSIs(s),
\end{equation}
där $\psi(s)$ är laplacetransformen av $\psi(t)$ och likvärdigt för $\Psi(s)$. Från definitionen \eqref{eq:nojump} fås 
\begin{equation}
\PSIs(s) = \frac{1-\psis (s)}{s}.
\end{equation}
Vi är nu redo att presentera \eqref{eq:four} där laplacetransformen av $P(n,t)$ utnyttjas och ger 
\begin{equation}
\Pks (k,s) = \sum_{n=0}^{\infty} \frac{1-\psis (s)}{s}[\psis (s)\pk (k)]^n,
\end{equation}
och slutligen genom att notera att detta är en geometrisk summa \todo[color=red, bordercolor=cyan,linecolor=orange]{motivera konvergens}
\begin{equation}\label{eq:master}
\Pks (k,s) = \frac{1-\psis (s)}{s(1-\psis (s)\pk (k))}.
\end{equation}
Detta är Montroll-Weiss ekvationen för CTRW.

Genom att applicera \eqref{eq:moment} på detta samband kan väntevärdet samt variansen för $X(t)$ beräknas. Väntevärdet blir 
\begin{equation}
\ev{X(s)} = -i\frac{\pd \Pks (k,s)}{\pd k}\bigg |_{k=0},
\end{equation}
som med \eqref{eq:master} ger 
\begin{equation}
\ev{X(s)} = \frac{1-\psis (s)}{s}\frac{\psis (s)(-i\frac{\pd \pk (k)}{\pd k})}{(1-\psis (s)\pk (k))^2}\bigg |_{k=0}.
\end{equation}
och ser vid en första anblick inte ut som framsteg. Men uttrycket förenklas eftersom
\begin{equation}
-i\frac{\pd \pk (k)}{\pd k}\bigg |_{k=0} = \ev{x}, \quad \pk (k)\big |_{k=0} = 1,
\end{equation}
där $\ev{x}$ väntevärdet av stegen och antas vara definierat.
Väntevärdet blir således 
\begin{equation}\label{eq:medels}
\ev{X(s)} = \frac{\psis (s)\ev{x}}{s(1-\psis (s))}.
\end{equation}
Analogt kan andramomentet beräknas enligt 
\begin{equation}
\ev{X(s)^2} = -\frac{\pd^2 \Pks (k,s)}{\pd k^2}\bigg |_{k=0},
\end{equation}
vilket efter en liten exercis ger
\begin{equation}\label{eq:vars}
\ev{X(s)^2} = \frac{\ev{x^2}\psis (s)}{s(1-\psis (s))}+\frac{2\ev{x}^2\psis (s)}{s(1-\psis (s))^2},
\end{equation}
där $\ev{x^2}< \infty$. Således kan väntevärdet samt andramomentet för $X(s)$ beskrivas i termer av transformen av sannolikhetsfördelningen $\psi(t)$. Både väntevärdet och andramomentet kan uttryckas något mer intuitivt i termer av väntevärde samt andramoment av antalet steg. Detta ses enligt följande 
\begin{equation}
   \ev{N(s)} = \frac{1-\psis (s)}{s}\sum_{N=0}^{\infty}N\psis (s)^N,
\end{equation}
som kan skrivas om som 
\begin{equation}
   \ev{N(s)} = \frac{1-\psis (s)}{s}\psis (s)\frac{\pd}{\pd \psis}\sum_{N=0}^{\infty}\psis (s)^N
\end{equation}
vilket efter lite algebra ger 
\begin{equation}
   \ev{N(s)} = \frac{\psis (s)}{s(1-\psis (s)}.
\end{equation}
Således kan \eqref{eq:medels} även skrivas som $\ev{X(s)}=\ev{N(s)}\ev{x}$, och väntevärdet för positionen som funktion av tid 
\begin{equation}\label{eq:medelt}
   \ev{X(t)} = \ev{N(t)}\ev{x}.
\end{equation}
Analogt ses att $\ev{N(s)^2}$ kan skrivas som
\begin{equation}
   \frac{1-\psis (s)}{s}\sum_{N=0}^{\infty}N^2\psis (s)^N = \frac{1-\psis (s)}{s}\bigg (\psis (s)\frac{\pd}{\pd \psis} \bigg )^2\sum_{N=0}^{\infty}\psis (s)^N,
\end{equation}
vilket kan förenklas till 
\begin{equation}
   \ev{N(s)^2} = \frac{\psis (s)}{s(1-\psis (s))}+\frac{2\psis (s)^2}{s(1-\psis (s))^2}.
\end{equation}
Givet detta kan \eqref{eq:vars} uttryckas som 
\begin{equation}
   \ev{X(s)^2} = \ev{N(s)}\ev{x^2}+\bigg (\ev{N(s)^2}-\ev{N(s)}\bigg )\ev{x}^2,
\end{equation}
och således
\begin{equation}\label{eq:vart}
      \ev{X(t)^2} = \ev{N(t)}\ev{x^2}+\bigg (\ev{N(t)^2}-\ev{N(t)}\bigg )\ev{x}^2.
\end{equation}






\section{Normal diffusion}
Vid normal diffusion är väntevärdet av tiden mellan steg $\tau\equiv\ev{t}$ finit och väntevärde samt varians för $X(t)$ kan beräknas i gränsen då $t\to\infty$ vilket är ekvivalent med $s\to0$. I denna gräns gäller för sannolikhetsfördelning 
\begin{equation}
\psis (s) \sim 1-\tau s.
\end{equation}
Givet denna fördelning fås från \eqref{eq:medels}
\begin{equation}
   \ev{X(s)}\sim \frac{\ev{x}(1-\tau s)}{\tau s^2}\sim\frac{\ev{x}}{\tau s^2}.
\end{equation}
På samma sätt ger \eqref{eq:vars}

Givet normal diffusion samt $\ev{x}=0$ fås \todo[color=red]{ska fyllas på här. }

\section{Subdiffusion}

Normal diffusion ses erhållas från CTRW när väntevärdet av tiden mellan steg är finit, $\ev{t}<\infty$. Det visar sig att subdiffusion erhålls om väntetiden mellan steg inte är definierad vilket demonstreras nedan. Antag att sannolikhetsfördelningen för väntetiderna i gräns mot stora tider blir 
\begin{equation}
\psi(t) \sim \frac{1}{t^{\alpha+1}}\quad 0<\alpha<1,
\end{equation}
vilket är en sannolikhetsfördelning med odefinerad väntetid. För att evaluera såväl väntevärde som andramoment nedan kommer Tauberians teorem\cite{Feller_prob1971} användas och lyder för $\alpha\geq0$:

\begin{equation}\label{eq:tauber}
   \psis (s)\underset{s\to 0}{\sim}\frac{1}{s^{\alpha+1}}\Longleftrightarrow \psi(t)\underset{t\to\infty}{\sim}\frac{t^{\alpha}}{\Gamma (\alpha+1)}
\end{equation}

Medelvärdet $\ev{X(s)}$ när $s\to 0$ för denna fördelning beräknas enligt \eqref{eq:medels} till
\begin{equation}
    \ev{X(s)} \sim \frac{1-As^{\alpha}}{As^{\alpha+1}} \sim \frac{1}{As^{\alpha+1}},
\end{equation}
och med \eqref{eq:tauber}
\begin{equation}
    \ev{X(t)} \sim \frac{\ev{x}t^{\alpha}}{A\Gamma (1+\alpha)}
\end{equation}
där $\Gamma (1+\alpha)$ är gammafunktionen. Således om väntevärdet av steglängden är nollskild är väntevärdet för positionen proportionell mot $t^{\alpha}$. Motsvarande beräkningar för andramoment fås från \eqref{eq:vars}
\begin{equation}
   \ev{X(s)^2} \sim \frac{(1-As^{\alpha})\ev{x^2}}{As^{\alpha+1}}+\frac{2(1-As^{\alpha})^2\ev{x}^2}{A^2s^{2\alpha+1}}.
\end{equation}

Betrakta två fall: $\ev{x}\neq0$ samt $\ev{x}=0$, för det första av dessa gäller 
\begin{equation}
   \ev{X(s)^2} \sim \frac{2\ev{x}^2}{A^2s^{2\alpha+1}}
\end{equation}
och enligt Tauberians teorem 
\begin{equation}
   \ev{X(t)^2} \sim \frac{2\ev{x}^2t^{2\alpha}}{A^2\Gamma (2\alpha+1)}.
\end{equation}
Variansen $\sigma_X^2 = \ev{X(t)^2}-\ev{X(t)}^2$ blir 
\begin{equation}
    \sigma_X^2 \sim \frac{2\ev{x}^2t^{2\alpha}}{A^2\Gamma (2\alpha+1)}-\frac{\ev{x}^2t^{2\alpha}}{A^2\Gamma (1+\alpha)^2}
\end{equation}


Då $\ev{x}=0$ gäller istället 

\begin{equation}
   \ev{X(s)^2}\sim \frac{\ev{x^2}}{As^{\alpha+1}},
\end{equation}
och motsvarande med avseende på tid 

\begin{equation}
   \ev{X(t)^2} \sim \frac{\ev{x^2}t^{\alpha}}{A\Gamma (\alpha+1)}.
\end{equation}

Variansen då $\ev{x}=0$ ges av 
\begin{equation}
   \sigma_X^2 \sim \frac{\ev{x^2}t^{\alpha}}{A\Gamma (\alpha+1)}.
\end{equation}

Eftersom $0<\alpha<1$ ses att får båda fallen så kommer fördelningen $\psi(t)\sim \frac{1}{t^{\alpha+1}}$ leda till ett subdiffusivt beteende, rörelsen blir alltså långsammare än vid klassisk brownsk rörelse. Det subdiffusiva beteendet ses i denna modell härstamma från att väntetiden mellan stegen är odefinierad. 


%Bara en liten kodsnutt som behövs när man kompilerar lokalt
%%% Local Variables: 
%%% mode: latex
%%% TeX-master: "00main.tex"
%%% End: 